% ===========================================================================
% STANDARD OPERATING PROCEDURE — 22 Constance St Unit Trust
% Monthly Reporting Pack Preparation
% Version 3.7 — Revised March 2026 (Step 7 Appendix C cross-references enhanced; distribution component formulae improved with dual-min equivalence and bound sufficiency proof; CGT Discount vs CGT Concession terminology clarification added; callout box header overlap fixed; visual refinements applied. Appendix D legislative references verified and corrected: TR 97/7A2 consolidation date, Effective Life Determination 2025, GSTR 2006/10, TD 2018/9)
% ===========================================================================
\documentclass[11pt,a4paper,openany]{article}

% --- Fonts: Palatino (textbook-standard serif) + Helvetica (sans headings) ---
\usepackage[T1]{fontenc}
\usepackage[utf8]{inputenc}
\usepackage{palatino}          % Palatino body text — classic textbook serif
\usepackage{mathpazo}          % Matching math font
\usepackage[scaled=0.92]{helvet} % Helvetica for sans-serif (headings, account codes)
\usepackage[protrusion=true,expansion=true,tracking=true,kerning=true,spacing=true,final]{microtype}  % Full microtypography
\usepackage{setspace}          % Line spacing control

% --- Page Layout ---
\usepackage[top=2.8cm, bottom=2.8cm, left=2.8cm, right=2.8cm, headheight=26pt]{geometry}
\setstretch{1.18}              % Slightly opened line spacing for readability

% --- Structure & Navigation ---
\usepackage{titlesec}
\usepackage{titletoc}
\usepackage{enumitem}
\usepackage{parskip}

% --- Tables ---
\usepackage{booktabs}
\usepackage{longtable}
\usepackage{tabularx}
\usepackage{array}
\usepackage{multirow}
\usepackage{float}
\usepackage{caption}

% --- Visual ---
\usepackage{xcolor}
\usepackage{colortbl}
\usepackage{tcolorbox}
\tcbuselibrary{skins, breakable}
\usepackage{fancyhdr}
\usepackage{lastpage}
\usepackage{graphicx}

% ============================================================
% COMPANY LOGO — replace the filename below with your image
% Supported formats: PDF, PNG, JPG (PDF recommended for best
% print quality). Place the file in the same folder as this
% .tex file, or provide the full path.
%
%   TO USE:  \companylogo            → renders the logo
%   TO SWAP: change "company_logo"   → your actual filename
%            change height=0.65cm    → adjust size as needed
% ============================================================
\newcommand{\companylogo}{%
  \IfFileExists{company_logo.pdf}{%
    \includegraphics[height=0.65cm]{company_logo.pdf}%
  }{\IfFileExists{company_logo.png}{%
    \includegraphics[height=0.65cm]{company_logo.png}%
  }{\IfFileExists{company_logo.jpg}{%
    \includegraphics[height=0.65cm]{company_logo.jpg}%
  }{%
    % ── Fallback: text placeholder when no logo file is found ──
    \small\sffamily\textcolor{navy}{\textbf{22 Constance St Unit Trust}}%
  }}}%
}
\usepackage{amsmath}
\usepackage{amssymb}
\usepackage{ragged2e}
\usepackage{calc}
\usepackage{hyperref}
\usepackage{xurl}
\usepackage{pdflscape}
% pdflscape (via lscape) calls \thispagestyle{empty} on every landscape page,
% suppressing headers/footers. Patch the internal macro to use fancy instead.
\makeatletter
\def\LS@makecol{\setbox\@outputbox\vbox{\pagestyle{fancy}\box\@outputbox}}
\makeatother
\usepackage{tikz}
\usetikzlibrary{shapes.geometric, shapes.symbols, arrows.meta, positioning, fit, backgrounds, calc}

% --- TikZ layers: keep connectors behind nodes to avoid 'crossed-out' boxes ---
\pgfdeclarelayer{arrows}
\pgfsetlayers{background,arrows,main}
\usepackage{rotating}
\usepackage{appendix}

% ===========================
% COLOUR PALETTE — refined
% ===========================
\definecolor{navy}{HTML}{0F2240}        % deeper, richer navy
\definecolor{steel}{HTML}{2C5F8A}       % deeper steel blue
\definecolor{steellight}{HTML}{4A82B0}  % lighter steel for accents
\definecolor{sky}{HTML}{C8DDF0}
\definecolor{palesky}{HTML}{E8F2FA}
\definecolor{warmgray}{HTML}{F5F4F2}
\definecolor{midgray}{HTML}{D2D2D2}
\definecolor{lightgray}{HTML}{EBEBEB}
\definecolor{textgray}{HTML}{4A4A4A}
\definecolor{gold}{HTML}{C49A00}        % accent gold for badges/rules
\definecolor{goldbg}{HTML}{FDF8E6}
\definecolor{critred}{HTML}{A82020}
\definecolor{critbg}{HTML}{FDF0F0}
\definecolor{critleft}{HTML}{C62828}
\definecolor{warnamber}{HTML}{B86E00}
\definecolor{warnbg}{HTML}{FEF8ED}
\definecolor{infoteal}{HTML}{167A6A}
\definecolor{infobg}{HTML}{EAF7F4}
\definecolor{contrabg}{HTML}{FFF8E7}
\definecolor{contraborder}{HTML}{C49A00}
\definecolor{tipgreen}{HTML}{276128}
\definecolor{tipbg}{HTML}{F0F8E8}

% ===========================
% HYPERREF
% ===========================
\hypersetup{
    colorlinks=true,
    linkcolor=steel,
    urlcolor=steel,
    citecolor=steel,
    hypertexnames=false,
    pdftitle={SOP — Monthly Reporting Pack — 22 Constance St Unit Trust},
    pdfauthor={UFS Fund Accounting Team},
}

% ===========================
% SECTION HEADINGS — numbered badge style
% ===========================
% Section: number in navy filled badge, title in navy, gold underrule
\titleformat{\section}
    {\Large\sffamily\bfseries\color{navy}}
    {\colorbox{navy}{\textcolor{white}{\hspace{3pt}\thesection\hspace{3pt}}}}{0.7em}{}
    [\vspace{2pt}{\color{gold}\rule{\textwidth}{1.2pt}}\vspace{-4pt}]

% Subsection: steel-blue left accent bar with number
\titleformat{\subsection}
    {\large\sffamily\bfseries\color{navy}}
    {\textcolor{steel}{\rule[-2pt]{3pt}{12pt}}\hspace{5pt}\textcolor{steel}{\thesubsection}}{0.5em}{}

% Subsubsection: small numbered label with steel dash accent
\titleformat{\subsubsection}
    {\normalsize\sffamily\bfseries\color{steel}}
    {\textcolor{steellight}{\rule[-1pt]{2pt}{9pt}}\hspace{4pt}\textcolor{steellight}{\thesubsubsection}}{0.4em}{}

\titlespacing*{\section}{0pt}{2.4em}{1.1em}
\titlespacing*{\subsection}{0pt}{1.6em}{0.7em}
\titlespacing*{\subsubsection}{0pt}{1.1em}{0.4em}

% ===========================
% HEADER / FOOTER — refined
% ===========================
\pagestyle{fancy}
\fancyhf{}
\fancyhead[L]{\companylogo}
\fancyhead[C]{\small\sffamily\textcolor{steel}{\textbf{22 Constance St Unit Trust}}}
\fancyhead[R]{\small\sffamily\textcolor{textgray}{SOP \textbullet{} Monthly Reporting Pack}}
\fancyfoot[L]{\small\sffamily\textcolor{textgray}{Confidential \textbullet{} Internal Use Only}}
\fancyfoot[C]{\small\sffamily\textcolor{gold}{\rule{1.5cm}{0.4pt}\hspace{4pt}\textbf{\thepage\ of \pageref{LastPage}}\hspace{4pt}\rule{1.5cm}{0.4pt}}}
\fancyfoot[R]{\small\sffamily\textcolor{textgray}{v3.7 \textbullet{} March 2026}}
\renewcommand{\headrulewidth}{0pt}
\renewcommand{\footrulewidth}{0pt}
% Custom double-rule header line
\renewcommand{\headrule}{%
  \vspace{-2pt}%
  \hbox to\headwidth{\color{navy}\leaders\hrule height 1.4pt\hfill}%
  \vspace{1.5pt}%
  \hbox to\headwidth{\color{gold}\leaders\hrule height 0.5pt\hfill}%
}
\setlength{\headheight}{28pt}

% ===========================
% CALLOUT BOXES — polished with icon labels and accent stripes
% ===========================

% CRITICAL — red left band + top accent stripe
\newtcolorbox{critical}[1][]{
    enhanced, breakable,
    colback=critbg, colframe=critleft,
    boxrule=0pt, leftrule=5pt,
    borderline north={1pt}{0pt}{critleft!40!white},
    drop shadow southeast,
    fonttitle=\sffamily\bfseries\small\color{white},
    title={\raisebox{-0.5pt}{\textbf{!}}\enspace CRITICAL\ifx&#1&\else\textcolor{white!70!critleft}{\enspace\textbf{·}}\enspace #1\fi},
    attach boxed title to top left={xshift=8pt, yshift=-\tcboxedtitleheight/2},
    boxed title style={
        colback=critleft, colframe=critleft,
        rounded corners=all, left=6pt, right=6pt, top=2pt, bottom=2pt
    },
    top=14pt, bottom=6pt, left=8pt, right=8pt,
    before skip=26pt, after skip=10pt,
    rounded corners=south
}

% WARNING — amber left band
\newtcolorbox{warning}[1][]{
    enhanced, breakable,
    colback=warnbg, colframe=warnamber,
    boxrule=0pt, leftrule=5pt,
    borderline north={1pt}{0pt}{warnamber!40!white},
    drop shadow southeast,
    fonttitle=\sffamily\bfseries\small\color{white},
    title={{\bfseries\sffamily\textbf{$\blacktriangleright$}}\enspace WARNING\ifx&#1&\else\textcolor{white!70!warnamber}{\enspace\textbf{·}}\enspace #1\fi},
    attach boxed title to top left={xshift=8pt, yshift=-\tcboxedtitleheight/2},
    boxed title style={
        colback=warnamber, colframe=warnamber,
        rounded corners=all, left=6pt, right=6pt, top=2pt, bottom=2pt
    },
    top=14pt, bottom=6pt, left=8pt, right=8pt,
    before skip=26pt, after skip=10pt,
    rounded corners=south
}

% NOTE — teal left band
\newtcolorbox{note}[1][]{
    enhanced, breakable,
    colback=infobg, colframe=infoteal,
    boxrule=0pt, leftrule=5pt,
    borderline north={1pt}{0pt}{infoteal!35!white},
    drop shadow southeast,
    fonttitle=\sffamily\bfseries\small\color{white},
    title={\textbf{i}\enspace NOTE\ifx&#1&\else\textcolor{white!70!infoteal}{\enspace\textbf{·}}\enspace #1\fi},
    attach boxed title to top left={xshift=8pt, yshift=-\tcboxedtitleheight/2},
    boxed title style={
        colback=infoteal, colframe=infoteal,
        rounded corners=all, left=6pt, right=6pt, top=2pt, bottom=2pt
    },
    top=14pt, bottom=6pt, left=8pt, right=8pt,
    before skip=26pt, after skip=10pt,
    rounded corners=south
}

% REVISION NOTE / CONTRADICTION — gold left band
\newtcolorbox{contradiction}[1][]{
    enhanced, breakable,
    colback=contrabg, colframe=contraborder,
    boxrule=0pt, leftrule=5pt,
    borderline north={1pt}{0pt}{contraborder!40!white},
    drop shadow southeast,
    fonttitle=\sffamily\bfseries\small\color{white},
    title={\textbf{$\star$}\enspace REVISION NOTE --- Contradiction Identified and Corrected\ifx&#1&\else\textcolor{white!70!contraborder}{\enspace\textbf{·}}\enspace #1\fi},
    attach boxed title to top left={xshift=8pt, yshift=-\tcboxedtitleheight/2},
    boxed title style={
        colback=contraborder, colframe=contraborder,
        rounded corners=all, left=6pt, right=6pt, top=2pt, bottom=2pt
    },
    top=14pt, bottom=6pt, left=8pt, right=8pt,
    before skip=26pt, after skip=10pt,
    rounded corners=south
}

% TIP — green left band
\newtcolorbox{tip}[1][]{
    enhanced, breakable,
    colback=tipbg, colframe=tipgreen,
    boxrule=0pt, leftrule=5pt,
    borderline north={1pt}{0pt}{tipgreen!35!white},
    drop shadow southeast,
    fonttitle=\sffamily\bfseries\small\color{white},
    title={\textbf{$\checkmark$}\enspace TIP\ifx&#1&\else\textcolor{white!70!tipgreen}{\enspace\textbf{·}}\enspace #1\fi},
    attach boxed title to top left={xshift=8pt, yshift=-\tcboxedtitleheight/2},
    boxed title style={
        colback=tipgreen, colframe=tipgreen,
        rounded corners=all, left=6pt, right=6pt, top=2pt, bottom=2pt
    },
    top=14pt, bottom=6pt, left=8pt, right=8pt,
    before skip=26pt, after skip=10pt,
    rounded corners=south
}

% ===========================
% CUSTOM COMMANDS
% ===========================
\newcolumntype{L}[1]{>{\raggedright\arraybackslash}p{#1}}
\newcolumntype{R}[1]{>{\raggedleft\arraybackslash}p{#1}}
\newcolumntype{C}[1]{>{\centering\arraybackslash}p{#1}}

% Account code styling: small-caps sans-serif with a subtle sky-tinted pill background
\newcommand{\acct}[1]{%
  \mbox{\colorbox{sky!45}{\strut\,\textsf{\textbf{\small #1}}\,}}}

\newcommand{\thead}[1]{\cellcolor{navy}\textcolor{white}{\sffamily\bfseries\small #1}}
\newcommand{\tabhead}{\rowcolor{navy}\color{white}\sffamily\bfseries\small}

% --- Academic legal citation commands ---
\newcommand{\legalcite}[2]{%
  \textsuperscript{\sffamily\scriptsize\textcolor{navy}{[\textit{#1}; \S\,\ref{#2}]}}}
\newcommand{\legalciteonly}[1]{%
  {\sffamily\small\textcolor{navy}{[\textit{#1}]}}}
\newcommand{\applies}{\textcolor{infoteal}{\bfseries\checkmark}}
\newcommand{\notapplies}{\textcolor{textgray}{$\times$}}
\newcommand{\deadline}[1]{%
  \par\vspace{1pt}\noindent{\sffamily\scriptsize\textcolor{critleft}{\bfseries$\triangleright$\,#1}}}
\newcommand{\secref}[2]{\hyperref[#1]{\textcolor{navy}{#2}}}
\newcommand{\legbox}[1]{%
  \begin{tcolorbox}[
    enhanced, breakable,
    colback=navy!3!white, colframe=navy!50!white,
    borderline west={2.5pt}{0pt}{navy},
    borderline north={0.5pt}{0pt}{navy!25!white},
    drop shadow southeast,
    fontupper=\sffamily\small\color{navy!85!black},
    left=10pt, right=8pt, top=6pt, bottom=4pt,
    before skip=22pt, after skip=8pt,
    title={\sffamily\bfseries\small\color{white}\S\enspace Legislative \& Standards Framework},
    attach boxed title to top left={yshift=-2.5mm, xshift=6mm},
    boxed title style={
      colback=navy, colframe=navy,
      sharp corners, left=5pt, right=5pt, top=2pt, bottom=2pt
    }]
  #1
  \end{tcolorbox}\vspace{3pt}}

% ===========================
% PROCEDURE ENVIRONMENT — numbered steps in a styled steel-blue box
% Usage: \begin{procedure}[Optional title]  \step ... \step ...  \end{procedure}
% ===========================
\newcounter{stepcount}
\newcommand{\step}{\refstepcounter{stepcount}%
  \par\vspace{3pt}%
  \noindent\hspace{-2pt}%
  \makebox[0pt][r]{\colorbox{steel}{\textcolor{white}{\sffamily\bfseries\scriptsize\,\thestepcount\,}}\hspace{6pt}}%
  \ignorespaces}

\newtcolorbox{procedure}[1][]{
  enhanced, breakable,
  colback=palesky, colframe=steel,
  boxrule=0pt, leftrule=4pt,
  borderline north={0.5pt}{0pt}{steel!40!white},
  fontupper=\sffamily\small,
  left=28pt, right=8pt, top=8pt, bottom=8pt,
  title={\sffamily\bfseries\small\color{white}\raisebox{-0.5pt}{\textbullet}\enspace PROCEDURE\ifx&#1&\else\textcolor{white!70!steel}{\enspace\textbf{·}}\enspace #1\fi},
  attach boxed title to top left={xshift=8pt, yshift=-\tcboxedtitleheight/2},
  boxed title style={colback=steel, colframe=steel, sharp corners,
    left=6pt, right=6pt, top=2pt, bottom=2pt},
  before upper={\setcounter{stepcount}{0}},
  before skip=12pt, after skip=10pt,
  rounded corners=south
}

% ===========================
% GLOBAL LIST STYLING (enumitem)
% ===========================
\setlist{
  topsep=4pt,
  partopsep=0pt,
  parsep=0pt,
  itemsep=3pt,
}
\setlist[enumerate]{
  leftmargin=*,
  label=\textcolor{steel}{\bfseries\arabic*.},
  itemsep=4pt,
  topsep=4pt,
}
\setlist[itemize]{
  leftmargin=*,
  label=\textcolor{steel}{\small$\blacktriangleright$},
  itemsep=3pt,
  topsep=3pt,
}
\setlist[enumerate,2]{
  label=\textcolor{steellight}{\bfseries(\alph*)},
  leftmargin=2em,
}
\setlist[itemize,2]{
  label=\textcolor{steellight}{\footnotesize$\bullet$},
  leftmargin=2em,
}

% ===========================
% SECTION OVERVIEW / EXEC SUMMARY BOX
% Usage: \begin{scopebox}[Section label]  content  \end{scopebox}
% ===========================
\newtcolorbox{scopebox}[1][]{
  enhanced, breakable,
  colback=navy!4!white, colframe=navy,
  boxrule=0pt,
  borderline west={4pt}{0pt}{navy},
  borderline north={1pt}{0pt}{gold},
  borderline south={0.5pt}{0pt}{navy!20!white},
  fontupper=\sffamily\small\color{textgray},
  left=10pt, right=10pt, top=8pt, bottom=8pt,
  title={\sffamily\bfseries\small\color{white}%
    \textbullet\enspace SCOPE \& PURPOSE%
    \ifx&#1&\else\textcolor{gold!80!white}{\enspace\raisebox{0.5pt}{\textbf{---}}}\enspace #1\fi},
  attach boxed title to top left={xshift=8pt, yshift=-\tcboxedtitleheight/2},
  boxed title style={colback=navy, colframe=navy, sharp corners,
    left=6pt, right=6pt, top=2pt, bottom=2pt},
  before skip=10pt, after skip=12pt,
  rounded corners=south
}

% ===========================
% VISUAL PHASE DIVIDER — horizontal navy + gold double rule with label
% Usage: \phasedivider{Phase 1: Title}
% ===========================
\newcommand{\phasedivider}[1]{%
  \vspace{8pt}%
  \noindent\textcolor{navy}{\rule{\textwidth}{1.2pt}}\\[-4pt]
  \noindent\textcolor{gold}{\rule{\textwidth}{0.5pt}}%
  \par\vspace{2pt}%
  \noindent{\sffamily\small\bfseries\textcolor{navy}{#1}}%
  \vspace{4pt}%
}

% Caption styling — gold label prefix
\captionsetup{font=small, labelfont={sf,bf,color=navy}, textfont={color=textgray},
              format=plain, skip=6pt, labelsep=quad}
\captionsetup[table]{position=top}    % Table captions always above
\captionsetup[figure]{position=below} % Figure captions always below

% ===========================================================================
% MATH IDENTIFIER MACROS
% Purpose: every multi-letter financial/tax variable renders in upright sans-serif
% so it reads as a named quantity, not a product of italic single letters.
% Convention: \msf{...} = upright sans-serif label inside math mode.
% ===========================================================================
\newcommand{\msf}[1]{\mathsf{#1}}               % upright sans-serif for identifiers

% ── Tax loss framework ──────────────────────────────────────────────────────
\newcommand{\TL}{T\!L}                           % loss pool (short, upright feel via IT)
\newcommand{\TLnew}{\msf{TL}_{\msf{new}}}
\newcommand{\TLprior}{\msf{TL}_{\msf{prior}}}
\newcommand{\TLapplied}{\msf{TL}_{\msf{applied}}}
\newcommand{\TLrem}{\msf{TL}_{\msf{rem}}}
\newcommand{\TLremaining}{\msf{TL}_{\msf{rem}}}  % alias

% ── Ordinary income framework ────────────────────────────────────────────────
\newcommand{\OI}{\msf{OI}}
\newcommand{\OD}{\msf{OD}}
\newcommand{\NOI}{\msf{NOI}}
\newcommand{\NOIpos}{\msf{NOI}^{+}}
\newcommand{\TOI}{\msf{TOI}}

% ── Net taxable income ────────────────────────────────────────────────────────
\newcommand{\NetTI}{\msf{NetTI}}

% ── Capital gains framework ──────────────────────────────────────────────────
\newcommand{\NCG}{\msf{NCG}}
\newcommand{\GCGlong}{\msf{GCG}_{\msf{long}}}
\newcommand{\GCGshort}{\msf{GCG}_{\msf{short}}}
\newcommand{\CGlongnet}{\msf{CG}_{\msf{long,net}}}
\newcommand{\CGshortnet}{\msf{CG}_{\msf{short,net}}}
\newcommand{\CLlong}{\msf{CL}_{\msf{long}}}
\newcommand{\CLshort}{\msf{CL}_{\msf{short}}}
\newcommand{\SBC}{\msf{SBC}}

% ── Distribution components ──────────────────────────────────────────────────
\newcommand{\Cdisc}{C_{\msf{disc}}}
\newcommand{\COI}{C_{\msf{OI}}}
\newcommand{\CCG}{C_{\msf{CG}}}
\newcommand{\CTD}{C_{\msf{TD}}}
\newcommand{\CROC}{C_{\msf{ROC}}}
\newcommand{\CSBC}{C_{\msf{SBC}}}

% ── Bounds and accounting measures ──────────────────────────────────────────
\newcommand{\TDmax}{\msf{TD}_{\max}}             % \TDmax → TD_max
\newcommand{\ROCmax}{\msf{ROC}_{\max}}
\newcommand{\APpos}{AP^{+}}                      % AP+ stays italic — single letter is fine

% ===========================
% TABLE OF CONTENTS STYLING
% ===========================
\usepackage{tocloft}
\setcounter{tocdepth}{3}        % Include subsubsections in TOC
\setcounter{secnumdepth}{3}     % Number subsubsections
% Section entries: navy bold number, steel title, dotted leader
\renewcommand{\cftsecfont}{\sffamily\bfseries\color{navy}}
\renewcommand{\cftsecpagefont}{\sffamily\bfseries\color{navy}}
\renewcommand{\cftsecleader}{\cftdotfill{\cftdotsep}}
\renewcommand{\cftsecdotsep}{3}
% Subsection entries: steel, smaller
\renewcommand{\cftsubsecfont}{\sffamily\small\color{steel}}
\renewcommand{\cftsubsecpagefont}{\sffamily\small\color{steel}}
\renewcommand{\cftsubsecleader}{\cftdotfill{\cftdotsep}}
% Subsubsection entries: textgray, footnotesize
\renewcommand{\cftsubsubsecfont}{\sffamily\footnotesize\color{textgray}}
\renewcommand{\cftsubsubsecpagefont}{\sffamily\footnotesize\color{textgray}}
\renewcommand{\cftsubsubsecleader}{\cftdotfill{\cftdotsep}}
\setlength{\cftbeforesecskip}{5pt}
\setlength{\cftbeforesubsecskip}{1pt}
\setlength{\cftbeforesubsubsecskip}{0pt}
\setlength{\cftsubsubsecindent}{3.8em}
\setlength{\cftsubsubsecnumwidth}{3em}

% ============================================================================
\begin{document}

% ============================================================================
% TITLE PAGE
% ============================================================================
\begin{titlepage}
\thispagestyle{empty}

% Full-height left accent stripe via TikZ (behind content)
\begin{tikzpicture}[remember picture, overlay]
  % Deep navy left stripe
  \fill[navy] (current page.north west) rectangle ([xshift=1.1cm]current page.south west);
  % Gold thin inner edge on the stripe
  \fill[gold] ([xshift=1.0cm]current page.north west) rectangle ([xshift=1.1cm]current page.south west);
  % Top banner bar
  \fill[navy!95!black] ([yshift=-0pt]current page.north west) rectangle ([yshift=-1.1cm]current page.north east);
  % Gold under-line on top banner
  \fill[gold] ([yshift=-1.1cm]current page.north west) rectangle ([yshift=-1.18cm]current page.north east);
  % Bottom accent band
  \fill[navy!90!black] ([yshift=1.6cm]current page.south west) rectangle (current page.south east);
  \fill[gold] ([yshift=1.6cm]current page.south west) rectangle ([yshift=1.68cm]current page.south east);
  % --- Decorative geometric accent: corner chevron pattern (lower-right) ---
  \begin{scope}[opacity=0.06]
    \foreach \i in {0,1,...,8} {
      \draw[navy, line width=2.5pt]
        ([xshift={-1.8cm-\i*0.9cm}, yshift={2.2cm}]current page.south east)
        -- ([xshift={-0.3cm}, yshift={2.2cm+\i*0.9cm+1.5cm}]current page.south east);
    }
  \end{scope}
  % --- Subtle corner accent dot cluster (upper-right) ---
  \begin{scope}[opacity=0.04]
    \foreach \x in {0,...,5} {
      \foreach \y in {0,...,3} {
        \fill[navy] ([xshift={-1.5cm-\x*0.5cm}, yshift={-1.8cm-\y*0.5cm}]current page.north east) circle (1.5pt);
      }
    }
  \end{scope}
  % CONFIDENTIAL label in top bar
  \node[anchor=west, text=white, font=\sffamily\small\bfseries, inner sep=0pt]
    at ([xshift=1.5cm, yshift=-0.55cm]current page.north west)
    {CONFIDENTIAL \textbullet{} FOR INTERNAL USE ONLY};
  % Version pill top-right
  \node[anchor=east, fill=gold, text=navy, font=\sffamily\bfseries\footnotesize,
        rounded corners=3pt, inner xsep=7pt, inner ysep=3pt]
    at ([xshift=-0.4cm, yshift=-0.55cm]current page.north east)
    {Version 3.7 \textbullet{} March 2026};
\end{tikzpicture}

\vspace*{0.5cm}
\hspace*{1.6cm}%
\begin{minipage}{\dimexpr\textwidth-1.6cm}

% Logo
\noindent\companylogo\par
\vspace{1.0cm}

% Main title block
\noindent{\fontsize{34}{40}\selectfont\sffamily\bfseries\textcolor{navy}{Standard Operating\\[4pt]Procedure}}\par
\vspace{0.3cm}
\noindent{\fontsize{15}{20}\selectfont\sffamily\textcolor{steellight}{Monthly Reporting Pack Preparation}}\par
\vspace{0.25cm}
\noindent{\sffamily\small\textcolor{textgray}{Comprehensive month-end accounting, reconciliation, and financial reporting procedures}}\par
\vspace{0.7cm}

% Gold rule
\noindent\textcolor{gold}{\rule{9cm}{1.5pt}}\par
\vspace{0.5cm}

% Fund identity
\noindent{\fontsize{14}{18}\selectfont\sffamily\bfseries\textcolor{navy}{22 Constance St Unit Trust}}\par
\vspace{0.1cm}
\noindent{\sffamily\small\textcolor{textgray}{Unlisted Unit Trust \textbullet{} 22--26A Constance Street, Fortitude Valley QLD 4006}}\par
\vspace{0.9cm}

% Fund details table
\renewcommand{\arraystretch}{1.35}
\begin{tabular}{@{}>{\sffamily\bfseries\small\color{navy}}l @{\hspace{1.4em}} >{\sffamily\small}l @{}}
ABN                 & 28 218 694 388 \\
TFN                 & 552 125 429 \\
Constitution Date   & 17 April 2019 \\
Trustee             & 22 Constance St Pty Ltd (wholly owned subsidiary of CPG) \\
Manager             & Centennial Property Group Pty Ltd (CPG), ACN 151 746 977 \\
Accounting Period   & 1 July to 30 June \\
Document Owner      & UFS Fund Accounting Team \\
\end{tabular}

\vspace{0.9cm}

% Key Personnel box — refined
\begin{tcolorbox}[
    enhanced,
    colback=palesky, colframe=steel,
    boxrule=0pt, leftrule=4pt,
    borderline north={0.5pt}{0pt}{steel!30!white},
    sharp corners,
    left=10pt, right=10pt, top=8pt, bottom=8pt,
    title={\sffamily\bfseries\small\color{white}\raisebox{-0.5pt}{\textbullet}\enspace Key Personnel},
    attach boxed title to top left={xshift=8pt, yshift=-\tcboxedtitleheight/2},
    boxed title style={colback=steel, colframe=steel, sharp corners,
      left=5pt, right=5pt, top=2pt, bottom=2pt}
]
\renewcommand{\arraystretch}{1.25}
\begin{tabular}{@{}>{\sffamily\bfseries\small\color{steel}}l @{\hspace{1.2em}} >{\sffamily\small}l @{}}
Preparer (UFS)            & Anthony Tran \\
1st Reviewer (UFS)        & Pantea Asgari \\
2nd Reviewer (UFS)        & Sonya Beatriz \\
Client Coordinator (UFS)  & Kim Tran \\
Client Contact (CPG)      & William Cassimatis --- wcassimatis@centennial.com.au \\
Client Contact (CPG)      & Ally Hill --- ahill@centennial.com.au \\
\end{tabular}
\end{tcolorbox}

\end{minipage}

\end{titlepage}

% ============================================================================
% TABLE OF CONTENTS
% ============================================================================
\pagestyle{empty}
\vspace*{0.2cm}
{\noindent\sffamily\Huge\bfseries\textcolor{navy}{Contents}\par}
\vspace{0.15cm}
\noindent\textcolor{gold}{\rule{3.5cm}{1.5pt}}\hspace{0.3em}\textcolor{navy}{\rule{\dimexpr\textwidth-3.8cm}{0.5pt}}
\vspace{0.08cm}
{\noindent\sffamily\small\textcolor{steel}{\itshape Monthly Reporting Pack --- Standard Operating Procedure}}\par
\vspace{0.15cm}
\tableofcontents
\clearpage
\pagestyle{fancy}
\pagenumbering{arabic}
\setcounter{page}{1}



% ============================================================================
\section{Fund Background and Investment Overview}
\label{sec:background}
% ============================================================================

\subsection{Property Description}

22--26 Constance Street, Fortitude Valley is a character commercial office building comprising two adjoining structures. The smaller front building fronts Constance Street; the rear Sub Station (dating to 1928) has been converted into creative commercial space.

\begin{table}[H]
\centering\small\sffamily
\caption{Property summary as at 31 January 2026}
\begin{tabularx}{\textwidth}{@{}>{\bfseries}L{4.5cm} X@{}}
\rowcolor{navy}\textcolor{white}{\bfseries Attribute} & \textcolor{white}{\bfseries Detail} \\
\midrule
Total Lettable Area (NLA) & 2,592 sqm \\
\rowcolor{warmgray}
Total Land Area & 1,539 sqm \\
Acquisition Price & \$8.77M (\$10.8M inclusive of acquisition costs) \\
\rowcolor{warmgray}
Acquisition Date & 23 July 2019 \\
Acquisition \$/sqm & \$3,383 per sqm \\
\rowcolor{warmgray}
Initial Yield at Acquisition & 2.05\% \\
Market Yield at Acquisition & 11.02\% \\
\rowcolor{warmgray}
Most Recent Valuation & Independent valuation --- \$13,200,000 as at 31 December 2024 \\
IP Carrying Value & \$13,590,288 (31 January 2026) \\
\rowcolor{warmgray}
St George Loan & \$5,000,000 facility, 1.95\% p.a., matures 15 June 2026 \\
\bottomrule
\end{tabularx}
\end{table}

\begin{critical}[St George Loan Maturity — 15 June 2026]
The St George loan matures on \textbf{15 June 2026}. Loan refinancing or sale of the asset must be resolved before maturity. Monitor loan status monthly and escalate to CPG if no written refinancing confirmation has been received. For St George interest and line fee recording, refer to Section~\ref{sec:stgeorge-entries}; for borrowing cost amortisation, refer to Section~\ref{sec:borrowing-amort}; for loan maturity reclassification considerations at financial statement date, refer to Appendix~\ref{app:estimates}.
\end{critical}

% --- 1.2 Lease Schedule ---
\subsection{Current Lease Schedule}
\label{sec:leases}

The following table summarises all current leases for the property. Total leased area is 692 sqm of 2,592 sqm NLA ($\sim$26.7\% occupancy).

\begin{table}[H]
\centering\small\sffamily
\caption{Current lease register as at January 2026}
\label{tab:leases}
\begin{tabularx}{\textwidth}{@{}L{2.8cm} C{1.1cm} R{1.6cm} R{1.6cm} L{3.8cm} L{1.5cm}@{}}
\rowcolor{navy}\textcolor{white}{\bfseries\sffamily Tenant} & \textcolor{white}{\bfseries\sffamily Area (sqm)} & \textcolor{white}{\bfseries\sffamily Rent (\$/mth)} & \textcolor{white}{\bfseries\sffamily Rent (\$/yr)} & \textcolor{white}{\bfseries\sffamily Term} & \textcolor{white}{\bfseries\sffamily Type} \\
\midrule
5Point Projects & 200 & 7,500 & 90,000 & 28 October 2024 -- 27 October 2027 & Gross \\
\rowcolor{warmgray}
Popgun Labs & 156 & 7,793 & 93,520 & 16 December 2024 -- 15 December 2026 & Net \\
Sony Music Ent. & 336 & 14,797 & 177,563 & Month-to-month & Semi-gross \\
\midrule
\textbf{Total} & \textbf{692} & \textbf{30,090} & \textbf{361,083} & & \\
\bottomrule
\end{tabularx}
\end{table}

\begin{table}[H]
\centering\small\sffamily
\caption{Lease concessions and charge schedule detail}
\label{tab:concessions}
\begin{tabularx}{\textwidth}{@{}L{2.8cm} L{4.5cm} R{2cm} L{4cm}@{}}
\rowcolor{navy}\textcolor{white}{\bfseries\sffamily Tenant} & \textcolor{white}{\bfseries\sffamily Charge Item} & \textcolor{white}{\bfseries\sffamily Amount (\$/yr)} & \textcolor{white}{\bfseries\sffamily Period} \\
\midrule
5Point Projects & Rent --- Office & 120,000 & 28 October 2024 -- 27 October 2026 \\
\rowcolor{warmgray}
5Point Projects & Concession --- Rent Free & $-$30,000 & 28 October 2025 -- 27 October 2026 \\
Popgun Labs & Rent --- Office & 115,920 & 16 December 2025 -- 15 December 2026 \\
\rowcolor{warmgray}
Popgun Labs & Concession --- Rent Free & $-$22,400 & 16 December 2024 -- 15 December 2026 \\
Sony Music & Rent --- Office & 177,563 & 1 March 2024 -- 28 February 2026 \\
\rowcolor{warmgray}
Sony Music & OGS --- Outgoing Recoveries & 46,393 & 1 July 2025 -- 30 June 2026 \\
\midrule
\multicolumn{2}{@{}l}{\textbf{Total annual concessions (rent-free)}} & \textbf{$-$52,400} & \textbf{= \$4,366.67/month} \\
\multicolumn{2}{@{}l}{\textbf{Total annual outgoings recovery}} & \textbf{46,393} & \textbf{= \$3,866.07/month} \\
\bottomrule
\end{tabularx}
\end{table}

\textbf{Security deposits:} 5Point Projects --- bank guarantee \$33,000; Popgun Labs --- bank guarantee \$61,600.

\begin{warning}[Sony Music Month-to-Month Holdover Risk]
\textbf{Sony Music} is on a month-to-month holdover. The original lease expired 29 February 2024 and the current charge schedule ends 28 February 2026. This tenant represents 336 sqm (49\% of leased area) and 49\% of gross rental income. Any vacancy arising from Sony's departure would materially affect the fund's income and ICR covenant position. Monitor and escalate to CPG. For leasing fee amortisation review considerations when lease status changes, see Appendix~\ref{app:estimates}.
\end{warning}

\begin{note}[Concessions Tied Directly to Journal Entries]
The lease concession amounts tie directly to the rent journal entries: total concessions of \$52,400 p.a.\ = \$4,366.67/month, recorded as \acct{71.02} Rental Abatement. Outgoings recovery of \$46,393 p.a.\ (Sony only) = \$3,866.07/month, recorded as \acct{71.01}. This confirms the lease data is consistent with the journal entry framework.
\end{note}

% --- 1.3 Fund Status ---
\subsection{Current Fund Status --- Asset Sale in Progress}
\label{sec:sale-status}

As at January 2026, the fund is actively pursuing the sale of the property. This introduces additional accounting and reporting considerations. For CGT implications of the sale (including Event~A1 timing, cost base, Div~43 adjustments, and the put/call option structure), see Appendix~\ref{sec:est-cgt}. For the full accruals and cut-off checklist as the sale approaches exchange and settlement, see Appendix~\ref{sec:est-accruals-settlement}.

\subsubsection{Account 709 --- Deferred Cost of Sale}

Sale-related costs are accumulated as a balance sheet asset (not expensed) while the sale is active. Components at 31 January 2026:

\begin{itemize}[nosep]
    \item CBRE: Technical due diligence --- \$13,400.00
    \item MP Realty Trust: Marketing campaign --- \$15,306.36
    \item K\&L Gates: Legal fees --- \$17,846.47
    \item MP Realty Trust: Commission --- \$82,812.50
    \item \textbf{Total deferred cost of sale: \$133,731.53}
\end{itemize}

\subsubsection{Account 307 --- Aborted Sales Campaign Cost}

Costs from a previously aborted sales campaign were expensed through the P\&L. \$20,119.39 was recognised in July 2025.

\subsubsection{Upon Sale Completion}

All deferred costs of sale will be de-recognised from the balance sheet and recognised against the gain or loss on disposal.

\begin{warning}[Deferred Cost of Sale — Do Not Expense While Active]
Deferred cost of sale (\acct{709}) must be reviewed each month. Do not expense these items through the P\&L while the sale is active. Once the sale is either completed or formally abandoned, seek CPG direction on accounting treatment.
\end{warning}

% --- 1.4 Cyclone ---
\subsection{Cyclone Damage --- Insurance Claim and Remediation Capex}
\label{sec:insurance}

The property sustained significant cyclone damage. The following accounting treatment must be maintained consistently each reporting period.

\textbf{Insurance Claim Income.} Recognised in \acct{73.02}. Proceeds are received directly into \acct{680} --- St George by the insurer (Steadfast IRS / Vision Re Partners). YTD to January 2026: \$214,136.10, comprising: September 2025 (\$160,602.00), November 2025 (\$10,707.00), December 2025 (\$42,827.10).

\textbf{Remediation Capex.} All remediation works are capitalised to \acct{781.01} --- Property Improvements. Major items include: roof/ceiling repairs and asbestos removal (FHS --- multiple claims); Tenancy 1D balcony remediation; additional structural repairs (\$23,561 --- FHS, July 2025); fire detection system replacement --- Ampac Fire Finder (\$8,572 accrual June 2025; \$5,592 October 2025; \$2,980 December 2025); Handler Property Group project management (\$6,125.52, August 2025); FHS major progress claims (\$204,184 + \$26,741 + \$18,903.50, August 2025).

\begin{note}[Monthly Insurance Claim and Capex Procedure]
Each month: (1) confirm insurance claim receipts credited to St George (\acct{680}) and record in \acct{73.02} (see Phase~2 recording procedure, Section~\ref{sec:stgeorge-entries}); (2) capitalise all remediation works to \acct{781.01} on the Capex tab (payments via NAB); (3) update fair value disclosure once a new valuation is received (see Appendix~\ref{app:estimates} for revaluation considerations). Do \textbf{not} net insurance claims against remediation capex, and do \textbf{not} confuse insurance premium expenses (NAB) with insurance claim receipts (St George). For the comprehensive accounting, income tax, CGT, and GST treatment of insurance claims (including post-settlement retention and cost base interactions), see Appendix~\ref{sec:est-insurance}.
\end{note}

% --- 1.5 Strategy ---
\subsection{Investment Strategy}
\label{sec:strategy}

\begin{enumerate}[nosep]
    \item Acquire a recently refurbished, majority-vacant character building at a significant discount to market value.
    \item Execute an aggressive leasing and marketing campaign to achieve full occupancy.
    \item Sell the fully leased asset or refinance with strong ICR and WALE metrics.
\end{enumerate}


The fund is currently in the sale process notwithstanding that full occupancy has not been achieved. The sale is being pursued in the context of the approaching St George loan maturity (15 June 2026).

% --- 1.6 Fees ---
\subsection{Fee Structure}
\label{sec:mgmtfee}

The fee structure below defines the contractual basis for each fee type. For the period-end fee size indicators --- including how each fee's base metric fluctuates with changes in income, GAV, and IRR --- refer to Appendix~\ref{sec:est-fee-size}. The performance fee waterfall mechanics and their interaction with the distribution components are addressed in Appendix~\ref{sec:est-distribution-waterfall}.

\begin{table}[H]
\centering\small\sffamily
\caption{Fund fee structure}
\begin{tabularx}{\textwidth}{@{}>{\bfseries}L{4.5cm} X@{}}
\rowcolor{navy}\textcolor{white}{\bfseries Fee Type} & \textcolor{white}{\bfseries Basis} \\
\midrule
Acquisition Fee & 2\% of Gross Asset Value (GAV) \\
\rowcolor{warmgray}
Disposal Fee & 1\% of GAV \\
Annual Management Fee & 5\% of net receivable income (see note below) \\
\rowcolor{warmgray}
Performance Fee (Tier 1) & 20\% of all returns above 8\% IRR \\
Performance Fee (Tier 2) & 40\% of all returns above 15\% IRR \\
\bottomrule
\end{tabularx}
\end{table}



% ============================================================================
\section{Bank Accounts and Cash Structure}
\label{sec:banking}
% ============================================================================

\begin{critical}[NAB vs. St George — Primary Bank Distinction]
The trust's primary operating bank is \textbf{NAB}, which processes all property-level income and expenses. The St George account (\acct{680}) is the lending facility; St George Bank debits interest charges and line fees directly from this facility account. It also receives insurance claim proceeds (credited directly by the insurer) and holds the \$1,325,000 sale deposit. No property operating transactions (rent, opex) are processed through St George.
\end{critical}

\subsection{Account Register}

\begin{table}[H]
\centering\small\sffamily
\caption{Bank account register}
\begin{tabularx}{\textwidth}{@{}L{2.8cm} X@{}}
\rowcolor{navy}\textcolor{white}{\bfseries\sffamily Account} & \textcolor{white}{\bfseries\sffamily Description} \\
\midrule
\acct{680} --- St George & St George Bank lending facility. \$5,000,000 drawn at 1.95\% p.a.; matures 15 June 2026. Interest charges and line fees are debited directly from this facility account by St George Bank (they do \emph{not} flow through NAB). Also receives insurance claim proceeds from insurer (Steadfast IRS / Vision Re Partners) and holds the \$1,325,000 sale deposit (matched by \acct{884} liability). No ordinary property operating transactions are processed through St George. \\
\rowcolor{warmgray}
\acct{685} --- NAB & NAB operating account. Receives all rent remittances and processes all property-level and trust-level expenses. Balance at 31 Jan 2026: \$71,714.32. \\
\acct{686} --- STA QLD & State-managed trust account (Queensland). CPG inter-entity transfers. Balance at 31 Jan 2026: nil. \\
\bottomrule
\end{tabularx}
\end{table}

\subsection{Cash Reconciliation Requirements}

\begin{enumerate}[nosep]
    \item \acct{680} (St George): Reconcile per bank statement to per balance sheet. Zero difference expected.
    \item \acct{685} (NAB): Reconcile per NAB bank statement to per balance sheet. Zero difference expected.
    \item \acct{686} (STA QLD): Reconcile per Yardi TB to per balance sheet (see Appendix~\ref{sec:est-sta} for full STA legal, accounting, and management framework).
    \item All reconciliations are performed in the \textbf{Cash} tab.
    \item The \textbf{Yardi vs.\ Bank Stat Rec} tab provides detailed NAB bank statement line-by-line reconciliation against the Yardi GL. All timing differences must be documented with narrative comments.
\end{enumerate}

% ============================================================================
\section{Yardi Trial Balance Import Process}
\label{sec:yardi}
% ============================================================================

\subsection{Overview}

Yardi is the property management and accounting system used by CPG. At period end, the Preparer downloads two trial balance movement exports from Yardi: \texttt{cs4001} (property-level accounts) and \texttt{22chts} (trust-level accounts) for 22 Constance Street. These exports are imported into the \textbf{Import\_Yardi} tab.

\subsection{Import Process}

\begin{enumerate}[nosep]
    \item The Preparer downloads the Yardi TB movement reports from Yardi: \texttt{cs4001} (property-level) and \texttt{22chts} (trust-level), both in TRNS/SPL/ENDTRNS format.
    \item The Preparer imports the data into the \textbf{Import\_Yardi} tab, which feeds \textbf{Yardi\_TB} and \textbf{Yardi\_Income Statement} tabs.
    \item The data is posted as a general journal (memo: ``Yardi import''), creating movements across accounts including: \acct{685} --- NAB, \acct{686} --- STA QLD, \acct{703} --- Deferred Income, \acct{706/707} --- Prepayments (Operating \& Insurance), \acct{71} --- Rental Income, \acct{700} --- Property Operating Expenses, \acct{882.02} --- Accrued Operating, and \acct{255} --- GST.
    \item The Preparer reconciles the Yardi TB to the reporting pack TB. The \textbf{TB} tab consolidates the Yardi import with UFS manual journals.
    \item The \textbf{Yardi vs.\ Bank Stat Rec} tab reconciles the NAB bank statement against Yardi GL movements.
\end{enumerate}

\subsection{Reconciliation Codes --- Yardi vs.\ Bank Statement}

The following five timing-difference cases govern both the Yardi~vs.\ Bank Statement reconciliation (Step~2, Section~\ref{sec:yardi-rec}) and the NAB cut-off entries (Phase~2, Step~3, Section~\ref{sec:cutoff}). The same classification is used throughout to ensure consistency.

\begin{table}[H]
\centering\small\sffamily
\caption{Timing difference classification codes (five cases)}
\label{tab:fivecases}
\begin{tabularx}{\textwidth}{@{}C{0.8cm} L{5.5cm} X@{}}
\thead{Case} & \thead{Condition} & \thead{Treatment} \\
\midrule
1 & Transaction in current period Yardi TB \textbf{matches} transaction in current period bank statement & No timing difference. Imported via Step~6b (Section~\ref{sec:yardi-import}). No Sundry Debtor entry required. \\
\rowcolor{warmgray}
2 & Transaction in \textbf{previous period} Yardi TB (\acct{NAB}) matches transaction in \textbf{current period} bank statement & DR \acct{685} --- NAB \quad CR \acct{Sundry Debtor}. \\
3 & Transaction in \textbf{next period} Yardi TB (\acct{NAB}) matches transaction in \textbf{current period} bank statement & DR \acct{Sundry Debtor} \quad CR \acct{685} --- NAB. \\
\rowcolor{warmgray}
4 & Transaction on bank statement does \textbf{not match} any transaction in previous, current, or next period Yardi TB & Request bank information from CPG. Post manual entries based on the nature identified once clarified. \\
5 & Transaction in current period Yardi TB does \textbf{not match} current or previous period bank statement & Record as Sundry Debtor (DR or CR as applicable) and await the next period bank statement. If aging is unreasonably long, request the nature from CPG. \\
\bottomrule
\end{tabularx}
\end{table}

Each reconciling item must be coded with a case number and narrative. The reconciliation must balance to zero for the period to be complete.

% ============================================================================
\section{Reporting Pack Workbook Structure}
\label{sec:workbook}
% ============================================================================

The reporting pack is maintained in an Excel workbook. Principal tabs:

{\small\sffamily
\begin{longtable}{@{}L{3.2cm} L{5cm} L{5.5cm}@{}}
\caption{Reporting pack tab reference}\\
\rowcolor{navy}\textcolor{white}{\bfseries Tab Name} & \textcolor{white}{\bfseries Purpose} & \textcolor{white}{\bfseries Key Inputs} \\
\midrule
\endfirsthead
\rowcolor{navy}\textcolor{white}{\bfseries Tab Name} & \textcolor{white}{\bfseries Purpose} & \textcolor{white}{\bfseries Key Inputs} \\
\midrule
\endhead
\bottomrule
\endfoot

Import\_Yardi & Raw Yardi GL import (TRNS/SPL) & Source for TB, PL, BS \\
\rowcolor{warmgray}
Cover Page & Summary cover for CPG distribution & ABN, TFN, Dashboard \& NAV links \\
CPG Dashboard & Executive dashboard for Fund Month End Meeting & P\&L, BS, unit price, capex, fees \\
\rowcolor{warmgray}
NAV per unit & Net Asset Value per unit & NTA adjusted/unadjusted (11,438,500 units) \\
Tax reconciliation & Tax income reconciliation & Taxable income, deferred, capital return \\
\rowcolor{warmgray}
STI calc / STI Adj & Lease incentive amortisation & Feeds PL leasing fee line \\
Sec.\ 40--880 / 25--25 & Tax deduction schedules & Input to tax reconciliation \\
\rowcolor{warmgray}
Tax depreciation & Depreciation schedule (tax) & Input to tax reconciliation \\
Cost base \& carrying value & Investment cost base tracking & Disposal calcs, CGT \\
\rowcolor{warmgray}
PL / BS / CFS & Financial statements & TB, accruals, prepayments, journals \\
TB & Trial Balance (Yardi + UFS) & Import\_Yardi + manual entries \\
\rowcolor{warmgray}
COA & Chart of Accounts reference & Account mapping \\
Yardi vs.\ Bank Stat Rec & NAB Yardi GL vs.\ bank statement & NAB bank statement \\
\rowcolor{warmgray}
GL & General Ledger detail & Yardi import, manual entries \\
Cash & Cash reconciliation (\acct{680}, \acct{685}, \acct{686}) & Bank statements, Yardi TB \\
\rowcolor{warmgray}
Receivables & \acct{662}, \acct{660}, \acct{709} & NAB report; sale cost tracking \\
Capex & Capex log (\acct{781.01}, \acct{781.02}) & Cyclone remediation, fire system \\
\rowcolor{warmgray}
Prepaid leasing fee & Leasing fee amortisation (\acct{706.07}, \acct{780.12}) & STI calc, CPG invoices \\
Prepayments\_operating & Operating prepayments (\acct{706.01}--\acct{706.07}) & Land tax, rates, insurance \\
\rowcolor{warmgray}
Borrowing cost prepaid & Borrowing cost amortisation (\acct{708.03}, \acct{708.04}) & St George loan costs \\
Payables & Trade payables (\acct{883}) and accruals & CPG invoices, NAB report \\
\rowcolor{warmgray}
Loan\_St George & Loan balance and interest & \$5M @ 1.95\%; maturity 15 Jun 2026 \\
\rowcolor{warmgray}
Accruals / Detailed Accruals & Monthly accruals (\acct{882.02}, \acct{882.16}) & NAB report, CPG invoices \\
Unitholders & Register and equity ledger & \$11,438,500 capital, distributions \\
\rowcolor{warmgray}
BAS\_Qtr / GST ledger & Quarterly BAS and GST workings & PL, invoices \\
\end{longtable}
}

% ============================================================================
\section{Monthly Reporting Process}
\label{sec:process}
% ============================================================================

The monthly reporting cycle follows three sequential phases comprising ten procedural steps.


\begin{table}[H]
\centering\small\sffamily
\caption{Monthly reporting timeline}
\begin{tabularx}{\textwidth}{@{}C{0.7cm} L{2.5cm} X L{2.2cm}@{}}
\rowcolor{navy}\textcolor{white}{\bfseries\sffamily Step} & \textcolor{white}{\bfseries\sffamily Responsible} & \textcolor{white}{\bfseries\sffamily Action} & \textcolor{white}{\bfseries\sffamily Timing} \\
\midrule
\multicolumn{4}{@{}l}{\cellcolor{navy!8}\sffamily\bfseries\small\textcolor{navy}{\raisebox{0pt}[9pt][3pt]{\quad Phase 1 \textbullet{} Source Documents, Reconciliation and TB Import}}} \\
\midrule
1 & CPG & Uploads St George bank statement, bank information responses, and supporting invoices to SharePoint. & BD1 \\
\rowcolor{warmgray}
2 & Client Coordinator (UFS) & Downloads NAB bank statement; archives in UFS Shared Folder. & 20th \& BD1 \\
3 & Preparer & Downloads Yardi TB (\texttt{cs4001} and \texttt{22chts}) from Yardi. Downloads St George bank statement from SharePoint. Reconciles Yardi NAB TB vs.\ NAB bank statement. & 20th \& BD1 \\
\rowcolor{warmgray}
4 & Preparer & Submits bank information request to CPG; follows up unidentified items. & 20th \& BD1 \\
5 & CPG & Provides bank information responses and uploads supporting invoices to SharePoint. & 20th \& BD1 \\
\rowcolor{warmgray}
6 & Preparer & Revises Yardi TB (trust-level and sale/leasing reclassifications); imports revised TB. & BD1 \\
\midrule
\multicolumn{4}{@{}l}{\cellcolor{navy!8}\sffamily\bfseries\small\textcolor{navy}{\raisebox{0pt}[9pt][3pt]{\quad Phase 2 \textbullet{} Reporting Pack Compilation}}} \\
\midrule
7 & Preparer & Compiles all tabs: property level, trust level, NAV, dashboard, distribution workings. Records insurance claim income and cyclone capex. & BD2--BD4 \\
\midrule
\multicolumn{4}{@{}l}{\cellcolor{navy!8}\sffamily\bfseries\small\textcolor{navy}{\raisebox{0pt}[9pt][3pt]{\quad Phase 3 \textbullet{} Internal Review and Client Distribution}}} \\
\midrule
\rowcolor{warmgray}
8 & Preparer & Sends pack to 1st and 2nd Reviewer; clears all comments. & BD4 \\
9 & 1st/2nd Reviewer & Reviews and signs off checklists (NAV, BAS, Distribution). & BD5 \\
\rowcolor{warmgray}
10 & 1st Reviewer & Sends approved NAV pack and monthly report to CPG. & BD5 \\
\bottomrule
\end{tabularx}
\end{table}

% ============================================================================
\section{Phase 1: Source Documents, Reconciliation and TB Import}
\label{sec:phase1}
% ============================================================================

Phase 1 covers all preparatory work before reporting pack compilation: collecting source documents, reconciling the Yardi NAB trial balance against the NAB bank statement, submitting the bank information request to CPG, revising the Yardi TB, and importing the revised TB. Steps must be completed in sequence --- the Preparer reconciles the NAB bank statement against the Yardi TB \emph{before} revising and importing the TB. All Phase~1 work is completed before Phase~2 (Section~\ref{sec:phase2}) begins.

\subsection{Steps 1--2 --- Collect Source Documents}

\subsubsection{Property-Level Source Documents (NAB)}

The Client Coordinator (UFS) downloads the NAB bank statement and archives it in the UFS Shared Folder. The NAB bank statement covers all property-level cash transactions including: rental income remittances (commercial rent, outgoings recovery); fund administration expenses; development costs and cyclone remediation capex payments; professional expenses (legal, valuation, consulting); insurance premium expenses; and all property operating expenses (Handler Holdings --- property management; Energy Australia --- electricity; Office of State Revenue --- land tax; council rates; Queensland Urban Utilities --- water; Grolife --- garden maintenance; Premier Fire --- fire safety; AJT Prestige / JJ Waste --- cleaning; security fees; leasing commissions).

\begin{note}[Insurance Premium Expense vs. Claim Receipts]
Insurance premium \emph{expenses} are paid via NAB (\acct{685}). Insurance claim \emph{proceeds} are received into St George (\acct{680}) directly from the insurer --- these do not appear on the NAB bank statement.
\end{note}

\subsubsection{Trust-Level Source Documents (NAB and St George)}

Trust-level cash transactions on NAB include: BAS payments; distribution payments; CPG STA QLD transfers; BPAY payments; and sale commission payments (MP Realty Trust).

St George bank statement covers: interest and line fee payments (debited from the facility); insurance claim proceeds credited by insurer (Steadfast IRS / Vision Re Partners); and the \$1,325,000 sale deposit received.

CPG uploads the St George bank statement to SharePoint. The Preparer downloads it from SharePoint as needed for reconciliation.

\subsubsection{Trust-Level Expense Accounts (Accruals --- Non-Cash)}
\label{sec:trust-accrual-accounts}

Trust-level expenses fall into two distinct settlement categories. \textbf{Most} trust-level expenses (management fees, admin fees, professional fees, etc.) are \textbf{not processed through any bank account} at the time of accrual --- they are posted as manual accrual journals to \acct{882.16} --- Fund Accruals (Balance Sheet liability) each period, and settled when CPG arranges payment. This is distinct from property-level operating accruals (\acct{882.02}) which are imported via the Yardi TB. \textbf{Two exceptions} --- interest (\acct{303.01}) and line fees (\acct{303.02}) --- are settled directly through St George (\acct{680}) and recorded on a \textbf{cash basis} with no accrual to \acct{882.16} (see Section~\ref{sec:stgeorge-entries}).

\textbf{Three primary source locations} must be checked each period to support these accruals:

\begin{enumerate}[nosep]
    \item \textbf{St George (CPG-managed bank):} The St George bank statement (downloaded from CPG SharePoint) is the \textbf{primary evidence} for interest and line fee charges. Both are debited directly from the St George facility account (\acct{680}) by St George Bank --- they do \emph{not} appear on the NAB bank statement and are \emph{not} paid via NAB. The St George statement confirms the exact debit dates and amounts for \acct{303.01} (Loan Interest) and \acct{303.02} (Line Fees). \textbf{Both are recorded on a cash basis:} the entry is posted directly when St George debits the facility --- DR \acct{303.01} / CR \acct{680} for interest and DR \acct{303.02} / CR \acct{680} for line fees. No month-end accrual is raised for either item.

    \item \textbf{NAB (Client Coordinator (UFS)):} The NAB bank statement (archived in UFS Shared Folder by the Client Coordinator (UFS)) confirms cash payments for trust-level costs such as BAS, distribution, and BPAY payments. Management fee and admin fee payments may also appear here when settled. NAB entries confirm settlement of previously accrued liabilities in \acct{882.16}.

    \item \textbf{Yardi (CPG-managed accounting system):} CPG records certain trust-level charges in Yardi (e.g.\ management fees, fund admin fees). The Yardi TB export (\texttt{22chts} --- trust-level code) will show these as P\&L charges. Per the SOP procedure (Step~6a, Section~\ref{sec:yardi-revise}), these are \emph{reclassified} from the Yardi P\&L to \acct{882.16} during the Yardi TB revision process. The Preparer must identify and reclassify each trust-level item before import.
\end{enumerate}

\textbf{Other supporting sources for accrual estimation:} Where a current invoice or Yardi entry is not yet available, accruals may be estimated from:
\begin{itemize}[nosep]
    \item \textbf{Contracts and engagement letters:} Fixed-fee arrangements (e.g.\ UFS fund admin \$1,250/month, audit fee schedule) are accrued monthly based on the agreed contractual amount.
    \item \textbf{Prior period patterns:} Where invoices are periodic or irregular (e.g.\ ASIC fees, registry fees, consultant fees), prior period amounts and timing provide the best estimate. Document the basis used in the accrual workpaper.
    \item \textbf{CPG invoice or instruction:} Management fees (\acct{301.01}) are accrued at 5\% of net receivable income, or per CPG invoice if one has been issued for the period.
\end{itemize}

\textbf{Trust-level expense accounts accrued to \acct{882.16}:}

\begin{table}[H]
\centering\small\sffamily
\caption{Trust-level expense accounts (303.01 and 303.02 on cash basis; all others accrued to \acct{882.16})}
\begin{tabularx}{\textwidth}{@{}L{2.5cm} L{3.5cm} L{2.5cm} X@{}}
\rowcolor{navy}\textcolor{white}{\bfseries Account} & \textcolor{white}{\bfseries Description} & \textcolor{white}{\bfseries Primary Source} & \textcolor{white}{\bfseries Basis / Notes} \\
\midrule
\acct{301.01} & Management Fees & Yardi (\texttt{22chts}) / CPG invoice & 5\% of net receivable income; reclassify from Yardi P\&L to \acct{882.16} \\
\rowcolor{warmgray}
\acct{301.02} & Fund Admin \& Tax Fees & Contract (UFS) & UFS fixed fee \$1,250/month; no CPI increase applied over the engagement \\
\acct{301.03} & Registry Fees & Invoice (CPG SharePoint) & Per invoice; if not received, estimate from prior period pattern \\
\rowcolor{warmgray}
\acct{301.04} & Disposal Fee & CPG confirmation (conditional) & 1\% of GAV; accrue only upon CPG confirmation --- see note below \\
\acct{301.05} & ASIC Fee & ASIC notice / prior pattern & Annual; accrue monthly or in full at year-end \\
\rowcolor{warmgray}
\acct{301.10} & Other Expenses & Various & Miscellaneous trust-level costs not otherwise classified \\
\acct{302.02} & Legal Fee & Invoice (CPG SharePoint) & Trust-level legal costs only --- sale-related legal fees go to \acct{709} \\
\rowcolor{warmgray}
\acct{302.03} & Valuation Fee & Invoice (CPG SharePoint) & Independent valuation costs \\
\acct{302.04} & Consultant Fee & Invoice / CPG instruction & Per engagement letter or CPG direction \\
\rowcolor{warmgray}
\acct{302.05} & Audit Fee & Contract / prior pattern & Annual audit; accrue monthly or at period end \\
\acct{303.01} & Loan Interest & St George statement / Loan\_St George tab & \textbf{Cash basis} --- DR \acct{303.01} / CR \acct{680} when St George debits the facility. No month-end accrual to \acct{882.16}. \\
\rowcolor{warmgray}
\acct{303.02} & Line Fees & St George statement / facility schedule & \textbf{Cash basis} --- DR \acct{303.02} / CR \acct{680} when St George debits the facility. No month-end accrual to \acct{882.16}. \\
\acct{303.03} & Establishment Fee Amort. & Borrowing cost prepaid tab & Straight-line over remaining facility term \\
\rowcolor{warmgray}
\acct{303.04} & Legal Fee Amort. & Borrowing cost prepaid tab & Straight-line over remaining facility term \\
\acct{303.05} & Performance Fee & CPG confirmation (conditional) & Tier 1/Tier 2 structure; accrue only when threshold is probable --- see note below \\
\bottomrule
\end{tabularx}
\end{table}

\begin{note}[Disposal and Performance Fees — Reserved Accounts]
\textbf{Disposal Fee (\acct{301.04}) and Performance Fee (\acct{303.05}):} These account numbers are \emph{reserved} in the Chart of Accounts because both fees are expected to be triggered upon completion of the property sale. Neither is currently being accrued. Do not post to these accounts without explicit CPG confirmation. Do not post property-level operating accruals to \acct{882.16}; those belong to \acct{882.02}.
\end{note}

\subsubsection{Yardi TB Export}

The Preparer downloads the Yardi TB movement export directly from the Yardi system. Two separate exports are required each period:

\begin{itemize}[nosep]
    \item \textbf{Property-level} (code: \texttt{cs4001}): Covers rental income, operating expenses, prepayments, receivables, and property-level accruals.
    \item \textbf{Trust-level} (code: \texttt{22chts}): Covers management fees, professional fees, fund administration charges, and trust-level transactions.
\end{itemize}

Both exports are held temporarily before revision (Step~6a, Section~\ref{sec:yardi-revise}) and import (Step~6b, Section~\ref{sec:yardi-import}).

\subsubsection{Document Archiving}

Source documents are stored across \textbf{three} locations, each managed by a different party:

\begin{enumerate}[nosep]
    \item \textbf{UFS Shared Folder (UFS-managed):} The Client Coordinator (UFS) downloads the NAB bank statement and archives it directly in the Trust's Shared Folder on the UFS internal network drive. This is the primary repository for UFS working files and the NAB bank statement.

    \item \textbf{CPG SharePoint (CPG-managed):} CPG uploads the following to the designated SharePoint location:
    \begin{itemize}[nosep]
        \item St George bank statement;
        \item Bank information request responses (explanations of unidentified NAB transactions);
        \item Supporting invoices corresponding to bank information requests (remediation invoices from FHS, Premier Fire, etc.; insurance remittance advices from Steadfast IRS / Vision Re Partners; sale-related invoices from CBRE, K\&L Gates, MP Realty Trust).
    \end{itemize}
    The Preparer accesses SharePoint to download these documents as needed for reconciliation and journal entry support.

    \item \textbf{Yardi (CPG-managed):} The Preparer downloads the Yardi TB movement export directly from the Yardi system. Two property codes are used:
    \begin{itemize}[nosep]
        \item \texttt{cs4001} --- Property-level accounts (rental income, operating expenses, prepayments, receivables);
        \item \texttt{22chts} --- Trust-level accounts (management fees, professional fees, fund administration).
    \end{itemize}
    Invoices and supporting documents attached to Yardi transactions are also accessible within the Yardi system.
\end{enumerate}

\begin{note}[File Locations and Folder Structure]
Each location serves a distinct purpose: the UFS Shared Folder holds the NAB bank statement and UFS working files; CPG SharePoint holds the St George bank statement, bank information responses, and CPG-sourced invoices; Yardi holds the trial balance exports and attached invoices. All three locations must be checked each period to ensure completeness of source documentation.
\end{note}

\subsection{Step 3 --- Reconcile Yardi NAB TB vs.\ NAB Bank Statement}
\label{sec:yardi-rec}

Before revising or importing the Yardi TB, the Preparer reconciles the Yardi NAB trial balance against the NAB bank statement in the \textbf{Yardi vs.\ Bank Stat Rec} tab. This identifies all timing differences between Yardi and the bank, which are coded using the five-case classification (Table~\ref{tab:fivecases}) and documented with narrative. The reconciliation must balance to zero before proceeding.

\subsection{Step 4 --- Bank Information Request}
\label{sec:bankreq}

After completing the initial reconciliation, the Preparer submits a bank information request to CPG covering all unidentified or unmatched transactions from the NAB bank statement. Queries are documented in the bank information request template (Excel), covering: missing invoices, unidentified payments, cut-off items, and insurance claim remittance details.

The request is submitted twice: on the 20th of the current month and on BD1 of the following month. CPG uploads responses and supporting invoices to SharePoint between the 20th and BD1. The Preparer downloads these from SharePoint.

Additional requests for the sale and cyclone context:
\begin{itemize}[nosep]
    \item \textbf{Cyclone remediation:} New invoices/progress claims from FHS, Premier Fire, etc.\ Confirm insurance claim amounts from Steadfast IRS / Vision Re Partners.
    \item \textbf{Asset sale:} Updated deferred cost of sale invoices (CBRE, K\&L Gates, MP Realty Trust). Confirm sale exchange/settlement dates.
    \item \textbf{Loan maturity:} Confirm whether the St George facility has been formally extended or refinancing is in progress.
\end{itemize}

\subsection{Step 6a --- Revise Yardi TB}
\label{sec:yardi-revise}

Once the bank information request is submitted and CPG responses received (or best available), the Preparer reviews and revises the raw Yardi TB for two categories of adjustments. Revisions are documented in the \textbf{Yardi\_TB} tab before import.

\subsubsection{Trust-Level Expenses --- Reclassify from P\&L to BS Accrual}

Yardi records all expenses in the P\&L and only accrues property-level (operating) expenses. It does not accrue Trust-level expenses. Our procedure requires Trust-level expenses to be held in \acct{882.16} (Fund Accruals --- BS) until paid.

\textbf{Action:} Identify all Trust-level expense items in the Yardi TB recorded as P\&L charges. Reverse these from the Yardi P\&L and reclassify to \acct{882.16}. The accrual is cleared when the actual payment is made. Property-level operating expense accruals (\acct{882.02}) remain as recorded by Yardi.

\begin{note}[Trust-Level Expenses Subject to Reclassification]
Trust-level expenses subject to reclassification include: management fees, fund admin fees, registry fees, ASIC fees, legal fees (trust-level), valuation fees, consultant fees, audit fees, and any conditional disposal/performance fees. Property-level operating expenses (council rates, insurance premiums, property management) stay in the P\&L.
\end{note}

\subsubsection{Sale-Related and Leasing Expenses --- Reclassify from P\&L to BS}

Invoices recorded by Yardi as operating expenses must be reviewed to determine their nature. The reclassification rules are:

\begin{enumerate}[nosep,leftmargin=*]
    \item \textbf{Invoice relates to the sale of the property} (e.g.\ commission, legal fee, advertising for sale): \\
    Reclassify from P\&L to \acct{709} --- Deferred Cost of Sale (BS asset). Do \emph{not} expense while the sale is active. Upon settlement, \acct{709} is de-recognised against the gain/loss on disposal.

    \item \textbf{Invoice relates to leasing} (e.g.\ leasing commission, tenant incentive, leasing agent fee): \\
    Reclassify from P\&L to \acct{780.12} --- Prepaid Leasing Fee (BS asset). Refer to the current lease register (Section~\ref{sec:leases}) to confirm the tenant and lease term against which the fee is amortised, and to Appendix~\ref{app:estimates} for lease status review considerations. At the same time, record the following contra entry to reflect the increase in investment property carrying value: DR \acct{780.30} --- Fair Value Adjustment / CR \acct{213} --- Gain/(Loss) on Revaluation.

    The leasing fee is then amortised monthly: DR \acct{700.26} --- Leasing Fee (P\&L) / CR \acct{780.12} --- Prepaid Leasing Fee. At each amortisation, simultaneously reverse the fair value entry: DR \acct{213} --- Gain/(Loss) on Revaluation / CR \acct{780.30} --- Fair Value Adjustment, so that the P\&L amortisation charge does not distort the fair value carrying amount.

    Net effect over the life of the leasing fee:
    \begin{itemize}[nosep]
        \item \acct{780.12} increases on capitalisation, decreases on amortisation --- nets to zero at end.
        \item \acct{780.30} increases on capitalisation, reverses on amortisation --- mirrors \acct{780.12} as its contra.
        \item \acct{213} (Gain/(Loss) on Revaluation) is credited on capitalisation, debited on amortisation --- net equity impact is zero over the period.
        \item \acct{700.26} (Leasing Fee) captures the periodic P\&L expense.
    \end{itemize}

    \item \textbf{Otherwise} (ordinary operating expense, not sale- or leasing-related): \\
    Leave as a normal operating expense in the P\&L --- no reclassification required.
\end{enumerate}

\begin{warning}[Invoice Classification — Sale, Leasing, or Operating]
The classification of invoices as sale-related, leasing-related, or ordinary operating is critical and affects both the P\&L and balance sheet. Refer to the invoice description, supplier name, and CPG instruction. When in doubt, request CPG confirmation before posting.
\end{warning}

\subsection{Step 6b --- Import Revised Yardi TB}
\label{sec:yardi-import}

After completing all revisions in Step~6a (Section~\ref{sec:yardi-revise}), the Preparer imports the revised Yardi TB into the \textbf{Import\_Yardi} tab following the process in Section~\ref{sec:yardi}. The import creates movements in all relevant accounts including: \acct{685} --- NAB, \acct{686} --- STA QLD, \acct{703} --- Deferred Income, \acct{706/707} --- Prepayments (Operating \& Insurance), \acct{71} --- Rental Income, \acct{700} --- Property Operating Expenses, \acct{882.02} --- Accrued Operating, and \acct{255} --- GST. The revised TB (post-reclassification) forms the base from which all UFS manual journals are layered in Phase~2 (Section~\ref{sec:phase2}).

% ============================================================================
\section{Phase 2: Reporting Pack Compilation}
\label{sec:phase2}
% ============================================================================

Phase~2 compiles the reporting pack using the imported Yardi TB (Phase~1) as its foundation. The subsections below follow the recommended completion order: property-level entries first, then bank cut-off adjustments, St George transactions, cyclone-related items, deferred cost of sale, loan and borrowing costs, trust-level accruals, and finally the NAV statement, dashboard, and distribution workings. Throughout Phase~2, the Preparer must apply the double-count prevention checkpoints (Appendix~\ref{sec:est-double-count}) to avoid duplicate recording between Yardi and UFS, and classify each expenditure correctly using the repair vs.\ capex guide (Appendix~\ref{sec:est-capex-guide}).

\subsection{Property Level (NAB-Sourced Entries)}

Property-level entries in Phase~2 fall into three categories based on their \textbf{source} and the \textbf{verification method} used to confirm their accuracy. Each category is described below with the source (where the data originates) and the reconciliation target (what it is checked against).

\subsubsection{Category A --- Source: Yardi TB Import \texorpdfstring{$\longrightarrow$}{->} Verified against: Yardi Export Data}

\textbf{What these are:} Balances that flow into the reporting pack \emph{directly from the Yardi TB import} (Step~6b, Section~\ref{sec:yardi-import}). The Preparer does not post separate manual journals for Category~A items --- the import itself creates all GL movements. Verification means confirming that the imported figures in the reporting pack match the original Yardi export file.

\begin{itemize}[nosep]
    \item \textbf{Rental income and outgoings recovery (\acct{71.xx}):} Source = Yardi \texttt{cs4001} property-level export. Verify imported TB balances tie back to the Yardi export figures.
    \item \textbf{Property operating expenses (\acct{700.xx}):} Source = Yardi \texttt{cs4001}. Covers: property management, council rates, electricity, water, garden maintenance, cleaning, fire safety, and security. Verify imported amounts match Yardi export line by line.
    \item \textbf{Trade receivables (\acct{662} Debtors; \acct{663} Unallocated Income):} Source = Yardi. Verify imported balances against the Yardi Aged Receivables report. For aging bad debt review considerations, see Appendix~\ref{app:estimates}.
    \item \textbf{Operating prepayments (\acct{706} --- Prepayments Operating; \acct{707} --- Prepayments Insurance):} Source = Yardi. Verify that the imported prepayment balances agree to the \textbf{Prepayments\_operating} tab schedule. For attention items regarding land tax and council rate amortisation and fine avoidance, see Appendix~\ref{app:estimates}.
    \item \textbf{Operating accruals --- Yardi-originated (\acct{882.02}):} Source = Yardi (property-level accruals already recorded by CPG). Imported as part of the TB; no separate UFS manual entry required for these Yardi-originated accruals.
    \item \textbf{GST (\acct{255}):} Source = Yardi. Verify imported GST balance agrees to Yardi.
\end{itemize}

\subsubsection{Category B --- Source: NAB Bank Statement \texorpdfstring{$\longrightarrow$}{->} Verified against: Yardi vs.\ Bank Statement Reconciliation}

\textbf{What these are:} Entries where the Preparer must compare the NAB bank statement against the Yardi GL to identify and record cut-off timing differences. These are not imported from Yardi; they require manual journal entries posted by UFS. The reconciliation is performed in the \textbf{Yardi vs.\ Bank Stat Rec} tab and the \textbf{Cash} tab.

\begin{itemize}[nosep]
    \item \textbf{Cash --- NAB (\acct{685}):} Source = NAB bank statement (UFS Shared Folder). Reconcile the NAB bank statement closing balance to the per-books balance in the \textbf{Cash} tab. Any unexplained difference is investigated in the Yardi vs.\ Bank Statement reconciliation.
    \item \textbf{Cut-off Sundry Debtor entries (Cases~2--5):} Source = timing differences identified in Phase~1 Step~3 (Table~\ref{tab:fivecases}). For each Case~2--5 item, post the required Sundry Debtor journal entry --- refer to Step~7b (Section~\ref{sec:cutoff}).
    \item \textbf{Cyclone remediation capex (\acct{781.01}):} Source = NAB bank statement (cash payments made via NAB) cross-referenced against remediation invoices, which may be sourced from: (a)~\textbf{CPG SharePoint} (invoices uploaded by CPG: FHS, Premier Fire, Handler Property Group, etc.); (b)~\textbf{Yardi} (invoices attached to Yardi transactions and accessible within the Yardi system); or (c)~\textbf{CPG email/bank information request response} for any amounts not yet appearing in Yardi or SharePoint. Confirm each payment on the NAB bank statement and match to a supporting invoice before capitalising in the \textbf{Capex} tab. Do not expense through the P\&L.
\end{itemize}

\subsubsection{Category C --- Source: CPG Instructions, Invoices, or Schedules \texorpdfstring{$\longrightarrow$}{->} Verified against: Supporting Documentation (not Yardi or Bank Statement)}

\textbf{What these are:} Entries posted entirely by UFS based on CPG instructions, third-party invoices (from CPG SharePoint), or predetermined amortisation/accrual schedules. These do \emph{not} originate from the Yardi TB import and are \emph{not} reconciled to a bank statement. Their accuracy depends on the supporting document reviewed by the Preparer.

\begin{itemize}[nosep]
    \item \textbf{Operating accruals --- UFS-originated (\acct{882.02}):} Source = invoices received, CPG instruction, or historical patterns. Posted by UFS where a property-level obligation exists but has not yet been recorded in Yardi for the current period. Document the basis in the accrual workpaper.
    \item \textbf{Deferred cost of sale (\acct{709}):} Sale-related invoices may be sourced from: (a)~\textbf{CPG SharePoint} (primary location --- CPG uploads sale campaign invoices from CBRE, K\&L Gates, and MP Realty Trust); or (b)~\textbf{Yardi} (invoices may also be attached to Yardi transactions under the \texttt{cs4001} or \texttt{22chts} codes and should be checked in Yardi if a corresponding SharePoint document has not yet been received). Updated monthly in the \textbf{Receivables} tab. Do not expense while the sale is active; hold on balance sheet.
    \item \textbf{Prepaid leasing fee (\acct{780.12}) and fair value adjustment (\acct{780.30}):} Source = leasing invoices from Yardi or CPG SharePoint. Capitalised and amortised monthly per the \textbf{Prepaid leasing fee} tab.
    \item \textbf{Distribution payable (\acct{882.03}):} Source = CPG/client-approved distribution instruction. Record only upon receipt of written CPG approval; no estimate or unilateral accrual.
    \item \textbf{Capex fair value check:} Source = latest independent valuation report. Updated in the \textbf{Capex} tab to reconcile IP carrying value (\acct{780}) to the most recent independent valuation plus post-valuation capex additions. For revaluation considerations under AASB~140, including impairment assessment where a potential sale price is below current carrying value, see Appendix~\ref{app:estimates}.
\end{itemize}

\subsection{NAB Bank Statement Cut-Off Entries}
\label{sec:cutoff}

Using the reconciliation completed in Phase~1 Step~3 (Section~\ref{sec:yardi-rec}, Table~\ref{tab:fivecases}), the Preparer records cut-off entries as Sundry Debtors. Only Cases~2--5 require manual journal entries; Case~1 items are already captured through the imported Yardi TB (Step~6b, Section~\ref{sec:yardi-import}).

\begin{note}[Five-Case Classification Applies Consistently]
The five-case classification is defined once in Table~\ref{tab:fivecases} (Section~\ref{sec:yardi}) and applies identically to both the Yardi~vs.\ Bank Statement reconciliation and these cut-off journal entries. Do not create separate codes.
\end{note}

\subsection{St George Account Transactions}
\label{sec:stgeorge-entries}

Record all St George-related transactions per the St George bank statement (downloaded from CPG SharePoint). Interest charges and line fees are debited \emph{directly} from the St George facility account (\acct{680}) by the bank --- they do \textbf{not} flow through NAB (\acct{685}). Insurance claim proceeds and the sale deposit are also received into \acct{680}. For the account register and facility terms, see Section~\ref{sec:banking}; for the corresponding trust-level recording procedure, see Section~\ref{sec:borrowing-amort}.

\begin{itemize}[nosep]
    \item \textbf{Interest expense} --- recorded to \acct{303.01} --- Loan Interest (\textbf{P\&L expense account}):
    \begin{itemize}[nosep]
        \item Recorded on a \textbf{cash basis} --- no month-end accrual is raised.
        \item When St George debits the facility: DR \acct{303.01} --- Loan Interest (P\&L) \; / \; CR \acct{680} --- St George.
        \item Confirm debit date and exact amount from the St George bank statement (downloaded from CPG SharePoint). Cross-reference against the \textbf{Loan\_St George} tab to verify the amount is consistent with the 1.95\% p.a.\ rate on the \$5,000,000 facility.
    \end{itemize}

    \item \textbf{Line fee} --- recorded to \acct{303.02} --- Line Fees (\textbf{P\&L expense account}):
    \begin{itemize}[nosep]
        \item Recorded on a \textbf{cash basis} --- no month-end accrual is raised.
        \item When St George debits the facility: DR \acct{303.02} --- Line Fees (P\&L) \; / \; CR \acct{680} --- St George.
        \item Confirm debit date and exact amount from the St George bank statement.
    \end{itemize}
    \item \textbf{Insurance claim proceeds received:} DR \acct{680} --- St George \; / \; CR \acct{73.02} --- Insurance Claim Income (proceeds credited directly to St George account by insurer; confirm via remittance advice from Steadfast IRS / Vision Re Partners). For the comprehensive accounting, tax, CGT, and GST treatment of insurance claim proceeds (including cost base interactions and post-settlement retention), see Appendix~\ref{sec:est-insurance}.
    \item \textbf{Sale deposit received:} DR \acct{680} --- St George \; / \; CR \acct{884} --- Other Creditors (10\% deposit held as liability in St George account until unconditional settlement).
    \item \textbf{Other items} (e.g.\ bank facility fees): classify per nature and record accordingly.
\end{itemize}

\begin{note}[St George Interest — Cash Basis, No Month-End Accrual]
Interest expense (\acct{303.01}) and line fees (\acct{303.02}) are recorded on a \textbf{cash basis} --- no month-end accrual to \acct{882.16} is raised for either item. Both are settled via \acct{680} (St George facility), not via \acct{685} (NAB). The entry is posted directly when St George debits the facility: DR \acct{303.01} or \acct{303.02} (P\&L) / CR \acct{680} --- St George. NAB (\acct{685}) is not involved in interest or line fee settlement. Insurance \emph{expenses} (premiums) are paid via NAB (\acct{685}), whereas insurance claim \emph{proceeds} and interest/line fee debits all flow through St George (\acct{680}).
\end{note}

\subsection{Cyclone Insurance Claim and Remediation Capex}

\begin{enumerate}[nosep]
    \item Record insurance claim proceeds in \acct{73.02} (P\&L other income). Proceeds are received into \acct{680} --- St George (not NAB). Confirm amounts with remittance advice from Steadfast IRS / Vision Re Partners: DR \acct{680} --- St George / CR \acct{73.02} --- Insurance Claim Income. Before posting, check the Yardi TB to ensure CPG has not already recorded the same claim (double-count risk --- see Appendix~\ref{sec:est-double-count}).
    \item Insurance premium expenses are paid via NAB (\acct{685}) and appear on the NAB bank statement. Do not confuse with insurance claim receipts.
    \item Capitalise all remediation invoices to \acct{781.01} in the \textbf{Capex} tab (paid via NAB). Do not expense through the P\&L. For each invoice, individually assess whether the expenditure is a repair (immediate P\&L expense) or capital improvement (capitalise to \acct{781.01}) --- refer to the classification guide in Appendix~\ref{sec:est-capex-guide}. For tax purposes, classify as Division~40 (plant), Division~43 (structural), or s25-10 (repair) accordingly.
    \item Update Capex tab Fair Value Check: carrying value must reconcile to the latest independent valuation plus post-valuation capex.
    \item Accrue outstanding remediation invoices where an obligation exists.
\end{enumerate}

\subsection{Deferred Cost of Sale}

\begin{enumerate}[nosep]
    \item Review \acct{709} monthly on the \textbf{Receivables} tab; add new sale-related invoices (see Section~\ref{sec:sale-status} for current balance detail).
    \item Do not amortise or expense while the sale is active.
    \item \textbf{If completed:} de-recognise \acct{709} against gain/loss on disposal.
    \item \textbf{If aborted:} expense to \acct{307} --- Aborted Sales Campaign Cost (see Section~\ref{sec:sale-status} for the prior aborted campaign reference).
\end{enumerate}

\subsection{Trust Level (Loan and Borrowing Costs)}
\label{sec:borrowing-amort}

For the period-end assessment of whether the St George loan should be classified as current or non-current at each financial statement date, see Appendix~\ref{sec:est-loan-reclass}. For s25-25 borrowing expense timing rules (including the immediate write-off triggered on early repayment at settlement), see Appendix~\ref{sec:est-s2525-s40880}.

\begin{enumerate}[nosep]
    \item Record St George loan interest on a \textbf{cash basis} when St George debits the facility, per the \textbf{Loan\_St George} tab and the St George bank statement (refer also to the detailed journal entry in Section~\ref{sec:stgeorge-entries}). No month-end accrual is raised:
    \begin{itemize}[nosep]
        \item DR \acct{303.01} --- Loan Interest (P\&L) \quad CR \acct{680} --- St George.
    \end{itemize}
    \item Record line fee on a \textbf{cash basis} when St George debits the facility, per the St George bank statement. No month-end accrual is raised:
    \begin{itemize}[nosep]
        \item DR \acct{303.02} --- Line Fees (P\&L) \quad CR \acct{680} --- St George.
    \end{itemize}
    \item Record monthly amortisation of prepaid borrowing costs on a straight-line basis over the remaining facility term --- \textbf{Borrowing cost prepaid} tab:
    \begin{itemize}[nosep]
        \item DR \acct{303.03} --- Establishment Fee Amortisation \quad CR \acct{708.03} --- Prepaid Establishment Fee.
        \item DR \acct{303.04} --- Legal Fee Amortisation \quad CR \acct{708.04} --- Prepaid Legal Fee.
    \end{itemize}
    \item Record CPG inter-group loan movements (\acct{882.08}) if applicable.
\end{enumerate}

\subsection{Trust-Level Accruals}
\label{sec:trust-accruals}

Record all Trust-level accruals in \acct{882.16} --- Fund Accruals. These are distinct from property-level operating accruals (\acct{882.02}). Standard items each period (see also the fee-to-metric dependency map in Appendix~\ref{sec:est-fee-size} for how each fee's base metric may fluctuate):

\textbf{Administration Costs (301):}
\begin{itemize}[nosep]
    \item DR \acct{301.01} --- Management Fees / CR \acct{882.16} --- 5\% of net receivable income (or per CPG invoice).
    \item DR \acct{301.02} --- Fund Admin \& Tax Fees / CR \acct{882.16} --- UFS fixed fee \$1,250/month (no CPI increase applied over the engagement).
    \item DR \acct{301.03} --- Registry Fees / CR \acct{882.16} --- per invoice or estimate.
    \item DR \acct{301.05} --- ASIC Fee / CR \acct{882.16} --- annual; accrue monthly or at year-end.
    \item DR \acct{301.10} --- Other Expenses / CR \acct{882.16} --- miscellaneous trust-level costs.
\end{itemize}

\textbf{Professional Fees (302):}
\begin{itemize}[nosep]
    \item DR \acct{302.02} --- Legal Fee / CR \acct{882.16} --- trust-level legal costs only (sale-related legal fees are capitalised to \acct{709}).
    \item DR \acct{302.03} --- Valuation Fee / CR \acct{882.16} --- independent valuation costs.
    \item DR \acct{302.04} --- Consultant Fee / CR \acct{882.16} --- per engagement.
    \item DR \acct{302.05} --- Audit Fee / CR \acct{882.16} --- accrue monthly or at period end.
\end{itemize}

\textbf{Conditional fees (sale in progress) --- require CPG confirmation before posting:}
\begin{itemize}[nosep]
    \item \textbf{Disposal Fee:} DR \acct{301.04} --- Disposal Fee / CR \acct{882.16} --- 1\% of GAV upon completion of sale.
    \item \textbf{Performance Fee:} DR \acct{303.05} --- Performance Fee / CR \acct{882.16} --- per Tier 1/Tier 2 structure (Section~\ref{sec:mgmtfee}). Accrue only when the performance threshold is probable.
\end{itemize}

\begin{note}[Disposal and Performance Fees — Reserved in Chart of Accounts]
Account numbers \acct{301.04} (Disposal Fee) and \acct{303.05} (Performance Fee) are \textbf{reserved and maintained in the Chart of Accounts} even though neither fee is currently being accrued. Both fees are expected to be triggered upon completion of the property sale. The accounts will be activated once CPG confirms the applicable threshold has been met. Do not post to these accounts without CPG confirmation.
\end{note}

\subsection{NAV Statement}

Insert into the NAV Statement: General Ledger (\textbf{GL}), Balance Sheet (\textbf{BS}), Profit \& Loss (\textbf{PL}), and Cash Flow Statement (\textbf{CFS}).

\subsection{CPG Dashboard}

\textbf{Section 1 --- Operating Earnings and Distribution.}
\[
\text{OE} = \text{Net Ordinary Income} + \text{Depreciation} + \text{FV Adj.} + \text{Gain on Sale} - \text{Property Sales}
\]
Monthly OE = current month YTD $-$ prior month YTD. Insurance claim income (\acct{73.02}) flows through Net Ordinary Income; confirm with CPG whether separate disclosure is required.

\begin{note}[Performance Fee Excluded from Operating Earnings Add-Back]
\textbf{Performance Fee is excluded from the Operating Earnings add-back.} Performance Fee is an expense to unitholders that directly reduces investor returns. Adding it back to OE would artificially inflate Operating Earnings and misrepresent the fund's distributable performance. Unlike non-cash items (depreciation, fair value adjustments) which are legitimately added back to reflect economic cash earnings, the Performance Fee represents a real economic cost borne by unitholders and must remain as a deduction. If the Performance Fee has been accrued, it is already captured as an expense within Net Ordinary Income and no further adjustment is required.
\end{note}

\textbf{Section 2 --- Balance Sheet.} Capital Contribution = \acct{500} (\$11,438,500). Debt = \acct{932} (\$5,000,000).
\[
\text{ICR} = \frac{\text{Net Property Income}}{\text{Interest Expenses}} \qquad \text{Gearing} = \frac{\text{Debt}}{\text{Total Assets}} \approx 36.8\%
\]

\textbf{Section 3 --- Unit Price.} Units on issue: 11,438,500. Unit Price = Net Asset / Units. NTA per unit = Adjusted NTA / Units.

\textbf{Section 4 --- Capex and Fees.} Capex from \textbf{Capex} tab; Management fee from \acct{301.01}; Leasing fees from \acct{700.26}.

\textbf{Section 5 --- Property Book Values.} IP value at month end (\acct{780} total: \$13,590,288 at January 2026). Most recent independent valuation: \$13,200,000 (31 December 2024).

\textbf{Section 6 --- Acquisitions/Divestments.} Update the Divestments row with exchange date, settlement date, and sale price when known.

\subsection{Distribution Workings}

Distribution is determined by CPG instruction. The Preparer prepares tax workings (refer: \textbf{Tax reconciliation}, \textbf{STI calc/Adj}, \textbf{Section 40--880/25--25}, \textbf{Tax depreciation} tabs). For the complete distribution component determination framework (ordinary income, net assessable capital gain, CGT discount component, tax deferred, ROC), refer to Appendix~\ref{sec:est-distribution-components}. For the tax loss carryforward rules and calculation formulae (including the opening loss pool, conditions for utilisation, and the $\max/\min$ absorption formulae), see Appendix~\ref{sec:est-taxlosses}. These are reviewed by both Reviewers, then sent to the Tax team. Distribution is recorded in \acct{882.03} only upon CPG/client approval.

% ============================================================================
\section{Phase 3: Internal Review and Client Distribution}
\label{sec:review}
% ============================================================================

\begin{enumerate}[nosep]
    \item The Preparer sends the completed pack to 1st and 2nd Reviewer via the UFS Shared Folder.
    \item Reviewers document comments; if issues are found, the pack is returned to the Preparer, who clears all comments and re-submits. This loop continues until all issues are resolved.
    \item NAV pack and monthly report must be sent to CPG by \textbf{BD5}.
    \item The 1st Reviewer sends the approved pack to CPG for client review.
    \item CPG comments are addressed and resubmitted until approval is confirmed.
    \item Once CPG approves, the reporting pack is finalised and archived in the UFS Shared Folder.
\end{enumerate}

\begin{tip}[Pre-Sign-Off Estimate Review Checklist]
Before sign-off, Reviewers should verify that all accounting estimate and period-end attention items in Appendix~\ref{app:estimates} have been considered for the current period. In particular: leasing fee alignment to lease status, loan current/non-current classification, IP revaluation assessment, aging receivables review, correct asset/liability sign classification, double-count prevention checkpoints (Appendix~\ref{sec:est-double-count}), repair vs.\ capex classification (Appendix~\ref{sec:est-capex-guide}), fee size indicator recalculations (Appendix~\ref{sec:est-fee-size}), and sales settlement accruals if applicable (Appendix~\ref{sec:est-accruals-settlement}).
\end{tip}

% ============================================================================
\section{Checklists}
\label{sec:checklists}
% ============================================================================

All checklists are signed off by the 2nd Reviewer.

\subsection{Monthly NAV Procedures}

{\small\sffamily
\begin{longtable}{@{}C{0.6cm} L{10.2cm} L{2.5cm}@{}}
\caption{Monthly NAV procedures}\\
\rowcolor{navy}\textcolor{white}{\bfseries\#} & \textcolor{white}{\bfseries Procedural Item} & \textcolor{white}{\bfseries Responsibility} \\
\midrule
\endfirsthead
\rowcolor{navy}\textcolor{white}{\bfseries\#} & \textcolor{white}{\bfseries Procedural Item} & \textcolor{white}{\bfseries Responsibility} \\
\midrule
\endhead
\bottomrule
\endfoot

1 & NAB bank statement (current period) downloaded by Client Coordinator (UFS) and archived in UFS Shared Folder & Client Coordinator (UFS) \\
\rowcolor{warmgray} 2 & St George bank statement downloaded from CPG SharePoint & Preparer \\
3 & Yardi TB exports downloaded from Yardi (\texttt{cs4001} property-level; \texttt{22chts} trust-level) & Preparer \\
\rowcolor{warmgray} 4 & Yardi NAB TB vs.\ NAB bank statement reconciliation completed (\textbf{Yardi vs.\ Bank Stat Rec} tab) with narrative & Preparer \\
5 & Bank information request submitted to CPG (20th and BD1) & Preparer \\
\rowcolor{warmgray} 6 & All CPG SharePoint documents received (bank information responses, remediation invoices, insurance remittances, sale-related invoices) & Preparer \\
7 & Yardi TB revised: Trust-level expenses reclassified to \acct{882.16}; sale/leasing invoices reclassified per procedure & Preparer \\
\rowcolor{warmgray} 8 & Revised Yardi TB imported into Import\_Yardi tab & Preparer \\
9 & Cash reconciled: \acct{680} (St George), \acct{685} (NAB), \acct{686} (STA QLD) to bank statements & Preparer \\
\rowcolor{warmgray} 10 & Trade receivables (\acct{662}) and unallocated income (\acct{663}) reconciled & Preparer \\
11 & Deferred cost of sale (\acct{709}) updated with new sale-related invoices & Preparer \\
\rowcolor{warmgray} 11a & Bank guarantee register reviewed: expiry dates checked; guarantees expiring within 60 days flagged to CPG; notes disclosure updated for all active guarantees (Appendix~\ref{sec:est-bankguarantees}) & Preparer \\
\rowcolor{warmgray} 12 & Insurance claim income (\acct{73.02}) recorded if receipt received (via St George \acct{680}); double-count vs.\ Yardi confirmed; CGT cost base Element~4 adjusted for structural reinstatement proceeds (Appendix~\ref{sec:est-insurance}) & Preparer \\
13 & Cyclone capex capitalised to \acct{781.01} in Capex tab (insurance expenses paid via NAB \acct{685}) & Preparer \\
\rowcolor{warmgray} 14 & Capex Fair Value Check updated (reconcile to latest independent valuation plus post-valuation capex) & Preparer \\
15 & Prepayments reconciled (\acct{706}, \acct{707}, \acct{708}) & Preparer \\
\rowcolor{warmgray} 16 & NAB cut-off bank statement entries recorded as Sundry Debtor (5 cases) & Preparer \\
17 & St George transactions recorded: interest, line fee, insurance claim, sale deposit & Preparer \\
\rowcolor{warmgray} 18 & Borrowing cost amortisation recorded (\acct{303.03}, \acct{303.04}) & Preparer \\
19 & Trust-level accruals recorded (\acct{882.16}): admin fee, registry, management fee, disposal fee if applicable & Preparer \\
\rowcolor{warmgray} 20 & Operating accruals recorded (\acct{882.02}) & Preparer \\
21 & CPG inter-group loan (\acct{882.08}) confirmed if applicable & Preparer \\
\rowcolor{warmgray} 22 & Distribution payable (\acct{882.03}) recorded if applicable & Preparer \\
23 & Dashboard KPIs linked and validated & Preparer \\
\rowcolor{warmgray} 24 & ICR calculated on 12-month rolling basis & Preparer \\
25 & NAV per unit calculated (unadjusted and adjusted NTA) & Preparer \\
\rowcolor{warmgray} 26 & GL, PL, BS, CFS inserted into NAV Statement & Preparer \\
27 & IP book value (\acct{780}) reconciled to latest independent valuation plus capex & Preparer \\
\rowcolor{warmgray} 28 & Lease schedule reviewed: tenant status, expiries, vacancy risk; leasing fee amortisation aligned to lease status (Appendix~\ref{app:estimates}) & Preparer \\
29 & St George maturity flagged if within 3 months; loan classification (current vs.\ non-current) assessed per Appendix~\ref{app:estimates} & Preparer \\
\rowcolor{warmgray} 30 & IP carrying value assessed against potential sale price; revaluation considered if indicators present (Appendix~\ref{app:estimates}) & Preparer \\
31 & Aging bad debt reviewed; receivables $>$60 days escalated to CPG & Preparer \\
\rowcolor{warmgray} 32 & Tax payable/receivable sign reviewed; reclassified if sign has changed & Preparer \\
33 & Revenue received in advance and supplier credits correctly classified & Preparer \\
\rowcolor{warmgray} 34 & Pack reviewed by 1st Reviewer; all comments cleared & 1st Reviewer \\
35 & Double-count prevention checkpoints verified: Yardi TB vs.\ UFS journals reconciled, no duplicate entries (Appendix~\ref{sec:est-double-count}) & Preparer / Reviewer \\
\rowcolor{warmgray} 36 & Cyclone remediation invoices individually classified as repair vs.\ capex; tax division assigned (Appendix~\ref{sec:est-capex-guide}) & Preparer \\
37 & Fee size indicators verified: management fee base, disposal fee GAV, performance fee IRR recalculated (Appendix~\ref{sec:est-fee-size}) & Preparer \\
\rowcolor{warmgray} 38 & Sales settlement accruals considered if sale is approaching exchange/settlement (Appendix~\ref{sec:est-accruals-settlement}) & Preparer \\
39 & Pack signed off by 2nd Reviewer & 2nd Reviewer \\
\rowcolor{warmgray} 40 & NAV pack and monthly report sent to CPG by BD5 & 1st Reviewer \\
\end{longtable}
}

\subsection{Quarterly BAS Procedures}

{\small\sffamily
\begin{longtable}{@{}C{0.6cm} L{10.2cm} L{2.5cm}@{}}
\caption{Quarterly BAS procedures}\\
\rowcolor{navy}\textcolor{white}{\bfseries\#} & \textcolor{white}{\bfseries Procedural Item} & \textcolor{white}{\bfseries Responsibility} \\
\midrule
\endfirsthead \endhead \bottomrule \endfoot

1 & GST ledger reviewed and reconciled for the quarter & Preparer \\
\rowcolor{warmgray} 2 & GST credits and output tax confirmed (incl.\ insurance GST) & Preparer \\
3 & BAS return workings reviewed by 1st Reviewer & 1st Reviewer \\
\rowcolor{warmgray} 4 & BAS lodged with ATO by due date & 1st / 2nd Reviewer \\
5 & BAS payment processed via NAB & CPG \\
\rowcolor{warmgray} 6 & BAS checklist signed off by 2nd Reviewer & 2nd Reviewer \\
\end{longtable}
}

\subsection{Quarterly Distribution Procedures}

{\small\sffamily
\begin{longtable}{@{}C{0.6cm} L{10.2cm} L{2.5cm}@{}}
\caption{Quarterly distribution procedures}\\
\rowcolor{navy}\textcolor{white}{\bfseries\#} & \textcolor{white}{\bfseries Procedural Item} & \textcolor{white}{\bfseries Responsibility} \\
\midrule
\endfirsthead \endhead \bottomrule \endfoot

1 & Tax reconciliation workings prepared (incl.\ insurance treatment and tax loss carryforward position per Appendix~\ref{sec:est-taxlosses}) & Preparer \\
\rowcolor{warmgray} 2 & Tax workings reviewed by 1st and 2nd Reviewer & 1st / 2nd Reviewer \\
3 & Tax workings sent to Tax team for second review & 2nd Reviewer \\
\rowcolor{warmgray} 4 & Tax team confirms distribution components; component formula verified per Appendix~\ref{sec:est-distribution-components} & Tax Team \\
5 & Components (cash, tax-deferred, capital return) sent to CPG & 1st Reviewer \\
\rowcolor{warmgray} 6 & CPG/client approval received & CPG \\
7 & Distribution recorded in \acct{882.03} and Unitholders tab & Preparer \\
\rowcolor{warmgray} 8 & Distribution checklist signed off by 2nd Reviewer & 2nd Reviewer \\
\end{longtable}
}

% ============================================================================
% APPENDIX A — PROCESS FLOWCHART  (Swim-Lane / Action-Document-Review Design)
% ============================================================================
\clearpage
\appendix
\section{Monthly Reporting Cycle --- Process Flowchart}
\label{app:flowchart}

Each numbered node corresponds to a step in Appendix~\ref{app:process-detail}.
Every action follows the control pattern: \textbf{Action~$\to$ Document~$\to$ Review~$\to$ Decision}.
All decision diamonds use the same convention: \textbf{Yes}~= issue found (loop back to fix/re-do); \textbf{No}~= no issue (proceed to next step).
Swim lanes indicate the responsible party.

\vspace{0.5em}

\noindent\textbf{Flowchart Legend}

\vspace{0.3em}
\noindent
\begin{tabular}{@{}p{3.2cm} >{\centering\arraybackslash}p{0.9cm} p{9.5cm}@{}}
\toprule
\textbf{Symbol} & \textbf{Icon} & \textbf{Description} \\
\midrule
Action / Step block &
  \raisebox{-0.15cm}{\begin{tikzpicture}[baseline]
    \node[rectangle, rounded corners=3pt, draw=steel, line width=0.7pt, fill=palesky,
          minimum width=0.8cm, minimum height=0.45cm, inner sep=1pt,
          font=\sffamily\fontsize{5}{6}\selectfont] {Step N};
  \end{tikzpicture}} &
  Numbered action performed by the responsible party. Colour indicates lane (blue = Preparer; teal = Reviewer; pink = CPG; grey = Coordinator). \\[4pt]

Document &
  \raisebox{-0.15cm}{\begin{tikzpicture}[baseline]
    \node[tape, tape bend top=none, tape bend height=0.13cm,
          draw=steel!50, line width=0.6pt, fill=warmgray,
          minimum width=0.8cm, minimum height=0.45cm, inner sep=1pt,
          font=\sffamily\fontsize{5}{6}\selectfont] {Doc};
  \end{tikzpicture}} &
  Output, file, tab, or record produced or verified at that step (wavy-bottom = standard document symbol). \\[4pt]

Decision diamond &
  \raisebox{-0.2cm}{\begin{tikzpicture}[baseline]
    \node[diamond, aspect=2, draw=warnamber, line width=0.7pt, fill=warnbg,
          minimum width=0.9cm, minimum height=0.5cm, inner sep=0pt,
          font=\sffamily\fontsize{4.5}{5.5}\selectfont, align=center] {Yes/\\[-2pt]No};
  \end{tikzpicture}} &
  Review gate: \textbf{Yes} = issue found → loop back to fix; \textbf{No} = clear → proceed to next step. \\[4pt]

Off-page connector &
  \raisebox{-0.2cm}{\begin{tikzpicture}[baseline]
    \node[regular polygon, regular polygon sides=5, shape border rotate=180,
          draw=navy, line width=0.7pt, fill=sky, text=navy,
          minimum size=0.5cm, inner sep=0pt,
          font=\sffamily\bfseries\fontsize{6}{7}\selectfont] {A};
  \end{tikzpicture}} &
  Flow continues to (\textit{label on right}) or resumes from (\textit{label on top}) another flowchart page. \\[4pt]

Synchronisation bar &
  \raisebox{0.05cm}{\begin{tikzpicture}[baseline]
    \node[rectangle, fill=navy, draw=navy,
          minimum width=0.8cm, minimum height=0.1cm, inner sep=0pt] {};
  \end{tikzpicture}} &
  All parallel tracks above must complete before the flow below may begin (BPMN parallel join). \\[4pt]

\midrule
\multicolumn{3}{@{}l@{}}{\textbf{Swim-lane colour key}} \\[2pt]
Preparer &
  \raisebox{-0.1cm}{\begin{tikzpicture}[baseline]
    \node[rectangle, rounded corners=2pt, draw=steel, line width=0.7pt, fill=palesky!20,
          minimum width=0.8cm, minimum height=0.35cm] {};
  \end{tikzpicture}} &
  Preparer (blue) — compiles the reporting pack. \\[3pt]
Reviewer &
  \raisebox{-0.1cm}{\begin{tikzpicture}[baseline]
    \node[rectangle, rounded corners=2pt, draw=infoteal, line width=0.7pt, fill=infobg!30,
          minimum width=0.8cm, minimum height=0.35cm] {};
  \end{tikzpicture}} &
  1st \& 2nd Reviewer (teal) — independently review and sign off. \\[3pt]
CPG &
  \raisebox{-0.1cm}{\begin{tikzpicture}[baseline]
    \node[rectangle, rounded corners=2pt, draw=critred!80, line width=0.7pt, fill=critbg!35,
          minimum width=0.8cm, minimum height=0.35cm] {};
  \end{tikzpicture}} &
  CPG / Client (pink) — uploads source documents; approves distributions. \\[3pt]
Coordinator &
  \raisebox{-0.1cm}{\begin{tikzpicture}[baseline]
    \node[rectangle, rounded corners=2pt, draw=steel, line width=0.7pt, fill=sky!30,
          minimum width=0.8cm, minimum height=0.35cm] {};
  \end{tikzpicture}} &
  Client Coordinator (UFS) (grey-blue) — downloads NAB statement; reviews bank info request. \\
\bottomrule
\end{tabular}

\vspace{0.5em}

% ============================================================================
% TIKZ STYLE DEFINITIONS
% ============================================================================
\tikzset{
  %% ---- Process / Action boxes (coloured by party) ----
  proc/.style={
    rectangle, rounded corners=4pt,
    draw=steel, line width=0.9pt, fill=palesky,
    font=\sffamily\fontsize{6.5}{8}\selectfont, minimum width=3.6cm, minimum height=1.2cm,
    align=center, text width=3.4cm, inner sep=3pt},
  proc/cpg/.style={
    rectangle, rounded corners=4pt,
    draw=critred!80, line width=0.9pt, fill=critbg,
    font=\sffamily\fontsize{6.5}{8}\selectfont, minimum width=3.6cm, minimum height=1.2cm,
    align=center, text width=3.4cm, inner sep=3pt},
  proc/rev/.style={
    rectangle, rounded corners=4pt,
    draw=infoteal, line width=0.9pt, fill=infobg,
    font=\sffamily\fontsize{6.5}{8}\selectfont, minimum width=3.6cm, minimum height=1.2cm,
    align=center, text width=3.4cm, inner sep=3pt},
  proc/coord/.style={
    rectangle, rounded corners=4pt,
    draw=steel, line width=0.9pt, fill=sky,
    font=\sffamily\fontsize{6.5}{8}\selectfont, minimum width=3.6cm, minimum height=1.2cm,
    align=center, text width=3.4cm, inner sep=3pt},
  %% ---- Document shape (tape = wavy-bottom, standard flowchart document symbol) ----
  docn/.style={
    tape, tape bend top=none, tape bend height=0.18cm,
    draw=steel!50, line width=0.6pt, fill=warmgray,
    font=\sffamily\fontsize{5.5}{7}\selectfont,
    minimum width=3.0cm, minimum height=0.9cm,
    align=center, text width=2.8cm, inner sep=4pt},
  %% ---- Decision diamond ----
  dec/.style={
    diamond, aspect=2.2,
    draw=warnamber, line width=0.9pt, fill=warnbg,
    font=\sffamily\fontsize{6.5}{8}\selectfont,
    minimum width=2.0cm, align=center, inner sep=2pt},
  %% ---- Loop / fix helper box ----
  lbox/.style={
    rectangle, rounded corners=3pt,
    draw=textgray, line width=0.6pt, fill=warmgray,
    font=\sffamily\fontsize{5.5}{7}\selectfont,
    minimum width=2.2cm, minimum height=0.7cm,
    align=center, text width=2.0cm, inner sep=2pt},
  %% ---- Off-page connector (pentagon) ----
  opc/.style={
    regular polygon, regular polygon sides=5,
    shape border rotate=180,
    draw=navy, line width=0.9pt, fill=sky, text=navy,
    font=\sffamily\bfseries\normalsize,
    minimum size=0.8cm, align=center, inner sep=0pt},
  %% ---- Terminal (start / end) ----
  term/.style={
    rectangle, rounded corners=6pt,
    fill=navy, text=white,
    font=\sffamily\bfseries\small,
    minimum width=5cm, minimum height=0.65cm,
    align=center, inner sep=4pt},
  %% ---- Arrows & labels ----
  ar/.style={-Latex, draw=textgray, line width=0.8pt, rounded corners=6pt, shorten >=3pt, shorten <=2pt},
  lb/.style={font=\sffamily\fontsize{6}{7.5}\selectfont,
             fill=white, inner sep=1pt, text=textgray},
  %% ---- Swim-lane label ----
  lanelbl/.style={font=\sffamily\bfseries\footnotesize, text=white,
                  rotate=90, anchor=center},
}

%% ---- Process node macro: #1=step, #2=action, #3=person (suppressed), #4=deadline (suppressed) ----
\newcommand{\PN}[4]{%
  {\fontsize{8}{9.5}\selectfont\bfseries\textcolor{navy}{#1}}\\[1pt]%
  {\fontsize{6}{7.5}\selectfont #2}%
}

%% ---- Document node macro: #1=document list (no "Docs:" prefix — shape conveys it) ----
\newcommand{\DN}[1]{%
  {\fontsize{5.5}{7}\selectfont #1}%
}

% ==========================================================================
%  PAGE 1 OF 5 — Phase 1a: Steps 1--3  (Source Documents & Reconciliation)
% ==========================================================================
\begin{landscape}
\vspace*{-0.4cm}
\begin{center}
{\sffamily\bfseries\large\textcolor{navy}{Phase 1a --- Source Document Collection \& Reconciliation (Steps 1--3)}}\par\vspace{0.25cm}

\resizebox{0.96\linewidth}{!}{%
\begin{tikzpicture}[every node/.style={font=\sffamily}]

%% ===== NODES =====
%% Layout:
%%   START terminal centred at top (x=12)
%%   CPG lane:         y=-2.5   S1 at x=4,  D1 at x=11
%%   Coordinator lane: y=-6.0   S2 at x=4,  D2 at x=11
%%   Preparer lane:    y=-10.0  S3 at x=4,  D3 at x=11
%%   Decision + gates: R3 at x=17, F3 at x=17 (below), OA at x=23, OB at x=23

% -- START --
\node[term] (START) at (12,0.5) {BEGIN};

% -- CPG Lane: Step 1 --
\node[proc/cpg] (S1) at (4,-2.5) {\PN{1}{Upload source docs to CPG SharePoint}{CPG}{BD1}};
\node[docn]     (D1) at (11,-2.5) {\DN{St George bank stmt, bank info responses, invoices (remediation, insurance, sale) $\to$ SharePoint}};

% -- Coordinator Lane: Step 2 --
\node[proc]  (S2) at (4,-6.0) {\PN{2}{Download NAB bank stmt; archive in UFS Shared Folder}{Client Coord.}{20th \& BD1}};
\node[docn]  (D2) at (11,-6.0) {\DN{NAB bank stmt $\to$ UFS Shared Folder}};

% -- Preparer Lane: Step 3 --
\node[proc]  (S3) at (4,-10.0) {\PN{3}{Download Yardi TB exports \& StG stmt; reconcile NAB TB vs bank stmt}{Preparer}{20th \& BD1}};
\node[docn]  (D3) at (11,-10.0) {\DN{cs4001, 22chts exports; Yardi vs Bank Stat Rec tab (5 cases)}};
\node[dec]   (R3) at (17,-10.0) {Differences\\remain?};
\node[lbox]  (F3) at (17,-13.0) {Investigate \&\\resolve differences\\{\textit{Preparer}}};

% -- Off-page connector: single exit to Phase 1b (Steps 4-6 are always required) --
\node[opc] (OB2) at (23,-10.0) {A};
\node[lb, right=0.1cm of OB2] {\textit{Ph.1b: Bank Info \& TB Import}};

%% ===== SWIM-LANE BACKGROUNDS =====
\begin{scope}[on background layer]
  % Left colour strip (fix: remove white gaps between lane labels by filling full lane height)
  \def\LaneStripL{-1.2}
  \def\LaneStripR{0}

  % Role strip colours (solid)
  \fill[critred!75] (\LaneStripL, 0.9) rectangle (\LaneStripR,-4.2);
  \fill[steel]      (\LaneStripL,-4.2) rectangle (\LaneStripR,-7.8);
  \fill[steel!65]   (\LaneStripL,-7.8) rectangle (\LaneStripR,-14.5);

  % Main lane backgrounds (light)
  \fill[critbg!35]  (0, 0.9) rectangle (25.5,-4.2);
  \fill[palesky!25] (0,-4.2) rectangle (25.5,-7.8);
  \fill[palesky!12] (0,-7.8) rectangle (25.5,-14.5);

  % Lane labels (text on the coloured strip)
  \node[lanelbl] at (-0.6,-1.65) {CPG};
  \node[lanelbl] at (-0.6,-6.00) {Coordinator};
  \node[lanelbl] at (-0.6,-11.15) {Preparer};

  % Lane separators (extend through the colour strip)
  \draw[midgray, dashed, line width=0.35pt] (\LaneStripL,-4.2) -- (25.5,-4.2);
  \draw[midgray, dashed, line width=0.35pt] (\LaneStripL,-7.8) -- (25.5,-7.8);
\end{scope}

%% ===== ARROWS =====
%% DEPENDENCY: S1 & S2 are INDEPENDENT (parallel from BEGIN).
%%             S3 is DEPENDENT on D1 and D2 (both source docs required).
%%             R3 gates Ph.1b exit; F3 loops back for re-investigation.
\begin{pgfonlayer}{arrows}

%% ── Independent parallel start: S1 and S2 ──
%% Left spine branches east into S1 and S2 only (S3 depends on their outputs).
\draw[draw=textgray, line width=0.8pt, rounded corners=6pt] (START.west) -- (0.5,0.5) -- (0.5,-6.0);
\draw[ar] (0.5,-2.5)  -- node[lb,above]{begin} (S1.west);   %% branch -> S1 (CPG upload)
\draw[ar] (0.5,-6.0)  -- node[lb,above]{begin} (S2.west);   %% branch -> S2 (Coordinator download)

%% S1 -> D1
\draw[ar] (S1.east) -- node[lb,above]{upload} (D1.west);

%% S2 -> D2
\draw[ar] (S2.east) -- node[lb,above]{archive} (D2.west);

%% ── Dependent join: D1 + D2 → S3 ──
%% CROSSING FIX: D1 exits from east edge (right side of document block),
%% bends right to x=15, drops south to y=-9.0 (BELOW D2's horizontal at y=-7.8),
%% then turns west to S3.north east. D2 drops south at x=11, turns west at y=-7.8.
%% The two paths never cross: D1 is at x=15 when passing y=-7.8, and at y=-9.0
%% when heading west — fully separated in both dimensions.

%% D1 → S3  (CPG docs feed Preparer)
%%   D1 centre at (11,-2.5), S3 at (4,-10.0).
%%   Route: D1.east → right to x=15 → south to y=-9.0 → west to S3.north east.
\draw[ar] (D1.east) -- node[lb,above]{feeds} (16,-2.5) -- (16,-8.0) -- (4,-8.0) -- (S3.north);

%% D2 → S3  (NAB stmt feeds Preparer)
%%   D2 centre at (11,-6.0), S3 at (4,-10.0).
%%   Route: D2.south → down at x=11 to y=-7.8 → west to x=4 → south into S3.north.
\draw[ar] (D2.south) -- node[lb,right]{feeds} (11,-7.5) -- (1.5,-7.5) -- (1.5,-10.0) -- (S3.west);

%% S3 -> D3 -> R3
\draw[ar] (S3.east) -- node[lb,above]{reconcile} (D3.west);
\draw[ar] (D3.north) -- node[lb,right]{review} (11,-8.5) -- (17,-8.5) -- (R3.north);

%% R3 No → OB2  (east: no differences → proceed to Phase 1b)
\draw[ar] (R3.east) -- node[lb,above]{No} (OB2.west);

%% R3 Yes → F3  (south: differences remain → investigate)
\draw[ar] (R3.south) -- node[lb,right]{Yes} (F3.north);

%% F3 → R3  (loop: re-test after investigation)
%%   Route: west to x=14.5 corridor, north to just below R3, east into R3.
%%   Offset y=-10.3 separates from D3→R3.west entry at y=-10.0.
\draw[ar] (F3.west) -- node[lb,above]{retest} (13.5,-13.0) -- (13.5,-10.0) -- (R3.west);

\end{pgfonlayer}
\end{tikzpicture}
}%

\end{center}
\end{landscape}

% ==========================================================================
%  PAGE 2 OF 5 — Phase 1b: Steps 4--6  (Bank Info Request & TB Import)
%  Three swimlanes: Preparer | Coordinator | CPG
%  MERGED: Preparer lane holds BOTH Step 4 (left) and Step 6 (right)
% ==========================================================================
\begin{landscape}
\vspace*{-0.4cm}
\begin{center}
{\sffamily\bfseries\large\textcolor{navy}{Phase 1b --- Bank Information Request \& TB Import (Steps 4--6)}}\par\vspace{0.25cm}

\resizebox{0.96\linewidth}{!}{%
\begin{tikzpicture}[every node/.style={font=\sffamily}]

%% ===== LAYOUT =====
%% Preparer lane   (y -1.2 to -4.5): Step 4 (x≈3--14) + Step 6 (x≈16--26)
%% Coordinator lane (y -4.5 to -8.0): review Excel; Yes=return notes; No=approve to CPG
%% CPG lane         (y -8.0 to -11.8): Step 5 respond + completeness gate

%% ===== NODES =====

\node[opc] (OA) at (1,-2.8) {A};
\node[lb, above=0.1cm of OA] {\textit{from Ph.1a}};

%% Preparer — Step 4 (left side)
\node[proc] (S4) at (5.5,-2.8) {\PN{4}{Prepare bank info request (Excel)}{Preparer}{20th \& BD1}};
\node[docn] (D4) at (11.5,-2.8) {\DN{Bank info request (Excel) with all queries drafted}};

%% Preparer — Step 6 (right side, same lane)
\node[proc] (S6) at (18,-2.8) {\PN{6}{Revise Yardi TB (reclass trust / sale / leasing); import}{Preparer}{BD1}};
\node[docn] (D6) at (24,-2.8) {\DN{Yardi\_TB revised; Import\_Yardi tab updated}};

%% Off-page B exit (bottom-right, Preparer lane level or below)
\node[opc] (OB) at (24,-6.2) {B};
\node[lb, right=0.1cm of OB] {\textit{Ph.2a: Prop.Level}};

%% TB review gate and fix box (in Coordinator lane zone, right side)
\node[dec]  (R6) at (24,-4.5) {Import\\errors?};
\node[lbox] (F6) at (18,-4.5) {Fix reclassification\\errors\\{\textit{Preparer}}};

%% Coordinator lane
\node[proc/coord] (SC) at (5.5,-6.2) {\PN{}{Review bank info request (Excel)}{Coordinator}{BD1}};
\node[dec]  (RC) at (11.5,-6.2) {Review\\notes?};
\node[lbox] (FCN) at (15.0,-6.2) {Return to Preparer\\with review notes};

%% CPG lane — Step 5
\node[proc/cpg] (S5) at (5.5,-10) {\PN{5}{Respond to bank info; upload invoices to SharePoint}{CPG}{20th--BD1}};
\node[docn]     (D5) at (12,-10) {\DN{Invoices, remittances, bank info responses $\to$ SharePoint}};
\node[dec]      (R5) at (20,-10) {Outstanding\\items?};
\node[lbox]     (F5) at (25,-10) {Follow up with\\CPG for\\outstanding items};

%% ===== SWIM-LANE BACKGROUNDS =====
\begin{scope}[on background layer]
  \fill[palesky!20]  (0,-1.3) rectangle (26.5,-4.2);
  \fill[sky!30]      (0,-4.2) rectangle (26.5,-8.0);
  \fill[critbg!35]   (0,-8.0) rectangle (26.5,-11.8);
  \node[lanelbl, fill=steel!65, minimum width=2.9cm, minimum height=0.85cm,
        rounded corners=2pt] at (-0.7,-2.75) {Preparer};
  \node[lanelbl, fill=steel, minimum width=3.8cm, minimum height=0.85cm,
        rounded corners=2pt] at (-0.7,-6.1) {Coordinator};
  \node[lanelbl, fill=critred!75, minimum width=3.8cm, minimum height=0.85cm,
        rounded corners=2pt] at (-0.7,-9.9) {CPG};
  \draw[midgray, dashed, line width=0.35pt] (0,-4.2)  -- (26.5,-4.2);
  \draw[midgray, dashed, line width=0.35pt] (0,-8.0)  -- (26.5,-8.0);
  %% Dashed divider between Step 4 zone and Step 6 zone within Preparer lane
  \draw[steel!30, dashed, line width=0.35pt] (15,-1.3) -- (15,-4.2);
\end{scope}

%% ===== ARROWS =====
%% DEPENDENCY CHAIN (strictly sequential):
%%   OA → S4 → D4 → SC → RC → [Yes: FCN → S4 loop] [No: → S5]
%%   S5 → D5 → R5 → [Yes: F5 → S5 loop] [No: → S6]
%%   S6 → D6 → R6 → [No: F6 → S6 loop] [Yes: → OB]
%%
%% ARROW-CROSSING STRATEGY:
%%   • FCN→S4 return: via far-left corridor x=0.8 (above Coord lane, drop into Preparer)
%%   • RC→S5: south at x=11.5, then left-turn at y=-8.0 boundary
%%   • R5→S6: via RIGHT corridor x=25.5 (up the right edge to avoid crossing Coord lane)
%%   • F5→S5: via bottom corridor y=-11.5
\begin{pgfonlayer}{arrows}

%% OA → S4  (entry)
\draw[ar] (OA.east) -- node[lb,above]{enter} (S4.west);

%% S4 → D4  (draft request)
\draw[ar] (S4.east) -- node[lb,above]{draft} (D4.west);

%% D4 → SC  (submit to Coordinator for review: drop to lane boundary, left, then into SC)
\draw[ar] (D4.south) -- node[lb,right]{submit} (11.5,-4.2) -- (5.5,-4.2) -- (SC.north);

%% SC → RC  (Coordinator checks)
\draw[ar] (SC.east) -- node[lb,above]{check} (RC.west);

%% RC Yes → FCN  (notes found — return to Preparer)
\draw[ar] (RC.east) -- node[lb,above]{Yes} (FCN.west);

%% RC No → S5  (approved — send to CPG)
%%   RC at (11.5,-6.2), S5 at (5.5,-10).
%%   Route: south to y=-8.0 lane boundary, west along boundary, south into S5.north.
\draw[ar] (RC.south) -- node[lb,right]{No} (11.5,-8.0) -- (5.5,-8.0) -- (S5.north);

%% S5 → D5  (CPG responds)
\draw[ar] (S5.east) -- node[lb,above]{respond} (D5.west);

%% D5 → R5  (review completeness)
\draw[ar] (D5.east) -- node[lb,above]{review} (R5.west);

%% R5 Yes → F5  (outstanding items — follow up)
\draw[ar] (R5.east) -- node[lb,above]{Yes} (F5.west);

%% F5 → S5  (loop: follow up, then CPG re-responds)
%%   Route: south to y=-11.5, west along bottom corridor, north into S5.south.
\draw[ar] (F5.south) -- node[lb,below]{chase} (25,-11.5) -- (5.5,-11.5) -- (S5.south);

%% R5 No → S6  (all info received — proceed to TB revision)
%%   AVOID crossing Coordinator lane by routing via RIGHT edge corridor.
%%   R5 at (18,-10), S6 at (18,-2.8).
%%   Route: east to x=25.5, north along right edge to y=-1.6, west to x=18, south into S6.north.
\draw[ar] (R5.north) -- node[lb,right]{No} (20,-8.2) -- (26.5,-8.2) -- (26.5,-1.6) -- (18,-1.6) -- (S6.north);

%% S6 → D6  (import revised TB)
\draw[ar] (S6.east) -- node[lb,above]{import} (D6.west);

%% D6 → R6  (review import quality)
\draw[ar] (D6.south) -- node[lb,right]{review} (R6.north);

%% R6 Yes → F6  (import errors found — fix reclassification)
\draw[ar] (R6.west) -- node[lb,above]{Yes} (F6.east);

%% F6 → S6  (loop: fix then re-import)
\draw[ar] (F6.north) -- node[lb,right]{redo} (S6.south);

%% R6 No → OB  (no errors — proceed to Phase 2a)
\draw[ar] (R6.south) -- node[lb,right]{No} (OB.north);

%% FCN → S4  (return notes to Preparer — loop back to revise bank info request)
%%   Route: FCN.north → up to lane boundary y=-4.2 → far-left to x=0.8 →
%%          up to y=-1.6 (above Preparer lane) → east to x=5.5 → south into S4.north.
%%   shorten >=0pt ensures arrowhead touches S4 edge (prevents "reversed" appearance).
%% FCN → S4  (return notes to Preparer — loop back to revise bank info request)
%%   Route: FCN.north → lane boundary y=-4.2 → far-left x=0.8 → up to y=-1.3
%%          (higher corridor gives longer vertical drop into S4.north for clean arrowhead)
%%          → east to x=5.5 → south into S4.north (drop of ~0.9 units).
%% FCN → S4: up from FCN.north → left at corridor y=-1.0 → down into S4.north
\draw[ar] (FCN.north) -- node[lb,above]{revise} (15.0,-1.0) -- (5.5,-1.0) -- (S4.north);

\end{pgfonlayer}

\end{tikzpicture}
}%

\end{center}
\end{landscape}

% ==========================================================================
%  PAGE 3 OF 5 — Phase 2a: Step 7 Sub-steps a--e (Property-Level Entries)
%  Dependencies: 7a -> 7b (sequential, NAB rec)
%                7c, 7d, 7e: independent (separate sources), run in parallel
%  All sub-steps feed into 7j (NAV) on page 4.
% ==========================================================================
\begin{landscape}
\vspace*{-0.4cm}
\begin{center}
{\sffamily\bfseries\large\textcolor{navy}{Phase 2a --- Reporting Pack Compilation: Property-Level (Steps 7a--7e)}}\par\vspace{0.25cm}

\resizebox{0.96\linewidth}{!}{%
\begin{tikzpicture}[every node/.style={font=\sffamily}]

%% ===== DEPENDENCY LOGIC =====
%% 7a -> 7b: dependent (NAB Yardi rec feeds cut-off entries)
%% 7c, 7d, 7e: independent of each other and of 7a/7b (different sources)
%% All complete before page 4 (7f-7j)
%% Layout: 7a/7b on top row; 7c/7d/7e on lower rows (parallel tracks)

%% ===== NODES =====

\node[opc] (OB) at (1,-2) {B};
\node[lb, above=0.1cm of OB] {\textit{from Ph.1b}};

%% Track 1 (top): 7a -> 7b (sequential, NAB-dependent)
\node[proc] (S7a) at (6,-2) {\PN{7a}{Record property-level entries (Categories A / B / C)}{Preparer}{BD2}};
\node[docn] (D7a) at (13,-2) {\DN{Import\_Yardi, Receivables, Cash, Prepayments tabs verified}};
\node[proc] (S7b) at (19,-2) {\PN{7b}{NAB cut-off Sundry Debtor entries (Cases 2--5)}{Preparer}{BD2}};
\node[docn] (D7b) at (24.5,-2) {\DN{Yardi vs Bank Rec tab; Sundry Debtor journals posted}};
\node[dec]  (R7b) at (24.5,-4.5) {Unmatched\\items?};
\node[lbox] (F7b) at (19,-4.5) {Resolve unmatched\\items; re-check\\5-case coding};

%% Track 2: 7c (independent — St George source)
\node[proc] (S7c) at (6,-6.5) {\PN{7c}{Record StG interest (303.01) \& line fee (303.02) --- cash basis}{Preparer}{BD2--3}};
\node[docn] (D7c) at (13,-6.5) {\DN{Loan\_StG tab; StG bank stmt from SharePoint}};

%% Track 3: 7d (independent — insurance/capex)
\node[proc] (S7d) at (6,-9.5) {\PN{7d}{Insurance claim income (73.02); cyclone capex (781.01)}{Preparer}{BD3}};
\node[docn] (D7d) at (13,-9.5) {\DN{Capex tab; StG stmt (claim receipts); NAB stmt (capex pmts)}};
\node[dec]  (R7d) at (19,-9.5) {FV update\\needed?};
\node[lbox] (F7d) at (19,-11.8) {Update carrying\\value to valuation\\+ post-val capex};

%% Track 4: 7e (independent — cost of sale)
\node[proc] (S7e) at (6,-13.5) {\PN{7e}{Update deferred cost of sale (709) from new invoices}{Preparer}{BD3}};
\node[docn] (D7e) at (13,-13.5) {\DN{Receivables tab; SharePoint / Yardi sale invoices}};

%% Off-page C exit
\node[opc] (OC) at (25,-9.5) {C};
\node[lb, right=0.1cm of OC] {\textit{Ph.2b: Trust-Level}};

%% ===== SWIM-LANE BACKGROUND =====
\begin{scope}[on background layer]
  \fill[palesky!20] (0,-0.5) rectangle (27,-15);
  \node[lanelbl, fill=steel!65, minimum width=14.5cm, minimum height=0.85cm,
        rounded corners=2pt] at (-0.7,-7.75) {\textcolor{white}{Preparer --- Step 7 Compilation (Property-Level: 7a--7e)}};
  %% Dashed separators between tracks
  \draw[midgray, dashed, line width=0.35pt] (0,-5.0)  -- (27,-5.0);
  \draw[midgray, dashed, line width=0.35pt] (0,-8.2)  -- (27,-8.2);
  \draw[midgray, dashed, line width=0.35pt] (0,-12.0) -- (27,-12.0);
\end{scope}

%% ===== ARROWS =====
%% DEPENDENCY MAP:
%%   7a → 7b: DEPENDENT (sequential — NAB rec feeds cut-off entries)
%%   7c, 7d, 7e: INDEPENDENT of each other and of 7a/7b (separate sources)
%%   All tracks converge at OC (join gate) before Phase 2b.
%%
%% ARROW-CROSSING STRATEGY:
%%   • 7a/7b: top row, straight east chain — no crossings.
%%   • R7b→OC: right corridor at x=25.3, drops south — DEDICATED east channel.
%%   • D7c→OC: east to x=22.0 (staggered left of R7b corridor), south to OC.
%%   • R7d→OC: straight east at same y — clean.
%%   • D7e→OC: east to x=22.0 (same staggered channel as D7c), north to OC.
%%   • F7d loop: west to x=3.5 corridor, north to S7d — avoids D7d→R7d path.
%% Left spine corridor at x=1.0 for clean parallel branching from OB.
\begin{pgfonlayer}{arrows}

%% ── Independent parallel start: OB branches to 7a, 7c, 7d, 7e ──
%% OB → S7a  (east: straight into track 1)
\draw[ar] (OB.east) -- node[lb,above]{parallel} (S7a.west);

%% OB spine south at x=1.0 to feed S7c, S7d, S7e (all independent)
\coordinate (spC) at (1.0,-6.5);
\coordinate (spD) at (1.0,-9.5);
\coordinate (spE) at (1.0,-13.5);

%% Vertical bus (no arrowhead, rounded corners)
\draw[draw=textgray, line width=0.8pt, rounded corners=6pt] (OB.south) -- (spE);
%% Branches with rounded corners
\draw[ar] (spC) -- node[lb,above]{parallel} (S7c.west);
\draw[ar] (spD) -- node[lb,above]{parallel} (S7d.west);
\draw[ar] (spE) -- node[lb,above]{parallel} (S7e.west);

%% ── Track 1: 7a → 7b (DEPENDENT sequential) ──
\draw[ar] (S7a.east) -- node[lb,above]{verify} (D7a.west);
\draw[ar] (D7a.east) -- node[lb,above]{feeds}  (S7b.west);
\draw[ar] (S7b.east) -- node[lb,above]{post} (D7b.west);
\draw[ar] (D7b.south) -- node[lb,right]{review} (R7b.north);

%% R7b Yes → F7b  (unmatched items found — resolve)
\draw[ar] (R7b.west) -- node[lb,above]{Yes} (F7b.east);
%% F7b → S7b  (loop: re-post after resolving)
\draw[ar] (F7b.north) -- node[lb,right]{redo} (S7b.south);

%% R7b No → OC  (balanced — proceed)
%%   Route: east from R7b at (24.5,-4.5) to x=25.3 dedicated channel, south to OC at (25,-9.5).
\draw[ar] (R7b.south) -- node[lb,right]{No} (24.5,-8.1) -- (25.4,-8.1) -- (OC.north east);

%% ── Track 2: 7c (INDEPENDENT — St George source) ──
\draw[ar] (S7c.east) -- node[lb,above]{record} (D7c.west);
%% D7c → OC  (east to staggered channel x=22.0, south to OC level, east into OC)
%%   Staggered left of R7b→OC channel (x=25.3) to prevent overlap.
\draw[ar] (D7c.east) -- node[lb,above]{done} (22.0,-6.5) -- (22.0,-8.5) -- (OC.north west);

%% ── Track 3: 7d (INDEPENDENT — insurance/capex) ──
\draw[ar] (S7d.east) -- node[lb,above]{capitalise} (D7d.west);
\draw[ar] (D7d.east) -- node[lb,above]{review} (R7d.west);

%% R7d Yes → F7d  (FV update needed — fix carrying value)
\draw[ar] (R7d.south) -- node[lb,right]{Yes} (F7d.north);
%% F7d → S7d  (loop: exit F7d.west → left to x=6 (S7d south x) → straight up into S7d.south)
\draw[ar] (F7d.west) -- node[lb,above]{redo} (6,-11.8) -- (S7d.south);

%% R7d No → OC  (east — R7d and OC share y=-9.5; straight line)
\draw[ar] (R7d.east) -- node[lb,above]{No} (OC.west);

%% ── Track 4: 7e (INDEPENDENT — cost of sale) ──
\draw[ar] (S7e.east) -- node[lb,above]{update} (D7e.west);
%% D7e → OC  (east to staggered channel x=22.0, north to OC level, east into OC)
%%   Uses same staggered channel as D7c but approaches from below.
\draw[ar] (D7e.east) -- node[lb,right]{done} (22.0,-13.5) -- (22.0,-10.5) -- (OC.south west);

\end{pgfonlayer}
\end{tikzpicture}
}%

\end{center}
\end{landscape}

% ==========================================================================
%  PAGE 4 OF 5 — Phase 2b: Step 7 Sub-steps f--j (Trust-Level & Outputs)
%  Logic (revised): 7f, 7g, 7h, 7i run in parallel from OC; all must complete
%                  before 7j (final compilation) → OD
% ==========================================================================
\begin{landscape}
\vspace*{-0.4cm}
\begin{center}
{\sffamily\bfseries\large\textcolor{navy}{Phase 2b --- Reporting Pack Compilation: Trust-Level \& Outputs (Steps 7f--7j)}}\par\vspace{0.25cm}

\resizebox{!}{0.80\textheight}{%
\begin{tikzpicture}[every node/.style={font=\sffamily}]

%% ===== NODES =====
\node[opc] (OC) at (1,-1.5) {C};
\node[lb, above=0.1cm of OC] {\textit{from Ph.2a}};

%% Track 1: 7f — Borrowing cost amortisation
\node[proc] (S7f) at (6,-1.5)  {\PN{7f}{Borrowing cost amortisation (303.03, 303.04)}{Preparer}{BD3}};
\node[docn] (D7f) at (13,-1.5) {\DN{Borrowing cost prepaid tab; straight-line schedule}};

%% Track 2: 7g — Trust-level accruals + support loop
\node[proc] (S7g) at (6,-4.6)  {\PN{7g}{Trust-level accruals (882.16) \& operating accruals (882.02)}{Preparer}{BD3}};
\node[docn] (D7g) at (13,-4.6) {\DN{Accruals tab; CPG invoices; contracts; prior-period patterns}};
\node[dec]  (R7g) at (19,-4.6) {Support\\missing?};
\node[lbox] (F7g) at (19,-6.4) {Obtain missing\\invoices / CPG\\confirmation};

%% Track 3: 7h — Inter-group loan + distribution payable
\node[proc] (S7h) at (6,-7.7)  {\PN{7h}{Inter-group loan (882.08) \& distribution payable (882.03)}{Preparer}{BD3}};
\node[docn] (D7h) at (13,-7.7) {\DN{Unitholders tab; CPG written approval for distribution}};

%% Track 4: 7i — Lease/IP/loan maturity checks + self-review loop
\node[proc] (S7i) at (6,-10.8)  {\PN{7i}{Lease review, IP book value check, loan maturity flag}{Preparer}{BD4}};
\node[docn] (D7i) at (13,-10.8) {\DN{Lease schedule; Capex tab (IP vs valuation); StG maturity flag}};
\node[dec]  (SR)  at (19,-10.8) {Self-review:\\gaps found?};
\node[lbox] (FSR) at (19,-12.6) {Identify gaps \&\\complete entries\\{\textit{Preparer}}};

%% Join/synchronisation bar — standard BPMN parallel join (thick horizontal bar, no text)
%% Spans x=19.5 to x=26 at y=-13.35; arrows land on top, output leaves from bottom centre
\node[rectangle, fill=navy, draw=navy, minimum width=6.5cm, minimum height=0.18cm,
      inner sep=0pt] (J7) at (22.75,-13.35) {};
\node[lb, right=0.05cm of J7, font=\sffamily\fontsize{5}{6}\selectfont,
      text=navy] {\textit{sync}};

%% Final: 7j — Compile outputs (NAV last) → OD
\node[proc] (S7j) at (6,-14.3)  {\PN{7j}{NAV Statement, Dashboard, ICR, unit price}{Preparer}{BD4}};
\node[docn] (D7j) at (13,-14.3) {\DN{NAV per unit; Dashboard; GL/PL/BS/CFS tabs; OE formula check}};

\node[opc] (OD) at (19,-14.3) {D};
\node[lb, above=0.1cm of OD] {\textit{Ph.3: Review}};

%% ===== SWIM-LANE BACKGROUND =====
\begin{scope}[on background layer]
  \fill[palesky!20] (0,-0.3) rectangle (26,-15.8);
  \node[lanelbl, fill=steel!65, minimum width=15.8cm, minimum height=0.85cm,
        rounded corners=2pt] at (-0.7,-8.0) {\textcolor{white}{Preparer --- Step 7 Compilation (Trust-Level \& Final Outputs: 7f--7j)}};
  %% Track separators
  \draw[midgray, dashed, line width=0.35pt] (0,-3.05) -- (26,-3.05);
  \draw[midgray, dashed, line width=0.35pt] (0,-6.15) -- (26,-6.15);
  \draw[midgray, dashed, line width=0.35pt] (0,-9.25) -- (26,-9.25);
  \draw[midgray, dashed, line width=0.35pt] (0,-12.55) -- (26,-12.55);
\end{scope}

%% ===== ARROWS =====
\begin{pgfonlayer}{arrows}

%% Parallel-start bus — matches Phase 2a OB pattern exactly
%% OC → S7f direct (same y level), then spine drops south branching to S7g, S7h, S7i
\coordinate (BUS2) at (1.0,-4.6);
\coordinate (BUS3) at (1.0,-7.7);
\coordinate (BUS4) at (1.0,-10.8);

%% OC → S7f  (direct east, same row)
\draw[ar] (OC.east) -- node[lb,above]{parallel} (S7f.west);

%% Vertical bus from OC south (no arrowhead, rounded corners)
\draw[draw=textgray, line width=0.8pt, rounded corners=6pt] (OC.south) -- (BUS4);

%% Branches with rounded corners ar style
\draw[ar] (BUS2) -- node[lb,above]{parallel} (S7g.west);
\draw[ar] (BUS3) -- node[lb,above]{parallel} (S7h.west);
\draw[ar] (BUS4) -- node[lb,above]{parallel} (S7i.west);

%% ----------------------------------------------------------
%% Convergence: each track drops straight down its own corridor into
%% the top of the sync bar (J7) at a staggered x position.
%% Bar spans ~x=19.5 to x=26 at y=-13.35.
%% Landing x positions (right to left): 25.5, 24.0, 22.5, 21.0
%% Output departs from bar bottom at x=22.75 (centre) → west into S7j.
%% ----------------------------------------------------------

%% Track 1 (7f) — corridor x=25.5, lands at bar top right
\draw[ar] (S7f.east) -- node[lb,above]{amortise} (D7f.west);
\draw[ar] (D7f.east) -- node[lb,above]{done} (25.5,-1.5) -- (25.5,-13.35);

%% Track 2 (7g) — corridor x=24.0, lands next left on bar
\draw[ar] (S7g.east) -- node[lb,above]{post} (D7g.west);
\draw[ar] (D7g.east) -- node[lb,above]{check} (R7g.west);
\draw[ar] (R7g.south) -- node[lb,right]{Yes} (F7g.north);
\draw[ar] (F7g.west) -- node[lb,above]{redo} (6,-6.4) -- (S7g.south);
\draw[ar] (R7g.east) -- node[lb,above]{No} (24.0,-4.6) -- (24.0,-13.35);

%% Track 3 (7h) — corridor x=22.5, lands next left on bar
\draw[ar] (S7h.east) -- node[lb,above]{record} (D7h.west);
\draw[ar] (D7h.east) -- node[lb,above]{done} (22.5,-7.7) -- (22.5,-13.35);

%% Track 4 (7i) — corridor x=21.0, lands leftmost on bar
\draw[ar] (S7i.east) -- node[lb,above]{check} (D7i.west);
\draw[ar] (D7i.east) -- node[lb,above]{review} (SR.west);
\draw[ar] (SR.south) -- node[lb,right]{Yes} (FSR.north);
\draw[ar] (FSR.west) -- node[lb,above]{fix} (6,-12.6) -- (S7i.south);
\draw[ar] (SR.east) -- node[lb,above]{No} (21.0,-10.8) -- (21.0,-13.35);

%% Sync bar → 7j: departs from bar bottom centre (x=22.75) → west to S7j.north
\draw[ar] (22.75,-13.44) -- node[lb,right]{sync} (22.75,-16.0) -- (6,-16.0) -- (S7j.south);

%% 7j → Outputs → OD
\draw[ar] (S7j.east) -- node[lb,above]{compile} (D7j.west);
\draw[ar] (D7j.east) -- node[lb,above]{exit} (OD.west);

\end{pgfonlayer}

\end{tikzpicture}
}%

\end{center}
\end{landscape}

% ==========================================================================
%  PAGE 5 OF 5 — Phase 3: Steps 8--10  (Review & Client Distribution)
%  Lanes: Preparer | 1st & 2nd Reviewer | CPG / Archive
% ==========================================================================
\begin{landscape}
\vspace*{-0.4cm}
\begin{center}
{\sffamily\bfseries\large\textcolor{navy}{Phase 3 --- Internal Review \& Client Distribution (Steps 8--10)}}\par\vspace{0.25cm}

\resizebox{0.96\linewidth}{!}{%
\begin{tikzpicture}[every node/.style={font=\sffamily}]

%% ===== LAYOUT =====
%% Preparer lane:           y = -1.8 to -5.0   (Step 8, CLR box)
%% 1st & 2nd Reviewer lane: y = -5.0 to -10.0  (Step 9, R9a, R9b, RET)
%% CPG / Archive lane:      y = -10.0 to -15.2 (Step 10, R10, RESUB, ARC)
%% End lane:                y = -15.2 to -16.8 (END terminal)

%% ===== NODES =====

\node[opc] (OD) at (1,-3) {D};
\node[lb, above=0.1cm of OD] {\textit{from Ph.2b}};

%% --- Preparer lane ---
\node[proc] (S8) at (6,-3) {\PN{8}{Send completed pack to 1st \& 2nd Reviewer}{Preparer}{BD4}};
\node[docn] (D8) at (13,-3) {\DN{Full reporting workbook via UFS Shared Folder}};
\node[lbox] (CLR) at (20,-3) {Preparer clears\\all comments;\\re-submits pack};

%% --- 1st & 2nd Reviewer lane ---
%% COMBINED LOGIC: single diamond replaces Reviewer comments? + Checks failed?
%%   S9 → R9 (issues found?) → Yes → FIX (return/clear) → S8 loop
%%                            → No  → D9 → S10
\node[proc/rev] (S9)  at (6,-7)    {\PN{9}{Review pack independently; document comments; sign checklists}{1st \& 2nd Reviewer}{BD5}};
\node[dec]      (R9)  at (13,-7)    {Issues\\found?};
\node[lbox]     (FIX) at (20,-7)    {Return to Preparer;\\clear comments \&\\address failed items};
%% SGN: small action block directly below R9
\node[lbox]     (SGN) at (13,-8.8)  {Sign\\Checklist};
%% D9: checklist doc moved to just above S10 in CPG lane
\node[docn]     (D9)  at (6,-8.8)   {\DN{NAV checklist, BAS checklist, Distribution checklist signed}};


%% --- CPG / Archive lane ---
\node[proc/rev] (S10) at (6,-11.5) {\PN{10a}{Send approved NAV pack \& monthly report to CPG}{1st Reviewer}{BD5}};
\node[docn]     (D10) at (13,-11.5) {\DN{NAV pack, monthly report sent to CPG}};
\node[dec]      (R10) at (18,-11.5) {CPG\\comments?};
\node[lbox]     (RESUB) at (23,-11.5) {Address CPG\\comments; resubmit\\revised pack};
\node[proc, minimum width=3.6cm] (ARC) at (8,-14.5) {\PN{10b}{Archive CPG-approved pack in UFS Shared Folder}{1st Reviewer}{BD5}};
\node[docn] (DARC) at (15,-14.5) {\DN{Final approved pack archived in UFS Shared Folder}};
\node[term, minimum width=4cm] (END) at (15,-16.3) {END};

%% ===== SWIM-LANE BACKGROUNDS =====
\begin{scope}[on background layer]
  \fill[palesky!20] (0,-1.8) rectangle (25.5,-5.0);
  \fill[infobg!30]  (0,-5.0) rectangle (25.5,-10.0);
  \fill[critbg!35]  (0,-10.0) rectangle (25.5,-15.2);
  \fill[palesky!5]  (0,-15.2) rectangle (25.5,-16.8);
  \node[lanelbl, fill=steel!65, minimum width=3.2cm, minimum height=0.85cm,
        rounded corners=2pt] at (-0.7,-3.4) {Preparer};
  \node[lanelbl, fill=infoteal, minimum width=5.0cm, minimum height=0.85cm,
        rounded corners=2pt] at (-0.7,-7.5) {1st \& 2nd Reviewer};
  \node[lanelbl, fill=critred!75, minimum width=5.2cm, minimum height=0.85cm,
        rounded corners=2pt] at (-0.7,-12.6) {CPG / Archive};
  \draw[midgray, dashed, line width=0.35pt] (0,-5.0)  -- (25.5,-5.0);
  \draw[midgray, dashed, line width=0.35pt] (0,-10.0) -- (25.5,-10.0);
  \draw[midgray, dashed, line width=0.35pt] (0,-15.2) -- (25.5,-15.2);
\end{scope}

%% ===== ARROWS =====
%% DEPENDENCY CHAIN (strictly sequential):
%%   OD → S8 → D8 → S9 → R9 → [Yes: FIX → S8] [No: → D9 → S10]

%%   S10 → D10 → R10 → [Yes: RESUB → S10] [No: → ARC → DARC → END]
%%
%% ARROW-CROSSING STRATEGY:
%%   Top corridor staggered: CLR return at y=-1.8, RET return at y=-1.2
%%   This prevents the two return loops from overlapping.
%%   R9a→CLR: right corridor at x=21.5, rises to y=-1.8, then west.
%%   RET→S8: right corridor at x=24, rises to y=-1.2, then west (above CLR route).
%%   R9b→S10: drops at x=18, goes west at y=-10.0 (above RESUB return at y=-13.0).
%%   RESUB→S10: bottom corridor at y=-13.0, west to x=3.5, north into S10.south.
%%   R10→ARC: south at x=18 to y=-13.5, west to ARC.north.
%%   D8→S9: south to lane boundary y=-5.2, west to S9 x, south into S9.north.
\begin{pgfonlayer}{arrows}

%% OD → S8  (entry)
\draw[ar] (OD.east) -- node[lb,above]{enter} (S8.west);

%% S8 → D8  (submit pack)
\draw[ar] (S8.east) -- node[lb,above]{submit} (D8.west);

%% D8 → S9  (DEPENDENCY: submitted pack triggers review)
%%   Route: D8 south to lane boundary, west to S9 x, south into S9.north.
\draw[ar] (D8.south) -- node[lb,right]{assign} (13,-5.2) -- (6,-5.2) -- (S9.north);

%% S9 → R9  (review feeds into single combined issues check)
\draw[ar] (S9.east) -- node[lb,above]{assess} (R9.west);

%% R9 Yes → FIX  (issues found — return to preparer to clear all)
\draw[ar] (R9.east) -- node[lb,above]{Yes} (FIX.west);

%% FIX → CLR  (Reviewer returns pack to Preparer; Preparer clears comments)
\draw[ar] (FIX.north) -- node[lb,right]{return} (CLR.south);

%% CLR → S8  (Preparer re-submits cleared pack — corridor y=-1.5, west to S8.north)
\draw[ar] (CLR.north) -- node[lb,above]{resubmit} (20,-1.5) -- (6,-1.5) -- (S8.north);

%% R9 No → SGN  (no issues — Sign Checklist action directly below R9)
\draw[ar] (R9.south) -- node[lb,right]{No} (SGN.north);

%% SGN → D9  (signing produces checklist doc — same row, to the left)
\draw[ar] (SGN.west) -- node[lb,above]{sign} (D9.east);

%% D9 → S10  (signed checklists feed into CPG send step — straight down)
\draw[ar] (D9.south) -- node[lb,right]{approve} (S10.north);

%% S10 → D10  (send to CPG)
\draw[ar] (S10.east) -- node[lb,above]{send} (D10.west);

%% D10 → R10  (CPG reviews)
\draw[ar] (D10.east) -- node[lb,above]{review} (R10.west);

%% R10 Yes → RESUB  (CPG comments — address and resubmit)
\draw[ar] (R10.east) -- node[lb,above]{Yes} (RESUB.west);

%% RESUB → S10  (loop: bottom corridor y=-13.0, west straight to S10.south — no acute angle)
\draw[ar] (RESUB.south) -- node[lb,below]{revise} (23,-13.0) -- (6,-13.0) -- (S10.south);

%% R10 No → ARC  (no CPG comments — archive)
%%   Route: exit R10.north → up to y=-10.6 → RIGHT to x=24.5 (clear of RESUB at x=23)
%%   → south to y=-13.6 (above DARC.north at y=-14.05) → west to x=5.5 (left of ARC)
%%   → south to y=-14.5 (ARC vertical centre) → east into ARC.west (middle of left edge).
%%   Final segment is a clean horizontal arrowhead into the left edge. Zero crossings.
\draw[ar] (R10.north) -- node[lb,right]{No} (18,-10.6) -- (24.5,-10.6) -- (24.5,-13.6) -- (5.5,-13.6) -- (5.5,-14.5) -- (ARC.west);

%% ARC → DARC  (archive the approved pack)
\draw[ar] (ARC.east) -- node[lb,above]{archive} (DARC.west);

%% DARC → END  (archived output closes cycle — straight down since END is directly below DARC)
\draw[ar] (DARC.south) -- node[lb,right]{close} (END.north);

\end{pgfonlayer}
\end{tikzpicture}
}%

\end{center}
\end{landscape}

% ============================================================================
\clearpage
\section{Detailed Process Action Narratives}
\label{app:process-detail}

This section provides the full narrative for each step, as referenced in the flowchart in Appendix~\ref{app:flowchart}.

\subsection*{Step 1 — CPG: Document Upload (BD1)}
CPG uploads the St George bank statement, bank information responses, and all supporting invoices (remediation, insurance, sale-related) to the designated CPG SharePoint location for Preparer access. For the complete list of required source documents and filing locations, see Section~\ref{sec:phase1} (Steps~1--2: Collect Source Documents). For the fund's current sale status, insurance claim position, and loan maturity context that inform document expectations, see Section~\ref{sec:sale-status} (Current Fund Status) and Section~\ref{sec:background} (Property Description).

\subsection*{Step 2 — Client Coordinator (UFS): NAB Statement (20th \& BD1)}
Downloads the NAB bank statement from NAB online banking. Archives it in the Trust's folder on the UFS Shared Folder for Preparer access. See Section~\ref{sec:phase1} for the full Phase~1 source document collection workflow and archiving requirements. The NAB statement is the primary source for the bank reconciliation (Step~3, Section~\ref{sec:yardi-rec}) and the cut-off analysis (Appendix~\ref{sec:est-cutoff}).

\subsection*{Step 3 — Preparer: Yardi Import and Bank Reconciliation (20th \& BD1)}
Downloads both Yardi TB movement exports from the Yardi system (\texttt{cs4001} and \texttt{22chts}). Downloads the St George bank statement from SharePoint. Reconciles the Yardi NAB TB against the NAB bank statement in the \textbf{Yardi vs.\ Bank Stat Rec} tab using the five-case classification (see Table~\ref{tab:fivecases} and the full reconciliation procedure at Section~\ref{sec:yardi-rec}). The reconciliation must balance to zero before proceeding. Apply the double-count prevention checkpoints (Appendix~\ref{sec:est-double-count}) to identify any items recorded in both Yardi and UFS.

\subsection*{Step 4 — Preparer: Bank Information Request (20th \& BD1)}
Prepares and submits a bank information request to CPG covering all unidentified or unmatched transactions. Queries cover: missing invoices, unidentified payments, cut-off items, insurance claim remittances, sale invoices. See the full bank information request procedure at Section~\ref{sec:bankreq}. For the ``incurred'' test governing when expenses are deductible and the related cut-off rules, see Appendix~\ref{sec:est-cutoff} and the legislative references at \S\ref{sec:ref-income-tax} (TR~97/7).

\subsection*{Step 5 — CPG: Invoice and Information Responses (20th to BD1)}
Provides written responses to bank information requests. Uploads all supporting invoices and remittance advices to CPG SharePoint. Confirms sale status and loan maturity position if requested (see Section~\ref{sec:sale-status} for sale context and Appendix~\ref{app:estimates} for loan maturity reclassification considerations). For the insurance claim accounting and CGT treatment of invoices received in this step, see Appendix~\ref{sec:est-insurance}. For withholding tax obligations triggered by distribution instructions, see Appendix~\ref{sec:est-wht} (Withholding Tax) and the legislative references at \S\ref{sec:ref-wht}.

\subsection*{Step 6 — Preparer: Yardi TB Revision and Import (BD1)}
Reviews the raw Yardi TB; reclassifies trust-level expenses from P\&L to \acct{882.16}; reclassifies sale-related invoices to \acct{709} and leasing invoices to \acct{780.12}. Imports the revised TB into the \textbf{Import\_Yardi} tab. For the full reclassification procedure see Section~\ref{sec:yardi-revise} (Step~6a); for the import procedure see Section~\ref{sec:yardi-import} (Step~6b). Each invoice must be classified as sale-related, leasing-related, or ordinary operating per Appendix~\ref{sec:est-capex-guide} (Capital vs.\ Revenue Classification) and the double-count prevention checkpoints at Appendix~\ref{sec:est-double-count}.

\subsection*{Step 7 — Preparer: Full Reporting Pack Compilation (BD2--BD4)}
Compiles the full reporting pack. Sub-steps 7a--7j follow the recommended order (see also the detailed Phase~2 compilation procedure at Section~\ref{sec:phase2}):
\begin{enumerate}[nosep,label=(\alph*)]
  \item Property-level entries (Categories A/B/C, Section~\ref{sec:phase2}) --- verify Yardi import, receivables, prepayments, and operating expenses. Apply double-count prevention checkpoints (Appendix~\ref{sec:est-double-count});
  \item NAB cut-off Sundry Debtor entries (Cases 2--5, Section~\ref{sec:cutoff}) --- cross-check against Yardi-imported receivables to avoid duplication. For cut-off timing rules, the ``incurred'' test, and late invoice risks near the sale, see Appendix~\ref{sec:est-cutoff} (Period Cut-Off) and Appendix~\ref{sec:est-cutoff-risks} (Common Cut-Off and Classification Errors);
  \item St George interest (\acct{303.01}) and line fee (\acct{303.02}) on a cash basis when debited from the facility --- no month-end accrual raised for either item (Section~\ref{sec:stgeorge-entries}). For the loan facility terms, maturity date, and current/non-current reclassification assessment at each financial statement date, see Appendix~\ref{sec:est-loan-reclass} (Loan Classification). For the borrowing cost amortisation regime (s25-25 and s40-880, including the immediate write-off on early repayment at settlement), see Appendix~\ref{sec:est-s2525-s40880};
  \item Insurance claim income (\acct{73.02}) and cyclone capex (\acct{781.01}) --- classify each remediation invoice as repair vs.\ capex per Appendix~\ref{sec:est-capex-guide} (Capital vs.\ Revenue Classification); for the comprehensive accounting, income tax, CGT cost base interaction, GST, and post-settlement retention treatment of insurance claims, see Appendix~\ref{sec:est-insurance} (Insurance Claims). For CGT implications of remediation works performed around the sale, see Appendix~\ref{sec:est-cgt} (Capital Gains Tax);
  \item Deferred cost of sale (\acct{709}) per invoices from SharePoint or Yardi --- check Yardi TB for CPG-posted entries before duplicating. For the de-recognition journal at settlement and interaction with the gain/loss on disposal calculation, see Appendix~\ref{sec:est-hfs-accounting} (Held-for-Sale Accounting);
  \item Borrowing cost amortisation (\acct{303.03}, \acct{303.04}) --- verify remaining facility term alignment (Appendix~\ref{sec:est-borrow-amort}); see also Section~\ref{sec:borrowing-amort} for the recording procedure;
  \item Trust-level accruals (\acct{882.16}, Section~\ref{sec:trust-accruals}) and operating accruals (\acct{882.02}) --- verify fee size indicators (Appendix~\ref{sec:est-fee-size}) and CPI adjustments (Appendix~\ref{sec:est-accruals-settlement});
  \item CPG inter-group loan (\acct{882.08}) and distribution payable (\acct{882.03}) if applicable --- record distribution only upon receipt of written CPG approval; for the complete distribution component determination framework (ordinary income, net assessable capital gain, CGT discount component, tax deferred, ROC), see Appendix~\ref{sec:est-distribution-components} (Distribution Components). For the distribution waterfall and priority of payments, see Appendix~\ref{sec:est-distribution-waterfall};
  \item Lease review, IP book value check, and St George loan maturity flag --- assess loan reclassification (Appendix~\ref{sec:est-loan-reclass}); confirm IP carrying value (\acct{780}) reconciles to the latest independent valuation plus post-valuation capex (Appendix~\ref{sec:est-reval}, Investment Property Revaluation); review lease status and amortisation alignment (Appendix~\ref{sec:est-leasing}, Lease Status Review);
  \item NAV Statement, Dashboard, ICR, unit price --- confirm OE formula, verify fee-to-metric dependencies are current.
\end{enumerate}

\subsection*{Step 8 — Preparer: Pack Submission for Review (BD4)}
Sends the completed pack to the 1st Reviewer and 2nd Reviewer via the UFS Shared Folder. If the pack is returned from review with comments (via Step~9), the Preparer clears all review comments and re-submits the updated pack. The pack does not advance to sign-off until all comments are resolved. Before submission, the Preparer should verify the pre-sign-off estimate review checklist (Section~\ref{sec:review}) and confirm all checklist items (NAV, BAS, Distribution) at Section~\ref{sec:checklists} are complete. All accounting estimates and period-end attention items (Appendix~\ref{app:estimates}) should be documented and ready for Reviewer inspection.

\subsection*{Step 9 — 1st and 2nd Reviewer: Review and Sign-Off (BD5)}
Both Reviewers review the pack independently. If issues are found, the pack is returned to the Preparer (via FIX), who clears comments (CLR) and re-submits (Step~8). This loop continues until no issues remain. Once all issues are resolved, the 2nd Reviewer signs off all checklists (NAV, BAS, Distribution) per the checklist procedures at Section~\ref{sec:checklists}. Reviewers confirm: all balances reconcile, accruals are supported and not double-counted (Appendix~\ref{sec:est-double-count}), P\&L items correctly classified (including repair vs.\ capex per Appendix~\ref{sec:est-capex-guide}), OE formula excludes Performance Fee, fee size indicators are current (Appendix~\ref{sec:est-fee-size}), loan maturity classification is assessed (Appendix~\ref{sec:est-loan-reclass}), and accounting estimate attention items (Appendix~\ref{app:estimates}) have been considered.

\subsection*{Step 10 — 1st Reviewer: Client Distribution and Archiving (BD5)}
\begin{enumerate}[nosep,label=(\alph*)]
  \item \textbf{Send approved NAV pack \& monthly report to CPG (BD5):} The 1st Reviewer sends the completed, signed-off NAV pack and monthly report to CPG. If CPG returns comments, the Preparer addresses the comments and the 1st Reviewer resubmits the revised pack to CPG. This loop continues until CPG approval is confirmed. For quarterly distribution periods, include the distribution workings and withholding tax calculations (see Appendix~\ref{sec:est-distribution-waterfall} and Appendix~\ref{sec:est-wht}); for the quarterly BAS and distribution procedures, see Section~\ref{sec:checklists}.
  \item \textbf{Archive CPG-approved pack in UFS Shared Folder (BD5):} Once CPG approval is confirmed (no further comments), the 1st Reviewer archives the final approved pack in the UFS Shared Folder as the permanent record for the period. The archive is the definitive version of the reporting pack and must not be modified after archiving. For document retention requirements and regulatory obligations, see the statutory fee and lodgement deadlines at \S\ref{sec:ref-statutory} (Appendix~D).
\end{enumerate}



\vfill
\begin{center}
\textcolor{navy}{\rule{6cm}{0.5pt}}\\[0.6em]

\end{center}

% ============================================================================
% APPENDIX C — ACCOUNTING ESTIMATES AND PERIOD-END ATTENTION ITEMS
% ============================================================================
\clearpage

\section{Accounting Estimates and Period-End Attention Items}
\label{app:estimates}

This appendix provides consolidated guidance on accounting estimates, professional judgements, and period-end attention items requiring regular review by the Preparer and Reviewers. The items addressed here are not routine journal entries; they demand awareness of the fund's evolving circumstances --- particularly the active property sale, approaching St George loan maturity, ongoing cyclone insurance claim, and the trust's pending wind-up --- and the exercise of informed professional judgement at each reporting date. Failure to address these items in a timely and considered manner may result in financial misstatement, regulatory exposure, or client dissatisfaction.

The appendix is structured into seven parts, organised in the logical sequence of accounting, tax, legal, and financial analysis: Part~I addresses the balance sheet in full, opening with the held-for-sale assessment (which frames how the property is presented and measured across all other items) before progressing through investment property valuation, receivables, guarantees, loan classification, and tax balances. Parts II through VII address the income statement, capital expenditure, CGT and cut-off, insurance, trust administration, and period-end controls:

\begin{enumerate}[nosep]
    \item \textbf{Balance Sheet Recognition and Measurement} --- held-for-sale assessment (business, accounting, tax, legal, and financial implications); investment property revaluation; aging bad debt provisioning; bank guarantee recognition, disclosure, and expiry monitoring; revenue received in advance and supplier credits; St~George loan current/non-current classification; tax payable vs.\ receivable reclassification; tax loss carryforward position and utilisation conditions.
    \item \textbf{Income Statement --- Recurring Amortisation and Fee Monitoring} --- leasing fees, statutory expenses, borrowing costs, fee size indicators, accruals.
    \item \textbf{Capital vs.\ Revenue Classification and Tax Deduction Regimes} --- capex vs.\ repair, component accounting, s25-25 and s40-880.
    \item \textbf{Capital Gains Tax and Period Cut-Off} --- CGT Events A1 and H1, cost base, works around sale, late invoices.
    \item \textbf{Insurance Claims} --- accounting, tax, CGT, GST, post-sale retention.
    \item \textbf{Trust Administration, Distributions, and Compliance} --- STA, distribution components, waterfall, WHT, unitholders, registry.
    \item \textbf{Controls, Reconciliation, and Other Period-End Items} --- double-count prevention, other estimates.
\end{enumerate}



% ============================================================================
% PART I: BALANCE SHEET RECOGNITION AND MEASUREMENT
% ============================================================================

\vspace{1.2cm}
\begin{tcolorbox}[
    enhanced, breakable,
    colback=navy, colframe=navy,
    boxrule=0pt,
    borderline south={2pt}{0pt}{gold},
    sharp corners,
    left=14pt, right=10pt, top=10pt, bottom=10pt,
    label={sec:est-part-bs}
]
{\sffamily\bfseries\footnotesize\textcolor{gold}{Part I\enspace\textbullet\enspace Appendix~C}}\par
\vspace{3pt}
{\sffamily\bfseries\large\textcolor{white}{Balance Sheet Recognition and Measurement}}
\end{tcolorbox}
\begin{tcolorbox}[
    enhanced, breakable,
    colback=palesky, colframe=navy!30!white,
    boxrule=0.4pt, sharp corners,
    left=10pt, right=10pt, top=6pt, bottom=6pt
]
\small\itshape This part addresses all period-end assessments that directly affect the balance sheet, arranged in order from the most strategically significant to supporting items. It opens with the held-for-sale assessment: whether 22~Constance Street meets the AASB~5 criteria and the implications across accounting, tax, legal, and financial reporting dimensions. It then addresses investment property fair value and revaluation triggers; the aging and provisioning of trade receivables; bank guarantee recognition, disclosure, and expiry monitoring; revenue received in advance and supplier credit reclassification; the current/non-current classification of the St~George borrowing as maturity approaches; the sign and correct classification of tax balances; and the trust's tax loss carryforward position and the conditions under which accumulated losses may be applied against future income.
\end{tcolorbox}
\vspace{0.3cm}

% ============================================================================
\subsection{Held-For-Sale Assessment --- Business, Accounting, Tax, Legal, and Financial Issues}
\label{sec:est-hfs}
\legbox{AASB~5 \textit{Non-current Assets Held for Sale and Discontinued Operations} \legalcite{AASB~5}{sec:ref-hfs}; AASB~140 \textit{Investment Property} (para.~57--58) \legalcite{AASB~140}{sec:ref-accounting}; AASB~101 \textit{Presentation of Financial Statements} \legalcite{AASB~101}{sec:ref-accounting}; s104-10 ITAA~1997 (CGT Event A1) \legalcite{s104-10}{sec:ref-cgt}; s40-295 ITAA~1997 (balancing adjustment) \legalcite{s40-295}{sec:ref-taxdep}; \textit{Corporations Act 2001} s601FC; \textit{Property Occupations Act 2014} (Qld).}

The fund is currently in an active sale process for 22~Constance Street, with a sale contract exchanged and a \$1,325,000 deposit received (held in St~George, \acct{680}, matched by \acct{884} liability). The approaching St~George loan maturity (15~June~2026) makes settlement critical. This section provides a comprehensive framework for the held-for-sale (HFS) assessment across five dimensions: business, accounting, tax, legal, and financial.

\begin{critical}[Held-for-Sale Assessment — Required at Every Reporting Date]
The held-for-sale assessment must be performed at \textbf{every reporting date} from the point the sale became probable. Once the criteria are met, the accounting, tax, and disclosure consequences described below apply immediately. Given the fund's contractual position, the Preparer and Reviewer must assess and document the HFS status each month.
\end{critical}

% -----------------------------------------------------------------------
\subsubsection{Business Dimension --- Strategic Context and Fund Wind-Up}
% -----------------------------------------------------------------------

The decision to sell 22~Constance Street reflects the fund's investment strategy (acquire, lease-up, sell or refinance) reaching its terminal phase. Key business considerations:

\begin{enumerate}[nosep]
    \item \textbf{Investment mandate:} The Constitution and Information Memorandum describe the fund as a finite-life vehicle. The sale is consistent with --- and required by --- the investment mandate. CPG as responsible entity must formally resolve to sell and communicate that decision to unitholders per the trust deed.

    \item \textbf{Probability of sale:} A binding sale contract is in place (subject to conditions). This meets the ``highly probable'' threshold at the business level. Management's ability to withdraw from the sale without significant penalty must be assessed in light of the contract terms. Once exchange has occurred, ``highly probable'' is satisfied.

    \item \textbf{Fund wind-up timeline:} Following settlement, the fund must distribute net proceeds, finalise tax returns, lodge the final PAYG Annual Investment Income Report and trust tax return, discharge the St~George facility, return any residual cash, and apply to deregister the trust (or allow it to lapse per the trust deed). Coordinate with CPG and the tax agent on the wind-up timeline.

    \item \textbf{Unitholder communication:} CPG must keep unitholders informed of the sale status, expected settlement date, and distribution of proceeds. Assess whether the trust deed or any side agreement requires a unitholder vote or consent before settlement.

    \item \textbf{Ongoing leasing and property management:} While the property is under contract, existing leases continue. Lease obligations (rent collection, maintenance, outgoings recovery) must continue until settlement. CPG must confirm the property management handover plan for the period between exchange and settlement.
\end{enumerate}

% -----------------------------------------------------------------------
\subsubsection{Accounting Dimension --- AASB~5 vs.\ AASB~140 Interaction}
\label{sec:est-hfs-accounting}
% -----------------------------------------------------------------------

The relationship between AASB~5 and AASB~140 is one of the most commonly misunderstood areas in property fund accounting. The resolution for 22~Constance Street is as follows:

\paragraph{AASB~5 classification criteria}

Under AASB~5 \textit{Non-current Assets Held for Sale and Discontinued Operations} (para.~6--12), a non-current asset shall be classified as held for sale when:
\begin{itemize}[nosep]
    \item Its carrying amount will be recovered principally through a sale transaction rather than continuing use; \textbf{and}
    \item The asset is \textbf{available for immediate sale} in its present condition; \textbf{and}
    \item The sale is \textbf{highly probable}: management is committed to a plan to sell; an active programme to locate a buyer is in place; the asset is being actively marketed at a reasonable price; the sale is expected to complete within 12~months; and it is unlikely the plan will be significantly changed or withdrawn.
\end{itemize}

For 22~Constance Street: the property is available for sale, a binding contract exists, and settlement is expected within 12~months. The ``highly probable'' criterion is met.

\paragraph{The AASB~140 carve-out --- the critical nuance}
\vspace{2pt}%
\begin{warning}[AASB 5 Measurement Does Not Apply to Fair Value IP]
\textbf{AASB~5 measurement requirements do NOT apply to investment property measured under the AASB~140 fair value model.} This is the single most important accounting point for this fund. AASB~5 (para.~5) explicitly scopes out investment property carried at fair value. Because 22~Constance Street is an investment property measured at fair value through profit or loss under AASB~140, the fund does \textbf{not} reclassify the property to a ``held for sale'' asset on the balance sheet and does \textbf{not} apply the AASB~5 measurement rule (lower of carrying amount and fair value less costs to sell).
\end{warning}

Specifically:
\begin{enumerate}[nosep]
    \item \textbf{No balance sheet reclassification:} The property remains in \acct{780} --- Investment Property (non-current asset), measured at fair value under AASB~140, until the date of derecognition (settlement). Do not move it to a ``current assets'' held-for-sale line.

    \item \textbf{Continued fair value measurement:} At each period end, the investment property continues to be assessed for changes in fair value per the AASB~140 procedures (Section~\ref{sec:est-reval}). The potential sale price is the most relevant benchmark for fair value once a binding contract exists.

    \item \textbf{Accounting depreciation:} Under the AASB~140 fair value model, no accounting depreciation is charged on the investment property. AASB~5's rule that ``depreciation ceases on classification as held for sale'' is therefore not applicable here.

    \item \textbf{Div~40/43 tax depreciation:} While accounting depreciation is not charged, \textbf{tax depreciation} (Div~40 plant and Div~43 building allowance) may still accrue until the date of disposal for tax purposes --- see the tax dimension below.

    \item \textbf{Disclosure despite measurement carve-out:} Notwithstanding the measurement carve-out, the notes to the financial statements must disclose that the property is under a binding sale contract, the expected settlement date, and the sale price (net of selling costs). AASB~140 (para.~75) requires disclosure of assets classified as held for sale or subject to disposal in the reconciliation of investment property movements.

    \item \textbf{AASB~110 post-balance date events:} If the sale contract was exchanged or settled \textbf{after} the reporting date, assess whether it is an adjusting or non-adjusting event under AASB~110. Settlement itself (derecognition of the property) is typically a non-adjusting event that requires disclosure if material.
\end{enumerate}

\paragraph{Balance sheet presentation summary}

\begin{table}[H]
\centering\small\sffamily
\caption{Held-for-sale: balance sheet treatment for 22~Constance Street}
\begin{tabularx}{\textwidth}{@{}L{4.5cm} X@{}}
\rowcolor{navy}\textcolor{white}{\bfseries Item} & \textcolor{white}{\bfseries Treatment} \\
\midrule
Investment Property (\acct{780}) & Remains in non-current assets at fair value (AASB~140). No reclassification to current held-for-sale. \\
\rowcolor{warmgray}
Sale deposit received (\acct{884}) & Remains as current liability (Other Creditors) until settlement. \\
Deferred cost of sale (\acct{709}) & Remains as non-current asset; de-recognised against gain/loss on disposal at settlement. \\
\rowcolor{warmgray}
Bank guarantees (tenant security) & Contingent assets (off-balance sheet); disclose per AASB~137; assign to purchaser at settlement. \\
St~George loan (\acct{680}) & Classified as current liability (matures 15~June~2026 --- within 12~months of reporting date). \\
\rowcolor{warmgray}
Trust-level accruals for disposal & Disposal fee (\acct{882.16}), performance fee, settlement adjustments --- accrue when probable and estimable. \\
\bottomrule
\end{tabularx}
\end{table}

\legbox{AASB~5 (para.~5(d) --- measurement carve-out for fair value IP) \legalcite{AASB~5}{sec:ref-hfs}; AASB~140 (para.~57--58 --- no reclassification until derecognition) \legalcite{AASB~140}{sec:ref-hfs}; AASB~101 (para.~54(j) --- HFS presentation line) \legalcite{AASB~101}{sec:ref-hfs}; AASB~2020-1/AASB~2022-6 (current/non-current liability classification) \legalcite{AASB~2020-1}{sec:ref-hfs}.}

% -----------------------------------------------------------------------
\subsubsection{Tax Dimension --- CGT, Depreciation, and Timing}
\label{sec:est-hfs-tax}
% -----------------------------------------------------------------------

\begin{enumerate}[nosep]
    \item \textbf{CGT Event A1 --- disposal of the property:} CGT Event~A1 (s104-10 ITAA~1997) happens when the fund disposes of the property. The CGT event time is the \textbf{earlier of settlement date or contract date}. In practice, for a property under contract:
    \begin{itemize}[nosep]
        \item If the contract is \textbf{unconditional} (all conditions satisfied), the CGT event time is the \textbf{contract date} (not settlement date). This is critical for timing of the CGT liability and for the distribution component calculations.
        \item If the contract is \textbf{conditional} at year end, CGT Event~A1 has not yet occurred. The event happens when the last condition is satisfied.
        \item Confirm the contract conditions with K\&L Gates. Record the CGT event date in the cost base and distribution workpapers.
    \end{itemize}

    \item \textbf{CGT discount:} The fund has held the property for more than 12~months; the 50\% CGT discount (Div~115 ITAA~1997) is available to resident individual unitholders and the trust itself. Corporate and non-resident unitholders cannot access the discount. See Section~\ref{sec:est-cgt} for the full calculation.

    \item \textbf{Div~40 balancing adjustment:} On disposal, a balancing adjustment (s40-295 ITAA~1997) applies to all Div~40 depreciating assets (plant and equipment). The balancing adjustment = \textbf{termination value} (sale proceeds allocated to plant) \textbf{less} adjustable value (cost less total depreciation). A positive adjustment is assessable income; a negative adjustment is a deduction. Engage the tax agent to allocate the contract price between land, building (Div~43), and plant (Div~40) in the final year's depreciation schedule.

    \item \textbf{Div~40 depreciation --- continuity until CGT event:} Div~40 depreciation on plant and equipment accrues until the \textbf{CGT event date}. Once the CGT event has occurred, the balancing adjustment replaces further depreciation deductions.

    \item \textbf{Div~43 building allowance:} If the fund claims the capital works deduction (Div~43), the deduction ceases at the earlier of the CGT event date or settlement. The purchaser picks up the Div~43 effective life from that point.

    \item \textbf{s40-880 blackhole expenditure:} Any s40-880 deductions relating to the fund's structure or the sale that have not yet been fully amortised must be assessed. The remaining unamortised portion may be immediately deductible if the relevant business activity ceases on wind-up (seek tax agent advice).

    \item \textbf{Tax return timing:} If settlement occurs in FY2026, the gain will be reported in the FY2026 trust tax return. If the contract became unconditional before 30~June~2025, the gain may fall in FY2025. Confirm with the tax agent which financial year the CGT event falls in.

    \item \textbf{GST on the sale:} The sale of commercial property is ordinarily a taxable supply for GST purposes, unless sold as a going concern or under the margin scheme. Confirm the GST treatment with the tax agent and K\&L Gates based on the sale contract terms. If the margin scheme applies, the GST calculation is materially different from the standard 1/11th of the sale price.
\end{enumerate}

% -----------------------------------------------------------------------
\subsubsection{Legal Dimension --- Vendor Obligations, Trust Deed, and Settlement}
\label{sec:est-hfs-legal}
% -----------------------------------------------------------------------

\begin{enumerate}[nosep]
    \item \textbf{Sale contract review:} K\&L Gates should review the sale contract for: conditions precedent; vendor disclosure obligations; warranty and indemnity provisions; make-good obligations; assignment of leases and bank guarantees; settlement mechanics; and any post-settlement adjustment formula (settlement statement).

    \item \textbf{Trust deed compliance:} Confirm that the disposal of the property complies with the trust deed:
    \begin{itemize}[nosep]
        \item Does the trust deed require a unitholder resolution to approve the sale?
        \item Does the trust deed specify how disposal proceeds must be distributed?
        \item Are there any pre-emptive rights in favour of unitholders or third parties?
    \end{itemize}

    \item \textbf{ASIC and responsible entity obligations:} CPG as responsible entity under \textit{Corporations Act 2001} s601FC has a duty to act in the best interests of unitholders. CPG must document its decision to sell, the basis for the sale price, and the process followed (arm's length, independent valuation, market testing).

    \item \textbf{Lease assignment at settlement:} All existing leases (Sony Music, 5Point Projects, Popgun Labs) pass to the purchaser at settlement. The sale contract should specify whether leases are assigned or novated, and the mechanics for transfer of security deposits and bank guarantees. Confirm assignment mechanics with K\&L Gates.

    \item \textbf{Bank guarantee assignment:} The tenant bank guarantees (\$33,000 --- 5Point Projects; \$61,600 --- Popgun Labs; Sony Music --- TBC) must be formally assigned to the purchaser at settlement. See Section~\ref{sec:est-bankguarantees} for the full bank guarantee framework.

    \item \textbf{St~George mortgage discharge:} The sale proceeds must discharge the St~George facility (\$5,000,000) on settlement. K\&L Gates must coordinate with St~George's solicitors for the discharge of mortgage and release of the bank's security interest.

    \item \textbf{Vendor warranties and post-settlement obligations:} The sale contract typically includes vendor warranties regarding the property's condition, leases, outgoings, and compliance. Claims under these warranties may survive settlement by 12--24~months. Assess any contingent liabilities arising from warranty exposure and disclose under AASB~137.

    \item \textbf{\textit{Property Occupations Act 2014} (Qld) --- STA:} The STA~QLD must be reconciled and closed as part of settlement. Any residual trust funds held by CPG must be remitted to the fund. Confirm the STA closure process with CPG and the approved auditor (see Section~\ref{sec:est-sta}).
\end{enumerate}

% -----------------------------------------------------------------------
\subsubsection{Financial Dimension --- NAV, ICR, Cash Flow, and Distribution Impact}
\label{sec:est-hfs-financial}
% -----------------------------------------------------------------------

\begin{enumerate}[nosep]
    \item \textbf{NAV per unit --- sale price vs.\ carrying value:} Once a binding sale contract exists, the most reliable indicator of the property's fair value is the net contract price (gross sale price less selling costs). If the net contract price is lower than the current carrying value (\$13,590,288 at 31~January~2026), a revaluation write-down is required under AASB~140. Update the NAV per unit calculation to reflect the expected net realisation and communicate the revised NAV to CPG and unitholders.

    \item \textbf{ICR and debt coverage:} The Interest Coverage Ratio (ICR) will move materially in the periods leading to settlement. Monitor the ICR monthly per the St~George covenant requirements.

    \item \textbf{Cash flow to settlement:} Prepare a cash flow forecast from the current date to the expected settlement date, covering: rental receipts; ongoing operating expenses; interest and line fee payments (St~George); sale-related costs (legal, agent, settlement adjustments); tax payments; and the mortgage discharge. Identify any shortfall risk.

    \item \textbf{Distribution of sale proceeds:} The distribution waterfall on sale is governed by the trust deed and the fund's Constitution. The typical sequence is:
    \begin{enumerate}[nosep]
        \item Repay St~George facility and discharge mortgage.
        \item Pay all fund creditors and accrued liabilities (management fee, disposal fee, legal, settlement costs).
        \item Pay any applicable performance fee (Tier~1: 20\% above 8\% IRR; Tier~2: 40\% above 15\% IRR).
        \item Return of unitholder capital (\$11,438,500).
        \item Distribute residual profit to unitholders.
    \end{enumerate}
    See Section~\ref{sec:est-distribution-waterfall} for the full waterfall framework.

    \item \textbf{Disposal fee:} CPG is entitled to a disposal fee of 1\% of GAV upon sale. Accrue in \acct{882.16} once CPG has confirmed the sale is probable and the GAV is determinable (typically upon exchange of contracts).

    \item \textbf{Performance fee:} Calculate the IRR from the fund's inception to the expected settlement date using the actual cash flows. If Tier~1 or Tier~2 thresholds are exceeded, accrue the performance fee. See Section~\ref{sec:est-fee-size} for the fee size indicator framework.

    \item \textbf{Deferred cost of sale (\acct{709}):} All pre-settlement sale costs are accumulated in \acct{709} and de-recognised at settlement against the gain/loss on disposal. Monitor \acct{709} for completeness at each month end.

    \item \textbf{Tax effect of the gain:} The capital gain flows through the distribution components to unitholders. The tax agent must provide the AMMA Statement components for the disposal year to enable unitholders to calculate their individual tax positions.
\end{enumerate}

\begin{note}[Held-for-Sale Monthly Checklist]
\textbf{Held-for-sale monthly checklist:} At each reporting date while the property is under contract, confirm: (1)~the sale contract is still in force and conditions precedent status has been updated; (2)~the property continues to be measured at fair value under AASB~140 (no AASB~5 reclassification); (3)~the notes disclosure includes the sale price, settlement date, and significant conditions; (4)~the deferred cost of sale (\acct{709}) is updated; (5)~the St~George maturity flag is active; (6)~cash flow to settlement has been reviewed; and (7)~the disposal fee and performance fee accruals have been assessed.
\end{note}

% ============================================================================
\subsection{Investment Property Revaluation --- Lower of Potential Sale Price and Carrying Value}
\label{sec:est-reval}
\legbox{AASB~140 \textit{Investment Property} \legalcite{AASB~140}{sec:ref-accounting}; AASB~101 \textit{Presentation of Financial Statements} \legalcite{AASB~101}{sec:ref-accounting}.}

Under AASB~140 \textit{Investment Property}, the property is carried at fair value with changes recognised in profit or loss through \acct{213} --- Gain/(Loss) on Revaluation. The following considerations apply at each reporting date:

\begin{enumerate}[nosep]
    \item \textbf{Independent valuation:} The most recent independent valuation is \$13,200,000 (31 December 2024). Post-valuation capex additions are added to arrive at the carrying value on the \textbf{Capex} tab (\$13,590,288 at 31 January 2026).

    \item \textbf{Potential sale price:} Where the fund is actively pursuing a sale and indicative offers or contract prices are available, compare the \textbf{potential sale price (net of selling costs)} to the current carrying value. If the net realisable sale price is \textbf{lower} than the carrying value, consider whether a write-down is required under AASB~140 to reflect fair value at the reporting date.

    \item \textbf{Assessment:} At each month-end, the Preparer should document in the \textbf{Capex} tab:
    \begin{itemize}[nosep]
        \item Current carrying value per books;
        \item Most recent independent valuation and date;
        \item Any indicative sale price or contract price (if available);
        \item Whether a write-down adjustment is required and, if so, the CPG-approved amount.
    \end{itemize}

    \item \textbf{CPG approval:} Any revaluation adjustment (upward or downward) must be supported by CPG instruction or an updated independent valuation. UFS does not make unilateral revaluation adjustments.
\end{enumerate}

\begin{critical}[Fair Value Assessment Heightened During Active Sale]
With an active sale campaign in progress and St George loan maturity approaching (15 June 2026), fair value assessment is heightened. If the sale proceeds (net of deferred cost of sale in \acct{709}) would result in a value below the current IP carrying amount, raise this with CPG immediately. This has direct implications for NAV per unit, ICR, and gearing calculations reported on the Dashboard.
\end{critical}


\subsection{Aging Bad Debt Review}
\label{sec:est-baddebt}

Trade receivables (\acct{662}) must be reviewed for recoverability at each reporting date. The following procedure applies:

\begin{enumerate}[nosep]
    \item \textbf{Monthly review:} Download the Yardi Aged Receivables report at each month-end. Identify all balances exceeding 60 days past due.

    \item \textbf{Escalation:} For balances exceeding 90 days, submit a query to CPG via the bank information request covering: tenant name, invoice reference, amount outstanding, and aging bracket. Request CPG's assessment of recoverability and any collection actions in progress.

    \item \textbf{Provisioning:} If CPG confirms a receivable is doubtful or irrecoverable, raise a provision for doubtful debts:
    \begin{itemize}[nosep]
        \item DR \acct{700.xx} --- Bad Debt Expense (P\&L) / CR \acct{664} --- Provision for Doubtful Debts (BS).
        \item When the debt is formally written off: DR \acct{664} / CR \acct{662}.
    \end{itemize}

    \item \textbf{Frequency:} Given the fund's low occupancy ($\sim$26.7\%) and concentration risk (Sony Music represents 49\% of rental income), aging review should be performed \textbf{monthly} rather than quarterly.

    \item \textbf{Security deposits:} Before provisioning, cross-reference against held bank guarantees (5Point Projects --- \$33,000; Popgun Labs --- \$61,600) which may partially or fully offset any shortfall.
\end{enumerate}


\subsection{Bank Guarantees --- Balance Sheet Recognition and Audit Risk}
\label{sec:est-bankguarantees}
\legbox{AASB~137 \legalcite{AASB~137}{sec:ref-insurance}; AASB~9 \legalcite{AASB~9}{sec:ref-accounting}.}

Bank guarantees are one of the most frequently missed balance sheet items in property fund accounting. They do not flow through the P\&L, they are not always visible in the Yardi TB, and they may not appear on any bank statement until they are called upon. Despite their off-balance-sheet nature under accounting standards, they create \textbf{material contingent liabilities and assets} that require disclosure and, in some cases, full recognition on the balance sheet. Failure to identify and correctly treat bank guarantees can distort NAV, mislead investors about the fund's financial position, and create audit findings.

\subsubsection{What Bank Guarantees Are in a Property Fund Context}

A bank guarantee in the 22 Constance Street context means a financial institution (typically NAB or another authorised deposit-taking institution) has provided an unconditional undertaking to pay a specified sum to a beneficiary (most commonly the fund, as landlord) on demand, without recourse to the guarantor (the tenant). Bank guarantees in property funds commonly arise as:

\begin{itemize}[nosep]
    \item \textbf{Security deposits from tenants:} Instead of (or in addition to) a cash bond, a tenant provides a bank guarantee as security for their lease obligations (rent payment, make-good, damage). The fund is the beneficiary; the tenant is the obligor; the issuing bank is the guarantor. Currently held: 5Point Projects (\$33,000 bank guarantee); Popgun Labs (\$61,600 bank guarantee). Combined: \$94,600.
    \item \textbf{Performance guarantees:} Provided by a contractor to guarantee completion of works. Relevant for FHS (cyclone remediation contractor) if such a guarantee was required under the construction contract.
    \item \textbf{Vendor guarantees to the purchaser:} The fund may be required to provide a bank guarantee to the purchaser as security for vendor obligations under the sale contract (e.g.\ make-good, defect rectification, warranty claims). This creates a \textbf{contingent liability} for the fund.
    \item \textbf{Lender-required guarantees:} St George may hold a bank guarantee as additional security under the facility agreement, beyond the registered mortgage.
\end{itemize}

\subsubsection{Accounting Treatment --- When to Recognise on the Balance Sheet}

\paragraph{Bank guarantees received (fund as beneficiary --- tenant security)}

Under AASB~137 \textit{Provisions, Contingent Liabilities and Contingent Assets}, a bank guarantee received by the fund as beneficiary is a \textbf{contingent asset} unless the fund has made a demand and the bank has accepted the obligation to pay, at which point it becomes a receivable. Standard treatment:

\begin{itemize}[nosep]
    \item \textbf{While the lease is current and no breach exists:} The bank guarantee is a contingent asset --- it is \textbf{not recognised on the balance sheet} but must be \textbf{disclosed in the notes} to the financial statements (amount, nature, beneficiary details, expiry date).
    \item \textbf{If a demand has been made (e.g.\ tenant in breach and fund has called the guarantee):} Recognise a receivable on the balance sheet: DR \acct{662} --- Trade Receivables (or a separate Guarantee Receivable account) / CR P\&L (Other Income or Gain on Guarantee Call). The receipt of the guarantee proceeds reduces the receivable.
    \item \textbf{If the guarantee expires without being called:} No P\&L entry. The contingent asset simply lapses. Update the notes disclosure to remove the expired guarantee.
\end{itemize}

\paragraph{Bank guarantees given (fund as obligor --- vendor obligations under sale contract)}

If the fund is required to provide a bank guarantee to the purchaser as a vendor warranty obligation:

\begin{itemize}[nosep]
    \item The bank guarantee represents a \textbf{contingent liability} for the fund.
    \item If it is possible but not probable that the guarantee will be called: disclose as a contingent liability in the notes (AASB~137.86).
    \item If it is probable that the guarantee will be called and the amount can be reliably estimated: \textbf{recognise a provision} on the balance sheet (DR P\&L Provision Expense / CR \acct{882.16} or a separate Provision account) per AASB~137.14.
    \item The fund may also be required to provide cash collateral to its bank in exchange for the guarantee, which creates a \textbf{restricted cash asset} on the balance sheet (DR Restricted Cash / CR \acct{685} NAB --- not available for general use).
\end{itemize}

\paragraph{GST on guarantee proceeds}

When a bank guarantee is called and proceeds are received from the bank, the GST treatment is as follows: the proceeds are generally \textbf{not subject to GST} (financial supply) if the bank pays under the guarantee without any supply being made. However, if the guarantee proceeds are in substitution for a taxable supply (e.g.\ rent), the GST treatment needs analysis with the tax agent.

\subsubsection{How to Spot Missing Bank Guarantees --- Detection Checklist}

Bank guarantees are uniquely difficult to track because they do not appear in the Yardi GL, do not create journal entries when received, and do not appear on bank statements until called. The following controls must be applied:

\begin{enumerate}[nosep]
    \item \textbf{Lease register review:} For every lease, check the lease document for the security deposit clause. Confirm whether the tenant has provided (a)~a cash bond, (b)~a bank guarantee, or (c)~both. Currently: 5Point Projects = \$33,000 bank guarantee; Popgun Labs = \$61,600 bank guarantee; Sony Music = confirm with CPG (month-to-month --- original security may have lapsed or been returned).
    \item \textbf{Physical guarantee document:} Obtain the original bank guarantee documents and confirm: issuing bank, guarantee number, beneficiary name (must be the fund or trustee), guaranteed amount, expiry date, and calling conditions (usually `unconditional and on demand').
    \item \textbf{Expiry date monitoring:} Bank guarantees have expiry dates. If a guarantee expires before the lease ends (or before make-good is completed), the security is lost. Add expiry dates to the lease register with a 60-day advance alert.
    \item \textbf{Yardi cross-check:} Search Yardi for any balance in receivable or deposit accounts that may represent a guarantee. Confirm whether CPG has recorded the guarantee anywhere in the system (some property managers track guarantees as off-ledger items in Yardi notes).
    \item \textbf{Sale contract review:} The sale contract for 22 Constance Street must be reviewed for provisions regarding assignment or return of tenant bank guarantees at settlement. Standard practice: tenant guarantees transfer to the purchaser at settlement (vendor no longer holds them). Confirm the process for novation of each guarantee with K\&L Gates.
    \item \textbf{Construction contract review:} Review the FHS remediation contract for any performance bank guarantee provided by FHS. If one exists, its status must be tracked through practical completion.
    \item \textbf{St George facility agreement review:} Confirm with CPG whether any additional bank guarantee has been provided to St George as supplementary security under the facility. If so, the cash collateral is a restricted asset.
    \item \textbf{Vendor guarantee obligations under sale contract:} Review the sale contract for any clause requiring the fund to provide a bank guarantee to the purchaser (e.g.\ warranty guarantee, make-good performance guarantee). If present, account for as contingent liability or provision as above.
\end{enumerate}

\begin{table}[H]
\centering\small\sffamily
\caption{Bank guarantee tracking register --- 22 Constance Street (as at January 2026)}
\label{tab:bg-register}
\begin{tabularx}{\textwidth}{@{}L{2.2cm} L{1.6cm} L{1.6cm} L{2.2cm} L{1.6cm} X@{}}
\rowcolor{navy}\textcolor{white}{\bfseries Guarantee} & \textcolor{white}{\bfseries Amount} & \textcolor{white}{\bfseries Issuing Bank} & \textcolor{white}{\bfseries Beneficiary} & \textcolor{white}{\bfseries Expiry} & \textcolor{white}{\bfseries Status / Action} \\
\midrule
5Point Projects --- lease security & \$33,000 & TBC & Fund / Trustee & TBC & Confirm expiry vs.\ lease end (Oct 2027). Transfer to purchaser at settlement. \\
\rowcolor{warmgray}
Popgun Labs --- lease security & \$61,600 & TBC & Fund / Trustee & TBC & Confirm expiry vs.\ lease end (Dec 2026). Transfer to purchaser at settlement. \\
Sony Music --- lease security & TBC & TBC & Fund / Trustee & TBC & Confirm with CPG --- month-to-month tenant; guarantee may have lapsed. \\
\rowcolor{warmgray}
FHS --- construction performance & TBC & TBC & Fund / Trustee & TBC & Confirm whether performance guarantee was required under FHS contract. \\
Vendor guarantee to purchaser & TBC & TBC & Purchaser & TBC & Review sale contract for requirement. Assess as contingent liability or provision. \\
\rowcolor{warmgray}
St George --- facility security & TBC & TBC & St George & TBC & Confirm from facility agreement. Track cash collateral as restricted asset if applicable. \\
\bottomrule
\end{tabularx}
\end{table}

\begin{critical}[Settlement Risk — Bank Guarantee Assignment]
\textbf{Settlement risk --- bank guarantee assignment:} At settlement, all tenant bank guarantees held by the fund must be formally assigned or novated to the purchaser as part of the settlement process. If the guarantee is not assigned, the fund retains the contingent liability for make-good and the purchaser has no security. Failure to assign guarantees is a common source of post-settlement disputes. Confirm assignment mechanics with K\&L Gates before settlement.
\end{critical}

\subsubsection{Balance Sheet Sign-Off Checklist --- Bank Guarantees}

\begin{table}[H]
\centering\small\sffamily
\caption{Monthly bank guarantee sign-off items}
\begin{tabularx}{\textwidth}{@{}C{0.6cm} X L{2.5cm} L{2.2cm}@{}}
\rowcolor{navy}\textcolor{white}{\bfseries\#} & \textcolor{white}{\bfseries Check} & \textcolor{white}{\bfseries Owner} & \textcolor{white}{\bfseries Frequency} \\
\midrule
1 & Lease register reviewed; bank guarantee details confirmed for each tenant (amount, bank, expiry, calling conditions) & Preparer & Monthly \\
\rowcolor{warmgray}
2 & Expiry dates checked; guarantees expiring within 60 days flagged to CPG for renewal request & Preparer & Monthly \\
3 & Notes to financial statements include all active bank guarantees held (as contingent assets) & Preparer / Reviewer & Monthly \\
\rowcolor{warmgray}
4 & Any guarantee called or demand made during the period: receivable recognised on balance sheet & Preparer & As occurs \\
5 & Sale contract reviewed for vendor guarantee obligations to purchaser; contingent liability assessed & Reviewer / Tax Advisor & At exchange \\
\rowcolor{warmgray}
6 & At settlement: all tenant guarantees formally assigned to purchaser; \acct{884} and guarantee records updated & Preparer / Legal & At settlement \\
7 & St George facility agreement reviewed for guarantee collateral; restricted cash confirmed if applicable & Preparer & Annually / at refinance \\
\bottomrule
\end{tabularx}
\end{table}

% ============================================================================
\subsection{Revenue Received in Advance and Supplier Credits}
\label{sec:est-ria}

Review the following accounts for correct classification between assets and liabilities:

\begin{enumerate}[nosep]
    \item \textbf{Deferred income (\acct{703}):} Where tenants pay rent in advance (i.e.\ rent received before the period to which it relates), the amount must be classified as a \textbf{current liability} --- Revenue Received in Advance (or Deferred Income). Verify that the Yardi-imported \acct{703} balance correctly represents the unearned portion at the reporting date.

    \item \textbf{Unallocated income (\acct{663}):} If receipts are received into the NAB account and have not yet been matched to a specific tenant or invoice, they should remain in \acct{663} as an unallocated credit. Once identified, reclassify to the appropriate income or liability account.

    \item \textbf{Supplier credits or overpayments:} If UFS identifies a credit note or overpayment to a supplier that has created a debit balance in a payable account (e.g.\ \acct{883} Creditors showing a net debit), reclassify the debit portion to a \textbf{current asset} --- Supplier Prepayment or Receivable, rather than netting it against the payables balance. This ensures accurate presentation of assets and liabilities.

    \item \textbf{Sale deposit (\acct{884}):} The \$1,325,000 deposit held in St George is matched by \acct{884} --- Other Creditors liability. Upon settlement or forfeiture, this will be de-recognised. Confirm the deposit status with CPG at each month-end.
\end{enumerate}


\subsection{Loan Classification --- Long-Term vs.\ Short-Term Debt}
\label{sec:est-loan-reclass}
\legbox{AASB~101 (current/non-current classification) \legalcite{AASB~101}{sec:ref-accounting}; AASB~9 (refinancing test) \legalcite{AASB~9}{sec:ref-accounting}.}

At each financial statement date, the classification of the St George loan between current and non-current liabilities must be reassessed under AASB~101 \textit{Presentation of Financial Statements}:

\begin{enumerate}[nosep]
    \item \textbf{General rule:} A borrowing is classified as \textbf{current} if it is due to be settled within 12 months after the reporting date, or if the entity does not have an unconditional right to defer settlement for at least 12 months.

    \item \textbf{Current status:} The St George facility (\$5,000,000) matures on \textbf{15 June 2026}. At any reporting date where the maturity date falls within 12 months, the loan must be reclassified from non-current (\acct{932} --- Long-Term Debt) to current liabilities on the Balance Sheet, unless a written refinancing agreement has been executed before the reporting date that extends the maturity beyond 12 months.

    \item \textbf{Reclassification entry (if required):}
    \begin{itemize}[nosep]
        \item DR \acct{932} --- Long-Term Borrowings / CR \acct{932C} --- Current Portion of Borrowings (or equivalent current liability account per the Chart of Accounts).
    \end{itemize}

    \item \textbf{Impact:} Reclassification to current affects the fund's liquidity ratios, ICR presentation, and gearing calculation on the Dashboard. Ensure the Dashboard reflects the correct classification.

    \item \textbf{Refinancing evidence:} If CPG provides a signed refinancing or facility extension letter, retain a copy in the UFS Shared Folder and document the revised maturity date in the \textbf{Loan\_St George} tab. The loan may remain classified as non-current only if the extension agreement is \textbf{unconditional and executed} before the reporting date.
\end{enumerate}

\begin{warning}[St George Loan — Reclassify if Within 12 Months of Maturity]
As at 31 January 2026, the St George loan maturity (15 June 2026) is within 5 months. Unless a formal refinancing agreement is in place, the loan should be presented as a \textbf{current liability} in the Balance Sheet. Confirm classification with CPG at each reporting date.
\end{warning}


\subsection{Tax Payable vs.\ Tax Receivable Reclassification}
\label{sec:est-tax-reclass}

At each financial statement date, review the sign of the tax balance to ensure correct classification:

\begin{enumerate}[nosep]
    \item \textbf{Tax payable (credit balance):} If the trust has a net tax obligation, present as a \textbf{current liability} --- Tax Payable.

    \item \textbf{Tax receivable (debit balance):} If the trust has a net tax refund position (e.g.\ due to PAYG instalments exceeding the assessed liability, or franking credit offsets), reclassify from Tax Payable to \textbf{current asset} --- Tax Receivable on the Balance Sheet.

    \item \textbf{GST:} Similarly, review \acct{255} (GST) at quarter-end. If the net GST position is a receivable (input credits exceed output tax), ensure it is presented as a current asset, not as a negative liability.

    \item \textbf{Timing:} This review is particularly important at 30 June (year-end) and at each BAS quarter-end (30 September, 31 December, 31 March) when tax instalments and GST refunds may change the sign of the balance.
\end{enumerate}


\subsection{Tax Loss Carryforward --- Rules, Conditions, and Calculations}
\label{sec:est-taxlosses}
\legbox{s8-1 ITAA~1997 (general deductions) \legalcite{s8-1}{sec:ref-income-tax}; Schedule~2F ITAA~1936 (trust loss testing --- income injection test); Schedule~2G ITAA~1936 (trust loss testing --- pattern of distributions test); s102-15 ITAA~1997 (capital losses offset only against capital gains); Division~265 ITAA~1997 (tax losses generally).}

Where the trust's allowable deductions for an income year exceed its assessable income, a tax loss arises. The treatment of that loss differs fundamentally from a corporate structure: a trust does not carry losses as a balance sheet item in the conventional sense. Rather, the accumulated loss pool ($TL$) is preserved as a memorandum figure and applied against future ordinary income, subject to the trust satisfying the loss testing rules in Schedules~2F and~2G of the ITAA~1936 in each year of utilisation.

The carried-forward loss pool has no legislative time limit --- losses accumulate indefinitely until fully absorbed. However, the right to apply them in any given year is conditional on satisfying both loss testing tests, as described below. The tax loss position feeds directly into the distribution component determination at Section~\ref{sec:est-distribution-components}: after ordinary deductions are applied and any remaining loss pool is absorbed, the resulting Taxable Ordinary Income ($\TOI$) forms the starting point for decomposing the trust's distribution into its income, capital gain, tax-deferred, and return-of-capital components.

A critical structural distinction must be observed: under s102-15 ITAA~1997, ordinary tax losses and capital losses are kept entirely separate. \textbf{Ordinary tax losses} ($TL$) can only be applied against future ordinary income --- they cannot shelter capital gains. \textbf{Capital losses} ($CL$) can only be applied against future capital gains --- they have no effect on ordinary income. The distribution component framework at Section~\ref{sec:est-cgt} addresses the capital loss pool in the context of the CGT calculation.

\begin{note}[Current Tax Loss Position — Nil]
As at the reporting date, the fund's tax loss position ($TL$) is nil --- there are no prior-year ordinary tax losses per the lodged tax returns. Given the fund's active sale process and approaching wind-up, any losses arising in FY2025 or FY2026 must be tracked carefully. Losses that remain unabsorbed at the time of the final tax return are permanently forfeited. Coordinate with the tax agent at each year-end to confirm the opening balance, any new loss $\mathcal{L}$ generated during the year, and the closing pool $\TLrem$ carried forward.
\end{note}

\subsubsection{Conditions for Utilisation --- Trust Loss Testing Rules}

Before ordinary carried-forward losses can be applied against current-year ordinary income, the trust must satisfy \textbf{both} of the following tests under Schedules~2F and~2G of the ITAA~1936:

\begin{enumerate}[nosep]
    \item \textbf{Income Injection Test (Schedule~2F):} The trust must not have received, directly or indirectly, an ``income injection'' --- a transaction or arrangement designed to artificially create or inflate the trust's income so that carried-forward losses can be absorbed. The test examines all transactions during the income year in which the loss is applied, identifying arrangements with no genuine commercial purpose beyond enabling loss utilisation. For 22~Constance Street, this test is satisfied where all income flows (rent, insurance proceeds, interest) arise from arm's-length commercial arrangements and no related-party scheme has been employed to introduce artificial income into the trust.

    \item \textbf{Pattern of Distributions Test (Schedule~2G):} The trust must not have varied its pattern of income distributions to beneficiaries in a manner that suggests the purpose of the variation was to facilitate loss utilisation. The test compares the proportional distribution entitlements of beneficiaries across years. For a fixed unit trust --- where entitlements are set by the Constitution in proportion to units held and cannot be varied at the trustee's discretion --- this test is ordinarily satisfied as a matter of the fund's legal structure.
\end{enumerate}

\begin{warning}[Fixed Unit Trust — Loss Testing Requirements]
As a \textbf{fixed unit trust} with mandatory proportionate distributions, 22~Constance Street will generally satisfy both loss testing requirements without specific tax planning. Nonetheless, the tax agent must formally confirm the loss testing outcome for each income year in which prior-year losses are proposed to be utilised. Carried-forward losses must not be applied without that confirmation.
\end{warning}

\subsubsection{Calculation Framework and Reference Formulae}

The loss absorption mechanics are integrated into the distribution component calculation at Section~\ref{sec:est-distribution-components}. The reference formulae below summarise how the loss pool is updated and applied each income year, and how the resulting Taxable Ordinary Income is derived. The worked numerical example that follows demonstrates the mechanics across a three-year cycle.

\begin{note}[Notation Key for Distribution Formulae]
\textbf{Notation.} Throughout the formulae below, $\OI$ denotes total ordinary assessable income (rent, interest, sundry receipts) and $\OD$ denotes total allowable ordinary deductions (interest, Div~40/43, repairs, management fees, s25-25, s40-880, land tax). The symbol $[\,x\,]^{+}$ denotes the positive part of $x$, defined as $\max(0,\, x)$; it evaluates to $x$ when $x > 0$ and to zero otherwise. Subscripts are used to distinguish successive states of the carried-forward loss pool ($TL$) within the same income year.
\end{note}

\begin{table}[H]
\centering\small\sffamily
\caption{Tax loss carryforward --- calculation reference table}
\label{tab:loss-cf}
\begin{tabularx}{\textwidth}{@{}L{5.8cm} X@{}}
\rowcolor{navy}\textcolor{white}{\bfseries Formula} & \textcolor{white}{\bfseries Meaning} \\
\midrule
$\NOI \;=\; \OI - \OD$ & Current-year net ordinary income; may be negative if deductions exceed income \\[2pt]
\rowcolor{warmgray}
$\mathcal{L} \;=\; [\,\OD - \OI\,]^{+}$ & New loss generated this year; zero unless $\OD > \OI$ \\[2pt]
$\NOI^{+} \;=\; [\,\OI - \OD\,]^{+}$ & Net positive ordinary income available to absorb losses; floored at zero \\[2pt]
\rowcolor{warmgray}
$\TLnew \;=\; \TLprior + \mathcal{L}$ & Accumulated loss pool entering the absorption step \\[2pt]
$\TLapplied \;=\; \min\!\bigl(\TLnew,\; \NOI^{+}\bigr)$ & Losses absorbed this year: the lesser of the pool and available income \\[2pt]
\rowcolor{warmgray}
$\TOI \;=\; \NOI^{+} - \TLapplied \;\geq\; 0$ & Taxable Ordinary Income after loss absorption; structurally non-negative \\[2pt]
$\TLrem \;=\; \TLnew - \TLapplied$ & Closing loss pool carried forward to the following income year \\[2pt]
\rowcolor{warmgray}
$\displaystyle TL_{(n)} \;=\; \sum_{k=1}^{n} \mathcal{L}_k \;-\; \sum_{k=1}^{n} TL_{\text{applied},k}$ & Cumulative loss pool after $n$ income years \\
\bottomrule
\end{tabularx}
\end{table}

\subsubsection{Worked Example --- Loss Carryforward Over Three Years}

\begin{table}[H]
\centering\small\sffamily
\caption{Loss carryforward worked example (three-year cycle)}
\begin{tabularx}{\textwidth}{@{}L{6cm} R{2.3cm} R{2.3cm} R{2.3cm}@{}}
\rowcolor{navy}\textcolor{white}{\bfseries Item} & \textcolor{white}{\bfseries Year~1} & \textcolor{white}{\bfseries Year~2} & \textcolor{white}{\bfseries Year~3} \\
\midrule
$\OI$ \quad Gross ordinary income & \$350,000 & \$380,000 & \$420,000 \\
\rowcolor{warmgray}
$\OD$ \quad Ordinary deductions & \$400,000 & \$310,000 & \$340,000 \\
$\NOI = \OI - \OD$ & $-\$50{,}000$ & \$70,000 & \$80,000 \\
\rowcolor{warmgray}
$\mathcal{L} = [\,\OD - \OI\,]^{+}$ \quad New loss & \$50,000 & \$0 & \$0 \\
$\NOI^{+} = [\,\OI - \OD\,]^{+}$ & \$0 & \$70,000 & \$80,000 \\
\rowcolor{warmgray}
$\TLprior$ \quad Opening loss pool & \$0 & \$50,000 & \$10,000 \\
$\TLnew = \TLprior + \mathcal{L}$ & \$50,000 & \$50,000 & \$10,000 \\
\rowcolor{warmgray}
$\TLapplied = \min\!\bigl(\TLnew,\, \NOI^{+}\bigr)$ & \$0 & \$50,000 & \$10,000 \\
$\TOI = \NOI^{+} - \TLapplied$ & \$0 & \$20,000 & \$70,000 \\
\rowcolor{warmgray}
$\TLrem$ \quad Closing loss pool & \$50,000 & \$0 & \$0 \\
Cumulative losses generated $\bigl(\textstyle\sum \mathcal{L}_k\bigr)$ & \$50,000 & \$50,000 & \$50,000 \\
\rowcolor{warmgray}
Cumulative losses absorbed $\bigl(\textstyle\sum TL_{\text{applied},k}\bigr)$ & \$0 & \$50,000 & \$60,000 \\
\bottomrule
\end{tabularx}
\end{table}

\begin{note}[Year 1 Tax Loss — Illustrated Example]
In Year~1, deductions exceed income by \$50,000 so a tax loss of $\mathcal{L} = \$50{,}000$ arises and $\NOI^{+} = 0$: no losses can be absorbed, and the pool accumulates to $\TLrem = \$50{,}000$. In Year~2, income exceeds deductions by \$70,000, so $\NOI^{+} = \$70{,}000$. The full prior-year pool is absorbed ($\TLapplied = \min(50k,\,70k) = \$50{,}000$), leaving $\TOI = \$20{,}000$. In Year~3 the opening pool is nil and the full $\NOI^{+} = \$80{,}000$ is taxable.
\end{note}

\subsubsection{Interaction with the Tax Reconciliation Workpaper}

The loss carryforward position must be reconciled in the \textbf{Tax reconciliation} tab each period:
\begin{enumerate}[nosep]
    \item Record the opening $\TLprior$ from the prior-year tax return (confirmed by the tax agent).
    \item Calculate the current-year $\NOI^{+}$, new loss $\mathcal{L}$ (if any), and $\TLapplied$ using the formulae above.
    \item Record the closing $\TLrem$ as the closing balance to be confirmed in the final tax return.
    \item \textbf{Cross-reference to distribution components:} $\TOI$ (Taxable Ordinary Income after loss absorption) feeds directly into Step~2 of the distribution component framework (Section~\ref{sec:est-distribution-components}). Confirm $\TOI$ is correctly carried from the Tax Reconciliation tab into the distribution waterfall calculation.
\end{enumerate}

\begin{critical}[22 Constance Street — Tax Loss Position Currently Nil]
\textbf{22 Constance Street specific note:} As at the reporting date, the tax loss position ($TL$) is nil --- no prior-year ordinary losses per the lodged tax returns. If the fund incurs a tax loss in FY2025 or FY2026 (e.g.\ from Div~40/43 deductions exceeding net rental income), the resulting $\mathcal{L}$ must be added to the pool. Given the fund's imminent sale and wind-up timeline, any $\TLrem$ not absorbed before the fund's final tax return is \textbf{permanently forfeited}. Coordinate with the tax agent to ensure the loss pool is maximised before wind-up.
\end{critical}




% ============================================================================
% PART II: INCOME STATEMENT --- RECURRING AMORTISATION AND FEE MONITORING
% ============================================================================

\vspace{1.2cm}
\begin{tcolorbox}[
    enhanced, breakable,
    colback=navy, colframe=navy,
    boxrule=0pt,
    borderline south={2pt}{0pt}{gold},
    sharp corners,
    left=14pt, right=10pt, top=10pt, bottom=10pt,
    label={sec:est-part-pl}
]
{\sffamily\bfseries\footnotesize\textcolor{gold}{Part II\enspace\textbullet\enspace Appendix~C}}\par
\vspace{3pt}
{\sffamily\bfseries\large\textcolor{white}{Income Statement --- Recurring Amortisation and Fee Monitoring}}
\end{tcolorbox}
\begin{tcolorbox}[
    enhanced, breakable,
    colback=palesky, colframe=navy!30!white,
    boxrule=0.4pt, sharp corners,
    left=10pt, right=10pt, top=6pt, bottom=6pt
]
\small\itshape This part covers recurring income statement items requiring monthly attention: amortisation schedules, statutory expense timing, borrowing cost alignment, fee base metric monitoring, and CPI-linked adjustments.
\end{tcolorbox}
\vspace{0.3cm}

\subsection{Leasing Fee Amortisation --- Lease Status Review}
\label{sec:est-leasing}

Leasing fees capitalised to \acct{780.12} are amortised over the term of the underlying lease (see Section~\ref{sec:yardi-revise} for the capitalisation and amortisation journal entries). The amortisation schedule \textbf{must be reviewed whenever the status of a lease changes}. Specific attention items:

\begin{enumerate}[nosep]
    \item \textbf{New lease executed:} When a new tenant lease is signed, any associated leasing commission or incentive invoice must be capitalised to \acct{780.12} and a new amortisation schedule created in the \textbf{Prepaid leasing fee} tab. Confirm the lease term against the signed lease agreement (Section~\ref{sec:leases}) and set the amortisation period accordingly.

    \item \textbf{Existing lease renewed or extended:} If a tenant renews or extends its lease term, review whether the unamortised leasing fee balance should be re-spread over the revised (extended) lease term. If additional leasing fees are incurred on renewal, capitalise the incremental amount and amortise over the new term. Document the revised schedule in the \textbf{Prepaid leasing fee} tab.

    \item \textbf{Lease terminated or tenant vacates:} If a lease is terminated early, surrendered, or the tenant vacates (including transition to holdover then departure), the \textbf{remaining unamortised balance} of the leasing fee must be \textbf{expensed immediately} to \acct{700.26} --- Leasing Fee. Simultaneously reverse the remaining fair value adjustment: DR \acct{213} / CR \acct{780.30}. Do not continue amortising a leasing fee for a lease that no longer exists.

    \item \textbf{Month-to-month holdover:} Where a tenant is on holdover (e.g.\ Sony Music, Section~\ref{sec:leases}), assess whether the original leasing fee has been fully amortised. If the holdover period extends beyond the original amortisation schedule, no further amortisation is required. If the tenant departs during holdover, apply the termination treatment above.
\end{enumerate}

\begin{warning}[Leasing Fee Amortisation — Review on Lease Status Change]
The lease schedule (Section~\ref{sec:leases}) and the \textbf{Prepaid leasing fee} tab must be reviewed together each month. Any change in lease status (new, renewed, expired, terminated) has a direct impact on the leasing fee amortisation and the investment property carrying value.
\end{warning}


\subsection{Statutory Operating Expense Timing --- Land Tax, Council Rates, and Fine Avoidance}
\label{sec:est-opex-timing}
\legbox{s8-1 ITAA~1997 \legalcite{s8-1}{sec:ref-income-tax}; \textit{Nilsen Development Laboratories} \legalciteonly{see \S\,\ref{sec:ref-income-tax}}; \textit{Land Tax Act 2010} (Qld) \legalciteonly{see \S\,\ref{sec:ref-statutory}}; \textit{City of Brisbane Act 2010} (Qld) \legalciteonly{see \S\,\ref{sec:ref-statutory}}. All period-end operating expense accruals must satisfy TR~97/7 \legalcite{TR~97/7}{sec:ref-income-tax} before recognition.}

Certain property operating expenses are assessed by government authorities on fixed schedules and carry penalties (interest or fines) for late payment. While UFS follows Yardi's treatment for these operating expenses and does \textbf{not} make independent adjustments to the amounts recorded by CPG in Yardi, the Preparer and Reviewers must \textbf{monitor} the following to avoid regulatory fines:

\begin{enumerate}[nosep]
    \item \textbf{Land tax (\acct{700.xx} --- Office of State Revenue):} Assessed annually by the Queensland Office of State Revenue. Verify that: (a) the prepayment balance in \acct{706} is being amortised over the correct assessment period; (b) no duplicate recording exists between the Yardi-imported prepayment and any UFS manual accrual in \acct{882.02}; (c) the payment due date has not passed without settlement --- late payment attracts interest at the statutory rate.

    \item \textbf{Council rates:} Assessed quarterly or half-yearly by Brisbane City Council. Apply the same checks as land tax: verify amortisation period, check for duplicates between prepayments and accruals, and confirm timely settlement.

    \item \textbf{Water rates (Queensland Urban Utilities):} Typically quarterly. Monitor for the same timing issues.

    \item \textbf{Insurance premiums (\acct{707}):} Annual renewal, prepaid and amortised monthly. Confirm renewal has been processed and the correct premium amount is recorded. Late renewal may result in coverage gaps.
\end{enumerate}

\begin{note}[UFS Role — Alert CPG on Operating Expense Treatment]
\textbf{Key principle:} UFS does not adjust Yardi's operating expense treatment. However, UFS is responsible for \textbf{alerting CPG} if the Preparer observes: (a) aging of an expense that could generate late payment interest or fines; (b) duplicate recording between a Yardi-imported prepayment/accrual and a UFS manual entry; or (c) a material discrepancy between the amortised expense and the authority's assessment notice. Document any concerns in the bank information request or a separate email to CPG.
\end{note}


\subsection{Borrowing Cost Amortisation --- Facility Term Alignment}
\label{sec:est-borrow-amort}

Prepaid borrowing costs (\acct{708.03} --- Establishment Fee; \acct{708.04} --- Legal Fee) are amortised on a straight-line basis over the \textbf{remaining facility term} (see Section~\ref{sec:borrowing-amort}). Attention items:

\begin{enumerate}[nosep]
    \item \textbf{Facility extension or refinancing:} If the St George facility is extended or refinanced, the amortisation period must be re-calculated over the new remaining term. Any unamortised balance is spread over the revised period --- it is \textbf{not} written off unless the facility is terminated entirely.

    \item \textbf{Facility termination (e.g.\ upon sale):} If the loan facility is repaid and terminated upon sale completion, any remaining unamortised borrowing cost must be \textbf{expensed immediately} to \acct{303.03}/\acct{303.04} in the period of termination.

    \item \textbf{Monthly verification:} At each month-end, verify that the amortisation charge in the \textbf{Borrowing cost prepaid} tab is consistent with the remaining months to maturity. As at 31 January 2026, there are approximately 4.5 months remaining to the 15 June 2026 maturity.
\end{enumerate}

\legbox{s25-25 ITAA~1997 (borrowing cost amortisation) \legalcite{s25-25}{sec:ref-borrowing}; s25-25(5) ITAA~1997 (immediate write-off on facility termination) \legalcite{s25-25(5)}{sec:ref-borrowing}; AASB~9 (modification vs.\ extinguishment --- 10\% test) \legalcite{AASB~9}{sec:ref-accounting}.}

% ============================================================================

\subsection{Fee Size Indicators --- Revenue, Profit, and Fund Metrics}
\label{sec:est-fee-size}

Several fund fees are calculated as a \textbf{percentage of a fluctuating base metric}. The Preparer must ensure that fee calculations are updated when the underlying metric changes. Failure to do so results in incorrect fee accruals that compound over time.

\begin{table}[H]
\centering\small\sffamily
\caption{Fee-to-metric dependency map --- verify base metric each period}
\label{tab:fee-metrics}
\begin{tabularx}{\textwidth}{@{}L{2.8cm} L{2.5cm} X@{}}
\rowcolor{navy}\textcolor{white}{\bfseries Fee} & \textcolor{white}{\bfseries Base Metric} & \textcolor{white}{\bfseries Fluctuation Risk and Attention} \\
\midrule
Management Fee (\acct{301.01}) & 5\% of net receivable income & Net receivable income = gross rental less abatements (\acct{71.02}). Changes with lease events (new tenant, vacancy, rent review, rent-free expiry). Recalculate each month. \\
\rowcolor{warmgray}
Disposal Fee (\acct{301.04}) & 1\% of GAV & GAV = total assets. Changes with IP revaluation, capex additions, and cash balance movements. Update GAV at point of accrual. \\
Performance Fee (\acct{303.05}) & 20\%/40\% of returns above IRR hurdles & IRR depends on: distributions paid, NAV at exit, total capital contributed. Recalculate with each new NAV. Highly sensitive to final sale price. \\
\rowcolor{warmgray}
Fund Admin \& Tax (\acct{301.02}) & Fixed \$1,250/month & Currently fixed. Verify at engagement renewal whether the fee becomes linked to fund size (e.g.\ NAV, number of properties, or transaction volume). \\
Audit Fee (\acct{302.05}) & Engagement letter / prior year & Audit fees may increase with fund complexity (cyclone claims, sale transaction, multiple valuations). Confirm current-year estimate with auditor. \\
\rowcolor{warmgray}
Valuation Fee (\acct{302.03}) & Per engagement & May increase if additional valuations are commissioned (e.g.\ pre-sale, post-cyclone). Accrue based on latest engagement fee, not prior year. \\
Legal Fee --- Trust (\acct{302.02}) & Per engagement & Increases significantly during sale process (due diligence, contract negotiation, settlement). Distinguish from sale legal fees in \acct{709}. \\
\rowcolor{warmgray}
Registry Fee (\acct{301.03}) & Per invoice or estimate & Typically stable but may change with unit transfers or redemptions. \\
\bottomrule
\end{tabularx}
\end{table}

\textbf{Key metrics to monitor each period and their impact on fees:}

\begin{enumerate}[nosep]
    \item \textbf{Revenue (gross rental income):} Drives management fee base. Note that \textit{gross} rental income (before abatements) differs from \textit{net receivable income} (after abatements). The management fee is 5\% of \textbf{net} receivable income. Total annual abatements are \$52,400 (Table~\ref{tab:concessions}). Any change in tenant status directly affects the management fee.

    \item \textbf{Revenue (net of abatement):} The actual management fee base. When rent-free periods expire (e.g.\ 5Point Projects concession ends 27 October 2026; Popgun Labs concession ends 15 December 2026), net receivable income \textbf{increases}, and the management fee increases correspondingly.

    \item \textbf{Profit / Net Ordinary Income:} Feeds the Operating Earnings (OE) calculation on the Dashboard. If profit fluctuates significantly (e.g.\ due to revaluation gains/losses, insurance income, or sale-related expenses), verify that the OE formula correctly adds back non-cash items (Section~\ref{sec:phase2}).

    \item \textbf{NAV and NTA per unit:} Directly affected by profit/loss, distributions, and revaluation adjustments. Changes in NAV affect the Performance Fee IRR calculation and the unit price reported to investors.

    \item \textbf{GAV (Gross Asset Value):} Total assets including IP, cash, receivables, and prepayments. The Disposal Fee is 1\% of GAV at the point of sale completion. GAV fluctuates with IP revaluation, capex additions, cash inflows (insurance, rent), and cash outflows (opex, loan repayments).

    \item \textbf{Investment property carrying value (\acct{780}):} Key input to GAV, gearing, and ICR. Changes with capex additions, fair value adjustments, and leasing fee capitalisation/amortisation.

    \item \textbf{Number of properties and occupancy:} While currently a single-property fund, occupancy ($\sim$26.7\%) affects revenue, management fees, and ICR. A tenant departure or new lease changes multiple fee calculations simultaneously.
\end{enumerate}

\begin{warning}[Significant Events — Trace Impact Across All Schedules]
When a \textbf{significant event} occurs (e.g.\ new lease signed, tenant vacates, IP revaluation, sale exchange), the Preparer should trace the impact through \textbf{all} fee calculations in the pack, not just the directly affected account. Use the fee-to-metric map (Table~\ref{tab:fee-metrics}) as a checklist.
\end{warning}

\legbox{\textit{Corporations Act 2001} s601FC (responsible entity duties --- arm's length pricing) \legalciteonly{see \S\,\ref{sec:ref-hfs}}; AASB~124 \textit{Related Party Disclosures} \legalcite{AASB~124}{sec:ref-accounting}; AASB~140 (management fee capitalisation criteria) \legalcite{AASB~140}{sec:ref-accounting}.}

% ============================================================================

\subsection{Accruals and Provisions --- Sales Settlement, Special Events, and CPI Adjustments}
\label{sec:est-accruals-settlement}

Beyond routine monthly accruals, the Preparer must consider whether \textbf{additional accruals or provisions} are required for non-routine events. These are often missed because they arise from irregular transactions rather than the regular Yardi import cycle.

\subsubsection{Sales Settlement Accruals and Provisions}

When the property sale reaches exchange or approaches settlement, the following accruals and provisions are commonly required but frequently overlooked:

\begin{enumerate}[nosep]
    \item \textbf{Vendor's adjustment statement items:} At settlement, adjustments are made for rates, taxes, and rent apportioned between vendor and purchaser. Prior to settlement, accrue estimated vendor adjustments:
    \begin{itemize}[nosep]
        \item Council rates (apportioned to settlement date);
        \item Water rates (apportioned to settlement date);
        \item Land tax (apportioned to settlement date --- note that land tax can be a material adjustment);
        \item Rent received in advance that must be apportioned to the purchaser;
        \item Outgoings recovery received in advance apportioned to the purchaser.
    \end{itemize}

    \item \textbf{Disposal fee accrual (\acct{301.04}):} 1\% of GAV, accrued upon CPG confirmation that the sale is probable. Do not accrue before CPG confirmation (see Section~\ref{sec:trust-accruals}).

    \item \textbf{Performance fee accrual (\acct{303.05}):} Tier~1 (20\% above 8\% IRR) and Tier~2 (40\% above 15\% IRR). Accrue only when the performance threshold is probable and the amount is reliably estimable. Obtain CPG's IRR calculation before accruing.

    \item \textbf{Deferred cost of sale de-recognition (\acct{709}):} At settlement, the full \acct{709} balance is de-recognised against the gain/loss on disposal. Prepare the disposal journal in advance: DR Cash (proceeds) / DR Accumulated Depreciation / DR \acct{709} release / CR Investment Property (\acct{780}) / CR or DR Gain/(Loss) on Disposal.

    \item \textbf{Loan repayment and break costs:} On settlement, the St George facility will be repaid. Accrue any break costs, early termination fees, or interest accrued to settlement date that have not yet been debited by St George. Expense remaining unamortised borrowing costs (see Section~\ref{sec:est-borrow-amort}).

    \item \textbf{Make-good provisions:} Review existing leases for make-good clauses. If the purchaser inherits the lease, no provision is required. If leases are terminated on sale, estimate and provide for make-good obligations.

    \item \textbf{Sale agent commission:} MP Realty Trust commission (currently in \acct{709} at \$82,812.50). Confirm whether the commission is subject to a success fee adjustment based on final sale price.

    \item \textbf{Legal completion costs:} K\&L Gates legal fees may increase at settlement (title transfer, mortgage discharge, settlement agent). Accrue an estimate if the final invoice has not been received.
\end{enumerate}

\begin{critical}[Vendor Adjustment Statement — Largest Source of Missed Accruals]
The vendor's adjustment statement is often the largest source of missed accruals at settlement. The Preparer should obtain a \textbf{draft settlement statement from K\&L Gates or CPG} as early as possible and reconcile each line item against the reporting pack accruals before the settlement month pack is finalised.
\end{critical}

\subsubsection{Special Events Requiring Accruals or Provisions}

\begin{enumerate}[nosep]
    \item \textbf{Cyclone or natural disaster (ongoing):} If further cyclone damage occurs or new remediation invoices are identified, accrue the estimated cost if an obligation exists at the reporting date. Cross-reference with insurance claim status --- only accrue the \textbf{net} exposure (cost less expected insurance recovery) unless the recovery is uncertain.

    \item \textbf{Tenant dispute or litigation:} If a tenant disputes an outgoings charge, rent increase, or make-good obligation, consider whether a provision for legal costs or potential settlement is required under AASB~137 \textit{Provisions, Contingent Liabilities and Contingent Assets}.

    \item \textbf{Regulatory assessments:} Unexpected government assessments (e.g.\ retrospective land tax reassessment, building compliance order, environmental remediation notice) should be accrued when the obligation becomes probable.

    \item \textbf{Post-cyclone insurance excess:} If the insurance policy carries an excess (deductible) that has not been settled, accrue the excess as a liability.
\end{enumerate}

\subsubsection{CPI-Linked Fee and Expense Adjustments}

Several fund expenses are subject to periodic CPI adjustments. The Preparer must verify whether CPI escalation has been applied at the relevant review date:

\begin{enumerate}[nosep]
    \item \textbf{Property management fee (Handler Holdings):} Review the management agreement for CPI escalation clauses. If an annual CPI increase applies, update the accrual base at the review date.

    \item \textbf{Tenant rent reviews:} Some leases contain CPI or market rent review clauses. When a review is triggered, update the rental income schedule in Yardi. Verify that the charge schedule in Table~\ref{tab:concessions} reflects the post-review rent.

    \item \textbf{UFS fund administration fee:} Currently fixed at \$1,250/month with no CPI increase applied over the engagement (per Section~\ref{sec:trust-accruals}). However, if the engagement letter is renewed, check whether CPI is applied to the new term.

    \item \textbf{Insurance premiums:} Renewal premiums typically increase with CPI or claims history. At renewal, update the prepayment amortisation schedule in \acct{707} to reflect the revised premium.

    \item \textbf{Council rates and land tax:} These are government-assessed and typically increase annually. UFS follows Yardi's treatment but should verify that the prepayment amortisation base matches the current assessment notice, not the prior year amount.
\end{enumerate}

\begin{note}[CPI Adjustments — Maintain Annual Review Calendar]
CPI adjustments are easily missed because they occur annually rather than monthly. The Preparer should maintain a \textbf{CPI review calendar} noting the review date for each CPI-linked expense and flag it in the month the adjustment is due.
\end{note}

% ============================================================================


% ============================================================================
% PART III: CAPITAL VS.  REVENUE CLASSIFICATION AND TAX DEDUCTION REGIMES
% ============================================================================

\vspace{1.2cm}
\begin{tcolorbox}[
    enhanced, breakable,
    colback=navy, colframe=navy,
    boxrule=0pt,
    borderline south={2pt}{0pt}{gold},
    sharp corners,
    left=14pt, right=10pt, top=10pt, bottom=10pt,
    label={sec:est-part-capex}
]
{\sffamily\bfseries\footnotesize\textcolor{gold}{Part III\enspace\textbullet\enspace Appendix~C}}\par
\vspace{3pt}
{\sffamily\bfseries\large\textcolor{white}{Capital vs.\ Revenue Classification and Tax Deduction Regimes}}
\end{tcolorbox}
\begin{tcolorbox}[
    enhanced, breakable,
    colback=palesky, colframe=navy!30!white,
    boxrule=0.4pt, sharp corners,
    left=10pt, right=10pt, top=6pt, bottom=6pt
]
\small\itshape This part provides the framework for classifying each item of expenditure as capital or revenue, the tax division (Div~40, Div~43, s25-10), component accounting under AASB~116/140, and the borrowing cost and blackhole expenditure deduction regimes.
\end{tcolorbox}
\vspace{0.3cm}

\subsection{Capex vs.\ Repair --- Detailed Assessment Criteria}
\label{sec:est-capex-guide}
\legbox{s25-10 ITAA~1997 \legalcite{s25-10}{sec:ref-repairs}; TR~97/23 \legalcite{TR~97/23}{sec:ref-repairs}; AASB~116.43 \legalcite{AASB~116}{sec:ref-accounting}; AASB~140 \legalcite{AASB~140}{sec:ref-accounting}. Case law: \textit{W Thomas \& Co} \legalciteonly{see \S\,\ref{sec:ref-repairs}}; \textit{Western Suburbs Cinemas} \legalciteonly{see \S\,\ref{sec:ref-repairs}}.}

Correct classification of each invoice as a repair (immediate deduction or P\&L expense) or capital expenditure (balance sheet asset with depreciation schedule) is critical for profit measurement, investment property carrying value, tax deductions, and NAV. The ATO's primary authority is \textit{Taxation Ruling TR~97/23 --- Income tax: repairs}, which draws a practical line between work that \textbf{restores efficiency of function} without altering the character of the asset (repair) and work that \textbf{improves or goes beyond restoration} (capital). All classification decisions must be supported by contemporaneous evidence and referenced to these criteria. For the statutory framework under AASB~116 and AASB~140, see Section~\ref{sec:est-capex-accounting}. For the tax division classification (Div~40, Div~43, s25-10), see Section~\ref{sec:est-capex-taxdivision}. For CGT implications of works around the sale, see Section~\ref{sec:est-cgt}.

\subsubsection{Accounting Classification (AASB~116 / AASB~140)}
\label{sec:est-capex-accounting}

Under AASB~116 \textit{Property, Plant and Equipment} (applied by analogy) and AASB~140 \textit{Investment Property}:

\begin{enumerate}[nosep]
    \item \textbf{Capitalise (add to \acct{780}/\acct{781})} if the expenditure:
    \begin{itemize}[nosep]
        \item Extends the useful life of the asset beyond its original assessment;
        \item Increases the asset's future economic benefits (higher rental yield, improved lettable area, enhanced building services);
        \item Replaces a major component that is separately depreciable under AASB~116.43 (component accounting);
        \item Constitutes a \textbf{betterment or enhancement} that goes beyond restoring the asset to its previous condition.
    \end{itemize}
    \item \textbf{Expense to P\&L (\acct{700.xx})} if the expenditure:
    \begin{itemize}[nosep]
        \item Restores the asset to its \textbf{original condition} without extending useful life or enhancing capacity;
        \item Is routine, periodic maintenance (cleaning, painting, minor plumbing, pest control);
        \item Is \textbf{day-to-day servicing} (AASB~116.12); or
        \item Does \textbf{not} result in future economic benefits in excess of those originally assessed.
    \end{itemize}
\end{enumerate}

\begin{warning}[Tax vs. Accounting Divergence on Repairs]
\textbf{Tax vs.\ accounting divergence:} An expenditure can be a `repair' for accounting purposes (restores original condition, no change in useful life) but `capital' for tax purposes (entirety replaced under TR~97/23). Both treatments must be assessed independently. In that scenario, the item is expensed to P\&L for accounting but is not deductible as s25-10 repair for tax --- it enters the Div~43 capital works schedule at 2.5\% p.a. Document both conclusions on the Capex Assessment Form.
\end{warning}

\subsubsection{Assessment Criteria A through E --- Applying TR~97/23}
\label{sec:est-capex-criteria}

The following five criteria must each be considered for every invoice above the materiality threshold (\$5,000 recommended). The Preparer must document conclusions on the \textbf{Capex vs.\ Repair Assessment Form}.

\paragraph{Criterion A --- Restore to Original / Previous Condition}

Work that remedies defects, damage, or deterioration \textit{returning the asset to its prior working condition} without improvement is repair-like under TR~97/23. The benchmark is the condition of the asset immediately before the damage or deterioration --- not the condition at acquisition.

\textit{Evidence required:} Before/after photos and PM defect reports; contractor scope of works (restoration language: `repair', `patch', `reinstate to pre-event condition' supports repair; `upgrade', `new system', `enhanced compliance' indicates capital); post-works performance compared to pre-damage performance (material improvement = capital indicator); like-for-like specification of replacement material.

\paragraph{Criterion B --- Extension of Useful Life}

Work enabling the asset or component to contribute to income production for a period \textit{longer than the original expected life} is an indicator of improvement and therefore capital. Assessed from objective indicators, not intuition.

\textit{Evidence required:} ATO effective life tables (Effective Life Determination~2025) --- compare effective life of replacement to item removed; asset register comparison (original vs.\ revised estimated useful life after works); engineering/QS written statement for works above \$20,000 confirming serviceable life of new component; manufacturer specification sheets confirming design life.

\paragraph{Criterion C --- Replacement of an Entirety}

TR~97/23 distinguishes repair (renewal of subsidiary parts of a whole) from capital (reconstruction of the whole or an `entirety'). Replacing the entirety of a separately identifiable part is capital \textit{even if the only purpose was to restore, not to improve}.

\textit{Applying the test:} First, identify the separately identifiable unit --- would a reasonable person describe the work as `replacing the [asset]' rather than `fixing part of the [asset]'? If yes, treat as capital. Second, assess the proportion replaced: $\geq$50\% of tiles, pipe runs, cables, or structural elements replaced in one project is likely an entirety; below 30\% in a single project is more defensible; 30--50\% is a grey zone requiring professional judgement and documented rationale.

\begin{warning}[ATO Aggregation Rule — Do Not Split a Single Scheme]
The ATO may aggregate nominally separate work orders that form part of a single scheme to replace an entirety. Do not split a roof replacement project into multiple small invoices to avoid the capital threshold --- economic substance governs. For 22 Constance Street: rebuilding 80\% of the roof structure = replacement of entirety $\to$ Division~43 capital works. Replacing 10\% of damaged tiles = repair (s25-10).
\end{warning}

\paragraph{Criterion D --- Initial Repairs}

Under TR~97/23 (and confirmed in \textit{FCT v Western Suburbs Cinemas Ltd}), costs to repair defects that \textbf{existed at acquisition} (or were effectively `bought in' via price) are capital --- not an immediate repair deduction. This is a very common error in property fund accounting.

\textit{Assessment:} Review the purchase contract, building and pest inspection report, and due diligence records. Any defect identified at acquisition that is subsequently `repaired' is an initial repair $\to$ capital. Also consider: was the property acquired `as is' with the defect reflected in the price? Were the works required to make the property lettable before the first lease? For 22 Constance Street, the cyclone occurred during the income-producing period --- the initial repair rule does \textit{not} apply to cyclone remediation.

\paragraph{Criterion E --- Repair vs.\ Improvement (Effect-Based Test)}

The ATO's practical focus in TR~97/23 is on \textbf{effect: restore vs.\ improve}. A repair merely returns the asset to working condition. An improvement leaves the asset in a better state than it was before the damage or deterioration.

\textit{Signs of improvement (capital indicators):} Better materials combined with better performance or standards (not merely modern equivalents --- like-for-like using current product is generally still a repair); added features not present in the original component (access control, energy monitoring, additional safety capabilities); compliance-driven upgrades where the result is a better asset than before (the motive is irrelevant --- the outcome governs).

\subsubsection{SOP Decision Checklist --- Apply to Every Invoice Above \$5,000}
\label{sec:est-capex-checklist}

\begin{enumerate}[nosep]
    \item \textbf{Identify the unit of property:} Which part, component, system, or building element does the work address?
    \item \textbf{Initial repair test:} Was the defect noted at acquisition or reflected in the purchase price? If yes $\to$ likely initial repair (capital).
    \item \textbf{Entirety test:} Does the work replace the whole or substantially the whole of a separately identifiable part? If yes $\to$ capital.
    \item \textbf{Improvement test:} Does the work restore only, or does it improve/upgrade? Use objective evidence (scope language, specs, performance change).
    \item \textbf{If capital --- classify into the correct tax regime:} Div~40 (depreciating plant), Div~43 (structural capital works), or CGT cost base Element~4 (see Section~\ref{sec:est-cgt-costbase}).
    \item \textbf{Document and retain:} Scope of works, photos, PM report, QS/engineer opinion if $>$\$20,000. Attach to invoice on file.
\end{enumerate}

\begin{critical}[Capex vs. Repair Assessment Form — Mandatory Above Threshold]
A \textbf{Capex vs.\ Repair Assessment Form} is required for any invoice above the materiality threshold. The form must include mandatory evidence fields: before/after photos, scope of works description, PM report reference, whether the defect existed at acquisition, and whether the work replaces an entirety. TR~97/23 criteria must be explicitly referenced. Without this form, the classification cannot be considered adequately supported for ATO scrutiny.
\end{critical}

\subsubsection{Tax Classification --- Div~40, Div~43, and s25-10}
\label{sec:est-capex-taxdivision}
\legbox{Division~40 ITAA~1997 \legalcite{Div~40}{sec:ref-taxdep}; Division~43 ITAA~1997 \legalcite{Div~43}{sec:ref-taxdep}; s40-285 ITAA~1997 \legalcite{s40-285}{sec:ref-taxdep}; Effective Life Determination~2025 \legalcite{Eff.\ Life Det.\ 2025}{sec:ref-taxdep}.}

\paragraph{Three-Step Classification Process}

\begin{enumerate}[nosep]
    \item \textbf{Step 1 --- Capital or revenue?} Apply TR~97/23 criteria (Section~\ref{sec:est-capex-criteria}). If revenue $\to$ Step 3. If capital $\to$ Step 2.
    \item \textbf{Step 2 --- Div~40 or Div~43?} A \textit{depreciating asset} (Div~40) is tangible, has a limited effective life, and declines in value over time (s40-30 ITAA~1997). \textit{Capital works} (Div~43) are structural improvements to buildings and their fixed infrastructure. If depreciating asset $\to$ Div~40. If structural $\to$ Div~43.
    \item \textbf{Step 3 --- Revenue deduction category:} Genuine repair restoring original condition $\to$ s25-10. Business operating expense (not a repair, not capital) $\to$ s8-1. Borrowing expense $\to$ s25-25. Residual business capital $\to$ s40-880.
\end{enumerate}

\paragraph{Division~40 --- Depreciating Assets}

\textit{What qualifies:} Separately identifiable equipment: air conditioning units, lifts, hot water systems, electrical switchboards, security cameras, generators; fixtures and fittings: carpet, blinds, removable partitions, fire extinguishers, emergency lighting; technology: building management systems, fire detection systems (Ampac), access control systems.

\textit{What does NOT qualify:} Land; buildings and structural components (walls, floors, roof structure, permanent plumbing, wiring embedded in structure) $\to$ Div~43; intangible assets.

\textit{Key concepts:} Effective life per ATO Effective Life Determination~2025. Low-value pool: items $<$\$1,000 pooled at 18.75\% in year of acquisition, 37.5\% thereafter (s40-425). Immediate deduction for items $\leq$\$300 used primarily to produce income (s40-80). On disposal: balancing adjustment event (s40-295) --- compare termination value to adjustable value; excess = assessable income; shortfall = additional deduction.

\paragraph{Division~43 --- Capital Works}

\textit{What qualifies:} Building structure and structural improvements (walls, floors, roof structure, ceilings, stairwells, foundations); permanent infrastructure (plumbing embedded in structure, wiring conduit, fixed electrical reticulation); external structural works (paving, driveways, retaining walls, structural landscaping); asbestos removal enabling future income-producing use.

\textit{What does NOT qualify:} Plant and equipment ($\to$ Div~40); works constituting repairs (immediately deductible under s25-10); original construction of 22 Constance Street Sub Station (pre-1985) --- only post-1992 renovation/capital works qualify. Engage a quantity surveyor (QS) to identify qualifying construction expenditure and commencement dates.

\textit{Rate and commencement rules:} Works commencing after 26 February 1992: 2.5\% p.a.\ over 40 years (commercial income-producing property). Works commencing 21 August 1984 -- 26 February 1992: 4\% over 25 years. Works before 18 July 1985: generally ineligible.

\textit{Critical --- cost base reduction:} All Div~43 deductions claimed reduce the CGT cost base dollar-for-dollar (s110-45(4) ITAA~1997). See Section~\ref{sec:est-cgt-div43} for mandatory action.

\paragraph{Section 25-10 --- Repairs (Immediate Deduction)}

All five conditions must be satisfied for s25-10 to apply: (1) the property is used or available for use to produce assessable income at the time of the repair; (2) the repair is to property used for income production; (3) the expenditure is a `repair' per TR~97/23 --- not an improvement, entirety, or initial repair; (4) the expenditure is not capital in nature under the general test; and (5) the initial repair exclusion does not apply.

\begin{table}[H]
\centering\small\sffamily
\caption{Classification summary: accounting vs.\ tax treatment of common expenditures}
\label{tab:capex-tax}
\begin{tabularx}{\textwidth}{@{}L{3cm} L{2cm} L{2cm} L{2cm} X@{}}
\rowcolor{navy}\textcolor{white}{\bfseries Expenditure} & \textcolor{white}{\bfseries Accounting} & \textcolor{white}{\bfseries Div~40} & \textcolor{white}{\bfseries Div~43} & \textcolor{white}{\bfseries s25-10 / Notes} \\
\midrule
New A/C unit & Capex \acct{781} & \checkmark & & Over effective life (Effective Life Determination~2025, ~10--15 yrs) \\
\rowcolor{warmgray}
Roof replacement (entire) & Capex \acct{781} & & \checkmark & 2.5\% p.a.\ over 40 years \\
Roof tile repair (partial, like-for-like) & Expense \acct{700} & & & \checkmark --- immediate in year incurred \\
\rowcolor{warmgray}
Fire detection system Ampac & Capex \acct{781} & \checkmark & & Effective life ~10 yrs (ATO tables) \\
Fire extinguisher annual service & Expense \acct{700} & & & \checkmark --- s8-1 operating expense \\
\rowcolor{warmgray}
Asbestos removal (structural) & Capex \acct{781} & & \checkmark & 2.5\% p.a.\ --- part of capital works cost \\
Emergency cyclone tarping & Expense \acct{700} & & & \checkmark --- immediate in year incurred \\
\rowcolor{warmgray}
Balcony structural reconstruction & Capex \acct{781} & & \checkmark & 2.5\% p.a.\ over 40 years \\
Carpet replacement (common areas) & Capex \acct{781} & \checkmark & & Effective life 8--10 years \\
\rowcolor{warmgray}
Interior repaint & Expense \acct{700} & & & \checkmark --- repair, restores condition \\
Electrical switchboard upgrade & Capex \acct{781} & \checkmark & & Effective life 20--25 years \\
\rowcolor{warmgray}
External structural doors and frames & Capex \acct{781} & & \checkmark & 2.5\% p.a.\ (structural fixture) \\
Hot water system replacement & Capex \acct{781} & \checkmark & & Effective life 12 years \\
\rowcolor{warmgray}
Garden landscaping upgrade & Capex \acct{781} & & \checkmark & 2.5\% p.a.\ (structural works) \\
Routine garden maintenance (Grolife) & Expense \acct{700} & & & \checkmark --- s8-1 operating \\
\rowcolor{warmgray}
Leasing commission & \acct{780.12} amortise & \multicolumn{2}{l}{Div~40 or s40-880} & Over lease term or 5-year write-off \\
\bottomrule
\end{tabularx}
\end{table}

\begin{tip}[Apply the Betterment Test]
Apply the \textbf{`betterment test'}: does the work leave the asset in a better condition than immediately before the work was needed? Yes $\to$ capital. No (merely restores prior condition) $\to$ repair. For tax, also apply the \textbf{`entirety test'}: if the work replaces the whole or substantially the whole of the asset or a separately identifiable component, it is capital regardless of whether it improves or merely restores.
\end{tip}

% ============================================================================

\subsection{Expenditure Classification --- Repair Expense Integrated into Asset Cost vs.\ Standalone Expense}
\label{sec:est-repair-integration}
\legbox{AASB~116.43 (component accounting) \legalcite{AASB~116}{sec:ref-accounting}; AASB~140 (capitalisation criteria) \legalcite{AASB~140}{sec:ref-accounting}; TR~97/23 (tax repair vs.\ capital) \legalcite{TR~97/23}{sec:ref-repairs}; s25-10 ITAA~1997 \legalcite{s25-10}{sec:ref-repairs}.}
% ============================================================================

This section supplements the repair vs.\ capital classification guide in Section~\ref{sec:est-capex-guide} by addressing a subtler question: when a repair or maintenance expenditure is \textbf{sufficiently related to a capital asset} that it should be \textbf{integrated into the asset's carrying cost} rather than treated as a standalone P\&L expense, and when that integration occurs, how should both the accounting and the tax treatment align. This analysis arises specifically in the context of component accounting under AASB~116/140, the derecognition of replaced components, and the interaction between an invoice's repair/capital classification and its financial statement presentation.

\subsubsection{The Core Question: Three Possible Treatments for Any Item of Expenditure}

Every expenditure related to an existing asset has three possible accounting outcomes:

\begin{enumerate}[nosep]
    \item \textbf{Standalone expense (P\&L):} The expenditure does not meet the asset recognition criteria of AASB~116 (or AASB~140 by analogy for investment property) --- it is a repair, maintenance, routine servicing, or operating cost. Expensed immediately to P\&L (\acct{700} operating expenses or trust-level expense). Deductible under s25-10 (repair) or s8-1 (operating expense) for tax.
    \item \textbf{Capitalised to asset (Balance Sheet --- new or replacement component):} The expenditure meets the recognition criteria --- it is probable that future economic benefits will flow and the cost can be reliably measured. It is a new component, an extension, or a replacement of a separately identifiable component whose old book value is derecognised. Capitalised to \acct{781.01} or \acct{781.02} (Capex tab). Deductible over effective life under Div~40 or Div~43 for tax.
    \item \textbf{Integrated repair cost --- capitalised but linked to existing component:} A repair that is sufficiently significant to enhance or extend the economic life of a specific existing component, such that the entire original component (or a portion of it) must be \textbf{derecognised} and replaced with the repaired/reinstated component at the new cost. This is the most nuanced treatment and requires component accounting under AASB~116.43.
\end{enumerate}

\subsubsection{Component Accounting --- AASB~116.43 Applied by Analogy to Investment Property}

AASB~140 (Investment Property, measured at fair value or cost) does not mandate component accounting in the way AASB~116 does for plant and equipment. However, the principles of AASB~116.43--116.47 are applied \textbf{by analogy} for investment property held at cost, and many funds apply them in practice. Under component accounting:

\begin{itemize}[nosep]
    \item Each \textbf{separately identifiable significant component} of a building is recognised, depreciated, and derecognised separately. For example: structure (roof, walls, floors), mechanical systems (HVAC, lifts), electrical systems, fit-out, and facades.
    \item When a component is \textbf{replaced}, the carrying amount of the old component is \textbf{derecognised} (removed from the balance sheet as a disposal), and the new component is \textbf{recognised} (capitalised at the cost of the replacement).
    \item If the carrying amount of the replaced component is not separately tracked (because component accounting was not applied on acquisition), the fund \textbf{estimates} the cost and accumulated depreciation of the old component to calculate its derecognised book value.
\end{itemize}

\begin{note}[Component Derecognition — Limited Impact for Fair Value IP]
For 22 Constance Street, which is measured at fair value under AASB~140 (not cost), the practical importance of component derecognition is limited for balance sheet purposes --- because fair value is assessed by an independent valuer who implicitly incorporates component condition. However, component accounting \textbf{remains important for the CGT cost base and tax depreciation schedules} (Div~40 and Div~43), even if not strictly required for AASB~140 balance sheet measurement.
\end{note}

\subsubsection{When to Integrate Repair Cost into Asset Cost (vs.\ Expense Separately)}

The following decision framework determines whether a repair cost should be integrated into the asset's carrying cost or expensed separately:

\begin{table}[H]
\centering\small\sffamily
\caption{Decision framework: repair integrated into asset cost vs.\ standalone expense}
\begin{tabularx}{\textwidth}{@{}L{4cm} L{3cm} X@{}}
\rowcolor{navy}\textcolor{white}{\bfseries Condition} & \textcolor{white}{\bfseries Treatment} & \textcolor{white}{\bfseries Rationale and Application} \\
\midrule
Repair restores original condition only; no identifiable component replaced; work is partial (not entire component) & \textbf{Standalone P\&L expense} & Does not meet asset recognition criteria. No component derecognition. Accounted for as period expense (\acct{700}). Tax: s25-10 or s8-1. \\
\rowcolor{warmgray}
Repair replaces an entire identifiable component (AASB~116.43 trigger) & \textbf{Capitalise new; derecognise old} & Old component book value derecognised (loss on disposal). New component capitalised at replacement cost. Tax: Div~40 (plant) or Div~43 (structural). \\
Repair enhances or extends the economic life of an existing component beyond its original expected life & \textbf{Capitalise the enhancement} & The portion attributable to future economic benefit beyond original life is an asset addition. The maintenance portion remains as expense. Apportion if mixed. Tax: Div~40 improvement. \\
\rowcolor{warmgray}
Repair invoice is a mixed invoice (partly restores, partly improves a single component) & \textbf{Apportion: expense portion to P\&L; capital portion to BS} & Must split the invoice between repair (s25-10) and capital (Div~40/43) components. Request itemised breakdown from contractor. If not obtainable, apply a reasonable estimate based on scope of works. Document the apportionment basis. \\
Repair is ``bundled'' with new installation in a single invoice (e.g., cost of removing old HVAC + new HVAC unit) & \textbf{Apportion: removal cost as expense; new unit as capital} & Removal of old asset is a cost of derecognition. New unit is the replacement capital cost. Both amounts should be identified separately on the invoice or estimated. \\
\rowcolor{warmgray}
Repair cost is immaterial (below \$5,000 per the fund's materiality threshold) & \textbf{Expense regardless of nature} & Materiality: applying component accounting below materiality is not cost-effective. Expense immediately. \\
Insurance-funded repair of structural damage & \textbf{Capital if replacing a capital component; expense if repair only} & If the structural damage reinstatement replaces a capital works component (Div~43), capitalise the reinstatement. The insurance proceeds reduce the cost base of the reinstated Div~43 works (Section~\ref{sec:est-insurance-cgt}). If the repair merely restores like-for-like without component replacement, expense and recognise insurance proceeds as income. \\
\bottomrule
\end{tabularx}
\end{table}

\subsubsection{Derecognition of Replaced Component --- Accounting Mechanics}

When a component is replaced (trigger: AASB~116.43), the Preparer must:

\paragraph{Step 1 --- Identify the replaced component's book value}

If the replaced component is separately tracked in the asset register (Capex tab), its book value is directly available. If not separately tracked, estimate the original cost of the replaced component and the accumulated depreciation at date of replacement:
\begin{itemize}[nosep]
    \item \textbf{Original cost estimate:} Use the quantity surveyor's depreciation schedule (if available) or a reasonable proportion of the building's original cost base attributable to that component.
    \item \textbf{Accumulated depreciation estimate:} Based on the estimated useful life and age of the old component.
    \item \textbf{Book value (carrying amount):} Original cost $-$ accumulated depreciation = book value of old component.
\end{itemize}

\paragraph{Step 2 --- Derecognise the old component}

\[
\text{DR } \textit{Accumulated Depreciation (old component)} \quad \text{CR } \textit{Asset Account (cost of old component)}
\]
\[
\text{DR } \textit{Loss on Disposal / P\&L} \quad \text{CR } \textit{Asset Account (remaining book value)}
\]

For investment property measured at fair value, the accumulated depreciation account may not exist separately (because fair value adjustments are made globally, not via accumulated depreciation). In this case, the derecognition is reflected as an adjustment in the next fair value assessment, not as a separate journal.

\paragraph{Step 3 --- Recognise the new component}

\[
\text{DR } \acct{781.01} \text{ --- Investment Property (new component cost)} \quad \text{CR } \acct{685} \text{ (or } \acct{882.02} \text{ if accrued)}
\]

\paragraph{Step 4 --- Update the tax depreciation schedule}

\begin{itemize}[nosep]
    \item For Div~40 (plant): trigger a \textbf{balancing adjustment event} (s40-295) for the old asset (termination value = nil if scrapped, or insurance proceeds if insurance-funded). Record any balancing adjustment gain or loss. Commence new Div~40 depreciation on the new component at its full cost, using the ATO effective life (Effective Life Determination~2025).
    \item For Div~43 (capital works): no balancing adjustment for structural works. Update the capital works schedule with the new works cost; commence 2.5\% p.a.\ deduction from date of completion. Adjust CGT cost base for the derecognised Div~43 deductions (s110-45(4)) and for the new component's addition to Element~1 of the cost base.
\end{itemize}

\subsubsection{Mixed Invoice Treatment --- Practical Guidance}

In practice, many remediation and maintenance invoices cover work that is \textbf{partly repair and partly capital}. The ATO's position (TR~97/23, paragraph~33) is that where expenditure can be attributed to both, it must be apportioned. The following approach applies:

\begin{enumerate}[nosep]
    \item \textbf{Request itemisation:} Ask the contractor/supplier to provide a detailed breakdown of the invoice identifying repair labour, repair materials, capital installation, and removal/disposal separately. If a detailed breakdown is not available, request a proportionate breakdown based on the scope of works.
    \item \textbf{Apply a reasonable basis:} If no breakdown is available, the Preparer may apportion based on: (a) the proportion of square metres affected (for area-based repairs); (b) the materials cost vs. labour cost split (capital is usually material-heavy; maintenance is labour-heavy); or (c) an engineer's or quantity surveyor's assessment.
    \item \textbf{Document the apportionment:} Record the apportionment basis on the Capex Assessment Form for the invoice, with supporting rationale. The ATO may challenge any apportionment that results in a disproportionate allocation to the more tax-advantaged category.
    \item \textbf{Threshold:} Where the repair portion of a mixed invoice is $<$20\% of the total, the Preparer may treat the full invoice as capital (practical simplification, subject to materiality). Where the capital portion is $<$20\%, treat the full invoice as a repair expense. In borderline cases, document the basis and seek tax agent guidance.
\end{enumerate}

\subsubsection{Interaction with the Capex Assessment Form}

Every invoice above the \$5,000 materiality threshold (Section~\ref{sec:est-capex-checklist}) must have a completed \textbf{Capex vs.\ Repair Assessment Form}. Where the classification decision is ``integrated repair'' (i.e., the repair replaces a component), the Form must additionally document:

\begin{itemize}[nosep]
    \item Whether the replaced component was separately tracked in the Capex tab;
    \item The estimated book value of the derecognised component (with calculation supporting the estimate);
    \item Whether a Div~40 balancing adjustment event is triggered for the old component;
    \item Whether the new component's Div~43 cost must be tracked separately from the existing Div~43 schedule;
    \item Whether the insurance proceeds received (if applicable) must be applied to reduce the CGT cost base of the reinstated works (Section~\ref{sec:est-insurance-cgt}).
\end{itemize}

\begin{tip}[Common Error — Misclassifying Replacement as Repair]
The most common error in practice is to \textbf{classify a replacement as a repair} because the end result looks the same as before (e.g., replacing a damaged roof). Under TR~97/23 and AASB~116, \textbf{replacing the whole of a separately identifiable component is always capital} --- even if the outcome is identical to the original condition. The distinction is not the outcome but the scope: \textit{restoring a part} = repair; \textit{replacing the whole} = capital. Similarly, the most common under-claimed deduction is failing to take the Div~40 balancing adjustment deduction for the old asset when it is scrapped on replacement.
\end{tip}

\begin{table}[H]
\centering\small\sffamily
\caption{Repair integration vs.\ standalone --- common scenarios for 22 Constance Street}
\begin{tabularx}{\textwidth}{@{}L{4.5cm} L{2.5cm} L{2.5cm} X@{}}
\rowcolor{navy}\textcolor{white}{\bfseries Scenario} & \textcolor{white}{\bfseries Accounting} & \textcolor{white}{\bfseries Tax} & \textcolor{white}{\bfseries Component Derecognition?} \\
\midrule
Cyclone damage: partial roof tile replacement (10\% of roof area) & P\&L expense & s25-10 repair & No; partial repair only \\
\rowcolor{warmgray}
Cyclone damage: entire roof structure replacement & Capitalise new; derecognise old roof & Div~43 (2.5\% p.a.) & Yes --- derecognise old roof component; balancing adjustment if Div~43 \\
Ampac fire detection system: faulty sensor replaced & P\&L expense & s8-1 or s25-10 & No; individual sensor $<$ system entirety \\
\rowcolor{warmgray}
Ampac fire detection system: entire system replaced (new model) & Capitalise; derecognise old system & Div~40 (effective life~10yrs) & Yes --- Div~40 balancing adjustment on old system \\
HVAC unit: refrigerant regas and service & P\&L expense & s25-10 / s8-1 & No \\
\rowcolor{warmgray}
HVAC unit: compressor replacement (entire unit) & Capitalise; derecognise old unit & Div~40 & Yes --- Div~40 balancing adjustment on old unit \\
Mixed invoice: patch-and-paint (repair) + new partition walls (capital) & Apportion: repair $\to$ P\&L; partitions $\to$ BS & Apportion: s25-10 + Div~40/43 & Only for partition walls if they replace existing partitions \\
\rowcolor{warmgray}
Insurance-funded: balcony structural reconstruction & Capitalise at reinstatement cost; insurance proceeds reduce Div~43 cost base & Div~43 (2.5\%); CGT cost base Element~4 adjusted for insurance proceeds & Yes --- derecognise old balcony component; new component recognised \\
Cyclone emergency tarping (temporary) & P\&L expense & s25-10 (immediate) & No \\
\rowcolor{warmgray}
Interior repaint (full building) & P\&L expense & s25-10 / s8-1 & No \\
Electrical switchboard: full replacement due to storm damage & Capitalise; derecognise old switchboard & Div~40 (20--25 year life) & Yes --- Div~40 balancing adjustment \\
\bottomrule
\end{tabularx}
\end{table}

\begin{critical}[Mandatory Cross-Checks After Every Classification Decision]
\textbf{Mandatory cross-checks after every repair/capital classification decision:}
\begin{enumerate}[nosep]
    \item If the item is classified as \textbf{capital}: has the Div~43 cost base impact been recorded? Has the old component (if replaced) been derecognised and a Div~40 balancing adjustment calculated?
    \item If the item is classified as \textbf{repair/expense}: has it been confirmed that the work is not the replacement of an entirety? If insurance-funded, is the insurance receipt correctly characterised (income vs.\ cost base reduction)?
    \item For \textbf{any} classification: confirm the amount has NOT been double-counted (recorded in both Yardi as an operating expense AND in the UFS Capex tab as capital). See Section~\ref{sec:est-double-count}.
\end{enumerate}
\end{critical}

% ===========================================================================

% ============================================================================

\subsection{Section 25-25 and Section 40-880 --- Comprehensive Deduction Guidance}
\label{sec:est-s2525-s40880}
\legbox{s25-25 ITAA~1997 (borrowing costs) \legalcite{s25-25}{sec:ref-borrowing}; s25-25(5) ITAA~1997 (immediate write-off on repayment) \legalcite{s25-25(5)}{sec:ref-borrowing}; s40-880 ITAA~1997 (blackhole expenditure) \legalcite{s40-880}{sec:ref-borrowing}.}

\subsubsection{Section 25-25 --- Borrowing Expenses}

Section 25-25 ITAA~1997 provides that borrowing expenses (costs of establishing a borrowing used to produce assessable income) are deductible. If the expense is \$100 or less, it is immediately deductible. If more than \$100, it is spread over the \textbf{lesser of the loan term and 5 years}, starting from the date the borrowed funds are first used.

\textit{What qualifies:} Loan establishment/application fees charged by St George; mortgage registration fees; legal fees paid by the borrower to the lender's solicitor for loan documentation; valuation fees required by the lender at loan establishment; mortgage broker commissions paid by the borrower; loan insurance premium required by lender as a condition of the loan.

\textit{What does NOT qualify:} Ongoing loan management fees ($\to$ deductible under s8-1); interest on the loan ($\to$ s8-1); stamp duty on the mortgage ($\to$ may be Element~2 CGT cost base addition).

\begin{table}[H]
\centering\small\sffamily
\caption{Section 25-25 timing rules}
\begin{tabularx}{\textwidth}{@{}L{3.5cm} L{4.5cm} X@{}}
\rowcolor{navy}\textcolor{white}{\bfseries Scenario} & \textcolor{white}{\bfseries Deduction Period} & \textcolor{white}{\bfseries Note} \\
\midrule
Expense is \$100 or less & Immediate in year incurred & Straightforward \\
\rowcolor{warmgray}
Expense $>$\$100, loan term $\leq$ 5 years & Spread over loan term (days basis) & Deduct proportionally each year \\
Expense $>$\$100, loan term $>$ 5 years & Spread over 5 years maximum & Any unamortised balance at year 6 is lost \\
\rowcolor{warmgray}
Loan repaid early (e.g., at settlement) & Remaining unamortised balance deductible in year of repayment (s25-25(5)) & CRITICAL --- immediate write-off of \acct{303.03}/\acct{303.04} upon St George repayment at settlement \\
Loan refinanced & Original costs continue their schedule; new loan's costs begin fresh schedule & Do not conflate the two schedules \\
\bottomrule
\end{tabularx}
\end{table}

\begin{critical}[Borrowing Costs Written Off on Loan Repayment at Settlement]
Upon repayment of the St George facility at settlement, the entire unamortised balance of borrowing costs (accounts \acct{303.03} and \acct{303.04}) becomes immediately deductible in the settlement FY under s25-25(5). Ensure the tax return reflects this full write-off. Maintain a Borrowing Cost Register with start date, loan term, amortisation period, and early repayment trigger. Do not continue amortising after the loan is repaid.
\end{critical}

\subsubsection{Section 40-880 --- Blackhole Business Expenditure (5-Year Write-Off)}

Section 40-880 ITAA~1997 is the `last resort' provision --- it makes certain \textbf{business capital expenditure deductible over 5 years} where it is not otherwise deductible and not included elsewhere (not Div~40, not Div~43, not CGT cost base). The five gate tests must all be confirmed before applying s40-880: (1) not deductible under ordinary rules (run through s8-1 first); (2) not part of Div~40 cost or Div~43 construction expenditure; (3) not included in CGT cost base (cost base inclusion and s40-880 are mutually exclusive); (4) business must be carried on or recently ceased (5-year window post-cessation); and (5) the expenditure is capital in nature.

\textit{Typical candidates for the fund:} Fund establishment costs (initial legal, ASIC registration, trust deed) --- 5 years from date of establishment; business structuring/restructure costs that are capital; post-sale wind-up costs (accounting, ASIC, final audit) --- 5 years from date of wind-up expenses.

\textit{What does NOT qualify:} Borrowing expenses $>$\$100 ($\to$ use s25-25 as the specific provision); land tax and council rates ($\to$ s8-1); stamp duty on acquisition ($\to$ Element~2 cost base); ongoing accounting/audit fees ($\to$ s8-1); insurance premiums ($\to$ s8-1).

\begin{note}[Legal Acquisition Fees — Not Deductible as s25-25 Borrowing Costs]
Legal fees paid to the fund's own solicitor for acquiring the property are NOT s25-25 borrowing expenses. If they are capital and not otherwise deductible, they may qualify under s40-880. However, if they are Element~2 incidental acquisition costs (stamp duty, registration, legal fees on purchase), they are cost base inclusions and NOT s40-880. The characterisation depends on the specific nature of the legal advice.
\end{note}

% ============================================================================


% ============================================================================
% PART IV: CAPITAL GAINS TAX AND PERIOD CUT-OFF
% ============================================================================

\vspace{1.2cm}
\begin{tcolorbox}[
    enhanced, breakable,
    colback=navy, colframe=navy,
    boxrule=0pt,
    borderline south={2pt}{0pt}{gold},
    sharp corners,
    left=14pt, right=10pt, top=10pt, bottom=10pt,
    label={sec:est-part-cgt}
]
{\sffamily\bfseries\footnotesize\textcolor{gold}{Part IV\enspace\textbullet\enspace Appendix~C}}\par
\vspace{3pt}
{\sffamily\bfseries\large\textcolor{white}{Capital Gains Tax and Period Cut-Off}}
\end{tcolorbox}
\begin{tcolorbox}[
    enhanced, breakable,
    colback=palesky, colframe=navy!30!white,
    boxrule=0.4pt, sharp corners,
    left=10pt, right=10pt, top=6pt, bottom=6pt
]
\small\itshape This part consolidates all CGT guidance for the property sale (Event~A1, cost base, Div~43 adjustment, Div~115 discount), the put/call option structure (Event~H1), works straddling the sale date, and all cut-off and late-invoice deductibility issues.
\end{tcolorbox}
\vspace{0.3cm}

\subsection{Capital Gains Tax --- Key Issues for Sale of 22 Constance Street}
\label{sec:est-cgt}
\legbox{s104-10 ITAA~1997 (A1) \legalcite{s104-10}{sec:ref-cgt}; s110-25 ITAA~1997 (cost base) \legalcite{s110-25}{sec:ref-cgt}; Division~115 ITAA~1997 (50\% discount) \legalcite{Div~115}{sec:ref-cgt}; s102-5 ITAA~1997 (capital losses) \legalcite{s102-5}{sec:ref-cgt}; s110-45(4) ITAA~1997 (Div~43 reduction) \legalcite{s110-45(4)}{sec:ref-taxdep}.}

This section provides guidance on CGT issues relevant to the sale of 22 Constance Street by the fund. All legislative references are to Divisions 100--118 ITAA~1997. This guidance should be read alongside the fund's tax agent's CGT schedule and does not replace specialist tax advice. For CGT Event H1 (granting of options and rights), see Section~\ref{sec:est-cgt-h1}. For insurance claim CGT interactions, see Section~\ref{sec:est-insurance-cgt}.

\subsubsection{CGT Event A1 --- Contract Date, Not Settlement Date}
\label{sec:est-cgt-timing}

The sale of real property constitutes CGT Event A1 (s104-10 ITAA~1997). The CGT event happens \textbf{when the contract is entered into}, not at settlement. This is the most commonly missed CGT point in property fund accounting.

If the sale contract is signed in June FY and settlement occurs in August, the capital gain is assessable in the \textbf{June FY}, even though cash settlement occurs in the next FY. SOP control: confirm the CGT year is based on contract date (CGT Event A1 timing), regardless of when accounting derecognition occurs at settlement.

The accounting gain/loss is typically recognised at settlement (when risks and rewards transfer under AASB~140), but the taxable capital gain is assessed in the FY of the contract date. A timing difference between accounting and tax may arise --- document in the tax reconciliation workpaper.

\begin{critical}[CGT Event Date — FY2026 if Contract Executed Before 30 June]
For 22 Constance Street: if the sale contract is executed before 30 June 2026, the capital gain is assessable in FY2026, regardless of settlement date. The tax agent must be notified on the day of exchange to ensure the correct FY is captured in the return. Flag the exchange date immediately --- same-day protocol.
\end{critical}

\subsubsection{Splitting the Sale --- CGT Asset vs.\ Div~40 Balancing Adjustments}

Property disposals involve \textbf{multiple CGT assets and Div~40 balancing adjustments} --- the entire sale proceeds cannot be treated as a single CGT amount.

Building and land $\to$ CGT Event A1 (proceeds for the CGT regime). Plant and equipment $\to$ Div~40 balancing adjustment event (s40-295): on disposal, compare termination value (proceeds allocated to that asset) to adjustable value (undeducted cost). If termination value $>$ adjustable value: assessable income (recapture). If termination value $<$ adjustable value: additional deduction.

ATO expects a reasonable allocation of proceeds between the CGT asset (land/building) and Div~40 assets, typically supported by the QS/tax depreciation report. Avoid assigning nil to Div~40 assets. Claim Div~40 decline in value up to the stop-holding day (settlement date) in the year of disposal.

\subsubsection{CGT Cost Base --- Five Elements and Key Traps}
\label{sec:est-cgt-costbase}

The cost base of a CGT asset comprises five elements (s110-25 ITAA~1997). For the full five-element table and cost base reduction rules in the context of the distribution component calculation, see Section~\ref{sec:dist-cost-base-property}. The table below summarises the key traps specific to CGT computation for 22~Constance Street:

\begin{table}[H]
\centering\small\sffamily
\caption{CGT cost base elements for 22 Constance Street}
\begin{tabularx}{\textwidth}{@{}C{0.8cm} L{3.5cm} L{4.5cm} X@{}}
\rowcolor{navy}\textcolor{white}{\bfseries Elem.} & \textcolor{white}{\bfseries Description} & \textcolor{white}{\bfseries Relevant Amounts} & \textcolor{white}{\bfseries Notes} \\
\midrule
1 & Acquisition cost & Original purchase price of the property & Include all amounts paid to acquire the asset \\
\rowcolor{warmgray}
2 & Incidental acquisition costs & Stamp duty, legal fees on purchase, registration fees & Often overlooked --- obtain from original acquisition file \\
3 & Ownership costs (non-deducted) & Rates, land tax, insurance ONLY if not deducted & For income-producing property this is generally nil \\
\rowcolor{warmgray}
4 & Capital enhancement costs & Capex additions (Div~43 capital works, \acct{781} additions) not previously deducted; adjusted for Div~43 deductions claimed & Div~43 deductions REDUCE cost base --- see Section~\ref{sec:est-cgt-div43} \\
5 & Incidental disposal costs & MP Realty commission (\$82,812.50), K\&L Gates legal fees, marketing, due-diligence costs borne by vendor (account \acct{709}) & Not deductible as income expense --- cost base only \\
\bottomrule
\end{tabularx}
\end{table}

\textbf{Common traps:} (1) Do not include holding costs (rates, land tax, insurance) in Element~3 if they have already been deducted under s8-1 --- for income-producing property, Element~3 is generally nil. (2) Selling costs (agent commission, conveyancing, legal fees on sale, advertising) are capital and add to cost base (Element~5) if not otherwise deductible --- do not also claim as income deductions.

\subsubsection{Division~43 Deductions Reduce the CGT Cost Base}
\label{sec:est-cgt-div43}

Section 110-45(4) ITAA~1997 requires the cost base to be reduced by any Div~43 amounts deducted since acquisition. Every dollar of Div~43 deduction claimed reduces the cost base dollar-for-dollar. This is one of the most frequently missed steps the ATO identifies on audit of property fund disposals. For the general mechanism of how s110-45 cost base reductions operate (including Division~43, recoupments, and input tax credits), see the ``Cost base reductions (s110-45)'' paragraph in Section~\ref{sec:dist-cost-base-property}.

\textbf{Mandatory action before tax return lodgement:} Obtain the full Div~43 schedule from the fund's quantity surveyor's tax depreciation report; confirm cumulative Div~43 claimed to the date of contract; record the cost base reduction as a reconciling item in the CGT schedule; include all cyclone remediation structural works capitalised to \acct{781} and deducted as Div~43.

\begin{critical}[Cost Base Must Be Reduced for All Div 43 Deductions]
Failure to reduce the cost base for Div~43 deductions results in understatement of the capital gain and exposes the fund to ATO audit adjustment, interest, and penalties.
\end{critical}

\subsubsection{Capital Proceeds}

Included in capital proceeds (s116-20 ITAA~1997): contract price; settlement adjustments received from the purchaser for rates, rent, and outgoings apportioned to the post-settlement period; deposit forfeited by a defaulting purchaser; non-cash consideration including assumed liabilities.

\textbf{NOT included:} Amounts received as genuine reimbursement of the vendor's pre-settlement costs; cyclone insurance proceeds (account \acct{73.02}) --- these are income receipts assessable in the year received under s6-5, unrelated to CGT Event A1.

\subsubsection{CGT Discount and Foreign Resident Withholding}

Individual and trust taxpayers (not companies) are entitled to the 50\% CGT discount under Div~115 ITAA~1997 where the asset has been held for at least 12 months before the CGT event date. Capital losses (current year or carried forward) must be offset against the capital gain \textbf{before} the 50\% discount is applied (s102-5). The discount capital gain must be distributed to beneficiaries separately on the AMMA statement.

Under the foreign resident capital gains withholding (FRCGW) regime (s14-200 Schedule~1 TAA~1953), if the purchaser of real property cannot confirm the vendor is an Australian resident, the purchaser must withhold \textbf{15\%} of the purchase price at settlement and remit it to the ATO (effective 1~January~2025; the previous \$750,000 threshold and 12.5\% rate no longer apply --- all property sales are now caught regardless of value). The fund (as vendor) should apply for an ATO clearance certificate at least 28~days before settlement to avoid withholding.

\subsubsection{Depreciation Deductions --- Contract Date vs.\ Settlement Date}
\label{sec:est-cgt-depdates}

The last date on which Div~40 and Div~43 deductions are claimable differs:

\textbf{Division~40 --- stop-holding day = settlement date:} Decline in value is calculated for the period up to and including the day the balancing adjustment event occurs --- the day the taxpayer stops holding the asset (s40-295). For real property, this is settlement date. Pro-rata calculation: Annual Div~40 deduction $\times$ (days from 1 July to settlement $\div$ 365).

\textbf{Division~43 --- deduction ends at settlement date:} The deduction is available for each income year in which the capital works are used to produce assessable income. This continues until settlement date. Pro-rata: Annual Div~43 $\times$ (days from 1 July to settlement $\div$ 365). All Div~43 deductions claimed (including the settlement-year pro-rata) reduce the CGT cost base and must be fed into the CGT schedule.

\begin{warning}[Cross-Year Settlement — CGT in FY2026, Prorations in FY2027]
If the contract is signed in June FY2026 and settlement occurs in September FY2027, the CGT gain is assessed in FY2026 BUT the Div~40 and Div~43 pro-rata deductions run to the settlement date in FY2027. These are two separate tax calculations in two different income years. Ensure the tax agent is aware of and prepares for both.
\end{warning}

% ============================================================================

\subsection{Capex, Repairs and Works Around the Sale --- CGT Implications}
\label{sec:est-cgt-works}

Works commenced or invoiced around the time of the property sale create classification challenges affecting both income tax deductibility and the CGT cost base.

\textbf{Works completed before contract date:} If repair $\to$ deductible as s25-10 in the income year the work was completed. If capex $\to$ added to cost base (Element~4) and depreciated under Div~40 or Div~43; Div~43 deductions claimed reduce the cost base. Invoices received after contract date but relating to pre-contract works still belong to the pre-contract period.

\textbf{Works straddling contract date:} Allocate costs to pre- and post-contract periods based on actual work completed (percentage of completion or time basis). Post-contract portion: if works are required by the contract of sale as a vendor obligation, these may be Element~5 incidental disposal costs (increasing cost base, reducing net capital gain). If voluntary post-contract works, they are neither deductible nor a cost base addition.

\textbf{Works after contract date:} Generally not deductible under s8-1 (no income-producing nexus); not Element~4 cost base additions (the CGT asset is already disposed of). Exception: Element~5 treatment if required by contract; or s40-880 for narrow categories of post-sale business costs.

\begin{table}[H]
\centering\small\sffamily
\caption{Cyclone remediation works straddling the sale --- classification}
\label{tab:cyclone-sale}
\begin{tabularx}{\textwidth}{@{}L{3cm} L{3.2cm} L{3.2cm} L{1.8cm}@{}}
\rowcolor{navy}\textcolor{white}{\bfseries Scenario} & \textcolor{white}{\bfseries Income Tax} & \textcolor{white}{\bfseries CGT Treatment} & \textcolor{white}{\bfseries Account} \\
\midrule
Structural remediation before contract date & Not s25-10 --- Div~43 at 2.5\% & Element~4 cost base addition; Div~43 deductions reduce cost base & \acct{781.01} \\
\rowcolor{warmgray}
Temporary repairs before contract date & s25-10 deductible in period incurred & No cost base effect (already deducted) & \acct{700.xx} \\
Works required by sale contract, post-contract & Not deductible under s8-1 & Element~5 incidental disposal cost & \acct{709} \\
\rowcolor{warmgray}
Voluntary works post-contract & Not deductible & No cost base effect & Notes only \\
Insurance proceeds received after settlement & Assessable income in year received (s6-5) & Not capital proceeds from sale & \acct{73.02} \\
\bottomrule
\end{tabularx}
\end{table}

% ============================================================================

\subsection{CGT Event H1 --- Granting a Right or Option over the Property}
\label{sec:est-cgt-h1}
\legbox{s104-155 ITAA~1997 (H1 event) \legalcite{s104-155}{sec:ref-cgt}; s134-1 ITAA~1997 (absorption) \legalcite{s134-1}{sec:ref-cgt}; Division~115 ITAA~1997 (no discount for freshly granted options) \legalcite{Div~115}{sec:ref-cgt}; TD~2017/D4 (PCO structure) \legalcite{TD~2017/D4}{sec:ref-loanswap}.}

CGT Event H1 is among the most commonly overlooked CGT events in property fund accounting because it arises \textbf{not from the sale of the property itself but from granting a right or option over it} --- an event that can occur many months before any sale contract is signed. For 22 Constance Street, the most likely H1 triggers are the grant of a put or call option to the purchaser, the granting of exclusivity rights, the grant of a right to inspect and conduct due diligence under a binding heads of agreement, or any contractual right allowing the purchaser to acquire the property at a specified price or on specified terms.

\subsubsection{What CGT Event H1 Is}

CGT Event H1 is defined in s104-155 ITAA~1997. It happens when an entity \textbf{grants an option} to another entity. The CGT event occurs at the time the option is granted, not when it is exercised or lapses. The capital gain or loss is calculated at the time of granting.

\begin{note}[CGT Event H1 (Option) vs. A1 (Disposal) — Distinct Events]
H1 is entirely distinct from CGT Event A1 (disposal of the property). A1 occurs when the contract of sale is entered into. H1 occurs earlier --- when the option or right is first granted. Both events may therefore fall in the \textbf{same} or \textbf{different} income years, creating separate CGT calculations and separate return entries.
\end{note}

\subsubsection{Triggers for CGT Event H1 in a Property Sale Context}

The following arrangements commonly give rise to H1 in a commercial property sale:

\begin{itemize}[nosep]
    \item \textbf{Put and call option deeds:} The fund grants the purchaser a \textit{call option} (right to compel the vendor to sell) in exchange for an option fee. This is the most common H1 trigger. The call option grant = H1. The subsequent exercise of the option = A1 (separate event, usually in a later period).
    \item \textbf{Exclusive dealing / exclusivity deeds:} A binding exclusivity agreement granting the purchaser the exclusive right to negotiate the purchase can itself be a `right' for H1 purposes. The ATO analyses the substance --- if the arrangement amounts to an option, H1 applies even if labelled differently.
    \item \textbf{Heads of agreement with binding elements:} If a heads of agreement is partly binding (e.g.\ exclusivity clause, right to conduct due diligence on the property at a defined price) it may contain a `right' attracting H1.
    \item \textbf{Pre-emptive rights granted to existing tenants:} If a tenant has a contractually granted right of first refusal or pre-emption to purchase the property, H1 may arise at the time that right is granted (not necessarily when exercised).
    \item \textbf{Side agreements or letters of intent:} Any side deed giving the purchaser optionality (right to withdraw without penalty, right to match a competing offer) may be analysed as an option for H1 purposes.
\end{itemize}

\subsubsection{How H1 is Calculated}

The capital gain under CGT Event H1 is:

\[
\text{Capital gain (H1)} = \text{Option fee received} - \text{Cost base of granting the option}
\]

\begin{itemize}[nosep]
    \item \textbf{Option fee (capital proceeds):} The premium or fee received from the purchaser for the grant of the option. In many commercial property transactions the option fee is a nominal amount (e.g.\ \$1 or \$10,000), but even a nominal amount triggers the H1 event and must be reported.
    \item \textbf{Cost base of granting the option:} The cost base of granting a right or option is generally the incremental costs incurred directly in relation to granting it --- e.g.\ legal fees specifically for drafting the option deed. Costs of the overall sale campaign allocated to the option grant period must be carefully distinguished from costs of the subsequent A1 event.
    \item \textbf{If the option is subsequently exercised (i.e.\ the sale contract is signed):} Under s134-1 ITAA~1997, the H1 gain is subsumed into the A1 calculation. The option premium becomes part of the capital proceeds for the A1 event, and the option fee and related H1 costs are added to the cost base of the A1 event. This prevents double taxation --- but the H1 event \textbf{must still be reported in the year it arose}, with a consequential amendment to or adjustment in the A1 year.
    \item \textbf{If the option lapses without exercise:} The H1 gain stands alone. There is no A1 event. The option fee is a standalone capital gain in the year the option was granted.
\end{itemize}

\begin{table}[H]
\centering\small\sffamily
\caption{CGT Event H1 vs.\ A1 --- timing and interaction}
\begin{tabularx}{\textwidth}{@{}L{2.5cm} L{3.5cm} L{3.5cm} X@{}}
\rowcolor{navy}\textcolor{white}{\bfseries Scenario} & \textcolor{white}{\bfseries H1 Event} & \textcolor{white}{\bfseries A1 Event} & \textcolor{white}{\bfseries Accounting / Tax Consequence} \\
\midrule
Option granted and exercised in same FY & Date option granted & Date sale contract signed & H1 gain subsumed into A1 (s134-1). Report both in same return. \\
\rowcolor{warmgray}
Option granted in FY2026, exercise in FY2027 & FY2026 --- report H1 gain in FY2026 return & FY2027 --- report A1 gain in FY2027 return & H1 reported in FY2026. A1 reported in FY2027. Option premium included in A1 capital proceeds; H1 costs added to A1 cost base via s134-1 adjustment. \\
Option granted but lapses (no exercise) & Date option granted & No A1 event & H1 gain stands. Option fee is a standalone capital gain. \\
\rowcolor{warmgray}
Exclusivity arrangement, no option fee & Possible H1 at date of exclusivity grant & Date of sale contract & If no consideration received for H1 event, capital gain may be nil, but H1 must still be assessed. \\
Pre-emptive right in lease (granted at lease commencement) & Date of lease commencement & Date of sale contract & H1 arose at lease commencement (possibly years ago). Confirm whether H1 was assessed at the time. If not, late amendment risk. \\
\bottomrule
\end{tabularx}
\end{table}

\subsubsection{CGT Discount Eligibility for H1}

The 50\% CGT discount (Div~115 ITAA~1997) is available for an H1 capital gain if the \textbf{right or option} has been held for at least 12 months before the CGT event. For most property transactions, the option is created at the time of grant and has been `held' since that moment --- therefore the discount is \textbf{not available} for H1 unless the option was granted more than 12 months before the grant itself (which is structurally impossible). In practice, \textbf{the 50\% CGT discount does not apply to H1 gains on a freshly granted option}, even if the underlying property has been owned for years.

\begin{critical}[CGT Discount Not Automatic on Event H1]
The CGT discount available on Event A1 (sale of the property) does NOT automatically apply to Event H1 (granting the option). These are separate CGT events with separate discount eligibility tests. The H1 gain (option fee) is likely fully assessable with no discount. The tax agent must assess each event independently.
\end{critical}

\subsubsection{Accounting Treatment of Option Fees and Deposits Received}

\begin{itemize}[nosep]
    \item \textbf{Option fee received (H1 trigger):} On receipt of the option fee, do \textbf{not} recognise as income immediately. Record as a \textbf{liability} (\acct{884} --- Other Creditors or a separately labelled account) until the accounting treatment is confirmed by the tax agent. The option fee is capital in nature for tax --- it is not revenue income. For accounting under AASB~140, it is not rental income; it forms part of the eventual gain on disposal. Do not post to P\&L income accounts.
    \item \textbf{10\% sale deposit received:} The \$1,325,000 currently held in \acct{680} (St George) matched by \acct{884} liability is the sale deposit, not an option fee. Retain as a liability until settlement. At settlement, the deposit is applied against the sale proceeds and \acct{884} is cleared against the gain on disposal.
    \item \textbf{If the option lapses:} The forfeited option fee becomes income or a capital receipt depending on its character. Seek tax agent advice before reclassifying to the P\&L.
\end{itemize}

\subsubsection{SOP Controls for H1 --- Notification and Recording Protocol}

\begin{enumerate}[nosep]
    \item \textbf{On any grant of an option, exclusivity right, or pre-emptive right:} Notify the tax agent on the \textbf{same day} as the grant. Provide a copy of the option deed or heads of agreement immediately.
    \item \textbf{Determine the consideration:} What was the option fee or premium paid? Even \$1 creates an H1 event. Record the receipt in \acct{884} (liability) --- not P\&L income.
    \item \textbf{Confirm the H1 year:} The tax agent must confirm which FY the H1 event falls in and whether it is reportable separately from A1.
    \item \textbf{Track costs of granting the option:} Legal fees directly incurred in drafting the option deed are costs of granting --- these reduce the H1 gain. Segregate from general sale costs which are A1 Element~5 costs.
    \item \textbf{Monitor for exercise or lapse:} Record the outcome (exercise or lapse) and advise the tax agent. If exercised, the s134-1 adjustment mechanism applies. If lapsed, the H1 gain is finalised.
    \item \textbf{Review existing leases for pre-emptive rights:} Review all leases in the fund (5Point Projects, Popgun Labs, Sony Music) for any right of first refusal or pre-emptive purchase right. If such rights exist and have not previously been assessed for H1, advise the tax agent.
\end{enumerate}

\begin{warning}[Most Common H1 Error — Option Fee Misrecorded]
\textbf{The most common H1 error:} Practitioners treat the option fee as a deposit or income, record it incorrectly in the P\&L, and fail to notify the tax agent of the H1 event. The consequence is that the H1 capital gain is not reported in the correct FY, penalty interest accrues on the shortfall, and the s134-1 adjustment in the A1 year is not made, resulting in double-counting of the option consideration. Establish same-day notification as a non-negotiable SOP requirement.
\end{warning}

% ============================================================================

\subsection{Cut-Off: Late Invoices, Post-Settlement Deductibility}
\label{sec:est-cutoff}
\legbox{TR~97/7 (``incurred) \legalcite{TR~97/7}{sec:ref-income-tax}; s8-1 ITAA~1997 \legalcite{s8-1}{sec:ref-income-tax}. Case law: \textit{Carden's Case}; \textit{New Zealand Flax Investments}; \textit{Nilsen Development Laboratories} \legalciteonly{see \S\,\ref{sec:ref-income-tax}}.}

Cut-off errors --- recording expenses in the wrong income year --- are among the most consequential tax errors in property fund accounting. The primary ATO authority is \textit{Taxation Ruling TR~97/7 --- Income tax: the meaning of `incurred'} (consolidated 2 July 2025, incl.\ TR~97/7A2 addendum). All cut-off decisions should be cross-referenced to Section~\ref{sec:est-cgt} for CGT implications.

\subsubsection{The `Incurred' Principle --- Not `Invoiced' or `Paid'}

For income tax, an outgoing is deductible when it is \textbf{`incurred'} --- meaning a definite legal obligation to pay has arisen --- not when an invoice is received or when cash is paid (s8-1 ITAA~1997 for general deductions). An accrual must be recorded at balance date for obligations that have arisen regardless of whether an invoice has been received.

Key principles from TR~97/7 and case law: the liability must be `definitively committed' (\textit{Carden's Case 1938}); a present, existing, and certain liability is required, though the precise amount may be estimated (\textit{New Zealand Flax Investments}); an unconditional obligation to pay is sufficient --- the liability need not be currently due for payment (\textit{Nilsen Development Laboratories 1981}).

\begin{note}[Accrual Cut-Off — Incurred Date, Not Invoice Date]
If the fund receives a service in June (financial year-end) but the invoice is not received until August, the expense is still `incurred' in June provided the obligation to pay existed as at 30 June. An accrual must be recorded at year-end. The late invoice merely confirms the quantum --- it does not determine the deduction year.
\end{note}

\subsubsection{Post-Settlement Invoice Register}

Maintain a \textbf{Post-Settlement Invoice Register} for a minimum of 60--90 days following settlement. For each invoice received after settlement, record and assign a tax bucket before posting. Available buckets:

\begin{itemize}[nosep]
    \item \textbf{Repair deduction (s25-10):} Genuinely a repair during the income-producing period.
    \item \textbf{Div~40 depreciating asset / balancing adjustment:} Plant and equipment disposed with the property --- balancing adjustment applies.
    \item \textbf{Div~43 capital works:} Structural improvements --- 2.5\% p.a.
    \item \textbf{Borrowing expense (s25-25):} Loan establishment or break costs.
    \item \textbf{Blackhole (s40-880):} Business-related capital, 5-year write-off.
    \item \textbf{CGT cost base / selling cost:} Incidental disposal costs (Element~5) --- not deductible as P\&L; reduces capital gain.
    \item \textbf{Other deductible (s8-1):} Ordinary business expense with income nexus.
\end{itemize}

\begin{warning}[No Double-Benefit — Deduction Cannot Also Enter Cost Base]
\textbf{No-double-benefit check (mandatory):} If an amount is deducted under any provision, confirm it is NOT also added to the CGT cost base. If Div~43 is claimed, confirm the cost base reduction is made. This is a recurring audit error. See also the double-count prevention guidance in Section~\ref{sec:est-double-count}.
\end{warning}

\subsubsection{Late Invoice Scenarios}

\textbf{Scenario A --- Work completed in prior FY, invoice arrives in current FY:} The expense belongs in the prior FY. Accrue at 30 June (DR Expense / CR Accrued Liabilities). When the invoice arrives, compare to accrual: if immaterial $\to$ adjust in current FY; if material $\to$ consider amendment to prior FY return. Retain: signed contract or PO dated before 30 June; email confirming completion before 30 June; practical completion certificate; delivery docket.

\textbf{Scenario B --- Work completed after year-end, invoice backdated:} If the obligation was not incurred until after 30 June, the expense belongs in the new FY regardless of invoice dating. Always verify actual completion date against physical evidence (practical completion certificate, on-site photos with timestamp metadata).

\begin{warning}[Contractor Invoice Dating — Use Completion Date, Not Start Date]
Contractors sometimes date invoices to the commencement date rather than the completion date, creating a false impression that the expense belongs in the prior FY. Always verify against the actual completion date.
\end{warning}

\textbf{Scenario C --- Post-settlement late invoices:} Often deductible if the liability was incurred while the property was held for income-producing purposes. Use the decision tree in Section~\ref{sec:est-postsale-decision}. Key post-settlement deduction options where income nexus has ceased: s25-25 immediate write-off of unamortised borrowing costs upon full repayment of the St George facility (accounts \acct{303.03}/\acct{303.04}); s40-880 blackhole expenditure for wind-up and compliance costs; CGT cost base Element~5 for costs required by the sale contract.

\begin{table}[H]
\centering\small\sffamily
\caption{Cut-off checklist --- late invoice risk management}
\begin{tabularx}{\textwidth}{@{}C{0.6cm} L{5cm} L{4cm} L{2.5cm}@{}}
\rowcolor{navy}\textcolor{white}{\bfseries\#} & \textcolor{white}{\bfseries Check} & \textcolor{white}{\bfseries Evidence Required} & \textcolor{white}{\bfseries Owner} \\
\midrule
1 & At 30 June: identify all WIP where invoice not yet received --- accrue estimated amount & Contract, WIP report, email confirming completion & Preparer \\
\rowcolor{warmgray}
2 & For each late invoice (July--Aug): confirm whether work completed before or after 30 June & Practical completion certificate, delivery docket, timestamped photos & Preparer \\
3 & For invoices spanning year-end: apportion between FYs based on \% completion or days & Project schedule, invoice breakdown & Preparer / Reviewer \\
\rowcolor{warmgray}
4 & Post-settlement invoice: assign to one of 7 tax buckets before posting & Register template --- bucket must be documented with rationale & Preparer \\
5 & No-double-benefit: if deducted, confirm NOT also in cost base; if Div~43 claimed, confirm cost base adjustment made & Tax reconciliation workpaper & Reviewer \\
\rowcolor{warmgray}
6 & If amendment required for material prior-FY error: document and action & Amended ITR, ATO acknowledgment & Tax Agent \\
\bottomrule
\end{tabularx}
\end{table}

\subsubsection{Post-Sale Invoice Deductibility Decision Tree}
\label{sec:est-postsale-decision}

Apply the following to every invoice received after settlement:

\begin{enumerate}[nosep]
    \item \textbf{Was the legal obligation definitively committed (TR~97/7) before income-producing use ceased (i.e., before settlement)?} If yes $\to$ deductible under s8-1 in the period incurred (pre-sale). Record as accrual; confirm quantum via invoice.
    \item \textbf{Is the cost a borrowing expense under s25-25?} (e.g., break fee, residual establishment cost) $\to$ immediate write-off if loan fully repaid (s25-25(5)).
    \item \textbf{Is the cost an incidental disposal cost?} (related to the CGT event, required by contract, legal fees of title transfer) $\to$ add to CGT cost base Element~5. Reduces capital gain. Not a P\&L deduction.
    \item \textbf{Is it a business-related capital cost not in CGT cost base and not otherwise deductible?} (wind-up compliance costs) $\to$ deductible under s40-880 over 5 years.
    \item \textbf{None of the above:} The cost is \textbf{non-deductible}. Record as non-deductible in the tax reconciliation. No P\&L expense, no cost base addition.
\end{enumerate}

\begin{critical}[CGT Event Date — Greatest Cut-Off Risk for This Fund]
The single greatest cut-off risk for this fund is the CGT event date. If the sale contract is executed close to 30 June, a one-day difference determines which FY the capital gain is assessed in, which income year all capex and repair invoices must be allocated to, and when s25-25 borrowing cost write-off occurs. Establish a \textbf{CGT event date notification protocol}: the tax agent receives same-day notification of exchange.
\end{critical}

% ============================================================================

\subsection{Post-Settlement Obligations and Benefits}
\label{sec:est-postsettlement}
\legbox{TR~95/25 (interest post-cessation) \legalcite{TR~95/25}{sec:ref-income-tax}; s104-10 ITAA~1997 (A1 event date) \legalcite{s104-10}{sec:ref-cgt}; s25-25(5) ITAA~1997 (write-off of borrowing costs) \legalcite{s25-25(5)}{sec:ref-borrowing}.}

Settlement of the property sale does not extinguish all obligations and benefits of the fund. Maintain a \textbf{`Post-Sale Obligations \& Benefits' schedule} for items remaining the fund's responsibility or benefit after settlement. Tax routing depends on whether each item is a deductible outgoing (nexus + `incurred') or a CGT disposal/ownership cost base or proceeds adjustment.

\begin{table}[H]
\centering\small\sffamily
\caption{Post-settlement obligations and benefits --- tax treatment reference}
\begin{tabularx}{\textwidth}{@{}L{3cm} L{3.5cm} L{4.5cm} L{2cm}@{}}
\rowcolor{navy}\textcolor{white}{\bfseries Item} & \textcolor{white}{\bfseries Nature and Timing} & \textcolor{white}{\bfseries Tax Treatment} & \textcolor{white}{\bfseries Account} \\
\midrule
Final outgoings adjustments (rates, land tax, water) per settlement statement & Vendor/purchaser split per settlement statement & Deductible in year incurred if vendor's share; CGT proceeds adjustment if purchaser's portion refunded & \acct{700.xx} / settlement adj. \\
\rowcolor{warmgray}
Tax agent fees --- CGT schedule, final FY return, wind-up & Post-settlement, relates to income-producing period & Deductible under s8-1 (relates to producing assessable income); or s40-880 if post-income-producing & \acct{882.16} \\
Accounting/audit fees --- final audit, financial statements & Part of fund wind-up & s40-880 blackhole 5-year; or s8-1 if deductible & \acct{882.16} \\
\rowcolor{warmgray}
K\&L Gates post-settlement legal fees (title, warranty) & Relates to sale obligations & Element~5 CGT cost base if warranty claim; otherwise s8-1 if income nexus & \acct{709}/\acct{882.16} \\
Make-good costs to tenants & Obligation crystallises on sale; accrue pre-settlement & Deductible under s8-1 if incurred in income-producing period; or Element~5 cost base addition & \acct{882.16}/\acct{709} \\
\rowcolor{warmgray}
Final land tax --- vendor's portion for year of sale & Apportioned in settlement statement & Deductible in year incurred & \acct{700.xx} \\
Unamortised borrowing costs (\acct{303.03}/\acct{303.04}) --- St George repaid at settlement & Immediately deductible upon full repayment (s25-25(5)) & Full write-off in settlement year --- do not continue amortising & \acct{303.03}/\acct{303.04} \\
\rowcolor{warmgray}
Cyclone insurance proceeds (Steadfast/Vision Re) received after settlement & Fund retains rights regardless of property sale & Assessable income in year received (s6-5); not CGT proceeds & \acct{73.02} \\
\bottomrule
\end{tabularx}
\end{table}

\textbf{Insurance considerations:} Notify the insurer immediately upon settlement and obtain a refund of any unexpired premium (DR Cash / CR Prepaid Insurance \acct{707}). The cyclone insurance claim belongs to the fund regardless of when settled --- ensure the claim is not assigned to the purchaser in the contract of sale. Maintain public liability cover until all pre-settlement claims are resolved or the limitation period has expired. Directors' and officers' liability cover should be maintained for a run-off period (typically 7 years) after fund wind-up.

\textbf{Final distribution and wind-up:} TFN withholding at 47\% applies to unitholders who have not provided their TFN. A final AMMA statement must be issued to all unitholders showing distribution character. Carried-forward capital losses must be offset against the capital gain before distribution. Review the trust deed for mandatory wind-up steps (final distribution, trustee resignation, ASIC deregistration) --- failure to follow the trust deed can result in the trustee having personal liability.

% ============================================================================
% ============================================================================

\subsection{Cut-Off Risks and Errors --- Risk Register}
\label{sec:est-cutoff-risks}

The following table identifies the most common and material cut-off and classification errors in property fund accounting near sale. All are recurring ATO audit targets. Cross-reference to Section~\ref{sec:est-cutoff} (late invoice procedure), Section~\ref{sec:est-capex-guide} (capex classification), and Section~\ref{sec:est-cgt} (CGT timing).

\textbf{High-risk error patterns:}
\begin{enumerate}[nosep]
    \item Repair deduction claimed for an initial repair --- should be capital (Div~40/43 or CGT cost base). Check TR~97/23 initial repair test against due diligence records.
    \item Entirety replacement treated as repair --- apply the `entirety' test per TR~97/23.
    \item CGT year set to settlement date instead of contract date --- CGT Event A1 timing is contract date.
    \item Div~43 claimed but CGT cost base not adjusted --- Div~43 claimed = cost base reduced, dollar-for-dollar (s110-45(4)).
    \item Late invoices not accrued to correct period --- `incurred' (TR~97/7) determines the deduction year, not the invoice date or payment date.
    \item Property not genuinely available for rent but holding costs still claimed --- deductibility requires actual renting or genuine availability; apportion for non-income periods.
    \item Div~40 balancing adjustment on disposal missed or proceeds not apportioned --- must apportion proceeds and calculate balancing adjustment for each depreciating asset.
    \item s25-25 amortisation continued after loan repaid --- immediate write-off required in year of full repayment (s25-25(5)).
    \item Selling costs deducted instead of added to CGT cost base --- agent commission, conveyancing, advertising are Element~5, not income deductions.
\end{enumerate}

\begin{table}[H]
\centering\small\sffamily
\caption{High-risk cut-off windows --- calendar for 22 Constance Street}
\begin{tabularx}{\textwidth}{@{}L{2.5cm} L{4.2cm} X@{}}
\rowcolor{navy}\textcolor{white}{\bfseries Risk Window} & \textcolor{white}{\bfseries Elevated Risks} & \textcolor{white}{\bfseries Required Actions} \\
\midrule
30 June FY cut-off & Cyclone invoices --- which FY? Borrowing cost amortisation --- is loan repaid before 30 June? Settlement if contract signed before 30 June & Full WIP review by 25 June; confirm with CPG: settlement date, loan repayment date, all cyclone invoices received and dated \\
\rowcolor{warmgray}
Day of exchange (contract date) & CGT event date established --- all costs after exchange potentially non-deductible or cost base. s25-25 write-off triggered on loan repayment & Same-day email to tax agent with contract date. Flag all in-progress works for cut-off assessment immediately \\
Settlement date & Final settlement statement apportionment. Insurance proceeds timing. Remaining borrowing costs write-off. Post-settlement late invoices commence & Reconcile settlement statement within 5 business days. Write off \acct{303.03}/\acct{303.04} immediately. Establish Post-Settlement Invoice Register \\
\rowcolor{warmgray}
31 October (FY return due) & All pre-settlement accruals confirmed. Late invoices assessed. CGT schedule finalised with Div~43 adjustment & Pre-lodgement checklist: all accruals vs.\ invoices, QS report available, CGT cost base calculation with Div~43 adjustment completed \\
\bottomrule
\end{tabularx}
\end{table}





% ============================================================================
% PART V: INSURANCE CLAIMS
% ============================================================================

\vspace{1.2cm}
\begin{tcolorbox}[
    enhanced, breakable,
    colback=navy, colframe=navy,
    boxrule=0pt,
    borderline south={2pt}{0pt}{gold},
    sharp corners,
    left=14pt, right=10pt, top=10pt, bottom=10pt,
    label={sec:est-part-insurance}
]
{\sffamily\bfseries\footnotesize\textcolor{gold}{Part V\enspace\textbullet\enspace Appendix~C}}\par
\vspace{3pt}
{\sffamily\bfseries\large\textcolor{white}{Insurance Claims}}
\end{tcolorbox}
\begin{tcolorbox}[
    enhanced, breakable,
    colback=palesky, colframe=navy!30!white,
    boxrule=0.4pt, sharp corners,
    left=10pt, right=10pt, top=6pt, bottom=6pt
]
\small\itshape This part provides comprehensive guidance on the cyclone insurance claim: accounting recognition, tax treatment by component, CGT cost base interaction, GST implications, post-sale retention of claim rights, and premium cut-off.
\end{tcolorbox}
\vspace{0.3cm}

\subsection{Insurance Claims --- Accounting, Tax, and CGT Treatment}
\label{sec:est-insurance}
\legbox{s6-5 ITAA~1997 (ordinary income --- lost rent and BI proceeds) \legalcite{s6-5}{sec:ref-insurance}; s20-20 to s20-30 ITAA~1997 (recoupments) \legalcite{s6-5}{sec:ref-insurance}; GSTR~2006/10 \legalcite{GSTR~2006/10}{sec:ref-insurance}; AASB~137 \legalcite{AASB~137}{sec:ref-insurance}. CGT: s104-20 (C1) \legalcite{s104-20}{sec:ref-cgt}; s104-25 (C2) \legalcite{s104-25}{sec:ref-cgt}.}

Insurance claim accounting is one of the most complex and error-prone areas for the 22 Constance Street fund because the cyclone insurance claim straddles multiple financial years, involves both income and capital components, creates interaction with the CGT cost base calculation, and continues to generate receipts potentially \textbf{after the property is sold}. This section consolidates all insurance-related accounting, tax, and CGT guidance and should be read alongside Section~\ref{sec:est-cgt} (CGT issues) and Section~\ref{sec:est-postsettlement} (post-settlement obligations).

\subsubsection{Overview --- Cyclone Insurance Claim (Steadfast IRS / Vision Re Partners)}

The fund holds an active cyclone insurance claim for damage sustained to 22 Constance Street. The insurer (Steadfast IRS / Vision Re Partners) is making progress payments directly into the St George account (\acct{680}). As at January 2026, YTD receipts total \$214,136.10 (September 2025: \$160,602; November 2025: \$10,707; December 2025: \$42,827.10). Further payments are expected.

The claim has two distinct components that must be tracked and accounted for separately:

\begin{itemize}[nosep]
    \item \textbf{Indemnity for lost rent / business interruption:} Compensation for rental income lost as a result of the cyclone damage causing uninhabitable or unleased space. This component is \textbf{ordinary income} and is assessed as assessable income in the year received.
    \item \textbf{Indemnity for physical damage / reinstatement:} Compensation for the cost of reinstating the building fabric (structural damage, HVAC, fire systems, fitout). The correct treatment of this component depends on whether the related repair costs were expensed or capitalised --- see Section~\ref{sec:est-insurance-taxtreatment} below.
\end{itemize}

\subsubsection{Accounting Recognition --- When to Recognise Insurance Proceeds}

Under AASB~137, insurance proceeds are recognised as an asset (receivable) \textbf{only when it is virtually certain that the insurer will pay}. The recognition threshold is high:

\begin{itemize}[nosep]
    \item \textbf{Do not accrue insurance income until the insurer has provided written acceptance of the claim and a determinable amount.} A verbal indication, an adjuster's estimate, or a solicitor's letter is insufficient for recognition. The trigger is a written settlement letter or a formal progress payment advice from the insurer.
    \item \textbf{When the insurer issues a progress payment advice or makes a direct credit to \acct{680}:} Recognise the income at that point --- DR \acct{680} St George / CR \acct{73.02} Insurance Claim Income (P\&L Other Income).
    \item \textbf{Do not accrue more than the confirmed amount.} Even if the fund expects a much larger settlement, only recognised amounts (confirmed in writing) may be accrued. The remainder stays as a contingent asset (disclosed in notes, not recognised).
    \item \textbf{Post-settlement receipts:} If insurance proceeds are received after the property has been sold, they are still the fund's income (the claim belongs to the fund, not the purchaser --- confirm this in the sale contract). Recognise in \acct{73.02} in the period received. See Section~\ref{sec:est-insurance-postsale} for CGT interaction.
\end{itemize}

\begin{warning}[Double-Count Risk — Check Yardi Before Posting Insurance Receipt]
\textbf{Double-count risk:} Check the Yardi TB before posting any insurance claim receipt. CPG may have already recorded the same receipt in Yardi (under \acct{73.02} or another income account). Posting the UFS manual journal without checking Yardi first will double-count the income. See Section~\ref{sec:est-double-count} for the double-count prevention procedure.
\end{warning}

\subsubsection{Tax Treatment of Insurance Proceeds --- Component Analysis}
\label{sec:est-insurance-taxtreatment}
\legbox{s6-5 ITAA~1997 \legalcite{s6-5}{sec:ref-insurance}; s20-20 to s20-30 ITAA~1997 (recoupments) \legalcite{s6-5}{sec:ref-insurance}; GSTR~2006/10 \legalcite{GSTR~2006/10}{sec:ref-insurance}; ATO~ID~2004/966 \legalcite{ATO~ID~2004/966}{sec:ref-insurance}.}

The tax treatment of insurance proceeds depends entirely on the character of the loss being indemnified. The ATO's position (derived from s6-5 and ss20-20 to 20-30 ITAA~1997, and confirmed in ATO ID~2004/966 and the recoupment provisions) is that the character of the insurance proceeds follows the character of the underlying loss:

\begin{table}[H]
\centering\small\sffamily
\caption{Tax treatment of insurance proceeds by component}
\label{tab:insurance-tax}
\begin{tabularx}{\textwidth}{@{}L{2.8cm} L{3cm} L{2.8cm} X@{}}
\rowcolor{navy}\textcolor{white}{\bfseries Component of Claim} & \textcolor{white}{\bfseries Character of Underlying Loss} & \textcolor{white}{\bfseries Tax Treatment} & \textcolor{white}{\bfseries Account / Notes} \\
\midrule
Lost rent / business interruption & Revenue loss (income forgone) & Assessable ordinary income (s6-5 ITAA 1997) in year received & \acct{73.02}. Separate from building indemnity. \\
\rowcolor{warmgray}
Reinstatement of structural damage (Div~43 scope) & Capital loss (structural asset damaged) & Reduces the Div~43 cost base of the reinstated works. Proceeds reduce the deductible cost of Div~43 (offset against cost base, not income) & Track separately. Reduces the cost base of the Div~43 capital works schedule. \\
Reinstatement of plant and equipment (Div~40 scope) & Capital loss (depreciating asset damaged or destroyed) & CGT Event C1 (loss/destruction) may apply (s104-20). Proceeds are `termination value' for the destroyed asset's balancing adjustment (s40-295). & Calculate Div~40 balancing adjustment for each destroyed asset. Proceeds reduce the loss. \\
\rowcolor{warmgray}
Temporary repairs (tarping, emergency waterproofing) reimbursed & Revenue loss (expense reimbursed) & Assessable income (reduces the net cost of the deductible repair expense) in year received. & \acct{73.02}. \\
Consequential costs (professional fees, project management for claim) reimbursed & Revenue loss (expense reimbursed) & Assessable income in year received, to the extent the underlying costs were deducted. & \acct{73.02}. \\
Total loss of a building or CGT asset & Capital loss (CGT asset destroyed) & CGT Event C1 (s104-20): proceeds are capital proceeds for the C1 event. Does not apply to partial damage. & Not relevant here (partial damage only). \\
\bottomrule
\end{tabularx}
\end{table}

\begin{note}[Partial Cyclone Loss — CGT Event C1 Does Not Apply]
For 22 Constance Street, the cyclone damage was \textbf{partial} (not total loss), so CGT Event C1 (loss/destruction of a CGT asset) does not apply to the property itself. However, CGT Event C1 \textbf{may} apply to individual Div~40 plant and equipment assets that were destroyed (e.g.\ if the Ampac fire detection system was destroyed rather than damaged). Confirm with the tax agent whether the Ampac replacement represents a destruction/replacement of the old asset (C1 + A1 for Div~40) or a repair.
\end{note}

\subsubsection{CGT Interaction --- Insurance Proceeds and the Cost Base}
\label{sec:est-insurance-cgt}

Insurance proceeds interact with the CGT cost base calculation in two important ways:

\paragraph{1. Proceeds for reinstated capital works reduce Element 4 cost base}

Where insurance proceeds are received as compensation for damage to structural capital works (items that would normally be classified as Div~43), the ATO's position is that the proceeds effectively reduce the net cost of reinstatement. The CGT cost base consequence is:

\begin{itemize}[nosep]
    \item The gross reinstatement cost is added to the CGT cost base (Element~4) as a capital enhancement cost.
    \item The insurance proceeds received for that component \textbf{reduce the Element~4 addition}, because the fund's net out-of-pocket cost is reduced by the insurance recovery.
    \item If Div~43 deductions are also claimed on the reinstated works, those further reduce the cost base under s110-45(4).
    \item Net CGT cost base addition = Gross reinstatement cost $-$ Insurance proceeds received + Div~43 deductions claimed.
\end{itemize}

\paragraph{2. Proceeds received after settlement --- NOT capital proceeds from the property sale}

Insurance proceeds received after the property has been sold are \textbf{not} capital proceeds from CGT Event A1 (disposal of the property). They are ordinary income receipts (or, in specific circumstances, proceeds from a separate CGT event relating to the insurance claim itself). Key points:

\begin{itemize}[nosep]
    \item \textbf{They do not increase the capital proceeds from A1:} The capital proceeds from A1 are fixed at the contract price plus settlement adjustments. Post-settlement insurance receipts are a separate income stream.
    \item \textbf{They do not increase the CGT cost base:} The cost base is fixed as at the CGT event date (contract date). Insurance proceeds received after that date cannot retroactively increase the cost base.
    \item \textbf{They are assessable income under s6-5 ITAA~1997} in the year received, regardless of when the damage occurred.
    \item \textbf{Potential CGT Event C2 (cancellation of a right):} If the insurance claim is settled by extinguishing the contractual right to claim (e.g.\ a lump-sum settlement in full and final settlement of all claims), a CGT Event C2 may arise (s104-25). The right to claim under the insurance policy is itself a CGT asset. Seek specific advice if a global settlement is proposed.
\end{itemize}

\begin{table}[H]
\centering\small\sffamily
\caption{Insurance proceeds --- CGT cost base and income interaction summary}
\begin{tabularx}{\textwidth}{@{}L{3cm} L{2.8cm} L{2.8cm} X@{}}
\rowcolor{navy}\textcolor{white}{\bfseries Proceeds Type} & \textcolor{white}{\bfseries Income Tax} & \textcolor{white}{\bfseries CGT Cost Base Impact} & \textcolor{white}{\bfseries Key Rule} \\
\midrule
Lost rent / BI proceeds (pre-settlement) & Assessable s6-5 in year received & No cost base effect & Revenue character \\
\rowcolor{warmgray}
Structural reinstatement (pre-settlement) & Reduces net cost of Div~43 capital works & Reduces Element~4 addition; Div~43 deductions further reduce & Net cost = gross capex $-$ insurance proceeds \\
Plant reinstatement (Div~40, pre-settlement) & Termination value for balancing adj. & Div~40 balancing adjustment; reduced gain or increased loss & s40-295 balancing event \\
\rowcolor{warmgray}
All proceeds post-settlement & Assessable s6-5 in year received & No impact on A1 cost base (fixed at contract date) & Not capital proceeds from A1 \\
Lump-sum full and final settlement & Partly income, partly CGT C2? & Seek advice & s104-25 CGT Event C2 risk \\
\bottomrule
\end{tabularx}
\end{table}

\subsubsection{GST on Insurance Proceeds}

The GST treatment of insurance proceeds is nuanced and frequently misunderstood:

\begin{itemize}[nosep]
    \item \textbf{Proceeds paid directly by the insurer to the insured (fund):} Generally \textbf{input-taxed financial supplies} --- the fund does \textbf{not} charge GST on the proceeds received, but the fund \textbf{may} be required to reduce its input tax credit (ITC) entitlement on the underlying remediation costs to the extent they are funded by insurance.
    \item \textbf{ATO's `reduced ITC' mechanism:} If the insurer pays the fund's remediation costs directly (or reimburses them), the fund's ITC entitlement on those costs may be reduced because the insurer (not the fund) effectively bears the cost. The ATO applies GSTR~2006/10 (insurance settlements and entitlement to input tax credits; updated 14~February~2026) for this analysis.
    \item \textbf{Where the insurer pays the contractor directly (not the fund):} If the insurer pays FHS directly for remediation, the fund may never have incurred the GST-inclusive cost --- ITCs may not be claimable on amounts paid directly to contractors by the insurer.
    \item \textbf{Practical step:} For each insurance payment received, confirm with the tax agent: (a) whether the fund's ITC position on underlying remediation costs must be adjusted; (b) whether any GST is payable on the proceeds (generally no); (c) how the proceeds interact with the quarterly BAS.
\end{itemize}

\subsubsection{Post-Sale Insurance Claim --- Retention of Rights}
\label{sec:est-insurance-postsale}

The fund retains the right to the cyclone insurance claim proceeds even after the property is sold. However, this must be \textbf{explicitly protected in the sale contract}. Key actions:

\begin{enumerate}[nosep]
    \item \textbf{Sale contract review:} Confirm with K\&L Gates that the sale contract contains an express provision that: (a) the fund retains all rights to the cyclone insurance claim; (b) the claim is not assigned, novated, or transferred to the purchaser; (c) the purchaser acknowledges that any insurance proceeds received after settlement belong to the fund (not the property).
    \item \textbf{Insurer notification:} Notify Steadfast IRS / Vision Re Partners of the property sale and confirm that claim payments will continue to be directed to the fund (or its nominated account) after settlement. Update the insurer's records with post-settlement fund contact and banking details.
    \item \textbf{Post-settlement receipt protocol:} Proceeds received after settlement are posted to \acct{73.02} (assessable income) in the period received. The fund's bank details (previously tied to St George \acct{680}) must be updated --- direct post-settlement proceeds to \acct{685} NAB or a new fund account designated for wind-up period.
    \item \textbf{Tax return filing obligation:} Even after the property is sold, the fund must continue to lodge income tax returns if it is receiving insurance income (assessable). Winding up the fund before all insurance proceeds are received and declared creates a compliance risk. Coordinate the wind-up timeline with the insurance settlement timeline.
    \item \textbf{Unresolved claims at wind-up:} If the insurance claim is not fully settled by the time the fund is wound up, specific steps are required to assign the remaining claim rights to the unitholders or to ensure the trustee retains capacity to collect and distribute the proceeds. Seek trust deed and legal advice.
\end{enumerate}

\begin{critical}[Do Not Cancel Insurance Policy Immediately After Settlement]
\textbf{Do not cancel the insurance policy immediately after settlement.} The cyclone claim is an \textit{in-flight claim} --- it was lodged before settlement and remains the fund's claim post-settlement. Cancelling the insurance policy after settlement \textbf{does not extinguish the in-flight claim}, but the fund must ensure the policy is not cancelled in a way that allows the insurer to dispute the claim's validity. Confirm the policy continuation terms for in-flight claims with the broker before settlement.
\end{critical}

\subsubsection{Insurance Premium Prepayment and Cut-Off}

Insurance premium expenses are paid via NAB (\acct{685}) and recorded in \acct{706.07} / \acct{707} as prepayments, amortised monthly. On settlement:

\begin{itemize}[nosep]
    \item The unexpired portion of the premium is a refund receivable from the insurer. Record: DR \acct{662} Receivable / CR \acct{707} Prepaid Insurance (reversing the unamortised balance).
    \item When the refund is received: DR \acct{685} NAB / CR \acct{662} Receivable.
    \item The refunded premium is \textbf{not} P\&L income --- it reverses a prepaid asset and reduces the insurance expense for the period.
    \item If the policy continues post-settlement (for the in-flight claim), apportion the premium: the portion relating to the post-settlement period in which no rental income is derived may not be deductible under s8-1 (no income nexus) but may qualify under s40-880 as a business-related cost. Seek tax agent advice.
\end{itemize}

\subsubsection{Monthly Insurance Checklist}

\begin{table}[H]
\centering\small\sffamily
\caption{Monthly insurance-related sign-off items}
\begin{tabularx}{\textwidth}{@{}C{0.5cm} L{8cm} L{2.5cm} L{2cm}@{}}
\rowcolor{navy}\textcolor{white}{\bfseries\#} & \textcolor{white}{\bfseries Check} & \textcolor{white}{\bfseries Owner} & \textcolor{white}{\bfseries Timing} \\
\midrule
1 & Check St George (\acct{680}) for any new insurance claim credit from Steadfast / Vision Re; record in \acct{73.02} with remittance advice support & Preparer & Monthly \\
\rowcolor{warmgray}
2 & Confirm no Yardi double-count: search for any CPG entry in \acct{73.02} for the same period before posting UFS journal & Preparer & Monthly \\
3 & Confirm insurance claim component split (lost rent vs.\ structural reinstatement vs.\ temp repairs) from insurer / loss adjuster & Preparer / Tax Advisor & As received \\
\rowcolor{warmgray}
4 & Review ITC position: for each insurer payment, confirm with tax agent whether the fund's GST ITC on underlying remediation costs must be adjusted & Tax Advisor & Quarterly \\
5 & Confirm insurer has not introduced any condition or notice that could affect claim continuity (e.g.\ policy terms, subrogation) & Preparer / CPG & Monthly \\
\rowcolor{warmgray}
6 & Update CGT cost base workpaper: reduce Element~4 by the insurance proceeds allocated to structural capital works & Preparer & As received \\
7 & Pre-settlement: confirm sale contract expressly retains insurance claim rights for the fund post-settlement & Legal (K\&L Gates) & Before exchange \\
\rowcolor{warmgray}
8 & Post-settlement: notify insurer of property sale; update payment direction to post-settlement fund account & Preparer / CPG & Day of settlement \\
9 & Post-settlement: continue lodging income tax returns while claim income is outstanding; coordinate wind-up timeline with claim timeline & Tax Agent & Ongoing \\
\bottomrule
\end{tabularx}
\end{table}



% ============================================================================


% ============================================================================
% PART VI: TRUST ADMINISTRATION, DISTRIBUTIONS, AND COMPLIANCE
% ============================================================================

\vspace{1.2cm}
\begin{tcolorbox}[
    enhanced, breakable,
    colback=navy, colframe=navy,
    boxrule=0pt,
    borderline south={2pt}{0pt}{gold},
    sharp corners,
    left=14pt, right=10pt, top=10pt, bottom=10pt,
    label={sec:est-part-trust}
]
{\sffamily\bfseries\footnotesize\textcolor{gold}{Part VI\enspace\textbullet\enspace Appendix~C}}\par
\vspace{3pt}
{\sffamily\bfseries\large\textcolor{white}{Trust Administration, Distributions, and Compliance}}
\end{tcolorbox}
\begin{tcolorbox}[
    enhanced, breakable,
    colback=palesky, colframe=navy!30!white,
    boxrule=0.4pt, sharp corners,
    left=10pt, right=10pt, top=6pt, bottom=6pt
]
\small\itshape This part covers trust-level administration: the statutory trust account framework, distribution component determination, distribution waterfall mechanics, withholding tax obligations, unitholder management, and registry services.
\end{tcolorbox}
\vspace{0.3cm}

\subsection{STA QLD (State Trust Account) --- Legal, Accounting, Tax and Management Obligations}
\label{sec:est-sta}
\legbox{\textit{Property Occupations Act 2014} (Qld) \legalciteonly{see \S\,\ref{sec:ref-wht}}; s8-1 ITAA~1997 (deductibility of STA costs) \legalcite{s8-1}{sec:ref-income-tax}.}

\acct{686} STA QLD is the \textbf{Queensland Property Management Trust Account} maintained by CPG (the property manager) under the \textit{Property Occupations Act 2014} (Qld) (formerly the \textit{Property Agents and Motor Dealers Act 2000}). It is a statutorily regulated account: CPG is legally required to hold all money received in connection with managing the property on behalf of the fund (the principal) in a dedicated trust account, segregated from CPG's own funds. The balance represents \textbf{money belonging to the fund} (or to third parties such as tenants, for bonds) that CPG holds in a fiduciary capacity.

\subsubsection{Legal Framework --- Queensland Regulated Trust Account}

The \textit{Property Occupations Act 2014} (Qld) imposes strict obligations on property managers holding client money:

\begin{itemize}[nosep]
    \item \textbf{Segregation:} All money received on behalf of a principal must be deposited into a separate trust account and must not be commingled with the property manager's own operating funds. CPG maintains the STA QLD specifically for this purpose.
    \item \textbf{Authorised financial institution:} The trust account must be maintained with an approved ADI (authorised deposit-taking institution). CPG's trust account is held with a QLD-licensed ADI (confirm with CPG).
    \item \textbf{Beneficiary's money:} The fund is the beneficial owner of the funds in the STA to the extent they represent the fund's rental income, outgoings recoveries, or other property receipts. Tenant bonds/security cash (if any) are held in trust for tenants until legitimately applied.
    \item \textbf{Disbursement obligations:} CPG may only disburse funds from the STA in accordance with the management agreement and the principal's instructions. Unauthorised disbursements constitute a breach of trust and a statutory offence.
    \item \textbf{Monthly statement and remittance:} CPG is required to account to the fund regularly (typically monthly) by providing a statement of receipts and disbursements, and remitting the net balance to the fund's account (NAB \acct{685}) after deducting authorised expenses.
    \item \textbf{Audit:} The trust account must be audited by a licensed auditor annually (or more frequently if required). The auditor must certify the trust account records are accurate and that no shortfalls or unauthorised withdrawals have occurred. The fund should receive confirmation from CPG that the STA audit has been completed each year.
    \item \textbf{CPG insolvency risk:} If CPG becomes insolvent, the money in the STA is protected because it is held in trust (not on CPG's balance sheet). However, there are practical risks if the trust account has a shortfall due to CPG error or misconduct. The Queensland Government maintains a \textbf{Property Occupations Fund} (a statutory compensation fund) that may provide relief for losses from trust account deficiencies.
\end{itemize}

\subsubsection{What Flows Through the STA QLD}

The STA QLD is the \textbf{first stop} for all property-level cash receipts before they reach the fund's NAB account. The following flows are normal:

\begin{table}[H]
\centering\small\sffamily
\caption{STA QLD --- typical receipts and disbursements}
\begin{tabularx}{\textwidth}{@{}L{4cm} L{3cm} X@{}}
\rowcolor{navy}\textcolor{white}{\bfseries Item} & \textcolor{white}{\bfseries Direction} & \textcolor{white}{\bfseries Notes} \\
\midrule
Tenant rent payments (5Point, Popgun, Sony) & Receipt into STA & Tenants direct-credit rent to CPG's trust account. CPG then remits net to fund's NAB. \\
\rowcolor{warmgray}
Outgoings recoveries (utilities, rates, body corp) & Receipt into STA & Collected from tenants as part of gross rent or separate levy. \\
Property operating expenses (repairs, cleaning, maintenance) & Disbursement from STA & CPG pays property-level operating invoices directly from STA. These appear in Yardi as account \acct{700} costs. \\
\rowcolor{warmgray}
CPG management fee & Disbursement from STA & CPG deducts its management fee (5\% of net rental income) directly from STA before remitting to NAB. \\
Net remittance to fund & Disbursement from STA to NAB \acct{685} & Monthly net remittance: gross receipts minus direct disbursements. This is the primary NAB inflow. \\
\rowcolor{warmgray}
Security deposits (cash bonds, if any) & Receipt into STA (held for tenant) & Held in trust for tenants; cannot be disbursed without legitimate grounds (breach, make-good, etc.). \\
GST collected on rent & Receipt into STA & GST is collected by CPG in STA and ultimately remitted via BAS. Must reconcile with \acct{255}. \\
\rowcolor{warmgray}
Sundry income (carpark, storage, licence fees) & Receipt into STA & All miscellaneous property income flows through STA before remittance to NAB. \\
\bottomrule
\end{tabularx}
\end{table}

\subsubsection{Accounting Treatment --- UFS Recording of STA QLD}

\paragraph{Balance sheet presentation}

The \acct{686} STA QLD balance represents the fund's \textbf{cash held in CPG's trust account} at a point in time. It is presented as a cash and cash equivalent on the fund's balance sheet (alongside \acct{685} NAB and \acct{680} St George). At month-end, this balance should represent the net amount in the STA that is owed to the fund (i.e., rental income collected but not yet remitted to NAB, less expenses paid but not yet charged back).

\paragraph{How Yardi captures STA movements}

All property-level receipts and disbursements are recorded by CPG in Yardi. When the Yardi TB is imported into the fund's workpapers:
\begin{itemize}[nosep]
    \item Rent income (\acct{71}) and property operating expenses (\acct{700}) are automatically captured from the Yardi TB.
    \item The \textbf{net} effect of these transactions flows through \acct{686} STA QLD in the TB.
    \item The STA balance represents the timing difference between when income/expenses are recognised in Yardi (accrual) and when the cash is actually remitted to the fund's NAB account (\acct{685}).
    \item When CPG remits the net balance to NAB: DR \acct{685} NAB / CR \acct{686} STA QLD (or the movement is already captured in the Yardi import).
\end{itemize}

\paragraph{Expected STA QLD balance}

The STA QLD balance should typically be low or nil at month-end if CPG remits frequently (weekly or monthly). A persistently elevated STA balance (e.g., several months of net rent not remitted) may indicate:
\begin{enumerate}[nosep]
    \item CPG has not yet remitted the period's net income to the fund.
    \item Yardi has not been updated to reflect a remittance that has already occurred.
    \item A discrepancy between what CPG holds and what the TB records.
    \item In the worst case, CPG is delaying remittances (cash management issue).
\end{enumerate}

The January 2026 balance was nil --- which is expected given the fund is in sale mode and CPG is likely remitting monthly with minimal operational activity.

\paragraph{Key journal entries}

\begin{table}[H]
\centering\small\sffamily
\caption{STA QLD common journal entries (post-Yardi import)}
\begin{tabularx}{\textwidth}{@{}L{4.5cm} L{3.5cm} X@{}}
\rowcolor{navy}\textcolor{white}{\bfseries Scenario} & \textcolor{white}{\bfseries Journal} & \textcolor{white}{\bfseries Notes} \\
\midrule
Tenant pays rent into STA; CPG records in Yardi & DR \acct{686} / CR \acct{71} & Captured in Yardi import. STA increases. \\
\rowcolor{warmgray}
CPG pays maintenance from STA; records in Yardi & DR \acct{700} / CR \acct{686} & Captured in Yardi import. STA decreases. \\
CPG remits net to NAB (\acct{685}) & DR \acct{685} / CR \acct{686} & Either via Yardi import or UFS manual journal if not in Yardi. \\
\rowcolor{warmgray}
CPG deducts management fee before remitting & DR \acct{301.01} / CR \acct{686} & Management fee is charged in Yardi; reduces STA balance. \\
Month-end STA balance = timing difference & \acct{686} debit balance & Represents cash owed to fund, not yet remitted. Verify against CPG trust account statement. \\
\rowcolor{warmgray}
Month-end STA \textit{credit} balance (unusual) & \acct{686} credit balance & Fund owes money to CPG trust (e.g., expenses paid from STA exceed receipts). Investigate immediately. \\
\bottomrule
\end{tabularx}
\end{table}

\begin{critical}[Credit Balance in STA QLD Account — Investigate Immediately]
\textbf{A credit balance in \acct{686} STA QLD is abnormal} and must be investigated immediately. It implies that CPG has disbursed more from the trust account on the fund's behalf than it has received in rent and other income for that period. This could indicate expenses paid but not yet reimbursed, a misclassification, or a data error in the Yardi import. Never carry a credit \acct{686} balance to the next period without explanation.
\end{critical}

\subsubsection{Tax Treatment of the STA QLD}

\paragraph{Interest on the trust account}

Under the \textit{Property Occupations Act 2014} (Qld), interest earned on a property management trust account is typically remitted to the \textbf{State government} (Queensland Revenue Office) and is NOT the income of either CPG or the fund. This is a distinctive feature of Queensland-regulated trust accounts: the interest `belongs' to the state statutory fund and is not distributable to the principal (the fund).

\begin{itemize}[nosep]
    \item The fund does \textbf{not} recognise any interest income from the STA QLD.
    \item The fund does \textbf{not} account for any GST in connection with STA interest remittances.
    \item CPG's trust account audit will reflect the interest collected and remitted to the state.
    \item No accounting entry is required in the fund's records for this interest. Do not accrue trust account interest as fund income.
\end{itemize}

\paragraph{GST in the trust account}

GST collected on rental income flows through the STA. The GST component is NOT the fund's income and must be clearly tracked separately:
\begin{itemize}[nosep]
    \item Gross rent collected (including GST) flows into STA.
    \item GST component is classified in \acct{255} (GST Liability) via Yardi.
    \item The net rent (excluding GST) is the fund's income (\acct{71}).
    \item At BAS time, the GST liability (\acct{255}) is discharged via NAB payment: DR \acct{255} / CR \acct{685} NAB.
    \item If the Yardi import does not automatically separate GST, the BAS reconciliation must do so. Never post GST-inclusive rent to \acct{71}.
\end{itemize}

\paragraph{Tenant bonds held in STA (if any)}

If cash security bonds are held in the STA for tenants (e.g., a tenant who provided a cash bond instead of a bank guarantee), the cash in trust is the \textbf{tenant's money}, not the fund's. It must be:
\begin{itemize}[nosep]
    \item Held separately and not included in the fund's cash for NAV purposes.
    \item Released only when legitimately applied (tenant breach, make-good costs, unpaid rent).
    \item On application: DR \acct{685} or \acct{700} / CR \acct{686} (applied against repair costs or rent shortfall) and the bond liability (if separately tracked) is reversed.
    \item On return to tenant: DR Bond Liability / CR \acct{686} (remit back to tenant through STA).
\end{itemize}

\subsubsection{Management Risks and Controls}

\begin{table}[H]
\centering\small\sffamily
\caption{STA QLD risk register and controls}
\begin{tabularx}{\textwidth}{@{}L{3.5cm} L{2.5cm} L{2cm} X@{}}
\rowcolor{navy}\textcolor{white}{\bfseries Risk} & \textcolor{white}{\bfseries Likelihood} & \textcolor{white}{\bfseries Impact} & \textcolor{white}{\bfseries Control} \\
\midrule
CPG remittance delay (cash not transferred to NAB) & Medium & Medium & Compare STA balance to expected net income each period; escalate to CPG if balance elevated without explanation. \\
\rowcolor{warmgray}
Yardi import does not capture all STA movements & Medium & Medium & Cross-check net STA movement per Yardi TB against CPG trust account statement monthly; any variance = Yardi missing entry. \\
CPG unauthorised disbursements from trust account & Low & High & Annual trust account audit by CPG's auditor; request copy of audit certificate each year; review CPG disbursements vs.\ authorised expenses. \\
\rowcolor{warmgray}
CPG insolvency / trust account shortfall & Very Low & Very High & Trust segregation provides protection; state compensation fund as backstop. Escalate to CPG principal relationship manager if any financial distress signals. \\
GST included in income accounts (not separated) & Medium & Medium & Check \acct{71} against gross rent less GST; \acct{255} balance should approximate quarterly BAS liability. \\
\rowcolor{warmgray}
Tenant bonds included in fund NAV & Low & Medium & Track cash bonds separately; they are not fund assets. Confirm nil cash bonds (only bank guarantees currently held). \\
STA account not closed at sale settlement & Low & Medium & CPG must formally close or transfer the trust account at settlement. All outstanding balances must be remitted to NAB before settlement; no active STA balance should exist at completion. \\
\bottomrule
\end{tabularx}
\end{table}

\subsubsection{STA QLD at Sale Settlement}

At settlement of the property sale, the STA QLD must be wound down:

\begin{enumerate}[nosep]
    \item \textbf{Final remittance:} CPG must remit the entire STA balance (all collected rent, outgoings recoveries, and other property income) to the fund's NAB account on or immediately before settlement. No residual balance should remain in the STA after the management agreement is terminated.
    \item \textbf{Outstanding invoices:} All property-level invoices that CPG has outstanding (final maintenance, final outgoings, management fee) must be settled from the STA before the final remittance.
    \item \textbf{Tenant bonds:} Any cash bonds held for tenants must be either returned to the tenants (if no outstanding claims) or applied to legitimate claims (unpaid rent, make-good) before settlement. The bank guarantees (5Point \$33,000, Popgun \$61,600) must be formally assigned to the purchaser at settlement (see Section~\ref{sec:est-bankguarantees}).
    \item \textbf{Management agreement termination:} The management agreement between the fund/trustee and CPG terminates at settlement. Confirm the notice period required and provide formal termination notice in writing, within the required timeframe.
    \item \textbf{Yardi access:} Confirm with CPG the date on which Yardi access will be terminated or transferred. All Yardi TB data up to the settlement date must be downloaded and archived before access is lost.
    \item \textbf{Final accounting:} CPG must provide a final trust account statement covering all transactions from the last remittance date to settlement. This is the basis for the final NAB reconciliation post-settlement.
    \item \textbf{Account closure:} The STA QLD (\acct{686}) balance should be nil by settlement. Post the final remittance journal: DR \acct{685} NAB / CR \acct{686} STA QLD = final balance. After this entry, \acct{686} should carry a zero balance for all post-settlement periods.
\end{enumerate}

\begin{warning}[Management Agreement Termination Before Settlement]
\textbf{If the management agreement is terminated before settlement} (e.g., CPG is replaced or the fund self-manages for the final wind-down period), CPG must immediately remit all trust account balances and close the STA. Any delay in remittance after termination is a breach of the \textit{Property Occupations Act}. The fund should seek written confirmation of the nil STA balance before releasing CPG from its management obligations.
\end{warning}


% ============================================================================
% ============================================================================


\subsection{Distribution Components --- Determination Framework for Fixed Distribution Amount}
\label{sec:est-distribution-components}
\legbox{s97 ITAA~1936 \legalcite{s97}{sec:ref-wht}; Division~6E ITAA~1936 \legalcite{Div~6E}{sec:ref-wht}; Division~115 ITAA~1997 \legalcite{Div~115}{sec:ref-wht}; s102-5 ITAA~1997 (method statement) \legalcite{s102-5}{sec:ref-wht}; Division~276--278 ITAA~1997 (AMIT) \legalcite{Div~276--278}{sec:ref-wht}; s134-1 ITAA~1997 \legalcite{s134-1}{sec:ref-wht}; Draft TR~2012/D1 \legalcite{TR~2012/D1}{sec:ref-wht}.}

When the fund distributes a \textbf{fixed distribution amount on demand} (i.e., a nominated dollar amount per unit that the trustee has committed to pay, independent of the exact tax result), the Preparer must decompose that distribution into its correct tax components. This section provides a complete framework: all variable definitions, step-by-step calculation logic, general min/max formulae, all edge cases, reconciliation to the CGT calculation file, and a worked numerical example.

All component amounts must sum exactly to the total distribution amount $D$.  The components are reported to unitholders on their distribution statement (and, if the trust qualifies as an AMIT, on the AMMA statement under Div~276).  Incorrect component categorisation exposes unitholders to under- or over-taxation and creates amendment obligations.

\begin{note}[TR 2012/D1 — Draft Ruling Status and Scope]
\textbf{TR~2012/D1 remains a Draft Taxation Ruling.}  It sets out the Commissioner's preferred view on the meaning of ``income of the trust estate'' in Division~6, but alternative interpretations exist and may, in appropriate circumstances, support a reasonably arguable position.  The Preparer should confirm with the tax agent which interpretation the fund follows and ensure the trust deed's income definition is consistent.
\end{note}

This section is structured as follows:
\begin{enumerate}[nosep]
\item Definitions of all distribution components.
\item Cost base concepts --- both the property's cost base (which drives the trust's CGT calculation) and the unitholder's unit cost base (which is affected by what components the distribution contains).
\item Variable definitions and step-by-step calculation framework.
\item General formulae with mathematical proof of completeness.
\item Edge cases, scenarios, and worked numerical example.
\item Reconciliation of the distribution CGT component to the CGT calculation file.
\item Summary decision tree and matrix.
\end{enumerate}


% ═══════════════════════════════════════════════════════════════════════
% PART A — COMPONENT DEFINITIONS
% ═══════════════════════════════════════════════════════════════════════

\subsubsection{Definition of All Distribution Components}

\begin{table}[H]
\centering\small\sffamily
\caption{Distribution component definitions}
\label{tab:dist-defs}
\begin{tabularx}{\textwidth}{@{}L{3.5cm} X@{}}
\rowcolor{navy}\textcolor{white}{\bfseries Component} & \textcolor{white}{\bfseries Definition} \\
\midrule
\textbf{Taxable Ordinary Income (C\textsubscript{OI})} & The unitholder's share of the trust's net ordinary assessable income: rental income, interest, sundry income, net of deductible ordinary expenses and applied prior-year tax losses.  Fully assessable to the unitholder at their marginal rate in the year of present entitlement (s97 ITAA~1936), regardless of when cash is received. \\
\rowcolor{warmgray}
\textbf{Net Assessable Capital Gain (C\textsubscript{CG})} & The unitholder's share of the trust's net capital gain calculated under the \textbf{s102-5 method statement}: gross capital gains, reduced by capital losses (Steps~1--2), then by the Div~115 CGT discount (Step~3), then by any Div~152 small business concessions (Step~4).  Assessable to the unitholder at their marginal rate.  Note: Under Subdiv~115-C, the trust-level net capital gain is attributed to beneficiaries who then apply the method statement at their own level (grossing up and re-applying their own discount). \\
\textbf{CGT Discount Component (C\textsubscript{disc})} & The amount excluded by the Div~115 50\% discount applied at the trust level to long-term capital gains.  This amount is \textbf{NOT} included in $D$ and is \textbf{NOT} assessable income of the trust.  It is reported separately on the distribution statement (or AMMA statement for AMITs) so the unitholder can apply their own Div~115 treatment at their level.  See the dedicated explanation below for why $\Cdisc = \CCG$ in the typical case and when they diverge. \\
\rowcolor{warmgray}
\textbf{Tax Deferred (C\textsubscript{TD})} & The portion of the distribution that exceeds the trust's net taxable income but falls within the trust's \textbf{accounting profit} --- a timing difference.  It arises because tax deductions (principally Div~40 and Div~43) reduce taxable income below accounting profit.  C\textsubscript{TD} is \textbf{not currently assessable}; instead it reduces the unitholder's cost base in their units.  The mechanism differs depending on whether the trust is an AMIT or a non-AMIT --- see the cost base subsection below.  When units are eventually sold, the lower cost base produces a larger capital gain: the tax is ``deferred'', not forgiven. \\
\textbf{Return of Capital (ROC) (C\textsubscript{ROC})} & The portion of the distribution that exceeds \textbf{both} the trust's net taxable income \textbf{and} its accounting profit.  ROC represents a literal return of invested capital.  It is not assessable income and is not a timing difference --- it permanently reduces the unitholder's cost base.  When the cost base reaches zero, further ROC distributions trigger an immediate capital gain (CGT Event E4 for non-AMITs, CGT Event E10 for AMITs).  The key distinction from C\textsubscript{TD}: tax deferred reverses when units are sold at a higher gain; ROC is a permanent capital recovery (no future tax unless cost base is exhausted). \\
\rowcolor{warmgray}
\textbf{CGT Concession Amount (C\textsubscript{SBC})} & The portion of the capital gain excluded from assessable income under Div~152 small business CGT concessions (e.g., the 50\% active asset reduction in Subdiv~152-C, the retirement exemption in Subdiv~152-D, or rollover relief in Subdiv~152-E).  This is distinct from the Div~115 CGT discount.  Div~152 concessions are permanent exclusions.  For the non-assessable part calculation under E4, CGT concession amounts are subject to specific adjustment rules in s104-71(4).  Not applicable to most passive property investment trusts unless the fund and its unitholders individually meet the basic conditions in Subdiv~152-A. \\
\textbf{Non-Assessable Non-Exempt (NANE)} & Amounts excluded from assessable income by a specific statutory provision (e.g., certain government grants, MIT cross-staple arrangement income under s12-437).  Rare in a property trust context.  NANE does \textbf{not} reduce the unitholder's cost base (s104-71(1) excludes it from the non-assessable part). \\
\bottomrule
\end{tabularx}
\end{table}


% ═══════════════════════════════════════════════════════════════════════
% PART B — COST BASE DEFINITIONS (NEW)
% ═══════════════════════════════════════════════════════════════════════

\subsubsection{Cost Base of the Property --- The Five Elements (s110-25 ITAA~1997)}
\label{sec:dist-cost-base-property}

The trust's capital gain on disposal of the property is determined by subtracting the property's \textbf{cost base} from the capital proceeds.  Understanding the cost base is essential because it directly determines the size of $\GCGlong$ (and hence $\CCG$) in the distribution component calculation.

The cost base of a CGT asset consists of \textbf{five elements} (s110-25). For the CGT-computation-specific traps (including the ATO audit risk for not reducing the cost base by Div~43 deductions), see Section~\ref{sec:est-cgt-costbase} and Section~\ref{sec:est-cgt-div43}.

\begin{table}[H]
\centering\small\sffamily
\caption{Five elements of cost base (s110-25 ITAA~1997) --- applied to the 22~Constance Street property}
\label{tab:cost-base-elements}
\begin{tabularx}{\textwidth}{@{}C{1.2cm} L{3cm} X L{3cm}@{}}
\rowcolor{navy}\textcolor{white}{\bfseries Elem.} & \textcolor{white}{\bfseries Statutory Name} & \textcolor{white}{\bfseries What It Includes} & \textcolor{white}{\bfseries 22~Constance Example} \\
\midrule
\textbf{1} & Acquisition cost (s110-25(2)) & Money paid or property given to acquire the asset.  Includes the purchase price and any market value substitution if applicable (s112-20). & Original purchase price of the property. \\
\rowcolor{warmgray}
\textbf{2} & Incidental costs (s110-25(3), s110-35) & Costs incurred in acquiring the asset \textbf{or} that relate to a CGT event.  Includes: legal fees, stamp duty, valuation fees, survey fees, conveyancing, agent commissions (on acquisition \textbf{and} on disposal), advertising costs, and borrowing costs to the extent they are not otherwise deductible.  \textbf{Note:} TD~2017/10 confirms incidental costs incurred \textit{after} the CGT event can still form part of Element~2 if they ``relate to'' the event. & Stamp duty on purchase; legal fees on acquisition and on sale; real estate agent commission on sale. \\
\textbf{3} & Ownership costs (s110-25(4)) & Costs of owning the asset (only if acquired after 20~August 1991): interest on borrowings to acquire the asset, insurance, maintenance, repairs.  \textbf{Important:} Element~3 is \textbf{excluded from the reduced cost base} (s110-55(2)).  In practice, Element~3 is usually nil for income-producing assets because these costs are claimed as revenue deductions under s8-1 and therefore cannot also form part of the cost base (s110-45(2)). & Typically nil --- interest and insurance are deducted under s8-1 as revenue expenses. \\
\rowcolor{warmgray}
\textbf{4} & Capital improvements (s110-25(5)) & Capital expenditure to increase or preserve the asset's value, or relating to installing or moving the asset.  \textbf{Note:} Since 1~July 2005, there is no requirement that the expenditure be ``reflected in the state or nature of the asset at the time of the CGT event'' (removed by Act No.~32 of 2006). & Capital works: cyclone remediation, structural upgrades, HVAC replacement (to the extent capitalised, not deducted under s25-10). \\
\textbf{5} & Title preservation (s110-25(6)) & Capital expenditure to establish, preserve, or defend title to the asset. & Legal costs defending title or resolving encumbrances. \\
\bottomrule
\end{tabularx}
\end{table}

\paragraph{Cost base reductions (s110-45)}

The cost base is reduced by amounts the taxpayer has deducted or \textbf{can deduct}:

\begin{itemize}[nosep]
\item \textbf{Division~43 capital works deductions (s110-45(2) and (4)):}  Amounts deducted or that \textit{could have been} deducted under Div~43 reduce both the cost base and the reduced cost base.  Per TD~2005/47, ``can deduct'' means the deduction is available \textit{and} the amendment period has not expired.  The ATO will accept no reduction only where the taxpayer lacked sufficient information to determine construction expenditure \textbf{and} did not claim any Div~43 deduction (PS~LA~2006/1).  \textbf{This is frequently identified in ATO audits --- ensure the CGT calculation file reflects all Div~43 claimed.}
\item \textbf{Recoupments (s110-45(3)):}  Expenditure is excluded to the extent of any recoupment received (e.g., insurance proceeds), unless the recoupment is included in assessable income.
\item \textbf{Net input tax credits (s103-30):}  GST input tax credits claimed reduce the cost base.
\end{itemize}

\paragraph{Reduced cost base (s110-55)}

The \textbf{reduced cost base} is used to calculate a capital \textit{loss} (if any).  It has the same five elements as the cost base \textbf{except} Element~3 (ownership costs) is excluded entirely.  Additionally, the reduced cost base does not include amounts that could have been deducted had the asset been used wholly to produce assessable income (s110-55(5)).

\begin{warning}[Confirm All Div 43 Deductions Removed from Cost Base]
For 22~Constance Street, confirm with the tax agent that \textbf{all} Div~43 deductions claimed over the life of the property have been removed from the cost base in the CGT calculation file.  Failure to do so will understate the capital gain and therefore understate the $\CCG$ distribution component.  See Section~\ref{sec:est-cgt-div43} for the detailed Div~43 interaction.
\end{warning}


\subsubsection{Cost Base of the Units --- How Distribution Components Affect the Unitholder}
\label{sec:dist-cost-base-units}

Separate from the property's cost base, each \textbf{unitholder} has a cost base for their \textit{units} in the trust.  Distribution components affect this unit cost base as follows:

\begin{table}[H]
\centering\small\sffamily
\caption{Effect of distribution components on the \textbf{unitholder's unit cost base}}
\label{tab:unit-cost-base-effect}
\begin{tabularx}{\textwidth}{@{}L{2.5cm} C{2cm} C{2cm} C{2cm} X@{}}
\rowcolor{navy}\textcolor{white}{\bfseries Component} & \textcolor{white}{\bfseries Assessable?} & \textcolor{white}{\bfseries Cost Base} & \textcolor{white}{\bfseries Reduced CB} & \textcolor{white}{\bfseries Mechanism} \\
\midrule
C\textsubscript{OI} & Yes & No change & No change & Assessed under s97 ITAA~1936 \\
\rowcolor{warmgray}
C\textsubscript{CG} & Yes & No change & No change & Assessed under Subdiv~115-C \\
C\textsubscript{disc} & No & No change & No change & Excluded from non-assessable part by s104-71(4) item~1 \\
\rowcolor{warmgray}
C\textsubscript{TD} & No & \textbf{Reduced} & \textbf{Reduced} & CGT Event E4 (non-AMIT) or s104-107B (AMIT) \\
C\textsubscript{ROC} & No & \textbf{Reduced} & \textbf{Reduced} & Same as C\textsubscript{TD} --- reduces further; if CB reaches zero, excess triggers capital gain \\
\rowcolor{warmgray}
C\textsubscript{SBC} & No & See note & See note & Specific adjustments per s104-71(4) items \\
NANE & No & No change & No change & Excluded by s104-71(1) --- tax-exempted amount \\
\bottomrule
\end{tabularx}
\end{table}

\paragraph{AMIT vs Non-AMIT --- Different cost base adjustment mechanisms}

The cost base adjustment mechanism depends critically on whether the trust has elected into the AMIT regime (Div~276--278):

\begin{itemize}[nosep]
\item \textbf{Non-AMIT trusts:}  The non-assessable part of a distribution triggers \textbf{CGT Event E4} (s104-70).  At the end of each income year (or just before another CGT event on the units), the cumulative non-assessable payments reduce the unit cost base.  If cumulative non-assessable payments exceed the cost base, the excess is a capital gain under E4.  The non-assessable part is adjusted by s104-71 to exclude tax-exempted amounts (s104-71(1)), tax-free amounts (s104-71(3)), and CGT concession amounts (s104-71(4)).
\item \textbf{AMIT trusts:}  CGT Event E4 \textbf{does not apply} (s104-70(1A)).  Instead, the annual cost base adjustment is calculated as the \textbf{AMIT cost base net amount} under s104-107B to s104-107E.  This is the difference between the AMIT cost base reduction amount (distributions received, broadly) and the AMIT cost base increase amount (amounts included in assessable income, broadly).  If the net amount exceeds the cost base, \textbf{CGT Event E10} (s104-107A) is triggered --- not E4.  The AMIT regime provides greater certainty because cost base adjustments are calculated formulaically rather than relying on the characterisation of each payment.
\end{itemize}

\begin{note}[AMIT Regime — Confirm Election Status with Tax Agent]
Confirm with the tax agent whether 22~Constance Street Unit Trust has elected into the AMIT regime.  If it is an AMIT, the AMMA statement (not the SDS) is the primary reporting document, and the cost base adjustment follows the AMIT cost base net amount methodology.  If it is a non-AMIT, the SDS is used and CGT Event E4 applies.
\end{note}


% ═══════════════════════════════════════════════════════════════════════
% PART C — VARIABLES AND CALCULATION
% ═══════════════════════════════════════════════════════════════════════

\subsubsection{Variable Definitions}

\begin{table}[H]
\centering\small\sffamily
\caption{Variables used in distribution component formulae}
\label{tab:dist-vars}
\begin{tabularx}{\textwidth}{@{}C{2.0cm} X@{}}
\rowcolor{navy}\textcolor{white}{\bfseries Variable} & \textcolor{white}{\bfseries Definition} \\
\midrule
$D$ & Fixed distribution amount on demand (total \$ for the period, or per unit).  Known at outset, set by trustee resolution. \\
\rowcolor{warmgray}
$\OI$ & Gross ordinary income: rent, interest, insurance business-interruption proceeds, sundry income. \\
$\OD$ & Ordinary deductions: interest expense, Div~40 depreciation, Div~43 building allowance, repairs (s25-10), management fees, professional fees, s25-25 borrowing cost amortisation, s40-880 blackhole expenditure, land tax, other deductible expenses. \\
\rowcolor{warmgray}
$TL$ & Carried-forward ordinary tax losses from prior years ($\geq 0$; nil if none).  Subject to the trust loss testing rules in Schedules~2F and 2G of ITAA~1936 (the income injection and pattern-of-distributions tests). \\
$\GCGlong$ & Gross capital gain on assets held $>$12 months (before discount).  For 22~Constance: the property sale gain = capital proceeds $-$ cost base (all 5 elements per s110-25, reduced per s110-45). \\
\rowcolor{warmgray}
$\GCGshort$ & Gross capital gain on assets held $\leq$12 months (no Div~115 discount).  For 22~Constance: likely nil (property held many years). \\
$CL$ & Total capital losses available (current year losses + prior year capital losses carried forward).  Capital losses can only offset capital gains, never ordinary income (s102-15). \\
\rowcolor{warmgray}
$\SBC$ & The \textbf{post-discount} reduction under Div~152 small business CGT concessions (i.e., the amount by which Step~4 of the s102-5 method statement further reduces the gain \textit{after} the Div~115 discount has been applied in Step~3).  Nil for most passive property investment trusts. \\
$AP$ & Accounting profit for the period (book net income under AASB~140 fair value model or AASB~116 cost model, as applicable).  May differ from net taxable income due to timing differences (Div~40/43) and recognition differences (fair value gains not taxable until realised). \\
\rowcolor{warmgray}
$\NetTI$ & Trust net taxable income = the trust's ``net income'' per s95(1) ITAA~1936 = total assessable income less allowable deductions.  This is the amount on which beneficiaries are assessed under s97. \\
\bottomrule
\end{tabularx}
\end{table}


\subsubsection{Step-by-Step Calculation --- General Framework}

\paragraph{Step 1: Net ordinary income (before tax losses)}
\[
\NOI = \max(0,\ \OI - \OD)
\]
If $\OI < \OD$: the current period has a tax loss.  Any new tax loss adds to carried-forward losses:
\[
\TLnew = TL + \max(0,\ \OD - \OI)
\]
\textit{Use $\TLnew$ in all subsequent steps.}

\paragraph{Step 2: Apply carried-forward tax losses to ordinary income}
\[
\TLapplied = \min(TL,\ \NOI)
\]
\[
\TOI = \NOI - \TLapplied \quad \text{(Taxable Ordinary Income after losses)}
\]
\[
\TLremaining = TL - \TLapplied \quad \text{(Losses carried forward to next year)}
\]

\begin{note}[Ordinary Tax Losses Cannot Offset Capital Gains in Most Trusts]
Ordinary tax losses can reduce ordinary income to nil, but the residual \textbf{cannot} be applied to capital gains in most trust structures (unlike companies).  The segregation of ordinary losses and capital losses is a fundamental feature of the trust taxation regime.  If the trust has no prior-year loss testing issues, $\TLremaining$ carries forward to offset future ordinary income.  Confirm the trust passes the loss testing rules in Schedules~2F/2G if losses are being utilised.
\end{note}

\paragraph{Step 3: Net capital gain calculation (s102-5 method statement)}

The net capital gain is calculated following the statutory method statement in s102-5 ITAA~1997:

\textit{Step~3a --- Reduce by current-year and prior-year capital losses (s102-5 Steps~1 and 2).}  Apply capital losses against short-term gains first (this preserves the Div~115 discount on long-term gains and is the ATO's recommended ordering):
\[
\CLshort = \min(CL,\ \GCGshort)
\]
\[
\CGshortnet = \GCGshort - \CLshort
\]
\[
\CLlong = \max(0,\ CL - \CLshort)
\]
\[
\CGlongnet = \max(0,\ \GCGlong - \CLlong)
\]

\textit{Step~3b --- Apply CGT discount (s102-5 Step~3, Div~115).}  The 50\% discount applies to discount capital gains (assets held $>$12 months) by individuals and trusts:
\[
\Cdisc = \CGlongnet \times 0.5 \quad \text{(CGT Discount Component --- reported but NOT part of $D$)}
\]

\textit{Step~3c --- Apply Div~152 small business concessions (s102-5 Step~4).}  Applied \textbf{after} the discount.  The \textbf{net capital gain} is the non-negative result:
\[
\NCG = \max\!\bigl(0,\;\; \CGshortnet + \CGlongnet \times 0.5 - \SBC\bigr)
\]
\textit{If the expression inside $\max$ is negative, the excess becomes a capital loss carried forward (not a negative net capital gain).  For 22~Constance Street, $\SBC = 0$ (no Div~152 eligibility), so $\NCG = \CGshortnet + \CGlongnet \times 0.5$ in practice.}

\paragraph{Step 4: Trust net taxable income}
\[
\NetTI = \TOI + \NCG
\]

\textit{This is the s95(1) net income.  Division~6E ensures that capital gains and franked distributions (if any) are not double-counted when assessing beneficiaries under s97.}

\paragraph{Step 5: Accounting profit bounds for tax-deferred and ROC}

Obtain the book accounting profit $AP$ for the period.  Define:
\[
AP^+ = \max(0,\ AP) \quad \text{(non-negative accounting profit)}
\]

The maximum tax-deferred component is bounded by \textbf{both} the tax-accounting gap \textbf{and} the residual distribution after taxable income.  The formula enforces whichever constraint binds first:
\[
\TDmax = \max\!\bigl(0,\;\; \min(D,\; AP^+) - \NetTI\bigr)
\]

\textit{Equivalently:}
\[
\TDmax = \max\!\Bigl(0,\;\; \min\!\bigl(D - \NetTI,\;\; AP^+ - \NetTI\bigr)\Bigr)
\]

\textit{Interpretation: when $D \leq AP^+$, the bound is $D - \NetTI$ (the full excess over taxable income is a timing difference).  When $D > AP^+$, the bound is $AP^+ - \NetTI$ (only the accounting-to-tax gap is a timing difference; the rest is ROC).  When $AP^+ \leq \NetTI$, the bound is zero (no timing excess exists --- accounting profit does not exceed taxable income).}

The maximum ROC component is bounded by the distribution in excess of accounting profit:
\[
\ROCmax = \max(0,\ D - AP^+)
\]

\textit{Interpretation: ROC arises only when the distribution exceeds the trust's book profit.  Any distribution beyond $AP^+$ is a literal return of invested capital.}

\begin{note}[Using AP+ to Prevent Negative Profit Anomaly in Distributions]
Using $AP^+$ rather than $AP$ prevents the anomaly where a negative accounting profit (e.g., from an AASB~140 fair value write-down) would inflate $\ROCmax$ beyond $D$.  When $AP < 0$ and $\NetTI > 0$ (unrealised accounting loss but positive taxable income), the entire distribution up to $\NetTI$ is taxable, and any excess is ROC --- there is no tax-deferred component because accounting profit does not exceed taxable income.
\end{note}

\begin{note}[Verification of Bound Sufficiency]
The bounds are sufficient to absorb $D$ in all cases.  Specifically:
\[
\NetTI + \TDmax + \ROCmax \;\geq\; D \qquad \text{for all } D \geq 0
\]
\textit{Proof sketch:}
If $D \leq AP^+$: sum $= \NetTI + (D - \NetTI) + 0 = D$.
If $D > AP^+ \geq \NetTI$: sum $= \NetTI + (AP^+ - \NetTI) + (D - AP^+) = D$.
If $AP^+ < \NetTI$: sum $= \NetTI + 0 + (D - AP^+) \geq D$ (since $\NetTI > AP^+$).
Hence the sequential allocation formulae in the next section never leave an unallocated residual (except $\CSBC$, which is zero for non-Div~152 trusts).
\end{note}


% ═══════════════════════════════════════════════════════════════════════
% PART D — FORMULAE AND PROOF
% ═══════════════════════════════════════════════════════════════════════

\subsubsection{General Distribution Component Formulae}

Using the variables from the steps above, the distribution components are determined by a \textbf{sequential allocation} (each component absorbs as much of $D$ as it can before the next):

\begin{alignat}{2}
\COI &= \min(D,\ \TOI) \label{eq:coi} \\[4pt]
\CCG &= \min\!\bigl(\max(0,\ D - \COI),\ \NCG\bigr) \label{eq:ccg} \\[4pt]
\CTD &= \min\!\bigl(\max(0,\ D - \COI - \CCG),\ \TDmax\bigr) \label{eq:ctd} \\[4pt]
\CROC &= \min\!\bigl(\max(0,\ D - \COI - \CCG - \CTD),\ \ROCmax\bigr) \label{eq:croc} \\[4pt]
\CSBC &= \max(0,\ D - \COI - \CCG - \CTD - \CROC) \label{eq:csbc}
\end{alignat}

\medskip
\textbf{Verification identity (must hold exactly):}
\[
\COI + \CCG + \CTD + \CROC + \CSBC = D
\]

\medskip
\textbf{Separately reported (NOT part of $D$):}
\[
\Cdisc = \CGlongnet \times 0.5 \quad \text{(CGT Discount --- AMMA or distribution statement)}
\]


\subsubsection{Mathematical Proof of Completeness and Non-Redundancy}
\label{sec:dist-proof}

The formulae (\ref{eq:coi})--(\ref{eq:csbc}) form a \textbf{partition} of $D$: they are exhaustive (sum to $D$), non-negative, and non-overlapping.  This is proved as follows.

\paragraph{Non-negativity.}  Each component is defined using $\min(\max(0, \cdot), \cdot)$ or $\max(0, \cdot)$.  Since $D \geq 0$, $\TOI \geq 0$, $\NCG \geq 0$, and the bounds are non-negative, every component $\geq 0$.

\paragraph{Exhaustiveness.}  Define the \textbf{residual} after each step:
\begin{align*}
R_0 &= D \\
R_1 &= D - \COI = D - \min(D, \TOI) = \max(0,\ D - \TOI) \geq 0 \\
R_2 &= R_1 - \CCG = R_1 - \min(R_1, \NCG) = \max(0,\ R_1 - \NCG) \geq 0 \\
R_3 &= R_2 - \CTD = R_2 - \min(R_2, \TDmax) \geq 0 \\
R_4 &= R_3 - \CROC = R_3 - \min(R_3, \ROCmax) \geq 0 \\
\CSBC &= R_4 = \max(0,\ R_4) = R_4
\end{align*}

Since $\CSBC = R_4 = D - \COI - \CCG - \CTD - \CROC$, the sum is identically $D$. \hfill $\square$

\paragraph{Non-redundancy.}  Each component draws from a distinct economic source:  $\COI$ from ordinary income, $\CCG$ from capital gains, $\CTD$ from the accounting--tax timing gap, $\CROC$ from capital return, and $\CSBC$ as a statutory concession residual.  The sequential allocation ensures no dollar of $D$ is assigned to more than one component.

\paragraph{When does $\CSBC > 0$?}  Only when $D$ exceeds the sum of all other available sources: $D > \TOI + \NCG + \TDmax + \ROCmax$.  For a passive property trust without Div~152 eligibility, this should not occur.  A positive $\CSBC$ in a non-Div~152 trust is a \textbf{red flag} --- it indicates either a calculation error or an undisclosed concession.  Investigate before finalising.


% ═══════════════════════════════════════════════════════════════════════
% PART E — CGT DISCOUNT EQUALITY AND DIVERGENCE
% ═══════════════════════════════════════════════════════════════════════

\subsubsection{Why the CGT Discount Component Normally Equals the Assessable CGT Component}

In the 22~Constance Street fund's typical scenario (single property, held for more than 12 months, eligible for full Div~115 50\% discount, no Div~152 concessions, no short-term assets):

\[
\text{Net capital gain after losses} = G \quad\Longrightarrow\quad
\Cdisc = 0.5G, \quad \CCG = 0.5G \quad\Longrightarrow\quad \Cdisc = \CCG
\]

They are equal because the 50\% discount removes exactly half and retains exactly half.

\begin{note}[Terminology Clarification --- ``CGT Discount'' vs ``CGT Concession'' on AMMA Statements]
Practitioners working across multiple CPG funds frequently observe that the \textbf{CGT Discount} ($\Cdisc$) and the \textbf{CGT Concession Amount} ($\CSBC$) reported on the AMMA or distribution statement appear to be ``always the same.''  This is a \textbf{terminological confusion} --- the two line items are fundamentally different in nature and in most CPG passive property trusts they are \textbf{not} the same:

\begin{itemize}[nosep]
    \item \textbf{CGT Discount Component ($\Cdisc$)} is the amount \emph{excluded} from assessable income by the Div~115 50\% discount.  It equals $\CGlongnet \times 0.5$.  It is \textbf{not} part of $D$ and is reported as a \textbf{separate memorandum item} on the AMMA statement.  For a single long-term property trust, $\Cdisc = \CCG$ (the assessable half and the discounted half are equal by construction --- see above).
    \item \textbf{CGT Concession Amount ($\CSBC$)} is the Div~152 small business CGT concession --- an entirely different statutory regime.  Div~152 provides \textbf{permanent exclusions} (50\% active asset reduction under Subdiv~152-C, retirement exemption under Subdiv~152-D, or rollover relief under Subdiv~152-E) that apply \emph{after} the Div~115 discount.  For passive property investment trusts like CPG funds, Div~152 is \textbf{almost always inapplicable}: the fund must individually satisfy the basic conditions in Subdiv~152-A (net asset test $\leq$\$6M or aggregated turnover $<$\$2M, and the CGT asset must be an ``active asset''), which a passive property investment trust holding a single leased commercial building will typically not meet.  Therefore, $\CSBC = 0$ in the vast majority of CPG fund distributions.
\end{itemize}

\textbf{What practitioners actually observe as ``always the same'' is $\Cdisc = \CCG$} (the CGT discount memorandum amount equals the assessable capital gain component), which holds whenever the fund has: (i)~only long-term capital gains (property held $>$12 months); (ii)~no capital losses reducing the discount base; and (iii)~no Div~152 concessions.  Since all three conditions are satisfied in a typical CPG single-property trust, the equality $\Cdisc = \CCG$ holds mechanically.

\textbf{Exceptions where $\Cdisc \neq \CCG$} are described in the table below.  Additionally, if the fund unexpectedly satisfies the Div~152 conditions (for example, if the net asset value falls below \$6M due to cyclone damage write-downs, and the property is reclassified as an active asset because the RE conducts an active leasing business), $\CSBC$ may become non-zero --- but this would be unusual and should be flagged to the tax agent immediately.
\end{note}

\subsubsection{Cases When CGT Discount Component Does Not Equal Assessable CGT}

\begin{table}[H]
\centering\small\sffamily
\caption{Scenarios where C\textsubscript{disc} $\neq$ C\textsubscript{CG}}
\begin{tabularx}{\textwidth}{@{}L{3.5cm} X L{3cm}@{}}
\rowcolor{navy}\textcolor{white}{\bfseries Scenario} & \textcolor{white}{\bfseries Reason} & \textcolor{white}{\bfseries Effect} \\
\midrule
Mixed holding periods (some assets $<$12 months) & Short-term gains are not discounted ($\Cdisc = 0$ for short-term gains), so the blended result diverges. & $\Cdisc < \CCG$ \\
\rowcolor{warmgray}
Div~152 small business concessions also apply & SBC reduces the assessable gain \textit{after} the discount (Step~4), but $\Cdisc$ reflects only the discount (Step~3). & $\Cdisc > \CCG$ \\
Capital losses applied unevenly & If losses are applied partly to long-term and partly to short-term gains, the discount base changes relative to the assessable base. & Variable \\
\rowcolor{warmgray}
AMIT attribution variances & Attributed amounts on the AMMA may differ from actual tax-year positions due to prior-year adjustment (``AMMA mismatch''). & Reconcile AMMA to return \\
Foreign source CGT gains & Foreign tax credits and treaty relief may alter the effective discount. & Seek tax advice \\
\rowcolor{warmgray}
Entity type change mid-year & If the trust loses discount eligibility (e.g., conversion to a company, death of sole unitholder in a bare trust). & Rare \\
\bottomrule
\end{tabularx}
\end{table}


% ═══════════════════════════════════════════════════════════════════════
% PART F — RECONCILIATION TO CGT CALC FILE
% ═══════════════════════════════════════════════════════════════════════

\subsubsection{Cases When CGT Distribution Component Does Not Equal CGT in the CGT Calculation File}

Even if $\CCG$ is calculated correctly, it may not equal the amount shown in the fund's CGT calculation workpaper:

\begin{enumerate}[nosep]
    \item \textbf{Distribution paid in a different period from CGT event:} CGT Event A1 occurs at contract date.  If the contract is signed in June 2026 but the distribution is not paid until September 2026, the CGT component is calculated for the June 2026 tax year, but the distribution cash flows in the September 2026 quarter.  The CGT component attributed to unitholders matches the tax year, not the payment date.
    \item \textbf{CGT calc file uses pre-discount gross gain; distribution uses post-discount assessable gain:}  The CGT file typically shows $G$ (gross); the distribution statement shows $G \times 50\%$ (assessable).  These differ by the discount factor --- not an error, but a common source of confusion.
    \item \textbf{Div~43 cost base reduction not yet updated in CGT file:}  If cumulative Div~43 deductions have not been subtracted from the cost base per s110-45(2) and (4), the gross gain in the CGT file is understated.  See Section~\ref{sec:est-cgt-div43}.
    \item \textbf{Capital losses applied in a different order:}  If the CGT file applies $CL$ against long-term gains but the distribution calculation applies $CL$ against short-term gains first, the assessable amounts differ.  The ATO's recommended approach: apply losses to short-term gains first to preserve the discount.
    \item \textbf{Element~5 selling costs not yet confirmed:}  If incidental disposal costs (Element~2 of the cost base) are estimated but final amounts differ (e.g., legal fees not yet invoiced), the CGT gain in the distribution will differ from the final tax return.
    \item \textbf{Distribution paid before final tax return:}  If facts change between payment and lodgement (e.g., additional selling costs confirmed, insurance proceeds allocated), the distribution statement must be reviewed and unitholders may need amended statements.
    \item \textbf{H1 gain not separated from A1 gain:}  If CGT Event H1 (option granted) arose in a prior year and the s134-1 cost base/capital proceeds adjustment has not been correctly incorporated into the A1 calculation, the distribution CGT component will overstate the A1 gain.
\end{enumerate}


% ═══════════════════════════════════════════════════════════════════════
% PART G — EDGE CASES
% ═══════════════════════════════════════════════════════════════════════

\subsubsection{All Edge Cases and Scenarios}

\begin{table}[H]
\centering\scriptsize\sffamily
\caption{Distribution component scenarios --- all cases}
\label{tab:dist-scenarios}
\begin{tabularx}{\textwidth}{@{}L{3.2cm} X L{3.2cm}@{}}
\rowcolor{navy}\textcolor{white}{\bfseries Scenario} & \textcolor{white}{\bfseries Analysis} & \textcolor{white}{\bfseries Dominant Components} \\
\midrule
$D = \NetTI$ (perfect match) & Distribution exactly equals taxable income.  No tax deferred, no ROC.  All of $D$ is taxable.  Unitholder pays tax on exactly what they receive. & $\COI$, $\CCG$ \\
\rowcolor{warmgray}
$D < \NetTI$ (distribution below taxable income) & Unitholder assessed on full $\NetTI$ share under s97, but only receives $D$ in cash.  The shortfall ($\NetTI - D$) must be funded from unitholder's own resources.  No $\CTD$, no $\CROC$.  All of $D$ is taxable.  See ``Tax Exceeds Cash'' below. & $\COI$, $\CCG$ \\
$\NetTI < D \leq AP^+$ (exceeds taxable, within accounting profit) & Excess $D - \NetTI$ is tax deferred (timing difference from Div~40/43).  $\CTD = D - \NetTI$. & $\COI$, $\CCG$, $\CTD$ \\
\rowcolor{warmgray}
$D > AP^+$ (exceeds accounting profit --- true ROC) & Excess above $AP^+$ is ROC.  $\CROC = D - AP^+$.  Also, if $AP^+ > \NetTI$, tax deferred arises for the middle band. & $\COI$, $\CCG$, $\CTD$, $\CROC$ \\
$AP < 0$, $\NetTI > 0$ (accounting loss but positive taxable income) & Arises when AASB~140 fair value write-down reduces book profit below zero, but the unrealised loss is not deductible.  $\TDmax = 0$ (no timing excess).  If $D > \NetTI$: $\CROC = D - \NetTI$ (since $AP^+ = 0$, no middle band). & $\COI$, $\CCG$, possibly $\CROC$ \\
\rowcolor{warmgray}
$AP < 0$, $\NetTI = 0$ (accounting and tax loss) & No assessable income.  Entire $D$ is ROC.  Unit cost base reduced by $D$. & $\CROC$ only \\
$TL > \NOI$ (large losses exceed ordinary income) & $\TOI = 0$, $\TLremaining = TL - \NOI$.  No ordinary income component.  If $\NCG$ exists, $\CCG$ dominates.  May be largely tax deferred or ROC. & $\CCG$, $\CTD$, $\CROC$ \\
\rowcolor{warmgray}
$\NCG > D$ (capital gain alone exceeds distribution) & $\CCG = D$ (formula constrains to $D$).  Trust has a larger net capital gain than it distributes.  Undistributed portion is retained; unitholders are still assessed on full s97 share of $\NetTI$. & $\CCG = D$ \\
$D = 0$ (no distribution, trust has taxable income) & All components = \$0.  But unitholders are still assessed on their s97 share of $\NetTI$ --- tax on income never received as cash. & No components; s97 assessment \\
\rowcolor{warmgray}
Sale year: large $\NCG$, prior-year ordinary $TL$ & $TL$ offsets ordinary income but not capital gains.  $\TOI$ eliminated; $\CCG$ dominates.  $D$ may be insufficient to cover tax. & $\CCG$, possibly $\CROC$ \\
H1 in prior year, A1 in current year & H1 distribution (option fee) in prior year: small, likely $\CCG$.  Current year A1: s134-1 adjusts A1 proceeds and cost base.  Do not double-count H1 consideration. & $\CCG$, $\COI$ \\
\rowcolor{warmgray}
Post-settlement insurance proceeds only & Trust has only s6-5 ordinary income in wind-up period.  No capital gain.  Distribution is entirely ordinary income. & $\COI$ \\
Multiple CGT events in same year (H1 and A1) & If both occur in same income year, calculate each gain separately, aggregate in Step~3, then allocate combined $\NCG$ into $D$.  s134-1 adjustment still applies to A1. & $\CCG$ (combined) \\
\bottomrule
\end{tabularx}
\end{table}


\subsubsection{Paying Tax on Undistributed Income (When $D < \NetTI$) --- Tax Exceeds Cash}

When the distribution is \textbf{below} the trust's net taxable income ($D < \NetTI$), unitholders face a cash shortfall:

\begin{itemize}[nosep]
    \item Under s97 ITAA~1936, unitholders are assessed on their \textbf{proportionate share of the trust's net income} regardless of how much cash they receive.
    \item A unitholder who receives $D$ per unit but is assessed on $\NetTI$ per unit must fund the tax on $(\NetTI - D)$ from their own resources.
    \item This situation commonly arises at settlement when the trust has a large capital gain ($\NCG$) but distributes only the cash received at settlement less debt repayment (which may be less than the taxable gain).
    \item \textbf{Example:} Settlement proceeds \$13.25M, debt repaid \$5M, net available \$8.25M.  If the trustee distributes the full \$8.25M among 11.4M units $=$ \$0.72/unit, but total $\NetTI = \$4.5$M $=$ \$0.39/unit, then the distribution of \$0.72/unit \textit{exceeds} taxable income, so tax deferred and/or ROC components arise (the opposite problem).
    \item The ``below profit'' scenario arises when the trustee retains proceeds to repay debt or reserves before distributing, such that $D < \NetTI$.  Unitholders may need to fund tax personally.
\end{itemize}

\begin{warning}[Distribution Adequacy for Anticipated Unitholder Tax Liability]
For the 22~Constance Street sale, confirm with the tax agent and CPG whether the distribution amount will cover the anticipated tax liability for unitholders.  The recommended practice is to estimate each major unitholder's tax liability and confirm sufficient liquidity before the distribution is set.
\end{warning}

\subsubsection{When $\NetTI > D$ --- Allocation of Shortfall}

\[
\text{Tax shortfall scenario:} \quad \NetTI > D
\]

In this case:
\begin{itemize}[nosep]
    \item $\COI = \min(D, \TOI)$
    \item $\CCG = \min(D - \COI, \NCG)$
    \item $\CTD = 0$ (no room --- $D \leq \NetTI$)
    \item $\CROC = 0$
    \item The sum $\COI + \CCG = D$ --- the entire distribution is taxable.
    \item But the unitholder's share of trust net income is $\frac{\NetTI}{\text{Units}} > \frac{D}{\text{Units}}$, so additional tax is owed beyond the cash received.
    \item The remaining undistributed net income ($\NetTI - D$) is taxed to unitholders as retained trust income under s97 --- they owe tax on income they never received in cash.
    \item Trustees should consider this when setting $D$.  If the capital gain is very large and $D$ is constrained by debt repayment, unitholders face a significant unpaid tax liability.
\end{itemize}


% ═══════════════════════════════════════════════════════════════════════
% PART H — WORKED EXAMPLE (NEW)
% ═══════════════════════════════════════════════════════════════════════

\subsubsection{Worked Numerical Example --- 22~Constance Street Indicative Sale}
\label{sec:dist-worked-example}

The following is an \textbf{illustrative example only} using plausible numbers.  Actual figures will differ.

\begin{table}[H]
\centering\small\sffamily
\caption{Worked example --- inputs}
\begin{tabularx}{\textwidth}{@{}L{6cm} R{2.5cm} X@{}}
\rowcolor{navy}\textcolor{white}{\bfseries Item} & \textcolor{white}{\bfseries Amount} & \textcolor{white}{\bfseries Source} \\
\midrule
Sale price (capital proceeds) & \$13,250,000 & Contract of sale \\
\rowcolor{warmgray}
Cost base (Elements 1--5, net of Div~43 reduction) & \$9,000,000 & CGT calc file \\
Ordinary income (rental, interest) & \$400,000 & Yardi trial balance \\
\rowcolor{warmgray}
Ordinary deductions (interest, Div~40, Div~43, fees) & \$350,000 & Yardi trial balance \\
Prior-year tax losses ($TL$) & \$0 & Tax return \\
\rowcolor{warmgray}
Capital losses ($CL$) & \$0 & Tax return \\
Accounting profit ($AP$) & \$3,500,000 & Financial statements \\
\rowcolor{warmgray}
Total units on issue & 11,400,000 & Registry \\
Distribution set by trustee ($D$) & \$8,250,000 & Trustee resolution \\
\bottomrule
\end{tabularx}
\end{table}

\paragraph{Step~1: $\NOI$}
$\NOI = \max(0,\ 400{,}000 - 350{,}000) = \$50{,}000$

\paragraph{Step 2: Apply tax losses}
$TL = 0 \Rightarrow \TOI = \$50{,}000$

\paragraph{Step 3: Net capital gain}
$\GCGlong = 13{,}250{,}000 - 9{,}000{,}000 = \$4{,}250{,}000$.  No short-term gains, no losses.
\[
\Cdisc = 4{,}250{,}000 \times 0.5 = \$2{,}125{,}000 \quad \text{(reported separately)}
\]
\[
\NCG = 4{,}250{,}000 \times 0.5 - 0 = \$2{,}125{,}000
\]

\paragraph{Step 4: $\NetTI$}
$\NetTI = 50{,}000 + 2{,}125{,}000 = \$2{,}175{,}000$

\paragraph{Step 5: Bounds}
$AP^+ = \max(0, 3{,}500{,}000) = \$3{,}500{,}000$.
\begin{align*}
\TDmax &= \max\!\bigl(0,\ \min(8{,}250{,}000,\ 3{,}500{,}000) - 2{,}175{,}000\bigr) \\
                  &= \max(0,\ 1{,}325{,}000) = \$1{,}325{,}000
\end{align*}
\[
\ROCmax = \max(0,\ 8{,}250{,}000 - 3{,}500{,}000) = \$4{,}750{,}000
\]

\paragraph{Apply formulae}

Substituting (abbreviating millions as M, thousands as k):
{\small
\begin{align*}
\COI &= \min(8.25\text{M},\ 50\text{k}) = \$50{,}000 \\
\CCG &= \min(8.25\text{M} - 50\text{k},\ 2.125\text{M}) = \$2{,}125{,}000 \\
\CTD &= \min(8.25\text{M} - 50\text{k} - 2.125\text{M},\ 1.325\text{M}) \\
       &= \min(6.075\text{M},\ 1.325\text{M}) = \$1{,}325{,}000 \\
\CROC &= \min(8.25\text{M} - 50\text{k} - 2.125\text{M} - 1.325\text{M},\ 4.75\text{M}) \\
        &= \min(4.75\text{M},\ 4.75\text{M}) = \$4{,}750{,}000 \\
\CSBC &= 8.25\text{M} - 50\text{k} - 2.125\text{M} - 1.325\text{M} - 4.75\text{M} = \$0
\end{align*}
}

\paragraph{Verification}
$50{,}000 + 2{,}125{,}000 + 1{,}325{,}000 + 4{,}750{,}000 + 0 = \$8{,}250{,}000 = D$ \quad \checkmark

\paragraph{Per-unit summary}

\begin{table}[H]
\centering\small\sffamily
\caption{Worked example --- per-unit distribution components}
\begin{tabularx}{\textwidth}{@{}L{5cm} R{2.5cm} R{2.5cm} X@{}}
\rowcolor{navy}\textcolor{white}{\bfseries Component} & \textcolor{white}{\bfseries Total} & \textcolor{white}{\bfseries Per Unit} & \textcolor{white}{\bfseries Tax Treatment for Unitholder} \\
\midrule
$\COI$ (Ordinary income) & \$50,000 & \$0.0044 & Assessable at marginal rate \\
\rowcolor{warmgray}
$\CCG$ (Net capital gain) & \$2,125,000 & \$0.1864 & Assessable at marginal rate (may apply own Div~115 discount) \\
$\CTD$ (Tax deferred) & \$1,325,000 & \$0.1162 & Not assessable now; reduces unit cost base \\
\rowcolor{warmgray}
$\CROC$ (Return of capital) & \$4,750,000 & \$0.4167 & Not assessable; reduces unit cost base \\
\textbf{Total $D$} & \textbf{\$8,250,000} & \textbf{\$0.7237} & \\
\midrule
$\Cdisc$ (CGT discount --- separate) & \$2,125,000 & \$0.1864 & Reported only; not included in $D$ \\
\bottomrule
\end{tabularx}
\end{table}

\paragraph{Interpretation.}  Of the \$0.7237 per unit distributed, only \$0.1908/unit is currently taxable (\$0.0044 + \$0.1864).  The remaining \$0.5329/unit (\$0.1162 tax deferred + \$0.4167 ROC) reduces the unitholder's cost base.  This is because the distribution (\$8.25M) far exceeds the trust's net taxable income (\$2.175M), with the excess arising from: (a) the debt repayment being funded from sale proceeds before distribution (creating the ROC component), and (b) Div~40/43 deductions exceeding their accounting equivalents (creating the tax-deferred component).


% ═══════════════════════════════════════════════════════════════════════
% PART I — DECISION TREE AND SUMMARY
% ═══════════════════════════════════════════════════════════════════════

\subsubsection{Summary: Determining Components --- Decision Tree}

\begin{enumerate}[nosep]
    \item \textbf{Calculate $\TOI$, $\NCG$, $\NetTI$} using the formulae above (Steps~1--4).
    \item \textbf{Calculate $AP^+$, $\TDmax$, $\ROCmax$} (Step~5).
    \item \textbf{Compare $D$ to $\NetTI$ and $AP^+$:}
        \begin{enumerate}[nosep]
            \item If $D \leq \NetTI$: all of $D$ is taxable ($\COI$, $\CCG$).  No $\CTD$, no $\CROC$.
            \item If $\NetTI < D \leq AP^+$: taxable portion = $\NetTI$; excess = $\CTD = D - \NetTI$.
            \item If $D > AP^+$ and $AP^+ \geq \NetTI$: taxable = $\NetTI$; tax deferred = $AP^+ - \NetTI$; ROC = $D - AP^+$.
            \item If $AP^+ < \NetTI$ (unusual --- accounting profit lower than taxable, e.g., fair value write-down): $\CTD = 0$; $\CROC = \max(0, D - \NetTI)$ if $D > \NetTI$.
        \end{enumerate}
    \item \textbf{Apply formulae (\ref{eq:coi})--(\ref{eq:csbc})} to derive each component.
    \item \textbf{Report $\Cdisc$ separately} on the AMMA / distribution statement.
    \item \textbf{Verify} $\COI + \CCG + \CTD + \CROC + \CSBC = D$.
    \item \textbf{Reconcile $\CCG$ to the CGT calculation file} for the year, explaining any differences per the reconciliation cases above.
    \item \textbf{Obtain tax agent sign-off} before issuing distribution statements to unitholders.
\end{enumerate}

\begin{table}[H]
\centering\scriptsize\sffamily
\caption{Distribution component determination --- summary matrix by scenario}
\begin{tabularx}{\textwidth}{@{}L{2.8cm} C{1.2cm} C{1.2cm} C{1.2cm} C{1.2cm} C{1.2cm} L{2.2cm}@{}}
\rowcolor{navy}
\textcolor{white}{\bfseries Scenario} & \textcolor{white}{\bfseries C\textsubscript{OI}} & \textcolor{white}{\bfseries C\textsubscript{CG}} & \textcolor{white}{\bfseries C\textsubscript{TD}} & \textcolor{white}{\bfseries C\textsubscript{ROC}} & \textcolor{white}{\bfseries C\textsubscript{disc}} & \textcolor{white}{\bfseries Unitholder tax outcome} \\
\midrule
$D = \NetTI$ & Yes & Yes & Nil & Nil & Separate & Tax matches cash \\
\rowcolor{warmgray}
$D < \NetTI$ & Yes & Yes & Nil & Nil & Separate & Tax exceeds cash --- shortfall \\
$\NetTI < D \leq AP^+$ & Yes & Yes & Yes & Nil & Separate & Deferred tax on future unit sale \\
\rowcolor{warmgray}
$D > AP^+$ & Yes & Yes & Yes & Yes & Separate & Cost base reduced; gain on disposal if CB $\to$ zero \\
$AP < 0$, $\NetTI > 0$ & Yes & Yes & Nil & Possible & Separate & No timing difference; ROC if $D > \NetTI$ \\
\rowcolor{warmgray}
$TL$ eliminates $\TOI$ & Nil & Yes & Possible & Possible & Separate & Capital-only distribution \\
$D = 0$ (no dist.) & Nil & Nil & Nil & Nil & Nil & Tax on trust income under s97; no cash \\
\bottomrule
\end{tabularx}
\end{table}


\subsection{Distribution Types, Waterfall Mechanics, Legal and Tax Framework}
\label{sec:est-distribution-waterfall}
\legbox{s97 ITAA~1936 \legalcite{s97}{sec:ref-wht}; Division~115 ITAA~1997 \legalcite{Div~115}{sec:ref-wht}; TR~2012/D1 \legalcite{TR~2012/D1}{sec:ref-wht}; s104-71 ITAA~1997 (tax deferred) \legalcite{s104-71}{sec:ref-sta}.}
% ============================================================================

This section supplements the distribution component determination framework in Section~\ref{sec:est-distribution-components} by providing a comprehensive treatment of the \textit{types} of distributions available to the fund, the \textbf{distribution waterfall} that governs the priority and sequencing of distributions, and the legal and tax issues associated with each.

\subsubsection{Types of Distributions}

\begin{table}[H]
\centering\small\sffamily
\caption{Distribution types --- description and accounting treatment}
\begin{tabularx}{\textwidth}{@{}L{2.5cm} X L{3.2cm}@{}}
\rowcolor{navy}\textcolor{white}{\bfseries Type} & \textcolor{white}{\bfseries Description} & \textcolor{white}{\bfseries Account Treatment} \\
\midrule
\textbf{Cash distribution} & Most common form. The fund pays a cash amount to each unitholder proportionate to their unit holding. Funded from operating cash (NAB \acct{685}). Declared by trustee resolution; paid on payment date. & DR \acct{882.03} / CR \acct{685} on payment \\
\rowcolor{warmgray}
\textbf{Distribution Reinvestment Plan (DRP)} & The unitholder elects to reinvest their distribution by receiving additional units at NAV per unit instead of cash. No cash leaves the fund. The distribution is ``paid'' via unit allotment. Rare in closely held unlisted trusts. & DR \acct{882.03} / CR \acct{500} (Capital) on allotment \\
\textbf{In-specie distribution} & The fund distributes an asset (e.g., real property, shares, an interest in another investment) rather than cash. The asset is transferred to unitholders at market value. Rarely used and typically requires a specific Constitution authority and unanimous unitholder consent. & DR \acct{882.03} / CR Asset Account at carrying value; difference to gain/loss \\
\rowcolor{warmgray}
\textbf{Return of capital (ROC)} & A distribution of $\CROC$ component (exceeds both accounting profit and taxable income). A literal return of unitholders' invested capital. Not assessable income; reduces unitholder's cost base. Used when the fund has no distributable income but wishes to return excess cash. & DR \acct{500} --- Capital / CR \acct{685} (effectively a capital return) \\
\textbf{Interim distribution} & A distribution made during the income year (e.g., monthly, quarterly). Based on estimated taxable income. Component breakdown is estimated; reconciled at year-end when final tax position is known. & DR \acct{882.03} / CR \acct{685} on each interim payment \\
\rowcolor{warmgray}
\textbf{Final distribution} & A distribution made at or after year-end to match the fund's final taxable income. May be the only distribution if the fund distributes annually. For 22 Constance Street: the final distribution is expected at or around settlement when the property is sold and the fund is being wound down. & DR \acct{882.03} / CR \acct{685} \\
\textbf{Capital proceeds distribution} & A distribution of capital proceeds from the sale of the investment property. The net proceeds (after loan repayment and costs) are returned to unitholders. A large portion will be $\CROC$ (return of original capital contribution) and $\CCG$ (net assessable capital gain). & DR \acct{882.03} / CR \acct{685} (combined capital + income proceeds) \\
\rowcolor{warmgray}
\textbf{Special / extraordinary distribution} & An ad-hoc distribution outside the normal distribution cycle, declared for a specific reason (e.g., insurance proceeds, unexpected cash surplus). Must be supported by a trustee resolution. Component breakdown must be determined per Section~\ref{sec:est-distribution-components}. & DR \acct{882.03} / CR \acct{685} \\
\bottomrule
\end{tabularx}
\end{table}

\subsubsection{Distribution Frequency and Legal Obligations}

The trust's Constitution governs distribution frequency and timing. For 22 Constance Street:

\begin{itemize}[nosep]
    \item \textbf{Constitution:} The Constitution should be reviewed for the prescribed distribution periods (typically: quarterly or annual for unlisted wholesale property trusts), the mechanism for determining distributable income, and the trustee's discretion to withhold distributions for risk management purposes.
    \item \textbf{Minimum distribution for income tax purposes (s97 position):} There is no mandatory minimum distribution under the ITAA for trusts (unlike companies for franking purposes). However, if the fund does \textbf{not} distribute at least the trust net income ($\NetTI$) in each tax year, unitholders will be assessed on their proportionate share of $\NetTI$ regardless --- creating phantom income. This is the ``distribution shortfall'' risk described in Section~\ref{sec:est-distribution-components}.
    \item \textbf{Trust deed vs. trustee discretion:} Many trust deeds give the trustee discretion as to the amount and timing of distributions (subject to a maximum of distributable income). Where the trustee has absolute discretion, the trust is at greater risk of tax challenges (trust stripping, s100A). For a fixed unit trust like 22 Constance Street, the distribution entitlement is proportionate to units held --- no streaming discretion.
    \item \textbf{Present entitlement:} Under s97 ITAA~1936, a unitholder's assessable trust income is limited to amounts to which they are ``presently entitled'' at year-end. For a fixed unit trust, present entitlement arises from the trust deed and is proportionate to unit holding. The declaration of distribution is the trustee's formal confirmation of entitlement.
\end{itemize}

\subsubsection{Distribution Waterfall --- Concept and Mechanics}

The ``waterfall'' describes the \textbf{priority order} in which available distributions are allocated across different classes of recipients (or tiers of return). A waterfall is most commonly encountered in private equity and real estate fund structures where the manager is entitled to a performance fee or carried interest only after investors have received a preferred return.

For 22 Constance Street Unit Trust, the relevant waterfall mechanics are embedded in the \textbf{fee structure} (Section~\ref{sec:mgmtfee}) and the order in which operating returns, management fees, and performance fees are applied before calculating the residual available for unitholder distribution. The fund's Constitution should be consulted for the definitive waterfall provisions.

\paragraph{Standard Real Estate Fund Waterfall --- Four-Tier Structure}

Most private real estate fund structures (including many Australian unlisted property trusts) follow a four-tier waterfall:

\begin{table}[H]
\centering\small\sffamily
\caption{Four-tier distribution waterfall --- standard structure}
\begin{tabularx}{\textwidth}{@{}C{0.7cm} L{3.5cm} X@{}}
\rowcolor{navy}\textcolor{white}{\bfseries Tier} & \textcolor{white}{\bfseries Priority} & \textcolor{white}{\bfseries Description} \\
\midrule
1 & \textbf{Return of Capital} & Investors (unitholders) receive back 100\% of their invested capital before any profit is shared. For 22 Constance Street: all \$11,438,500 of capital contributions (\acct{500}) must be returned to unitholders first. \\
\rowcolor{warmgray}
2 & \textbf{Preferred Return (Hurdle)} & Investors receive a cumulative preferred return on their unreturned capital, calculated at a specified hurdle rate. For 22 Constance Street: 8\% IRR hurdle per the fee structure (Section~\ref{sec:mgmtfee} --- Performance Fee Tier 1 applies above 8\% IRR). Below the 8\% IRR threshold, 100\% of returns go to unitholders (no performance fee). \\
3 & \textbf{Manager Catch-Up} & Once the preferred return hurdle is reached, the manager (CPG) receives a disproportionate share of the next tranche of returns until the manager has caught up to its pro-rata target share of total profits. For some structures: 100\% to manager until parity; for others: 80/20 split during catch-up. Confirm the specific catch-up provisions in the Constitution. \\
\rowcolor{warmgray}
4 & \textbf{Carried Interest / Performance Fee} & Remaining returns split between investors and manager per the carried interest/performance fee rate. For 22 Constance: Tier 1 (8--15\% IRR): 20\% performance fee to CPG, 80\% to unitholders. Tier 2 ($>$15\% IRR): 40\% performance fee to CPG, 60\% to unitholders. \\
\bottomrule
\end{tabularx}
\end{table}

\paragraph{Waterfall Calculation --- Worked Example for 22 Constance Street}

Assume: total capital invested \$11,438,500; total net proceeds on sale and wind-up \$16,000,000 (hypothetical); total holding period 7 years.

\begin{enumerate}[nosep]
    \item \textbf{Tier 1 --- Return of Capital:} Unitholders receive \$11,438,500. Remaining proceeds: \$16,000,000 $-$ \$11,438,500 = \$4,561,500.
    \item \textbf{Tier 2 --- Preferred Return (8\% IRR):} Calculate the cumulative preferred return on invested capital over the holding period. At 8\% IRR compound annually for 7 years on \$11,438,500, the total cumulative return required = approximately \$3,000,000 (indicative; must be calculated on actual cash flows per an IRR model). If \$4,561,500 exceeds this, the hurdle is met.
    \item \textbf{Tier 3 --- Manager Catch-Up:} If the Constitution includes a catch-up, CPG receives a disproportionate share of the next tranche until it holds its share of the ``carried interest'' pool. Confirm the catch-up mechanics with CPG.
    \item \textbf{Tier 4 --- Carried Interest:} Any remaining proceeds above the hurdle and catch-up: split 80\% to unitholders and 20\% to CPG (Tier 1) or 60\%/40\% (Tier 2 if total IRR exceeds 15\%).
\end{enumerate}

\begin{note}[IRR vs. Cash-on-Cash Return — Waterfall Threshold Basis]
\textbf{IRR vs. cash-on-cash return:} The waterfall thresholds (8\% and 15\%) are based on Internal Rate of Return (IRR) --- the compound annual return considering the \textit{timing} of all cash flows (capital contributions, interim distributions, final distribution). A simple cash multiple or yield is not the same as IRR. The Preparer must use an IRR model (such as the fund's financial model or a dedicated Excel IRR calculation) that incorporates all actual cash flows with their actual dates to determine whether the IRR threshold has been reached. Do not approximate IRR with a simple interest calculation.
\end{note}

\subsubsection{Legal Issues --- Distribution Declarations and Trustee Obligations}

\begin{itemize}[nosep]
    \item \textbf{Trustee resolution:} Every distribution must be authorised by a formal written resolution of the trustee (22 Constance St Pty Ltd). The resolution must specify: the amount per unit (or total amount), the record date, the payment date, and the income year to which it relates.
    \item \textbf{Unitholders' rights:} Fixed unit trust unitholders have a vested, proportionate entitlement to the trust income per their unit holding. The trustee cannot selectively vary the entitlement between unitholders of the same class.
    \item \textbf{Breach of trust:} A trustee who pays distributions in excess of distributable income (without Constitution authority for ROC distributions) or who pays distributions to the wrong unitholders is in breach of trust and personally liable for any loss.
    \item \textbf{Freezing of distributions in insolvency:} If the trust is insolvent (liabilities exceed assets), distributions must not be made to unitholders until all creditors are paid. The trustee risks personal liability and breach of duty if distributions are made while insolvent.
\end{itemize}

\subsubsection{Tax Issues --- Distribution Declarations}

\begin{itemize}[nosep]
    \item \textbf{Record date:} The record date determines which unitholders are entitled to the distribution. For s97 ITAA~1936 purposes, the present entitlement arises at the year-end (30 June) or, for interim distributions, at the declared record date.
    \item \textbf{s100A --- Reimbursement agreements:} Where a distribution is declared but arrangements exist such that the unitholder does not benefit (e.g., the distribution is used to offset a related-party loan or the proceeds are immediately returned to the trust), s100A ITAA~1936 may apply to treat the unitholder as never having been presently entitled. This is a significant anti-avoidance provision. Confirm all distributions represent genuine economic benefits to unitholders.
    \item \textbf{GST on distributions:} Cash distributions and capital returns from a trust to its members are not subject to GST (they are not consideration for a supply). No GST adjustment is required on distribution journal entries.
    \item \textbf{Performance fee --- expense vs. distribution:} The performance fee paid to CPG is a \textbf{trust expense} (DR \acct{303.05} / CR \acct{882.16}) --- it is NOT a distribution to CPG as a beneficiary. It is a contractual fee for services. It reduces the fund's distributable income available to unitholders.
    \item \textbf{Interim distribution --- component reconciliation:} Interim distributions are based on estimated components. At year-end, if the estimated components differ from the actual final components (e.g., the preliminary estimate of $\COI$ vs. $\CTD$ is revised when the final Div~40 depreciation schedule is completed), the tax statement issued to unitholders must reflect the \textbf{actual final components} for the full year, not the interim estimates. Issue a corrected tax statement if the interim estimate was materially wrong.
\end{itemize}

\begin{table}[H]
\centering\small\sffamily
\caption{Distribution checklist --- pre-payment controls}
\begin{tabularx}{\textwidth}{@{}C{0.6cm} L{9cm} L{3cm}@{}}
\rowcolor{navy}\textcolor{white}{\bfseries\#} & \textcolor{white}{\bfseries Check} & \textcolor{white}{\bfseries Owner} \\
\midrule
1 & Trustee resolution formally declares distribution, record date, and payment date & CPG / Trustee \\
\rowcolor{warmgray}
2 & Distribution amount confirmed against distributable income (not to exceed distributable income unless ROC is Constitution-authorised) & Preparer \\
3 & Component breakdown ($\COI$, $\CCG$, $\CTD$, $\CROC$) calculated per Section~\ref{sec:est-distribution-components} and reviewed by 2nd Reviewer & Preparer / Reviewer \\
\rowcolor{warmgray}
4 & Register snapshot taken at record date; each unitholder's entitlement calculated & Registry / Preparer \\
5 & TFN/ABN status confirmed for all unitholders; 47\% withholding applied where TFN absent & Registry \\
\rowcolor{warmgray}
6 & Foreign residency status confirmed; non-resident WHT and MIT WHT calculated where applicable & Registry / Tax Agent \\
7 & EFT payment file prepared and reviewed by 1st Reviewer before release & Preparer / 1st Reviewer \\
\rowcolor{warmgray}
8 & Distribution payable posted in accounting system: DR \acct{882.03} / CR \acct{685} (net) + CR WHT payable & Preparer \\
9 & Distribution statements issued to unitholders within 3 business days of payment & Registry \\
\rowcolor{warmgray}
10 & WHT remitted to ATO via next Activity Statement & Preparer / Tax Agent \\
11 & Unitholder cost base records updated for $\CTD$ and $\CROC$ components & Registry / Tax Agent \\
\bottomrule
\end{tabularx}
\end{table}

% ============================================================================

\subsection{Withholding Tax --- Framework, Kinds, and Compliance Obligations}
\label{sec:est-wht}
\legbox{s12-140 TAA~Sch~1 (TFN withholding) \legalcite{s12-140}{sec:ref-wht}; Subdiv~12-H TAA~Sch~1 (MIT~WHT) \legalcite{Subdiv~12-H}{sec:ref-wht}; s202 ITAA~1936 (TFN quotation) \legalcite{s202}{sec:ref-wht}; FATCA IGA \legalcite{FATCA}{sec:ref-wht}; OECD~CRS \legalcite{CRS}{sec:ref-wht}.}
% ============================================================================

Withholding tax (WHT) is a mechanism under which the trustee deducts tax from distributions (or investment income) \emph{at source} and remits it to the ATO on behalf of the recipient, before the net amount is paid. The trustee acts as the collecting agent; the underlying tax liability always belongs to the recipient unitholder. WHT obligations for a fixed-unit unlisted trust such as the 22 Constance Street Unit Trust arise across four distinct regimes, each with its own rate, trigger, reporting form, and payment cycle. Failure to withhold, or withholding at the wrong rate, creates personal liability for the trustee under the PAYG withholding provisions (Division~12 TAA~1953).

\begin{critical}[Withholding Tax — Personal Liability of the Trustee]
WHT obligations are the \textbf{trustee's personal liability}. If the trustee fails to withhold and remit correctly, the ATO may recover the unremitted WHT from the trustee directly, in addition to any penalties and interest. The trustee cannot rely on the unitholder having paid the underlying tax independently. A WHT check must be completed before \textbf{every} distribution is paid.
\end{critical}

\subsubsection{Types of Withholding Applicable to the Fund}

\paragraph{1. Tax File Number (TFN) Withholding (Part VA ITAA~1936; s12-140 TAA~1953)}

Where an Australian-resident unitholder has not provided their TFN (or ABN for business unitholders) to the trustee by the time a distribution is paid, the trustee \textbf{must withhold} at the top marginal rate plus Medicare Levy: currently \textbf{47\%} of the distribution. This applies irrespective of the unitholder's actual marginal tax rate.

\begin{table}[H]
\centering\small\sffamily
\caption{TFN withholding trigger and rate}
\begin{tabularx}{\textwidth}{@{}L{4.5cm} X@{}}
\rowcolor{navy}\textcolor{white}{\bfseries Item} & \textcolor{white}{\bfseries Detail} \\
\midrule
Legislative basis & Part~VA ITAA~1936; ss12-140 TAA~1953 \\
\rowcolor{warmgray}
Rate & 47\% of gross distribution (top rate + Medicare Levy as at 2026) \\
Trigger & Unitholder has not quoted a valid TFN or ABN before payment \\
\rowcolor{warmgray}
Exempt & Unitholder is exempt from quoting TFN (e.g., certain government entities) \\
Remittance & Via Activity Statement (IAS or BAS) in the reporting period of payment \\
\rowcolor{warmgray}
Account (ATO) & W3 --- Amounts withheld where no ABN/TFN quoted \\
Recovery by unitholder & Via income tax return --- the withheld amount is a credit against the unitholder's assessed tax liability \\
\rowcolor{warmgray}
Registry action & Before every distribution run, confirm TFN/ABN held on register for every unitholder. Flag missing TFNs to CPG and trustee for resolution before payment. \\
\bottomrule
\end{tabularx}
\end{table}

\begin{note}[No WHT Required for Common Fund Investments]
The trustee is NOT required to withhold at 47\% if the investment is held through a common fund (e.g., a pooled fund where the investment manager holds TFNs on behalf of investors). However, for a directly managed unlisted unit trust with a named unitholder register (as is the case for 22 Constance Street), the trustee must verify TFN/ABN status individually for each registered unitholder.
\end{note}

\paragraph{2. Non-Resident Withholding --- Ordinary Income Distributions (Division~11A ITAA~1936; s12-315 TAA~1953)}

Where a distribution of \textbf{ordinary income} (rental income, interest, management fees in the trust's hands) is made to a \textbf{foreign-resident unitholder}, the trustee must withhold at the applicable non-resident rate before remitting the net amount.

\begin{table}[H]
\centering\small\sffamily
\caption{Non-resident withholding --- ordinary income distributions}
\begin{tabularx}{\textwidth}{@{}L{4.5cm} X@{}}
\rowcolor{navy}\textcolor{white}{\bfseries Item} & \textcolor{white}{\bfseries Detail} \\
\midrule
Legislative basis & Division~11A ITAA~1936; s12-315 TAA~1953 \\
\rowcolor{warmgray}
Standard rate & 30\% of the gross assessable ordinary income component \\
Treaty-reduced rate & Lower rate if Australia has a Double Tax Agreement (DTA) with the unitholder's country of tax residency. Common DTA rates: USA 15\%, UK 15\%, Japan 15\%, Germany 15\%, New Zealand 15\%. Confirm applicable DTA before applying a reduced rate. \\
\rowcolor{warmgray}
Trigger & Unitholder is a non-resident of Australia (foreign resident for tax purposes) \\
Applicable component & Applies to the \textbf{ordinary income component} ($\COI$) of the distribution only. Capital components have different treatment (see MIT WHT below). \\
\rowcolor{warmgray}
Remittance & Via Activity Statement (IAS); generally monthly or quarterly \\
ATO form & PAYG withholding --- Non-resident withholding \\
\rowcolor{warmgray}
Annual reporting & PAYG Payment Summary --- Foreign Residents (or AMMA statement if AMIT) \\
Documentation required & Trustee must hold evidence of foreign residency (statutory declaration, foreign tax identification, passport copy); must retain DTA eligibility documentation \\
\bottomrule
\end{tabularx}
\end{table}

\paragraph{3. MIT Withholding on Fund Payments (Division~840-M ITAA~1997; s12-385 TAA~1953)}

The \textbf{Managed Investment Trust (MIT)} WHT regime is the most commercially significant regime for property funds. Where the trust qualifies as a MIT (see eligibility criteria below), distributions of \textbf{capital gains and other fund payments} to \textbf{foreign-resident unitholders} are subject to the MIT WHT rate (not the standard 30\% non-resident WHT rate). The MIT regime is significantly more favourable for certain foreign investors.

\begin{table}[H]
\centering\small\sffamily
\caption{MIT withholding --- fund payments to foreign residents}
\begin{tabularx}{\textwidth}{@{}L{4.5cm} X@{}}
\rowcolor{navy}\textcolor{white}{\bfseries Item} & \textcolor{white}{\bfseries Detail} \\
\midrule
Legislative basis & Division~840-M ITAA~1997; s12-385 to s12-395 TAA~1953 \\
\rowcolor{warmgray}
MIT eligibility tests & (1) Trust is a managed investment scheme under the \textit{Corporations Act 2001} or a unregistered managed investment scheme meeting prescribed criteria; (2) trustee is an Australian resident; (3) trust has a sufficient number of members (or qualifies as a wholesale fund); (4) trust is not a closely held trust exceeding prescribed thresholds. An unlisted unit trust held by a small number of closely related unitholders may \textbf{not} qualify. Confirm MIT status with the fund's tax agent annually. \\
Rate --- Non-DTA countries & 30\% (same as non-resident WHT) \\
\rowcolor{warmgray}
Rate --- DTA eligible countries (reduced) & 15\% for countries with a DTA with Australia that reduces the MIT WHT rate. Key reduced-rate countries: USA, UK, Canada, Germany, France, Netherlands, Switzerland, Japan, New Zealand, Singapore. \\
Rate --- DTA eligible countries (nil) & Some treaties may reduce to 0\% for certain sovereign wealth funds, pension funds, or government entities. \\
\rowcolor{warmgray}
Applies to & ``Fund payments'' = amounts from Australian sources that are not included in the fund's assessable income but are paid to foreign residents, including: (a) capital gains on disposal of Australian real property; (b) tax-deferred amounts; (c) return of capital above cost base. Also applies to ordinary income where MIT has made a s275-10 election. \\
Cross-income component & Where the distribution is a blended amount, the MIT WHT applies only to the MIT ``fund payment'' portion. The ordinary income component ($\COI$) distributed to a foreign resident remains subject to the standard non-resident WHT (s12-315), not MIT WHT. \\
\rowcolor{warmgray}
Election & The trustee may make a s275-10 choice to have the MIT WHT regime apply to all distributions (including ordinary income) to foreign residents. This simplifies administration but must be made by the trustee before the first distribution of the year. \\
Remittance & Via IAS; monthly if WHT $>$\$1,000 per month for most funds. \\
\rowcolor{warmgray}
Annual tax reporting & MIT Annual Statement (MITAS) or AMMA if the fund also qualifies as an AMIT. \\
\bottomrule
\end{tabularx}
\end{table}

\begin{warning}[MIT Eligibility Uncertainty for This Fund]
\textbf{22 Constance Street --- MIT eligibility uncertainty:} An unlisted unit trust with a small, closely-related unitholder group may not meet the MIT membership tests. If the trust does NOT qualify as a MIT, the more onerous standard non-resident WHT regime (30\% on all distributions to foreign residents, regardless of component type) applies. The fund's tax agent must confirm MIT eligibility before any WHT rate concession is applied to distributions to non-resident unitholders. Applying the lower MIT rate without confirmed eligibility creates a primary WHT underpayment liability for the trustee.
\end{warning}

\paragraph{4. Attribution MIT (AMIT) Withholding (Division~276 ITAA~1997)}

Where the fund elects into the AMIT regime, the WHT obligations follow the \textbf{attributed amounts} rather than actual cash distributions. For foreign-resident unitholders in an AMIT, the WHT is calculated on the attributed amounts communicated via the AMMA statement. Attribution variances in a subsequent year affect the next year's WHT calculation.

\paragraph{5. Interest Withholding (Division~11A ITAA~1936 --- Interest)}

If the fund pays interest to a foreign-resident lender (e.g., a foreign-owned entity provides a loan at interest), a 10\% withholding obligation applies to the gross interest payment (s128B ITAA~1936). This applies to the fund if there is a related-party loan from a non-resident entity, not to the St George facility (St George is an Australian bank). Confirm with the tax agent if CPG or a related non-resident entity has advanced any loans to the trust.

\subsubsection{WHT Calculation Process --- Step by Step}

\begin{enumerate}[nosep]
    \item \textbf{Pre-distribution unitholder review:} Before every distribution run, the Preparer (or Registry) must confirm for each unitholder: (a) Australian resident vs. foreign resident status; (b) TFN/ABN on register; (c) DTA eligibility if foreign resident; (d) MIT eligibility of the trust (confirm with tax agent annually).
    \item \textbf{Determine gross distribution per unitholder:} Total distribution amount $D$ per unit $\times$ units held = gross distribution entitlement per unitholder.
    \item \textbf{Apply TFN withholding:} For any Australian-resident unitholder without a TFN/ABN on file: WHT = Gross distribution $\times$ 47\%. Net payment = Gross $-$ WHT.
    \item \textbf{Apply non-resident WHT to ordinary income:} For foreign-resident unitholders: WHT$_{\OI}$ = $\COI$ per unitholder $\times$ applicable rate (30\% standard, or DTA rate). This applies to the ordinary income component of the distribution only.
    \item \textbf{Apply MIT WHT to fund payments:} For foreign-resident unitholders where trust qualifies as MIT: WHT$_{MIT}$ = (Distribution $-$ $\COI$ component) $\times$ applicable MIT rate (15\% DTA or 30\% non-DTA). Note: if MIT election made under s275-10, apply MIT rate to the entire distribution.
    \item \textbf{Net payment:} Net distribution paid to foreign unitholder = Gross distribution $-$ WHT$_{\OI}$ $-$ WHT$_{MIT}$.
    \item \textbf{Remittance:} Total WHT collected across all unitholders is remitted to the ATO via the next Activity Statement. Record: DR \acct{882.03} --- Distribution Payable / CR \acct{685} (net paid to unitholders) and CR \acct{255} or separate WHT payable account (ATO remittance). Confirm account coding with the tax agent.
    \item \textbf{Annual WHT reporting:} Issue tax statements (PAYG Payment Summaries or AMMA statements) to unitholders by 14 July following the income year.
\end{enumerate}

\begin{table}[H]
\centering\small\sffamily
\caption{WHT rate summary by unitholder type and distribution component}
\begin{tabularx}{\textwidth}{@{}L{3.8cm} L{2.5cm} L{2.5cm} X@{}}
\rowcolor{navy}\textcolor{white}{\bfseries Unitholder Category} & \textcolor{white}{\bfseries Ordinary Income ($\COI$)} & \textcolor{white}{\bfseries Capital / Fund Payment} & \textcolor{white}{\bfseries Notes} \\
\midrule
Australian resident --- TFN on file & Nil & Nil & Assessed via ITR \\
\rowcolor{warmgray}
Australian resident --- No TFN & 47\% & 47\% & Top rate; recoverable via ITR \\
Foreign resident --- Non-DTA & 30\% & 30\% & Standard non-resident rate \\
\rowcolor{warmgray}
Foreign resident --- DTA country (MIT eligible) & 15\% (or DTA rate) & 15\% (or DTA rate) & MIT WHT rate; confirm DTA \\
Foreign resident --- DTA country (non-MIT) & DTA rate & 30\% & Only MIT rate concession for fund payments \\
\rowcolor{warmgray}
Super fund (complying) & Nil & Nil (15\% rate in fund's own hands) & Trust distributes gross; fund pays own tax \\
SMSF & Nil & Nil & Same as super fund \\
\rowcolor{warmgray}
Foreign sovereign / pension fund & 0--15\% & 0--15\% & Confirm specific DTA treaty article \\
\bottomrule
\end{tabularx}
\end{table}

\subsubsection{Legal Obligations and Penalties}

The trustee's legal obligations under PAYG WHT are strict liability in nature. Key provisions:

\begin{itemize}[nosep]
    \item \textbf{s16-20 TAA~1953 (Duty to withhold):} The payer (trustee) must withhold an amount from a withholding payment and pay it to the Commissioner.
    \item \textbf{s16-70 TAA~1953 (Penalty for failure to withhold):} A penalty equal to the amount that should have been withheld applies automatically for failure to withhold --- no discretion to waive unless ``shortfall amount'' provisions apply.
    \item \textbf{s16-75 TAA~1953 (Penalty for failing to pay withheld amounts):} A 10\% annual General Interest Charge applies on overdue WHT from the day after the due date.
    \item \textbf{GIC (General Interest Charge, s8AAB ITAA~1953):} Applied from the due date of remittance to the actual payment date. Currently approximately 11--12\% p.a. (indexed quarterly).
    \item \textbf{Penalty units:} Administrative penalties apply for late lodgement of WHT reports and Activity Statements.
\end{itemize}

\begin{tip}[Primary WHT Risk — Foreign Unitholder Identification]
For the 22 Constance Street fund, where all current unitholders are believed to be Australian residents with TFNs on file, the primary risk is foreign-resident status at the time of the final distribution (sale proceeds distribution). Unitholders should be asked to re-confirm their residency status immediately before the final distribution at settlement, as the tax treatment is fundamentally different for foreign residents, and the WHT rate differential (0\% vs 15--30\%) is material on a distribution of this size.
\end{tip}

% ============================================================================

\subsection{Unitholder Management --- Kinds, Legal Framework, and Tax Treatment}
\label{sec:est-unitholders}
\legbox{AML/CTF Act 2006 (Cth) \legalcite{AML/CTF}{sec:ref-wht}; FATCA (US--Australia IGA) \legalcite{FATCA}{sec:ref-wht}; OECD~CRS \legalcite{CRS}{sec:ref-wht}; s202 ITAA~1936 \legalcite{s202}{sec:ref-wht}; s97 ITAA~1936 \legalcite{s97}{sec:ref-wht}.}
% ============================================================================

Unitholders are the beneficiaries of the 22 Constance Street Unit Trust. The trust's Constitution confers rights on unitholders proportionate to their unit holdings: rights to income distributions, capital distributions, voting, and transfer. Understanding the type of each unitholder is essential for correct tax treatment of distributions (see Section~\ref{sec:est-distribution-components}), WHT obligations (Section~\ref{sec:est-wht}), registry reporting (Section~\ref{sec:est-registry}), and AMIT eligibility analysis.

\subsubsection{Types of Unitholders}

\begin{table}[H]
\centering\small\sffamily
\caption{Unitholder types --- characteristics and tax treatment}
\begin{tabularx}{\textwidth}{@{}L{3.2cm} L{3.8cm} X@{}}
\rowcolor{navy}\textcolor{white}{\bfseries Type} & \textcolor{white}{\bfseries Key Characteristics} & \textcolor{white}{\bfseries Tax Treatment of Distributions} \\
\midrule
\textbf{Resident individual} & Natural person, Australian tax resident. Highest volume in retail unlisted trusts. & $\COI$ assessable at marginal rate (up to 47\% + Medicare Levy). $\CCG$ assessable; may apply personal Div~115 50\% discount where eligible. $\CTD$ reduces unit cost base (s104-71). $\CROC$ reduces cost base (no immediate tax). \\
\rowcolor{warmgray}
\textbf{Resident company} & Australian company (Pty Ltd, Ltd). Common for wholesale investors, SPVs, related-party investors. & $\COI$ assessable at 30\% (or 25\% base rate for small companies). Company \textbf{cannot} apply the Div~115 discount --- $\Cdisc$ passes through but the company cannot benefit. $\CCG$ fully assessable at the company rate. \\
\textbf{Complying superannuation fund (including SMSF)} & Regulated super fund under SIS Act 1993. Common investor in property income funds. & $\COI$ assessable at 15\% concessional rate. For assets held $>$12 months in the accumulation phase, $\CCG$ eligible for 33.33\% Div~115 discount (not 50\%). Capital gains in pension phase: tax-free (nil rate). $\CTD$ reduces cost base in the fund's hands. \\
\rowcolor{warmgray}
\textbf{Resident trust} & A discretionary trust or unit trust investing as a unitholder. Common for family group investments. & The trust is assessed on its trust net income under s97 ITAA~1936. Each component ($\COI$, $\CCG$, $\Cdisc$) flows through to the investing trust's beneficiaries with its character preserved for CGT discount purposes (Div~115 streaming). \\
\textbf{Partnership} & Less common; partnership unitholders require each partner to be assessed individually on their share. & Trust components flow to each partner proportionately. Each partner assessed on their share at their individual rate. Div~115 discount applies at the partner level. \\
\rowcolor{warmgray}
\textbf{Foreign resident individual} & Non-resident natural person; holds units in an Australian property trust. & Subject to Australian non-resident WHT (Section~\ref{sec:est-wht}). Taxable on Australian-sourced income. Normally assessed under s115-215 ITAA~1997 for CGT discount on Australian real property gains. \\
\textbf{Foreign resident company} & Non-resident company. Corporate foreign resident rate 30\%. No Div~115 discount. & Subject to Australian non-resident WHT. Fully assessed on Australian-sourced income and capital gains. MIT WHT regime may apply if fund is MIT-eligible (15\% DTA countries). \\
\rowcolor{warmgray}
\textbf{Nominee / custodian} & Unit registered in name of a custodian (e.g., a bank or fund administrator) on behalf of underlying investors. & WHT based on underlying beneficial owner's residency status, not the custodian. AML/KYC must trace through to beneficial owner. FATCA/CRS look-through applies. \\
\bottomrule
\end{tabularx}
\end{table}

\subsubsection{Unit Register --- Legal Requirements}

The fund's Constitution requires the trustee (22 Constance St Pty Ltd) to maintain a \textbf{register of unitholders}. For an unlisted unit trust that is not a registered managed investment scheme, the register is maintained by the trustee or its appointed registry agent and is not lodged with ASIC. However, the following obligations apply:

\begin{enumerate}[nosep]
    \item \textbf{Mandatory register contents:} Full name and address of each unitholder; number of units held; class of units (if applicable); date of acquisition; consideration paid; TFN/ABN (collected with consent under privacy legislation); any transfer history.
    \item \textbf{Accuracy:} The register must be kept current. Any change in unitholder details (address, name change on marriage/death, transfer, partial redemption) must be updated within a reasonable time after written notification.
    \item \textbf{Access:} The Constitution typically grants unitholders the right to inspect the register. The trustee must respond to written requests within a prescribed period (usually 5 business days).
    \item \textbf{Privacy:} Unitholder data is subject to the \textit{Privacy Act 1988} (Cth). TFN data is additionally subject to strict handling restrictions under Part~VA ITAA~1936 --- it may not be disclosed except for the purpose of withholding tax compliance.
    \item \textbf{Retention:} Register records must be retained for 7 years after the fund is wound up (\textit{Evidence Act 1995}; ATO record-keeping requirements s262A ITAA~1936).
\end{enumerate}

\subsubsection{Unitholder Onboarding --- AML/CTF Obligations}

The trustee and CPG (as fund manager) may have obligations under the \textit{Anti-Money Laundering and Counter-Terrorism Financing Act 2006} (AML/CTF Act 2006) where the fund is providing ``designated services'' (financial services including interests in certain financial products). Where AML obligations apply:

\begin{itemize}[nosep]
    \item \textbf{Customer Identification Procedure (CIP):} Verify the identity of each unitholder before or as soon as practicable after the unitholder's first investment. For individuals: photo ID + proof of address. For companies: ASIC extract + beneficial owner verification. For trusts: trust deed extract + trustee identity. For SMSFs: SMSF establishment deed + trustee identity + ATO ABN confirmation.
    \item \textbf{Beneficial ownership:} Where units are held by a company or trust, trace through to the natural person(s) with ultimate beneficial ownership ($>$25\% interest, or effective control).
    \item \textbf{Ongoing monitoring:} Monitor for unusual transactions; re-verify identity if there are material changes (e.g., new beneficial owner, address change to high-risk jurisdiction).
    \item \textbf{AML/CTF Program:} The entity providing designated services must have a documented AML/CTF Program and appoint an AML/CTF Compliance Officer.
    \item \textbf{AUSTRAC:} Suspicious matter reports (SMRs) and threshold transaction reports (TTRs, for cash transactions $\geq$\$10,000) must be submitted to AUSTRAC as required.
\end{itemize}

\begin{note}[AML Obligations for Wholesale Unlisted Unit Trust]
For a wholesale unlisted unit trust such as 22 Constance Street where all unitholders are sophisticated investors known to the trustee, AML obligations may be streamlined through simplified due diligence. However, \textbf{foreign-resident unitholders always require full enhanced due diligence}, including confirmation that the source of investment funds does not originate from a high-risk jurisdiction.
\end{note}

\subsubsection{FATCA and CRS --- Foreign Resident Reporting}

\begin{table}[H]
\centering\small\sffamily
\caption{FATCA and CRS obligations for unitholder reporting}
\begin{tabularx}{\textwidth}{@{}L{3cm} X@{}}
\rowcolor{navy}\textcolor{white}{\bfseries Regime} & \textcolor{white}{\bfseries Obligation} \\
\midrule
\textbf{FATCA} (Foreign Account Tax Compliance Act --- USA) & Where the fund holds accounts of US Persons (US citizens, US tax residents, or certain US-controlled entities), the fund (or its Australian Financial Institution status) must report account details to the ATO, which passes them to the IRS under the Australia--US IGA. Report annually using FATCA XML format to the ATO. \\
\rowcolor{warmgray}
\textbf{CRS} (Common Reporting Standard --- OECD) & Requires Australian Financial Institutions (which may include the fund if it qualifies as an Investment Entity) to identify account holders who are tax residents of other CRS jurisdictions, and report their account balances, income received, and proceeds from redemptions annually to the ATO (who exchanges with the relevant foreign tax authority). Over 100 countries participate in CRS. \\
Classification test & Whether the 22 Constance Street fund is an ``Investment Entity'' for FATCA/CRS purposes depends on whether its gross income is primarily ($>$50\%) from financial asset management and investment activities. A property-owning trust may qualify. Confirm classification with the fund's compliance adviser. \\
\rowcolor{warmgray}
Due diligence & Must obtain self-certification from all unitholders regarding their tax residency. For pre-existing unitholders, obtain by the next reporting period. \\
Reporting deadline & FATCA: 31 July each year (for prior year). CRS: 31 July each year. \\
\rowcolor{warmgray}
Penalty & Up to \$250,000 per offence under the \textit{Tax Administration Act 1953} and \textit{Income Tax Assessment Act 1997} for failure to report. \\
\bottomrule
\end{tabularx}
\end{table}

\subsubsection{Transfer of Units --- Legal and Tax Issues}

Unit transfers in an unlisted unit trust require the written consent of the trustee (under most standard Constitutions). Key legal and tax issues on transfer:

\begin{itemize}[nosep]
    \item \textbf{Stamp duty:} Transfer of units in a unit trust that holds real property may attract \textbf{landholder duty} in Queensland (under the \textit{Duties Act 2001} (Qld)). Where the trust holds Queensland land with an unencumbered value $\geq$\$2,000,000 (which 22 Constance Street far exceeds), the acquisition of a ``significant interest'' (generally $\geq$50\% in an unlisted trust, or $\geq$90\% in a listed trust) by a person triggers duty on the acquisition of the proportionate land value. Obtain a stamp duty opinion from K\&L Gates before approving any unit transfer.
    \item \textbf{CGT to transferor:} The transfer is a CGT disposal (Event A1) for the outgoing unitholder. The capital gain = sale price of units $-$ cost base of units (reduced by prior $\CTD$ amounts received). Div~115 discount available if units held $>$12 months.
    \item \textbf{Cost base to transferee:} The incoming unitholder's cost base in the units = consideration paid = market value at transfer (arm's-length transaction). If not arm's-length, market value substitution rules may apply (s116-30 ITAA~1997).
    \item \textbf{Pre-transfer distribution:} If a distribution is declared before transfer but paid after, the distributable amount belongs to the unitholder on the register at the record date (not the payment date). Confirm record date for all transfers in progress at distribution time.
\end{itemize}

\subsubsection{s97 Assessment vs. AMIT Attribution --- Unitholder Tax Positions}

\begin{table}[H]
\centering\small\sffamily
\caption{Comparison: s97 trust regime vs.\ AMIT attribution for unitholders}
\begin{tabularx}{\textwidth}{@{}L{3.5cm} X X@{}}
\rowcolor{navy}\textcolor{white}{\bfseries Feature} & \textcolor{white}{\bfseries s97 Trust Regime (non-AMIT)} & \textcolor{white}{\bfseries AMIT Regime} \\
\midrule
Tax basis & Unitholder assessed on proportionate share of ``trust net income'' under s97, regardless of actual distribution received & Unitholder assessed on ``attributed amounts'' per AMMA statement; attributed amounts may differ from cash distributions \\
\rowcolor{warmgray}
Shortfall risk & If $D < \NetTI$, the unitholder pays tax on income not received in cash (``phantom income'' risk) & Attribution amounts are fixed on AMMA; timing differences resolved through attribution variances in subsequent year \\
Over-distribution & If $D > \NetTI$: excess is tax-deferred or ROC --- no immediate tax to unitholder & AMIT over-attribution resolved via next-year variance \\
\rowcolor{warmgray}
Amendments & If trust's tax return is amended, unitholders must each individually amend their own returns & Under AMIT, amendment of AMMA statement by trust re-attributes to unitholders for the relevant year; simpler process \\
Annual statement & Tax statement with components ($\COI$, $\CCG$, $\Cdisc$, $\CTD$, $\CROC$) & AMMA (Attribution MIT Member Annual Statement) \\
\rowcolor{warmgray}
Entitlement & Distribution must equal ``present entitlement'' each year or s97 position may be challenged & Attribution amounts may be less than or greater than cash distributions without risk of double tax \\
Applies to 22 Constance & Yes (current position) & Only if MIT eligibility confirmed and AMIT election made \\
\bottomrule
\end{tabularx}
\end{table}

% ============================================================================

\subsection{Registry Services --- Types, Legal Obligations, and Tax Compliance}
\label{sec:est-registry}
% ============================================================================

Registry services refer to the administrative functions required to maintain the fund's unitholder register, process unit transactions, prepare distribution statements, and meet legal reporting obligations to unitholders and regulators. For the 22 Constance Street Unit Trust, registry is currently managed as part of the fund administration service by UFS, with CPG maintaining records in Yardi (\textbf{Unitholders} tab) for the unit register and equity ledger. The account code for registry fees is \acct{301.03} --- Registry Fees.

\subsubsection{Types of Registry Services}

\paragraph{1. Unit Register Maintenance}

The most fundamental registry function. Encompasses: maintaining an accurate, up-to-date register of all unitholders (name, address, units held, date of acquisition, consideration, TFN/ABN); recording every change to unitholder details in writing; maintaining the historical register for audit and ATO record-keeping purposes; and providing the trustee with a signed certification of the register at year-end.

\begin{table}[H]
\centering\small\sffamily
\caption{Registry services --- types and legal obligations}\label{tab:registry-services}
\begin{tabularx}{\textwidth}{@{}L{2.8cm} X L{2.8cm}@{}}
\rowcolor{navy}\textcolor{white}{\bfseries Service Type} & \textcolor{white}{\bfseries Description} & \textcolor{white}{\bfseries Legal Basis} \\
\midrule
Register maintenance & Maintain current register; process all changes in writing within 5 business days & Trust Constitution; \textit{Evidence Act 1995} \\
\rowcolor{warmgray}
Subscription processing & Process new unit applications: collect application forms, verify AML/KYC, allot units, issue holding statements & \textit{AML/CTF Act 2006}; trust Constitution \\
Transfer processing & Process unit transfer forms; confirm trustee consent; update register; issue updated holding statement to transferee & \textit{Duties Act 2001} (Qld); Constitution \\
\rowcolor{warmgray}
Redemption / withdrawal & Process redemption requests per the Constitution; confirm trustee approval; calculate redemption price; process payment; update register & Trust Constitution \\
Distribution payment processing & Calculate entitlements per register; apply WHT where required; arrange EFT payments; reconcile distribution payable account & Division~12 TAA~1953; Part~VA ITAA~1936 \\
\rowcolor{warmgray}
Tax statement preparation & Prepare and issue distribution tax statements or AMMA statements to unitholders by 14 July each year & s16-153 TAA~1953; Division~276 ITAA~1997 \\
AML/KYC services & Initial CIP verification on investor onboarding; ongoing monitoring; SMR/TTR filing to AUSTRAC & \textit{AML/CTF Act 2006} \\
\rowcolor{warmgray}
FATCA/CRS reporting & Collect tax residency self-certifications; identify reportable accounts; prepare and lodge FATCA/CRS reports with ATO annually & FATCA IGA; OECD CRS; Division~396 TAA~1953 \\
ASIC lodgements & Annual ASIC review fees (if applicable); change of trustee or constitution lodgements; any MIS regulatory obligations & \textit{Corporations Act 2001} (Cth) \\
\rowcolor{warmgray}
Unit pricing support & Maintain NAV per unit calculation; confirm unit price on each transaction date for subscriptions and redemptions & Trust Constitution; ASIC RG~94 (if applicable) \\
Investor communications & Dispatch NAV reports, monthly reports, distribution statements, capital call notices, and tax statements to unitholders & Trust Constitution \\
\rowcolor{warmgray}
Meeting administration & Prepare and dispatch notices of unitholder meetings; record minutes; manage proxy forms; maintain minute book & Trust Constitution \\
Holding statement issuance & Issue holding statements to unitholders on initial allotment, on every transfer, and annually upon request & Trust Constitution \\
\rowcolor{warmgray}
Register certification & Annual written certification by the registry agent that the register is complete, accurate, and compliant with the Constitution & Audit; trust Constitution \\
\bottomrule
\end{tabularx}
\end{table}

\paragraph{2. Distribution Payment Processing --- Detailed Procedure}

Distribution payment processing is the most time-sensitive registry function. At each distribution date, the registry must:

\begin{enumerate}[nosep]
    \item Obtain the trustee's formal resolution to declare the distribution (amount, record date, payment date, component breakdown).
    \item Take a certified copy of the register as at the record date.
    \item Calculate each unitholder's entitlement: units held $\times$ cents per unit = gross entitlement.
    \item Apply WHT calculations per Section~\ref{sec:est-wht} for each unitholder (TFN withholding, non-resident WHT, MIT WHT).
    \item Confirm bank account details for each unitholder. Where details are absent or changed, request updated authority before payment.
    \item Prepare an EFT payment file for approval by the trustee.
    \item Process payments and reconcile: DR \acct{882.03} --- Distribution Payable / CR \acct{685} NAB (net paid to unitholders) and CR WHT payable (ATO remittance).
    \item Issue distribution statements to unitholders within 3 business days of payment.
    \item Remit WHT to ATO via the next Activity Statement.
\end{enumerate}

\paragraph{3. Tax Statement Preparation --- Annual Obligations}

By 14 July following each income year, the registry must prepare and dispatch individual tax statements to each unitholder (or AMMA statements if AMIT election is in place). The statement must disclose, by component: $\COI$ (ordinary income); $\CCG$ (assessable capital gain); $\Cdisc$ (CGT discount component); $\CTD$ (tax deferred); $\CROC$ (return of capital); $\CSBC$ (CGT concession component, if applicable); gross foreign income if any; WHT amounts deducted and remitted; and the date and amount of each payment.

\begin{critical}[Tax Statements Are Legal Documents — Zero Error Tolerance]
Tax statements issued to unitholders are \textbf{legal documents} upon which each unitholder prepares their individual tax return. Any error in component classification creates an amendment obligation for both the trust (via the amended tax return) and every affected unitholder (via individual amended returns). Incorrectly categorising $\CTD$ as $\COI$ (over-assessment) or $\COI$ as $\CTD$ (under-assessment) is the most common error. All tax statements must be reviewed by the fund's tax agent before dispatch.
\end{critical}

\paragraph{4. Registry Fees --- Account Treatment}

Registry fees (\acct{301.03}) are trust-level administration costs accrued monthly to \acct{882.16} --- Fund Accruals:
\begin{itemize}[nosep]
    \item DR \acct{301.03} --- Registry Fees / CR \acct{882.16} --- Fund Accruals (per invoice or monthly estimate).
    \item GST: Registry services provided by an Australian supplier are standard-rated (10\% GST). Input tax credit claimable to the extent the fund makes taxable supplies.
    \item Tax: Registry fees are deductible under s8-1 ITAA~1997 as a cost incurred in managing the fund's investment income-producing activities. Deductible in the year incurred per TR~97/7.
\end{itemize}

\subsubsection{Legal and Regulatory Obligations in Registry}

\begin{table}[H]
\centering\small\sffamily
\caption{Registry --- key legal obligations and references}
\begin{tabularx}{\textwidth}{@{}L{3.5cm} X@{}}
\rowcolor{navy}\textcolor{white}{\bfseries Obligation} & \textcolor{white}{\bfseries Detail and Reference} \\
\midrule
Privacy & Unitholder data governed by \textit{Privacy Act 1988} (Cth) --- Australian Privacy Principles (APPs). TFNs specifically restricted: may only be used for tax purposes and must not be disclosed to any other party. Registry agents must have a Privacy Policy and a process for handling access and correction requests. \\
\rowcolor{warmgray}
Record retention & All registry records, transaction records, correspondence, and tax statements must be retained for 7 years after fund wind-up (s262A ITAA~1936). Digital records acceptable if ATO-compliant. \\
PPSR & Any security interest granted over units must be registered on the Personal Property Securities Register (PPSR) under the \textit{Personal Property Securities Act 2009} (Cth) for the security holder to perfect their interest. The trustee's approval of any pledge or charge over units must be noted on the register. \\
\rowcolor{warmgray}
Unclaimed monies & Where a distribution payment is returned undelivered or a unitholder cannot be located, the unclaimed amount may need to be reported and remitted to the relevant State or Territory government's unclaimed monies fund after the prescribed period of inactivity (typically 7 years). \\
Death of unitholder & On the death of a unitholder, units vest in the personal representative (executor or administrator). The registry must be updated on production of probate or letters of administration. The transfer of units from a deceased estate may trigger stamp duty assessment in Queensland. \\
\rowcolor{warmgray}
Bankruptcy / insolvency & On the bankruptcy of a unitholder, the trustee in bankruptcy is entitled to deal with the units. The registry must note the appointment of the trustee in bankruptcy on the register. \\
\bottomrule
\end{tabularx}
\end{table}

\legbox{AML/CTF Act 2006 (Cth) \legalcite{AML/CTF}{sec:ref-wht}; FATCA (US--Australia IGA) \legalcite{FATCA}{sec:ref-wht}; OECD~CRS \legalcite{CRS}{sec:ref-wht}; s202 ITAA~1936 (TFN quotation) \legalcite{s202}{sec:ref-wht}; \textit{Duties Act 2001} (Qld) --- transfer duty on unit transfers \legalciteonly{see \S\,\ref{sec:ref-loanswap}}; \textit{Corporations Act 2001} s169A (register maintenance) \legalciteonly{see \S\,\ref{sec:ref-hfs}}.}

% ============================================================================


% ============================================================================
% PART VII: CONTROLS, RECONCILIATION, AND OTHER PERIOD-END ITEMS
% ============================================================================

\vspace{1.2cm}
\begin{tcolorbox}[
    enhanced, breakable,
    colback=navy, colframe=navy,
    boxrule=0pt,
    borderline south={2pt}{0pt}{gold},
    sharp corners,
    left=14pt, right=10pt, top=10pt, bottom=10pt,
    label={sec:est-part-controls}
]
{\sffamily\bfseries\footnotesize\textcolor{gold}{Part VII\enspace\textbullet\enspace Appendix~C}}\par
\vspace{3pt}
{\sffamily\bfseries\large\textcolor{white}{Controls, Reconciliation, and Other Period-End Items}}
\end{tcolorbox}
\begin{tcolorbox}[
    enhanced, breakable,
    colback=palesky, colframe=navy!30!white,
    boxrule=0.4pt, sharp corners,
    left=10pt, right=10pt, top=6pt, bottom=6pt
]
\small\itshape This part addresses the double-count prevention framework for Yardi/UFS recording conflicts and residual period-end estimates and judgements not covered by the preceding parts.
\end{tcolorbox}
\vspace{0.3cm}

\subsection{Double Counting Prevention --- Yardi vs.\ UFS Recording Conflicts}
\label{sec:est-double-count}

Double counting is one of the most common and material errors in property trust accounting where two systems (Yardi and UFS) maintain overlapping records. The following patterns are the most frequent sources of double counting and the recommended checkpoints to prevent them.

\subsubsection{Pattern 1: Sales Settlement Statement Transactions}

Yardi does \emph{not} frequently record settlement statement transactions (vendor adjustments, disposal costs, settlement proceeds). UFS typically records these manually. However, CPG may occasionally record settlement items in Yardi \textbf{after} UFS has already posted them.

\textbf{Checkpoint:} Before posting any settlement-related journal, search the Yardi TB (\texttt{cs4001} and \texttt{22chts}) for matching entries. Compare: (a)~the Yardi TB movement for the period; (b)~the UFS manual journals posted; (c)~the settlement statement line items. Any item appearing in \textbf{both} Yardi and UFS journals is double-counted.

\subsubsection{Pattern 2: Outgoings Charged to Tenants vs.\ Expenses Paid}

This is the most insidious double-count pattern. The sequence is:
\begin{enumerate}[nosep]
    \item CPG records the outgoings charge to the tenant in Yardi → appears as \acct{662} Trade Receivable (debit) and \acct{71.01} Outgoings Recovery (credit) in the Yardi TB.
    \item When the \textbf{invoice} is received (e.g.\ water rates from Queensland Urban Utilities), it may also be recorded in Yardi as a property expense → appears as \acct{700.xx} Operating Expense (debit) and \acct{882.02} or \acct{883} (credit).
    \item When UFS \textbf{pays} the invoice via NAB, it records the payment → DR \acct{882.02} or \acct{883} / CR \acct{685} NAB.
\end{enumerate}

\textbf{Where double counting occurs:} If UFS records the \textbf{expense} again (from the NAB bank statement) without realising Yardi has already recorded the same expense in the TB import, the cost is double-counted in the P\&L. Alternatively, if the outgoings recovery is recorded \textbf{and} the expense is recorded but not offset, net income is understated.

\textbf{Checkpoint:} For each property operating expense line (\acct{700.xx}), verify:
\begin{itemize}[nosep]
    \item Is this expense already imported from Yardi (Category~A)?
    \item Has UFS posted a separate manual journal for the same invoice?
    \item Is the outgoings recovery (\acct{71.01}) correctly matched against the expense?
    \item Has the payment (NAB) been recorded against the existing accrual (\acct{882.02}), not as a new expense?
\end{itemize}

\subsubsection{Pattern 3: Invoice Received vs.\ Cash Payment}

When an invoice is received and accrued (DR Expense / CR \acct{882.02} Accrual), and subsequently the cash payment appears on the NAB bank statement, the payment should \textbf{clear the accrual} (DR \acct{882.02} / CR \acct{685} NAB), not create a second expense entry.

\textbf{Checkpoint:} For every NAB payment, determine: is this settling a \textbf{previously accrued} liability, or is this a \textbf{new} expense not yet recorded? If the expense was already accrued (in Yardi or UFS), the payment clears the accrual. If the expense is new (e.g.\ ad hoc repair), the payment creates the expense entry directly.

\subsubsection{Pattern 4: Accounts Receivable and Accounts Payable Overlap}

Double-counting risk arises when:
\begin{itemize}[nosep]
    \item A receivable is raised in Yardi (\acct{662}) for rent due, \textbf{and} UFS raises a separate receivable for the same tenant for a sundry debtor (Cases~2--5). Check that the Sundry Debtor entries do not duplicate Yardi rent receivables.
    \item A payable is recorded in Yardi (\acct{883}), \textbf{and} UFS posts an accrual in \acct{882.02} or \acct{882.16} for the same obligation. Before posting any manual accrual, check whether Yardi has already recorded the liability.
    \item Insurance claim income (\acct{73.02}) is recorded when the remittance is received in St George, \textbf{and} also imported from Yardi if CPG has recorded the claim in Yardi. Check the Yardi TB for any insurance income entries before posting the St George receipt.
\end{itemize}

\subsubsection{Pattern 5: Tax Reconciliation --- Non-Deductible Expenses vs.\ Amortisation Schedules}

The Tax Reconciliation tab links to the P\&L and to amortisation/accrual schedules. Double counting can occur when:
\begin{itemize}[nosep]
    \item A non-deductible expense in the P\&L (e.g.\ entertainment, depreciation) is added back in the tax reconciliation, \textbf{but} the corresponding amortisation schedule also feeds the tax tab separately. This creates a double add-back or double deduction.
    \item Accruals in \acct{882.16} are linked to the tax tab as estimated expenses, \textbf{and} the corresponding P\&L expense account is also linked. The same expense is captured twice.
    \item The STI (lease incentive) amortisation flows to both the P\&L and the tax schedule. Ensure the tax schedule uses only the \textbf{tax-deductible} amortisation amount (which may differ from the accounting amortisation under AASB~16).
\end{itemize}

\textbf{Checkpoint:} For each item in the Tax Reconciliation tab, trace the source: does it come from (a)~the P\&L directly, (b)~an amortisation schedule, or (c)~an accrual workpaper? If the same underlying transaction feeds through more than one path, it is double-counted.

\subsubsection{Master Double-Count Prevention Checklist}

\begin{table}[H]
\centering\small\sffamily
\caption{Double-count prevention checkpoints --- verify each period}
\label{tab:double-check}
\begin{tabularx}{\textwidth}{@{}C{0.6cm} X L{3.5cm}@{}}
\rowcolor{navy}\textcolor{white}{\bfseries\#} & \textcolor{white}{\bfseries Checkpoint} & \textcolor{white}{\bfseries Accounts at Risk} \\
\midrule
1 & Compare Yardi TB import totals (by account) against UFS manual journals for the same accounts. Any account with entries from \textbf{both} sources must be investigated. & All accounts \\
\rowcolor{warmgray}
2 & For each NAB payment, confirm whether it clears an existing accrual or creates a new expense. & \acct{882.02}, \acct{882.16}, \acct{700.xx} \\
3 & Cross-check outgoings recovery (\acct{71.01}) against the corresponding expense (\acct{700.xx}). The recovery should not exceed the expense, and the expense should not be recorded twice (once as recovery, once as cost). & \acct{71.01}, \acct{700.xx} \\
\rowcolor{warmgray}
4 & Before posting any settlement-related journal, search Yardi TB for matching entries from CPG. & \acct{709}, \acct{780}, disposal accounts \\
5 & Verify that Sundry Debtor entries (Cases~2--5) do not duplicate Yardi-imported receivables. & \acct{660}, \acct{662} \\
\rowcolor{warmgray}
6 & In the Tax Reconciliation tab, trace each line item to a single source (P\&L or schedule, not both). & Tax reconciliation \\
7 & Reconcile the Yardi TB total against the reporting pack TB total. The difference should equal only UFS manual journal entries. Any unexplained difference indicates a double-count or omission. & TB tab \\
\rowcolor{warmgray}
8 & For insurance claims, confirm the entry appears in \textbf{either} Yardi \textbf{or} UFS St George journals --- not both. & \acct{73.02}, \acct{680} \\
\bottomrule
\end{tabularx}
\end{table}

\begin{tip}[TB Reconciliation — Most Effective Double-Count Control]
The single most effective double-count control is the \textbf{TB reconciliation}: Yardi TB total $+$ UFS manual journals $=$ Reporting Pack TB total. If this equation does not hold, investigate every account where the Yardi balance plus UFS journals does not equal the pack balance.
\end{tip}

% ============================================================================

\subsection{Other Period-End Estimates and Judgements}
\label{sec:est-other}

The following additional items require periodic attention. These supplement the detailed guidance in Sections~\ref{sec:est-accruals-settlement}--\ref{sec:est-capex-guide} above.

\begin{enumerate}[nosep]
    \item \textbf{Lease incentive straight-lining (AASB~16):} If lease incentives (rent-free periods) are material, verify that the \textbf{STI calc / STI Adj} tabs correctly spread the incentive over the lease term. A change in lease term (renewal or early termination) requires recalculation of the straight-line adjustment. This also affects the tax reconciliation (Section~40-880 / 25--25 tabs).

    \item \textbf{Contingent liabilities (AASB~137):} At each reporting date, consider whether any contingent liabilities require disclosure, including: pending litigation, warranty claims from the sale process, environmental remediation obligations beyond current cyclone works, potential make-good obligations under existing leases, or bank guarantee calls.

    \item \textbf{Going concern assessment (AASB~101):} Given the approaching loan maturity (15 June 2026) and active sale process, the going concern assumption should be assessed at each period-end. If there is material uncertainty about the fund's ability to continue as a going concern (e.g.\ no refinancing secured and sale not completed), document this for Reviewer and CPG discussion. This may require disclosure in the financial statements.

    \item \textbf{Related party transactions (AASB~124):} CPG inter-group loan movements (\acct{882.08}), management fees (\acct{301.01}), and any disposal fee (\acct{301.04}) are related party transactions. Ensure these are disclosed appropriately and priced at arm's length or per the trust deed. The fee-to-metric dependencies (Section~\ref{sec:est-fee-size}) should be checked for arm's length pricing.

    \item \textbf{Post-balance date events (AASB~110):} After the reporting date but before pack submission, consider whether any significant events have occurred (e.g.\ sale exchange, tenant departure notice, insurance settlement, loan refinancing confirmation) that require adjustment (adjusting events) or disclosure (non-adjusting events) in the current period pack.

    \item \textbf{Impairment of receivables (AASB~9):} The expected credit loss model under AASB~9 may require a forward-looking provision for trade receivables, even where no specific debtor is identified as doubtful. Given the fund's tenant concentration risk, consider whether a general provision is warranted (see also Section~\ref{sec:est-baddebt}).

    \item \textbf{Segment reporting and fund wind-up considerations:} If the sale completes and the fund moves to wind-up, additional reporting obligations may arise. Prepare for potential final tax return, final distribution calculation, and wind-up accounts. Obtain CPG instruction on timing.
\end{enumerate}

\begin{note}[Appendix Scope — Professional Judgement Required]
This appendix is not exhaustive. The Preparer and Reviewers should exercise professional judgement to identify any additional accounting estimates or period-end considerations that arise from changes in the fund's circumstances. In particular, as the property sale progresses toward settlement, the accruals and provisions in Section~\ref{sec:est-accruals-settlement} will become increasingly critical. Any new item identified should be added to the Monthly NAV Procedures checklist (Section~\ref{sec:checklists}) and documented for future periods.
\end{note}

\legbox{AASB~110 \textit{Events After the Reporting Period} \legalcite{AASB~110}{sec:ref-accounting}; AASB~101 \textit{Presentation of Financial Statements} (going concern) \legalcite{AASB~101}{sec:ref-accounting}; AASB~137 \textit{Provisions, Contingent Liabilities and Contingent Assets} \legalcite{AASB~137}{sec:ref-accounting}; AASB~124 \textit{Related Party Disclosures} \legalcite{AASB~124}{sec:ref-accounting}; AASB~9 (ECL model) \legalcite{AASB~9}{sec:ref-accounting}; AASB~16 \textit{Leases} \legalcite{AASB~16}{sec:ref-accounting}.}



% ============================================================================

% ============================================================================
\section{Legislative and Standards References}
\label{app:references}
% ============================================================================

This section consolidates all legislative, regulatory, and accounting standards references cited throughout this SOP. Each entry identifies the provision or standard, describes its relevance to the fund's operations, cross-references the substantive treatment in this document, and indicates whether it is \textbf{currently applicable} to the 22~Constance Street Unit Trust (\applies) or retained for completeness only (\notapplies).

The references are organised into twelve categories: (1)~income tax deductions and timing; (2)~repairs vs.\ capital expenditure; (3)~tax depreciation (Div~40/43); (4)~capital gains tax; (5)~borrowing expenses and blackhole expenditure; (6)~withholding tax and unitholder obligations; (7)~loan swap, refinancing, and put/call structures; (8)~statutory fees, levies, and government obligations; (9)~insurance, GST, and bank guarantees; (10)~STA, distributions, and trust accounts; (11)~accounting standards (general); and (12)~held-for-sale and presentation standards.

\begin{note}[Legislative References — ITAA 1997 Unless Otherwise Stated]
All legislative references are to the \textit{Income Tax Assessment Act 1997} (ITAA~1997) unless otherwise stated. ATO rulings and determinations are accessible via \texttt{ato.gov.au}. AASB standards are issued by the Australian Accounting Standards Board and available at \texttt{aasb.gov.au}. References marked with a dagger~(\dag) were updated or supplemented by ATO guidance issued after 1~January~2024. Entries containing a {\sffamily\textcolor{critleft}{\bfseries\scriptsize$\triangleright$\,Deadline}} marker identify time-critical compliance obligations --- these represent filing dates, payment due dates, or application windows where late action attracts penalties or interest.
\end{note}

% ============================================================================
\subsection{Income Tax --- General Deductions and Timing}
\label{sec:ref-income-tax}
% ============================================================================

\begingroup
\small\sffamily
\setlength{\LTpre}{4pt}
\setlength{\LTpost}{4pt}
\renewcommand{\arraystretch}{1.2}
\begin{longtable}{@{}L{3.0cm} p{\dimexpr\linewidth-3.0cm-3.3cm-1.0cm-6\tabcolsep\relax} L{3.3cm} C{1.0cm}@{}}
\caption{Income tax deduction and timing references --- applicability to 22 Constance Street} \label{tab:ref-income-tax} \\
\rowcolor{navy}
\textcolor{white}{\bfseries Reference} &
\textcolor{white}{\bfseries Relevance to this SOP} &
\textcolor{white}{\bfseries SOP Section} &
\textcolor{white}{\bfseries\scriptsize\applies} \\
\midrule
\endfirsthead
\multicolumn{4}{@{}l@{}}{\small\itshape\textcolor{navy!70!black}{\tablename~\thetable{} (cont.)}} \\[2pt]
\rowcolor{navy}
\textcolor{white}{\bfseries Reference} &
\textcolor{white}{\bfseries Relevance to this SOP} &
\textcolor{white}{\bfseries SOP Section} &
\textcolor{white}{\bfseries\scriptsize\applies} \\
\midrule
\endhead
\midrule
\multicolumn{4}{@{}r@{}}{\itshape\textcolor{navy!70!black}{Continued on next page\ldots}} \\
\endfoot
\bottomrule
\endlastfoot
\textbf{s8-1 ITAA 1997} & General deduction: outgoings incurred in gaining or producing assessable income. Governs deductibility of all property operating expenses, management fees, interest on the St~George facility, legal costs, and most fund-level expenses. The ``nexus'' test requires the outgoing to be incurred for a purpose sufficiently connected to the production of assessable income. & See \S\ref{sec:est-opex-timing}; \S\ref{sec:est-cutoff} & \applies \\
\rowcolor{warmgray}
\textbf{TR~97/7\dag} (consolidated 2~July~2025; incl.\ TR~97/7A2 addendum) & ATO's primary ruling on the meaning of ``incurred''. An outgoing is deductible when a definite, present, and certain legal obligation to pay arises --- not when an invoice is received or when cash is paid. The TR~97/7A2 addendum (2 July 2025) clarifies when a \textit{loss} (as distinct from an outgoing) is incurred, following the AAT decision in \textit{Bowerman v Commissioner of Taxation} [2023] AATA~3547. The ruling governs all cut-off and period-end accrual decisions. Every accrued liability in the fund's accounts must satisfy this test before recognition as a deduction. & See \S\ref{sec:est-cutoff}; \S\ref{sec:est-accruals-settlement} & \applies \\
\textbf{\textit{Carden's Case}} (1938) & The liability must be ``definitely committed'' and not merely contingent. Provides the foundational case law underpinning TR~97/7 and all subsequent ``incurred'' analysis. A provision for a contingent liability that has not crystallised into a legal obligation is not deductible. & See \S\ref{sec:est-cutoff} & \applies \\
\rowcolor{warmgray}
\textbf{\textit{New Zealand Flax Investments}} & The obligation need not be immediately payable; a present, existing, and certain obligation satisfies the ``incurred'' test even if the precise amount must be estimated. Governs accrual of invoices not yet received. & See \S\ref{sec:est-cutoff}; \S\ref{sec:est-accruals-settlement} & \applies \\
\textbf{\textit{Nilsen Development Laboratories}} (1981) & An unconditional obligation to pay is sufficient --- the liability need not be due for payment in the current income year. Relevant to accruals for annual charges (e.g., land tax, annual management fees) that are payable after year end. & See \S\ref{sec:est-opex-timing} & \applies \\
\rowcolor{warmgray}
\textbf{TR~95/25} & Interest deductibility continues after the income-producing activity ceases, provided the nexus existed at the time the interest was incurred. Applies to St~George interest accrued after the property is under contract but before settlement, and to any residual interest after settlement if the loan is not immediately repaid. & See \S\ref{sec:stgeorge-entries}; \S\ref{sec:est-loan-swap} & \applies \\
\textbf{TR~2000/2} & Apportionment of interest and other expenses where funds are used partly for income-producing and partly for other purposes. Relevant if the St~George facility is ever drawn in excess of the property acquisition cost or if funds are redrawn for non-income-producing purposes. & See \S\ref{sec:stgeorge-entries} & \applies \\
\rowcolor{warmgray}
\textbf{TR~2004/4} & Interest nexus analysis for complex borrowing arrangements. Governs the deductibility of interest on refinanced facilities and establishes that interest on a facility used to replace an income-producing borrowing retains deductibility. & See \S\ref{sec:est-loan-swap} & \applies \\
\end{longtable}
\endgroup


% ============================================================================
\subsection{Repairs vs.\ Capital Expenditure}
\label{sec:ref-repairs}
% ============================================================================

\begingroup
\small\sffamily
\setlength{\LTpre}{4pt}
\setlength{\LTpost}{4pt}
\renewcommand{\arraystretch}{1.2}
\begin{longtable}{@{}L{3.0cm} p{\dimexpr\linewidth-3.0cm-3.3cm-1.0cm-6\tabcolsep\relax} L{3.3cm} C{1.0cm}@{}}
\caption{Repairs and capital expenditure references --- applicability to 22 Constance Street} \label{tab:ref-repairs} \\
\rowcolor{navy}
\textcolor{white}{\bfseries Reference} &
\textcolor{white}{\bfseries Relevance to this SOP} &
\textcolor{white}{\bfseries SOP Section} &
\textcolor{white}{\bfseries\scriptsize\applies} \\
\midrule
\endfirsthead
\multicolumn{4}{@{}l@{}}{\small\itshape\textcolor{navy!70!black}{\tablename~\thetable{} (cont.)}} \\[2pt]
\rowcolor{navy}
\textcolor{white}{\bfseries Reference} &
\textcolor{white}{\bfseries Relevance to this SOP} &
\textcolor{white}{\bfseries SOP Section} &
\textcolor{white}{\bfseries\scriptsize\applies} \\
\midrule
\endhead
\midrule
\multicolumn{4}{@{}r@{}}{\itshape\textcolor{navy!70!black}{Continued on next page\ldots}} \\
\endfoot
\bottomrule
\endlastfoot
\textbf{s25-10 ITAA 1997} & Immediate deduction for repairs to income-producing property. Five conditions must all be satisfied: (1) structural, mechanical or physical defect; (2) the property was used for income-producing purposes; (3) deduction not otherwise available; (4) cost is revenue, not capital; (5) the work restores rather than improves. & See \S\ref{sec:est-capex-guide}; \S\ref{sec:est-capex-criteria} & \applies \\
\rowcolor{warmgray}
\textbf{TR~97/23} & ATO's primary ruling distinguishing repairs (restores efficiency of function without changing the character of the asset) from capital improvements (goes beyond restoration --- the ``betterment test''). Also sets out the ``entirety test'' and the ``initial repair rule'' (cost to repair a defect existing at acquisition is capital). Mandatory reference for every Capex vs.\ Repair Assessment Form. & See \S\ref{sec:est-capex-criteria}; \S\ref{sec:est-capex-checklist} & \applies \\
\textbf{\textit{W Thomas \& Co Pty Ltd v FCT}} & Key case law on the ``entirety'' test: replacement of the whole, or substantially the whole, of a separately identifiable part is capital regardless of whether it merely restores the asset. A replaced roof covering the whole of a building is capital; patching a section is a repair. Governs the cyclone damage expenditure classification. & See \S\ref{sec:est-capex-criteria}; \S\ref{sec:insurance} & \applies \\
\rowcolor{warmgray}
\textbf{\textit{FCT v Western Suburbs Cinemas Ltd}} & Confirms the initial repair rule: costs to remedy defects that existed at the date of acquisition are capital expenditure, not immediately deductible under s25-10. Applies when the fund acquires property with pre-existing defects. & See \S\ref{sec:est-capex-criteria} & \applies \\
\textbf{AASB~116 \textit{Property, Plant and Equipment}} & Capitalisation vs.\ expense criteria under accounting standards (applied by analogy to investment property). Component accounting under AASB~116.43: where a component of an asset is replaced, the old component is derecognised and the new component is capitalised regardless of whether the transaction is a ``repair'' for tax purposes. & See \S\ref{sec:est-capex-accounting}; \S\ref{sec:est-repair-integration} & \applies \\
\rowcolor{warmgray}
\textbf{AASB~140 \textit{Investment Property}} & Accounting standard for investment property. Governs when subsequent expenditure is capitalised (para 10--12): capitalise if it is probable that future economic benefits will flow and the cost can be reliably measured; expense otherwise. Fair value model: all capitalised costs are subsumed into the fair value carrying amount at the next revaluation. & See \S\ref{sec:est-reval}; \S\ref{sec:est-capex-accounting} & \applies \\
\end{longtable}
\endgroup


% ============================================================================
\subsection{Tax Depreciation --- Division~40 and Division~43}
\label{sec:ref-taxdep}
% ============================================================================

\begingroup
\small\sffamily
\setlength{\LTpre}{4pt}
\setlength{\LTpost}{4pt}
\renewcommand{\arraystretch}{1.2}
\begin{longtable}{@{}L{3.0cm} p{\dimexpr\linewidth-3.0cm-3.3cm-1.0cm-6\tabcolsep\relax} L{3.3cm} C{1.0cm}@{}}
\caption{Tax depreciation references --- applicability to 22 Constance Street} \label{tab:ref-taxdep} \\
\rowcolor{navy}
\textcolor{white}{\bfseries Reference} &
\textcolor{white}{\bfseries Relevance to this SOP} &
\textcolor{white}{\bfseries SOP Section} &
\textcolor{white}{\bfseries\scriptsize\applies} \\
\midrule
\endfirsthead
\multicolumn{4}{@{}l@{}}{\small\itshape\textcolor{navy!70!black}{\tablename~\thetable{} (cont.)}} \\[2pt]
\rowcolor{navy}
\textcolor{white}{\bfseries Reference} &
\textcolor{white}{\bfseries Relevance to this SOP} &
\textcolor{white}{\bfseries SOP Section} &
\textcolor{white}{\bfseries\scriptsize\applies} \\
\midrule
\endhead
\midrule
\multicolumn{4}{@{}r@{}}{\itshape\textcolor{navy!70!black}{Continued on next page\ldots}} \\
\endfoot
\bottomrule
\endlastfoot
\textbf{Division~40 ITAA~1997} (ss40-30 to 40-445) & Depreciating assets: plant and equipment with a limited effective life. Governs decline in value calculations, effective life determinations, low-value pooling (s40-425), immediate deduction for assets $\leq$\$300 (s40-80), and balancing adjustment events on disposal (ss40-285 to 40-305). An annual quantity surveyor report reconciles the Div~40 register for each year. & See \S\ref{sec:est-capex-taxdivision}; \S\ref{sec:est-cgt-div43} & \applies \\
\rowcolor{warmgray}
\textbf{s40-285 ITAA~1997} & Balancing adjustment on disposal of a Div~40 depreciating asset. The difference between the termination value (usually sale proceeds allocated to plant) and the adjustable value (tax written-down value) is either deductible (if TDV $>$ TV) or assessable (if TV $>$ TDV). Requires allocation of the global sale price between Div~40 assets and the CGT asset. & See \S\ref{sec:est-capex-taxdivision}; \S\ref{sec:est-cgt} & \applies \\
\textbf{Effective Life Determination~2025\dag} & Income Tax Assessment (Effective Life of Depreciating Assets) Determination 2025 (Legislative Instrument). Current authority for assigning statutory effective life to all Div~40 plant and equipment. Replaced the previous practice of annual taxation rulings: TR~2022/1 (the last such ruling, applicable from 1~July~2022) was formally withdrawn on 31~October~2025. All effective life tables (Tables~A and~B) are now maintained in the Determination, which is periodically updated. Accessed via the ATO Legal Database and the Depreciation and Capital Allowances Tool. & See \S\ref{sec:est-capex-taxdivision} & \applies \\
\rowcolor{warmgray}
\textbf{Division~43 ITAA~1997} (ss43-1 to 43-260) & Capital works deductions: structural improvements to commercial buildings and fixed infrastructure. Rate: 2.5\% p.a.\ over 40 years for works commencing after 26~February~1992. An annual quantity surveyor report allocates construction costs between Div~40 and Div~43. Outstanding Div~43 deductions reduce the CGT cost base under s110-45(4). & See \S\ref{sec:est-capex-taxdivision}; \S\ref{sec:est-cgt-div43} & \applies \\
\textbf{s110-45(4) ITAA~1997} & Division~43 deductions claimed reduce the CGT cost base dollar-for-dollar. Mandatory adjustment before calculating the capital gain on disposal. The tax agent's quantity surveyor report (tax depreciation schedule) must be used to calculate the cumulative Div~43 adjustment. & See \S\ref{sec:est-cgt-div43} & \applies \\
\end{longtable}
\endgroup


% ============================================================================
\subsection{Capital Gains Tax}
\label{sec:ref-cgt}
% ============================================================================

\begingroup
\small\sffamily
\setlength{\LTpre}{4pt}
\setlength{\LTpost}{4pt}
\renewcommand{\arraystretch}{1.2}
\begin{longtable}{@{}L{3.0cm} p{\dimexpr\linewidth-3.0cm-3.3cm-1.0cm-6\tabcolsep\relax} L{3.3cm} C{1.0cm}@{}}
\caption{CGT legislative references --- applicability to 22 Constance Street} \label{tab:ref-cgt} \\
\rowcolor{navy}
\textcolor{white}{\bfseries Reference} &
\textcolor{white}{\bfseries Relevance to this SOP} &
\textcolor{white}{\bfseries SOP Section} &
\textcolor{white}{\bfseries\scriptsize\applies} \\
\midrule
\endfirsthead
\multicolumn{4}{@{}l@{}}{\small\itshape\textcolor{navy!70!black}{\tablename~\thetable{} (cont.)}} \\[2pt]
\rowcolor{navy}
\textcolor{white}{\bfseries Reference} &
\textcolor{white}{\bfseries Relevance to this SOP} &
\textcolor{white}{\bfseries SOP Section} &
\textcolor{white}{\bfseries\scriptsize\applies} \\
\midrule
\endhead
\midrule
\multicolumn{4}{@{}r@{}}{\itshape\textcolor{navy!70!black}{Continued on next page\ldots}} \\
\endfoot
\bottomrule
\endlastfoot
\textbf{s104-10 ITAA~1997} & CGT Event~A1: disposal of a CGT asset. The event occurs when the contract for sale is \textit{entered into} (i.e., at exchange), not at settlement. In a PCO structure, the event occurs at the date of option exercise (the exercise constitutes the formation of a sale contract). & See \S\ref{sec:est-cgt-timing}; \S\ref{sec:est-putcall} & \applies \\
\rowcolor{warmgray}
\textbf{s110-25 ITAA~1997} & CGT cost base: five elements --- (1)~acquisition price; (2)~incidental acquisition costs (stamp duty, legal fees, agent's fees); (3)~ownership costs not otherwise deducted (holding costs such as rates, land tax, insurance premiums if not deductible); (4)~capital enhancement expenditure; (5)~incidental disposal costs. Div~43 deductions reduce Element~1 via s110-45(4). & See \S\ref{sec:est-cgt-costbase} & \applies \\
\textbf{s110-45(4) ITAA~1997} & Div~43 capital works deductions claimed since acquisition reduce the cost base. The quantum surveyor's cumulative Div~43 figure must be deducted before finalising the cost base. & See \S\ref{sec:est-cgt-div43} & \applies \\
\rowcolor{warmgray}
\textbf{s116-20 ITAA~1997} & Capital proceeds: the total consideration received or receivable for the CGT event. Includes settlement adjustments (rent, rates, land tax), the GST-exclusive component of the price, and any earnest money or deposit. In a PCO structure, the option fee is added to A1 proceeds under s134-1. & See \S\ref{sec:est-cgt-costbase} & \applies \\
\textbf{Division~115 ITAA~1997} & 50\% CGT discount for individuals, trusts, and complying superannuation funds holding a CGT asset for at least 12~months before the CGT event. Not available to companies. The discount component must be separately identified in the trust's tax return and in the unitholder's annual tax statement. & See \S\ref{sec:est-cgt-timing}; \S\ref{sec:est-distribution-components} & \applies \\
\rowcolor{warmgray}
\textbf{s102-5 ITAA~1997} & Capital losses from other CGT assets must be applied against the gross capital gain \textit{before} the 50\% Div~115 discount is applied. Reduces the net discountable gain. & See \S\ref{sec:est-cgt} & \applies \\
\textbf{s14-200 Schedule~1 TAA~1953\dag} & Foreign resident capital gains withholding (FRCGW): as of 1~January~2025, the withholding rate is \textbf{15\%} of the purchase price and applies to all property sales regardless of value (the previous \$750,000 threshold has been removed). The purchaser must withhold unless the vendor produces a valid ATO clearance certificate. Apply for the certificate at least 28~days before settlement. \deadline{Apply 28+ days before settlement} & See \S\ref{sec:est-investing-obligations} & \applies \\
\rowcolor{warmgray}
\textbf{Division~152 ITAA~1997} & Small business CGT concessions. Generally \textit{not applicable}: requires either the maximum net asset value test (entity's net assets $\leq$\$6m) or the $\leq$\$2m aggregated turnover test. A property investment trust of this scale with net assets exceeding \$6m does not qualify. & See \S\ref{sec:est-cgt} & \notapplies \\
\textbf{Division~118-B ITAA~1997} & Main residence exemption. Does NOT apply to investment property held by a trust for income-producing purposes. Not relevant to 22~Constance Street. & See \S\ref{sec:est-cgt} & \notapplies \\
\rowcolor{warmgray}
\textbf{s104-155 ITAA~1997} & CGT Event~H1: granting of a right or option over a CGT asset. The event occurs when the option deed is \textit{executed} (the grant), not when the option is exercised or lapses. The option fee received is the capital proceeds for H1. No 50\% CGT discount applies to a freshly granted option (12-month holding test cannot be met). & See \S\ref{sec:est-cgt-h1}; \S\ref{sec:est-putcall} & \applies \\
\textbf{s134-1 ITAA~1997} & Absorption of H1 into A1 on exercise: when the option is exercised, the H1 option fee is added to the A1 capital proceeds and H1 costs are added to the A1 cost base. Prevents double taxation across the two CGT events but does not reverse the H1 assessment in the H1 income year. & See \S\ref{sec:est-cgt-h1}; \S\ref{sec:est-putcall} & \applies \\
\rowcolor{warmgray}
\textbf{s104-20 ITAA~1997} & CGT Event~C1: loss or destruction of a CGT asset. Applies where an individual depreciating asset (e.g., Ampac fire detection equipment) is destroyed rather than merely damaged. Insurance proceeds from the insurer for the destroyed asset are the C1 capital proceeds. Distinguished from mere damage (which is a revenue receipt). & See \S\ref{sec:est-insurance-cgt} & \applies \\
\textbf{s104-25 ITAA~1997} & CGT Event~C2: cancellation, surrender, or similar termination of a contractual right. May arise on a ``full and final settlement'' of the insurance claim that extinguishes the fund's contractual right to claim further under the policy. Seek specific tax advice before agreeing to any global settlement of the Steadfast/Vision~Re claim. & See \S\ref{sec:est-insurance-cgt} & \applies \\
\rowcolor{warmgray}
\textbf{Div~115 ITAA~1997 (H1 discount)} & The 50\% CGT discount does NOT apply to CGT Event~H1 on a freshly granted option. The ``right'' has been held only since the moment of grant --- the 12-month holding period test cannot be satisfied. The A1 discount is assessed separately and is available if the property has been held for $>$12~months. & See \S\ref{sec:est-cgt-h1} & \applies \\
\end{longtable}
\endgroup


% ============================================================================
\subsection{Borrowing Expenses and Blackhole Expenditure}
\label{sec:ref-borrowing}
% ============================================================================

\begingroup
\small\sffamily
\setlength{\LTpre}{4pt}
\setlength{\LTpost}{4pt}
\renewcommand{\arraystretch}{1.2}
\begin{longtable}{@{}L{3.0cm} p{\dimexpr\linewidth-3.0cm-3.3cm-1.0cm-6\tabcolsep\relax} L{3.3cm} C{1.0cm}@{}}
\caption{Borrowing expense and blackhole expenditure references --- applicability to 22 Constance Street} \label{tab:ref-borrowing} \\
\rowcolor{navy}
\textcolor{white}{\bfseries Reference} &
\textcolor{white}{\bfseries Relevance to this SOP} &
\textcolor{white}{\bfseries SOP Section} &
\textcolor{white}{\bfseries\scriptsize\applies} \\
\midrule
\endfirsthead
\multicolumn{4}{@{}l@{}}{\small\itshape\textcolor{navy!70!black}{\tablename~\thetable{} (cont.)}} \\[2pt]
\rowcolor{navy}
\textcolor{white}{\bfseries Reference} &
\textcolor{white}{\bfseries Relevance to this SOP} &
\textcolor{white}{\bfseries SOP Section} &
\textcolor{white}{\bfseries\scriptsize\applies} \\
\midrule
\endhead
\midrule
\multicolumn{4}{@{}r@{}}{\itshape\textcolor{navy!70!black}{Continued on next page\ldots}} \\
\endfoot
\bottomrule
\endlastfoot
\textbf{s25-25 ITAA~1997} & Borrowing expense deduction: costs of establishing a borrowing used to produce assessable income. Where the expense exceeds \$100, it is spread over the lesser of the loan term and 5~years, commencing from the date the borrowed funds are first used to produce assessable income. & See \S\ref{sec:est-s2525-s40880}; \S\ref{sec:est-borrow-amort} & \applies \\
\rowcolor{warmgray}
\textbf{s25-25(5) ITAA~1997} & Upon full repayment of a borrowing, the entire unamortised balance of borrowing costs becomes \textit{immediately deductible} in the income year of repayment. Upon repayment of the St~George facility at settlement, all unamortised amounts in \acct{708.03} and \acct{708.04} must be written off to \acct{303.03}/\acct{303.04} in the same income year. \deadline{Write off in income year of loan repayment} & See \S\ref{sec:est-s2525-s40880}; \S\ref{sec:borrowing-amort} & \applies \\
\textbf{s40-880 ITAA~1997} & Blackhole business expenditure: 5-year straight-line deduction for business-related capital expenditure not otherwise deductible, not in CGT cost base, and satisfying the five gate tests. Candidates include fund establishment costs, wind-up advisory costs, and certain refinancing advisory fees. & See \S\ref{sec:est-s2525-s40880} & \applies \\
\rowcolor{warmgray}
\textbf{Division~820 ITAA~1997} & Thin capitalisation rules: targets multinational groups, foreign-controlled entities, and Australian entities with overseas investments above the minimum capital threshold. Not currently applicable: the fund is Australian-resident, its investors are domestic, and no cross-border financing is in place. Becomes relevant if a foreign-resident lender provides a facility or if foreign investors are admitted to the trust. & See \S\ref{sec:est-loan-swap} & \notapplies \\
\textbf{ss177A--177G ITAA~1936 (Part~IVA)} & General anti-avoidance provision. Relevant to the PCO structure if the principal purpose of the arrangement is to obtain a tax benefit (e.g., deferring a capital gain into a lower-tax year). Relevant to the loan structure if related-party borrowings are at above-market rates inflating interest deductions. & See \S\ref{sec:est-loan-swap}; \S\ref{sec:est-putcall} & \applies \\
\end{longtable}
\endgroup


% ============================================================================
\subsection{Withholding Tax and Unitholder Obligations}
\label{sec:ref-wht}
% ============================================================================

\begingroup
\small\sffamily
\setlength{\LTpre}{4pt}
\setlength{\LTpost}{4pt}
\renewcommand{\arraystretch}{1.2}
\begin{longtable}{@{}L{3.0cm} p{\dimexpr\linewidth-3.0cm-3.3cm-1.0cm-6\tabcolsep\relax} L{3.3cm} C{1.0cm}@{}}
\caption{Withholding tax and unitholder obligation references --- applicability to 22 Constance Street} \label{tab:ref-wht} \\
\rowcolor{navy}
\textcolor{white}{\bfseries Reference} &
\textcolor{white}{\bfseries Relevance to this SOP} &
\textcolor{white}{\bfseries SOP Section} &
\textcolor{white}{\bfseries\scriptsize\applies} \\
\midrule
\endfirsthead
\multicolumn{4}{@{}l@{}}{\small\itshape\textcolor{navy!70!black}{\tablename~\thetable{} (cont.)}} \\[2pt]
\rowcolor{navy}
\textcolor{white}{\bfseries Reference} &
\textcolor{white}{\bfseries Relevance to this SOP} &
\textcolor{white}{\bfseries SOP Section} &
\textcolor{white}{\bfseries\scriptsize\applies} \\
\midrule
\endhead
\midrule
\multicolumn{4}{@{}r@{}}{\itshape\textcolor{navy!70!black}{Continued on next page\ldots}} \\
\endfoot
\bottomrule
\endlastfoot
\textbf{s12-140 TAA Sch~1} & PAYG withholding at the top marginal rate (47\%) on distributions to Australian resident unitholders who have not quoted their TFN. The trustee must withhold at the time of payment and remit via the quarterly BAS. \deadline{Remit via quarterly BAS by 28th of month following quarter} & See \S\ref{sec:est-wht}; \S\ref{sec:est-financial-obligations} & \applies \\
\rowcolor{warmgray}
\textbf{Subdiv~12-H TAA Sch~1 (ss12-385 et seq.)} & MIT withholding tax: trustee must withhold on distributions to \textit{non-resident} unitholders. Rate: 15\% for EEOI-country residents; 30\% for non-EEOI; 10\% for clean building MIT (energy-efficient, post-2012). Penalty for failure to withhold: 100\% of the amount not withheld. Annual Investment Income Report (AIIR) due 31~October. \deadline{AIIR: 31 October annually} & See \S\ref{sec:est-wht}; \S\ref{sec:est-financial-obligations} & \applies \\
\textbf{s202 ITAA~1936} & TFN quotation by resident beneficiaries. Failure to quote TFN triggers mandatory 47\% withholding. The trustee must maintain a TFN register for all unitholders and verify TFN status at each distribution. & See \S\ref{sec:est-wht}; \S\ref{sec:est-unitholders} & \applies \\
\rowcolor{warmgray}
\textbf{AML/CTF Act 2006 (Cth)} & Anti-Money Laundering and Counter-Terrorism Financing Act 2006. Requires the trustee to verify the identity of each unitholder, conduct an ongoing risk assessment, monitor for unusual transactions, and report to AUSTRAC. Applies on admission of new unitholders and upon transfer of units. & See \S\ref{sec:est-unitholders} & \applies \\
\textbf{FATCA (US--Australia IGA)} & Foreign Account Tax Compliance Act: requires Australian financial institutions (including wholesale trusts with US-person investors) to identify, report, and, where applicable, withhold on payments to US-person account holders. Administered under the Australia--US IGA (effective 2014). & See \S\ref{sec:est-unitholders} & \applies \\
\rowcolor{warmgray}
\textbf{Common Reporting Standard (OECD~CRS)} & Multilateral automatic exchange of financial account information. The fund constitutes a ``financial account'' for CRS purposes if it has non-resident account holders. Requires collection of CRS self-certification forms from all unitholders and annual reporting to the ATO for transmission to relevant foreign tax authorities. & See \S\ref{sec:est-unitholders} & \applies \\
\textbf{\textit{Property Occupations Act 2014} (Qld)} & Primary statutory authority for Queensland property management trust accounts (STA~QLD). Governs CPG's obligations to maintain a segregated STA, provide monthly statements, remit funds, and submit to annual audit. \deadline{Monthly STA statements; annual STA audit} & See \S\ref{sec:est-sta} & \applies \\
\rowcolor{warmgray}
\textbf{s97 ITAA~1936} & Unitholders assessed on their proportionate share of the trust's net income regardless of distributions made. The basis for the distribution shortfall risk: if distributions fall short of net taxable income, unitholders are taxed on an amount they have not received in cash. & See \S\ref{sec:est-distribution-components}; \S\ref{sec:est-wht} & \applies \\
\textbf{Div~276--278 ITAA~1997} & Attribution Managed Investment Trust (AMIT) regime. If the trust qualifies as an AMIT, attribution replaces the s97 net income assessment: taxable amounts are attributed to each member individually. Attribution variances arise when attributed amounts differ from actual distributions. & See \S\ref{sec:est-distribution-components} & \applies \\
\rowcolor{warmgray}
\textbf{s104-71 ITAA~1997} & Tax-deferred distributions reduce the unitholder's cost base in their units. The tax deferred component ($\CTD$) is not immediately taxable for the unitholder but reduces their cost base, increasing their capital gain on future disposal. & See \S\ref{sec:est-distribution-components} & \applies \\
\textbf{TR~2012/D1} & ATO draft guidance on the character of trust distributions: distinguishes ordinary income, net capital gain (CGT discount component), tax deferred, and return of capital in blended distributions. Relevant to correctly labelling each component in the annual tax statement (AMMA Statement). & See \S\ref{sec:est-distribution-components}; \S\ref{sec:est-distribution-waterfall} & \applies \\
\rowcolor{warmgray}
\textbf{Div~115 ITAA~1997 (discount through trust)} & The 50\% CGT discount flows through the trust to resident individual unitholders as a ``CGT discount component''. Corporate unitholders cannot access the discount. Superannuation fund unitholders receive a 33.33\% discount instead. The discount component must be separately disclosed in the annual tax statement. & See \S\ref{sec:est-distribution-components} & \applies \\
\textbf{Div~152 ITAA~1997 (small business)} & Small business CGT concessions. Not currently applicable: the fund's net assets exceed the \$6m maximum net asset value threshold. No small business concession component arises in distributions. & See \S\ref{sec:est-distribution-components} & \notapplies \\
\end{longtable}
\endgroup


% ============================================================================
\subsection{Loan Swap, Refinancing, and Put-Call Option Structure}
\label{sec:ref-loanswap}
\label{sec:est-loan-swap}
\label{sec:est-putcall}
% ============================================================================

\begingroup
\small\sffamily
\setlength{\LTpre}{4pt}
\setlength{\LTpost}{4pt}
\renewcommand{\arraystretch}{1.2}
\begin{longtable}{@{}L{3.0cm} p{\dimexpr\linewidth-3.0cm-3.3cm-1.0cm-6\tabcolsep\relax} L{3.3cm} C{1.0cm}@{}}
\caption{Loan swap, refinancing, and PCO legislative references --- applicability to 22 Constance Street} \label{tab:ref-loanswap} \\
\rowcolor{navy}
\textcolor{white}{\bfseries Reference} &
\textcolor{white}{\bfseries Relevance to this SOP} &
\textcolor{white}{\bfseries SOP Section} &
\textcolor{white}{\bfseries\scriptsize\applies} \\
\midrule
\endfirsthead
\multicolumn{4}{@{}l@{}}{\small\itshape\textcolor{navy!70!black}{\tablename~\thetable{} (cont.)}} \\[2pt]
\rowcolor{navy}
\textcolor{white}{\bfseries Reference} &
\textcolor{white}{\bfseries Relevance to this SOP} &
\textcolor{white}{\bfseries SOP Section} &
\textcolor{white}{\bfseries\scriptsize\applies} \\
\midrule
\endhead
\midrule
\multicolumn{4}{@{}r@{}}{\itshape\textcolor{navy!70!black}{Continued on next page\ldots}} \\
\endfoot
\bottomrule
\endlastfoot
\textbf{AASB~9 \textit{Financial Instruments} (paras 3.3.1--3.3.3)} & Derecognition and modification test for financial liabilities: the ``10\% test'' determines whether a refinancing or renegotiation constitutes an extinguishment of the old liability (creating a new liability) or a mere modification. A change of lender is always an extinguishment regardless of economic similarity. & See \S\ref{sec:est-loan-swap} & \applies \\
\rowcolor{warmgray}
\textbf{AASB~7 \textit{Financial Instruments: Disclosures}} & Requires disclosure of interest rate risk, liquidity risk, maturity profile of financial liabilities, and fair value of financial instruments. The St~George facility (maturity 15~June~2026) and any interest rate swap must be disclosed. & See \S\ref{sec:est-loan-swap} & \applies \\
\textbf{s104-155 ITAA~1997 (H1)} & CGT Event~H1: granting of each option in the PCO deed. Two separate H1 events may arise on PCO execution (call option and put option). The option fee received is H1 capital proceeds; no Div~115 CGT discount applies to a freshly granted option. & See \S\ref{sec:est-putcall}; \S\ref{sec:est-cgt-h1} & \applies \\
\rowcolor{warmgray}
\textbf{s104-10 ITAA~1997 (A1)} & CGT Event~A1 in a PCO: the event date is the option \textit{exercise} date (not settlement), because the exercise constitutes the entry into a contract for sale within the meaning of s104-10(3)(b). & See \S\ref{sec:est-putcall} & \applies \\
\textbf{s134-1 ITAA~1997} & Absorption mechanism: on option exercise, H1 consideration is absorbed into A1 proceeds and H1 costs are added to A1 cost base. Prevents double taxation but the H1 assessment in the H1 year stands. & See \S\ref{sec:est-putcall}; \S\ref{sec:est-cgt-h1} & \applies \\
\rowcolor{warmgray}
\textbf{TD~2018/9} & Tax determination (final version of draft TD~2017/D4): interest deductibility for a trust beneficiary on borrowings on-lent interest free to the trustee. \textbf{Note:} This determination is cited for completeness where a unitholder has on-lent capital into the trust at zero interest. It does not deal with CGT Event~H1 mechanics --- H1 timing and the PCO framework flow from s104-155, s134-1, and s104-10(3)(b) above. & See \S\ref{sec:est-putcall}; \S\ref{sec:est-unitholders} & \applies \\
\textbf{s9-5 GST Act} & The grant of an option over real property is a taxable supply. GST = option fee~$\times$~1/11. The margin scheme election for the underlying property supply must be made \textit{before} the taxable supply occurs (i.e., before option exercise). & See \S\ref{sec:est-putcall}; \S\ref{sec:est-investing-obligations} & \applies \\
\rowcolor{warmgray}
\textbf{\textit{Duties Act 2001} (Qld)} & Queensland transfer duty on the PCO deed (dutiable transaction): rate \$38,025 + 5.75\% of dutiable value exceeding \$1,000,000. Due within 30~days of execution. At \$13.3m: approximately \$745,275. Penalty: QRO Unpaid Tax Interest (11.78\% p.a., compounding daily). \deadline{Within 30 days of PCO deed execution} & See \S\ref{sec:est-investing-obligations}; \S\ref{sec:est-putcall} & \applies \\
\textbf{PPSR Act 2009 (Cth)} & Personal Property Securities Act: new lender's security over personal property (bank accounts, receivables) must be registered to be perfected. Unregistered security interests are vulnerable on insolvency. & See \S\ref{sec:est-loan-swap} & \applies \\
\end{longtable}
\endgroup


% ============================================================================
\subsection{Statutory Fees, Levies, and Government Obligations}
\label{sec:ref-statutory}
\label{sec:est-statutory-fees}
\label{sec:est-financial-obligations}
\label{sec:est-investing-obligations}
\label{sec:est-operating-obligations}
\label{sec:est-deadline-calendar}
\label{sec:est-penalty-rates}
% ============================================================================

\begingroup
\small\sffamily
\setlength{\LTpre}{4pt}
\setlength{\LTpost}{4pt}
\renewcommand{\arraystretch}{1.2}
\begin{longtable}{@{}L{3.0cm} p{\dimexpr\linewidth-3.0cm-3.3cm-1.0cm-6\tabcolsep\relax} L{3.3cm} C{1.0cm}@{}}
\caption{Statutory fees, levies, and government obligation references --- applicability to 22 Constance Street} \label{tab:ref-statutory} \\
\rowcolor{navy}
\textcolor{white}{\bfseries Reference} &
\textcolor{white}{\bfseries Relevance to this SOP} &
\textcolor{white}{\bfseries SOP Section} &
\textcolor{white}{\bfseries\scriptsize\applies} \\
\midrule
\endfirsthead
\multicolumn{4}{@{}l@{}}{\small\itshape\textcolor{navy!70!black}{\tablename~\thetable{} (cont.)}} \\[2pt]
\rowcolor{navy}
\textcolor{white}{\bfseries Reference} &
\textcolor{white}{\bfseries Relevance to this SOP} &
\textcolor{white}{\bfseries SOP Section} &
\textcolor{white}{\bfseries\scriptsize\applies} \\
\midrule
\endhead
\midrule
\multicolumn{4}{@{}r@{}}{\itshape\textcolor{navy!70!black}{Continued on next page\ldots}} \\
\endfoot
\bottomrule
\endlastfoot
\textbf{\textit{Land Tax Act 2010} (Qld)} & Queensland land tax assessed at midnight 30~June each year. Trustee/company rate: \$1,450 + 1.7\textcent/\$ over \$350,000 (first band). Penalty: QRO UTI at 11.78\% p.a.\ (compounding daily) for late payment; up to 75\% penalty tax for failure to notify QRO within one month of subsequent 30~June. Land tax clearance certificate required before settlement. \deadline{Annual: midnight 30~June; clearance cert before settlement} & See \S\ref{sec:est-operating-obligations}; \S\ref{sec:est-opex-timing} & \applies \\
\rowcolor{warmgray}
\textbf{\textit{Taxation Administration Act 2001} (Qld) --- UTI} & QRO Unpaid Tax Interest (UTI) applies to unpaid Queensland state taxes (land tax, transfer duty). Rate: 11.78\% p.a.\ (2025--26); confirmed annually. Compounding daily from the payment due date. Distinguished from the ATO's GIC (which applies to Commonwealth taxes). & See \S\ref{sec:est-penalty-rates} & \applies \\
\textbf{\textit{City of Brisbane Act 2010} (Qld)} & Brisbane City Council general rates and fire levy. Quarterly notices; payment due within 30~days. Late payment interest: 12.12\% p.a.\ (2025--26), compounding daily from the day after the due date. Rates constitute a first charge over the land if unpaid for $>$3 years. \deadline{Quarterly: within 30 days of notice} & See \S\ref{sec:est-operating-obligations}; \S\ref{sec:est-opex-timing} & \applies \\
\rowcolor{warmgray}
\textbf{\textit{Corporations Act 2001} (Cth), s1351} & ASIC annual review fee: \$329 p.a.\ (2025--26, standard proprietary company). Due within 2~months of anniversary of registration. Late penalty: +\$98 (up to 1~month); +\$411 (over 1~month). Non-payment triggers deregistration of the trustee company --- a critical compliance risk. \deadline{Within 2 months of registration anniversary} & See \S\ref{sec:est-operating-obligations} & \applies \\
\textbf{s8AAD TAA~1953 (GIC)\dag} & ATO General Interest Charge: 10.65\% p.a.\ (January--March~2026 quarter; 90-day BAB rate + 7\%). Compounding daily on all unpaid ATO liabilities (income tax, PAYG, GST, FBT). \textbf{No longer tax-deductible from 1~July~2025} (Treasury Laws Amendment (Tax Incentives and Integrity) Act 2025). \deadline{Pay ATO liabilities by due date --- GIC accrues immediately} & See \S\ref{sec:est-financial-obligations}; \S\ref{sec:est-penalty-rates} & \applies \\
\rowcolor{warmgray}
\textbf{s280-105 TAA Sch~1 (SIC)} & Shortfall Interest Charge: applies to income tax shortfalls arising on amended assessments. Rate: base rate (90-day BAB) + 3\% p.a.\ (6.65\% January--March~2026). SIC is replaced by GIC after 21~days of non-payment. Not deductible from 1~July~2025. \deadline{21 days to pay SIC before GIC applies} & See \S\ref{sec:est-financial-obligations}; \S\ref{sec:est-penalty-rates} & \applies \\
\textbf{Div~284 TAA Sch~1} & Administrative shortfall penalties: 25\% (no reasonably arguable position); 50\% (no reasonably arguable position, recklessness); 75\% (intentional disregard). Voluntary disclosure before audit: up to 80\% reduction. & See \S\ref{sec:est-financial-obligations} & \applies \\
\rowcolor{warmgray}
\textbf{\textit{Land Title Act 1994} (Qld)} & Queensland Titles Registry: registration of new mortgage (\$195--\$780); discharge of existing mortgage (\$195 approx.). Effected via PEXA. No stamp duty on mortgage registration (abolished Qld, 1~July~2012). Priority determined by date of registration. & See \S\ref{sec:est-investing-obligations}; \S\ref{sec:est-loan-swap} & \applies \\
\textbf{Div~14-D TAA Sch~1\dag} & Foreign resident capital gains withholding (FRCGW): rate 15\% (from 1~January~2025; threshold removed). Purchaser must withhold at settlement; remit within 7~days. Clearance certificate eliminates withholding. Apply at least 28~days before settlement. Failure to withhold: purchaser liable for full amount + GIC. \deadline{Apply for clearance cert 28+ days before settlement; remit within 7 days} & See \S\ref{sec:est-investing-obligations} & \applies \\
\end{longtable}
\endgroup


% ============================================================================
\subsection{Insurance, GST, and Bank Guarantees}
\label{sec:ref-insurance}
% ============================================================================

\begingroup
\small\sffamily
\setlength{\LTpre}{4pt}
\setlength{\LTpost}{4pt}
\renewcommand{\arraystretch}{1.2}
\begin{longtable}{@{}L{3.0cm} p{\dimexpr\linewidth-3.0cm-3.3cm-1.0cm-6\tabcolsep\relax} L{3.3cm} C{1.0cm}@{}}
\caption{Insurance, GST, and bank guarantee references --- applicability to 22 Constance Street} \label{tab:ref-insurance} \\
\rowcolor{navy}
\textcolor{white}{\bfseries Reference} &
\textcolor{white}{\bfseries Relevance to this SOP} &
\textcolor{white}{\bfseries SOP Section} &
\textcolor{white}{\bfseries\scriptsize\applies} \\
\midrule
\endfirsthead
\multicolumn{4}{@{}l@{}}{\small\itshape\textcolor{navy!70!black}{\tablename~\thetable{} (cont.)}} \\[2pt]
\rowcolor{navy}
\textcolor{white}{\bfseries Reference} &
\textcolor{white}{\bfseries Relevance to this SOP} &
\textcolor{white}{\bfseries SOP Section} &
\textcolor{white}{\bfseries\scriptsize\applies} \\
\midrule
\endhead
\midrule
\multicolumn{4}{@{}r@{}}{\itshape\textcolor{navy!70!black}{Continued on next page\ldots}} \\
\endfoot
\bottomrule
\endlastfoot
\textbf{s6-5 ITAA~1997 + ss20-20 to 20-30} & Ordinary income (s6-5) captures lost rent and business interruption proceeds. The recoupment provisions (ss20-20 to 20-30) apply where the fund receives insurance proceeds that reimburse an amount previously deducted (e.g.\ deductible repair costs). The character of the proceeds follows the character of the loss indemnified. Capital insurance proceeds (for permanent structural damage) are not ordinary income and fall into the CGT framework. & See \S\ref{sec:est-insurance-taxtreatment} & \applies \\
\rowcolor{warmgray}
\textbf{GSTR~2006/10\dag} (updated 14~Feb~2026) & ATO GST Ruling --- insurance settlements and entitlement to input tax credits. Governs the reduced input tax credit mechanism where the insurer funds reinstatement works: if the insurer pays the contractor directly, the insured fund's ITC entitlement on those costs may be reduced. Also addresses apportionment where insurance proceeds cover only part of the GST-inclusive reinstatement cost and the fund tops up the balance. & See \S\ref{sec:est-insurance-taxtreatment} & \applies \\
\textbf{GSTR~2006/1} & ATO GST Ruling --- guarantees and indemnities. Confirms that the grant of a bank guarantee is a financial supply under Division~40 GST Act (input-taxed). Proceeds paid under a bank guarantee by the issuing bank to the fund (as beneficiary) are generally not subject to GST. & See \S\ref{sec:est-bankguarantees} & \applies \\
\rowcolor{warmgray}
\textbf{s6-5 ITAA~1997} & Assessable income: ordinary income includes lost rent insurance proceeds (business interruption) and insurance proceeds reimbursing deductible operating expenses. Distinguish from capital insurance proceeds (for permanent structural damage) which are capital receipts. & See \S\ref{sec:est-insurance-taxtreatment} & \applies \\
\textbf{AASB~137} & Provisions, Contingent Liabilities and Contingent Assets: governs recognition of the insurance claim receivable (contingent asset $\to$ probable receivable when receipt is virtually certain). Bank guarantees held as beneficiary are also contingent assets under AASB~137. & See \S\ref{sec:est-bankguarantees}; \S\ref{sec:est-insurance} & \applies \\
\rowcolor{warmgray}
\textbf{AASB~9 \textit{Financial Instruments}} & Financial guarantee contracts given by the fund (e.g., guarantees to the landlord or St~George) are measured at fair value on initial recognition and subsequently at the higher of the ECL provision and the premium received. Bank guarantees held as beneficiary are off-balance-sheet until called. & See \S\ref{sec:est-bankguarantees} & \applies \\
\textbf{ATO~ID~2004/966} & Interpretive Decision: insurance proceeds received for destruction of a depreciating asset constitute the ``termination value'' for the s40-295 balancing adjustment. Relevant if any Div~40 plant is destroyed (vs.\ merely damaged) in the cyclone. & See \S\ref{sec:est-insurance-cgt} & \applies \\
\rowcolor{warmgray}
\textbf{s104-155 ITAA~1997} & CGT Event~H1: option or right granted. See CGT references (\S\ref{sec:ref-cgt}). Also relevant in insurance context: a ``right to claim'' under an insurance policy is a CGT asset; its disposal (e.g., by agreeing to a global settlement that extinguishes further rights) may trigger a CGT event. & See \S\ref{sec:est-cgt-h1}; \S\ref{sec:est-insurance-cgt} & \applies \\
\textbf{s104-25 ITAA~1997} & CGT Event~C2: cancellation of a right. A global settlement of the cyclone insurance claim that extinguishes the contractual right to claim further under the Steadfast~IRS/Vision~Re policy may trigger C2. Seek specific tax advice before agreeing to a full and final settlement. & See \S\ref{sec:est-insurance-cgt} & \applies \\
\end{longtable}
\endgroup


% ============================================================================
\subsection{STA, Distribution, and Trust Account References}
\label{sec:ref-sta}
% ============================================================================

\begingroup
\small\sffamily
\setlength{\LTpre}{4pt}
\setlength{\LTpost}{4pt}
\renewcommand{\arraystretch}{1.2}
\begin{longtable}{@{}L{3.0cm} p{\dimexpr\linewidth-3.0cm-3.3cm-1.0cm-6\tabcolsep\relax} L{3.3cm} C{1.0cm}@{}}
\caption{STA, distribution, and trust account references --- applicability to 22 Constance Street} \label{tab:ref-sta} \\
\rowcolor{navy}
\textcolor{white}{\bfseries Reference} &
\textcolor{white}{\bfseries Relevance to this SOP} &
\textcolor{white}{\bfseries SOP Section} &
\textcolor{white}{\bfseries\scriptsize\applies} \\
\midrule
\endfirsthead
\multicolumn{4}{@{}l@{}}{\small\itshape\textcolor{navy!70!black}{\tablename~\thetable{} (cont.)}} \\[2pt]
\rowcolor{navy}
\textcolor{white}{\bfseries Reference} &
\textcolor{white}{\bfseries Relevance to this SOP} &
\textcolor{white}{\bfseries SOP Section} &
\textcolor{white}{\bfseries\scriptsize\applies} \\
\midrule
\endhead
\midrule
\multicolumn{4}{@{}r@{}}{\itshape\textcolor{navy!70!black}{Continued on next page\ldots}} \\
\endfoot
\bottomrule
\endlastfoot
\textbf{\textit{Property Occupations Act 2014} (Qld)} & Primary statutory authority for Queensland regulated property management trust accounts. Governs CPG's obligations to maintain a segregated STA~QLD, issue monthly statements, remit funds within prescribed periods, and submit to annual audit by an approved auditor. & See \S\ref{sec:est-sta} & \applies \\
\rowcolor{warmgray}
\textbf{s97 ITAA~1936} & Unitholders are assessed on their proportionate share of the trust's net income regardless of distributions actually made. The principal mechanism creating the distribution shortfall risk (where $D < \text{Net}TI$, unitholders are taxed on undistributed income). & See \S\ref{sec:est-distribution-components}; \S\ref{sec:est-wht} & \applies \\
\textbf{Div~115 ITAA~1997} & 50\% CGT discount for individuals and trusts with assets held $>$12 months. Basis for the CGT discount component ($C_\text{disc}$) flowing to resident individual unitholders. Corporate and non-resident unitholders cannot access the discount. & See \S\ref{sec:est-distribution-components} & \applies \\
\rowcolor{warmgray}
\textbf{Div~152 ITAA~1997} & Small business CGT concessions. Not currently applicable --- the fund's net assets exceed the \$6m maximum net asset value threshold. & See \S\ref{sec:est-distribution-components} & \notapplies \\
\textbf{s104-71 ITAA~1997} & Tax-deferred distributions reduce the cost base of units in the unitholder's hands. The tax-deferred component ($\CTD$) arises from building allowances and other non-cash deductions distributed to unitholders in excess of the unitholder's proportionate share of taxable income. & See \S\ref{sec:est-distribution-components} & \applies \\
\rowcolor{warmgray}
\textbf{Div~276--278 ITAA~1997 (AMIT)} & Attribution Managed Investment Trust regime. If the trust qualifies, taxable amounts are attributed individually to each member rather than assessed under s97. Attribution variances may create differences between attributed and distributed amounts. Qualification requires the trust to be a ``widely held'' MIT. & See \S\ref{sec:est-distribution-components} & \applies \\
\textbf{TR~2012/D1} & ATO draft guidance on the character of trust distributions. Distinguishes ordinary income, net capital gain, tax deferred, and return of capital in composite distributions. Informs the labelling of each component in the fund's AMMA Statement issued to unitholders. & See \S\ref{sec:est-distribution-components}; \S\ref{sec:est-distribution-waterfall} & \applies \\
\rowcolor{warmgray}
\textbf{s134-1 ITAA~1997} & When a CGT Event~H1 option is exercised, the H1 consideration is absorbed into A1 proceeds and H1 costs are added to the A1 cost base. Prevents double-counting in the distribution component calculation across the H1 year and A1 year. & See \S\ref{sec:est-cgt-h1}; \S\ref{sec:est-putcall} & \applies \\
\end{longtable}
\endgroup


% ============================================================================
\subsection{Accounting Standards}
\label{sec:ref-accounting}
% ============================================================================

\begingroup
\small\sffamily
\setlength{\LTpre}{4pt}
\setlength{\LTpost}{4pt}
\renewcommand{\arraystretch}{1.2}
\begin{longtable}{@{}L{3.0cm} p{\dimexpr\linewidth-3.0cm-3.3cm-1.0cm-6\tabcolsep\relax} L{3.3cm} C{1.0cm}@{}}
\caption{Accounting standards references --- applicability to 22 Constance Street} \label{tab:ref-accounting} \\
\rowcolor{navy}
\textcolor{white}{\bfseries Reference} &
\textcolor{white}{\bfseries Relevance to this SOP} &
\textcolor{white}{\bfseries SOP Section} &
\textcolor{white}{\bfseries\scriptsize\applies} \\
\midrule
\endfirsthead
\multicolumn{4}{@{}l@{}}{\small\itshape\textcolor{navy!70!black}{\tablename~\thetable{} (cont.)}} \\[2pt]
\rowcolor{navy}
\textcolor{white}{\bfseries Reference} &
\textcolor{white}{\bfseries Relevance to this SOP} &
\textcolor{white}{\bfseries SOP Section} &
\textcolor{white}{\bfseries\scriptsize\applies} \\
\midrule
\endhead
\midrule
\multicolumn{4}{@{}r@{}}{\itshape\textcolor{navy!70!black}{Continued on next page\ldots}} \\
\endfoot
\bottomrule
\endlastfoot
\textbf{AASB~116} \textit{Property, Plant and Equipment} & Capitalisation criteria and component accounting (AASB~116.43): where a component of a complex asset is replaced, the replaced component is derecognised (with any resulting gain or loss in P\&L) and the new component is capitalised. Applied by analogy to investment property expenditure classification. & See \S\ref{sec:est-capex-accounting}; \S\ref{sec:est-repair-integration} & \applies \\
\rowcolor{warmgray}
\textbf{AASB~140} \textit{Investment Property} & Investment property recognition, fair value measurement, and disclosure. All direct costs of acquisition (excluding borrowing costs) are capitalised. Subsequent expenditure is capitalised only if it increases the asset's economic benefits above the existing standard of performance. Fair value is assessed at each period end; any revaluation gain/loss is in P\&L. & See \S\ref{sec:est-reval}; \S\ref{sec:est-capex-accounting} & \applies \\
\textbf{AASB~9} \textit{Financial Instruments} & Expected credit loss (ECL) model for trade receivables and loans. Also governs: modification vs.\ extinguishment analysis for the St~George facility (paras 3.3.1--3.3.3); fair value measurement of interest rate swaps; recognition of financial guarantee contracts. & See \S\ref{sec:est-baddebt}; \S\ref{sec:est-loan-swap} & \applies \\
\rowcolor{warmgray}
\textbf{AASB~101} \textit{Presentation of Financial Statements} & Going concern assessment (required at every period end). Loan current/non-current reclassification: the St~George facility maturing 15~June~2026 must be presented as current if no refinancing or extension is in place. \dag~\textbf{AASB~2020-1 and AASB~2022-6 amendments (effective 1~January~2024):} These amendments clarify that a liability is classified as non-current only if the entity has an \emph{unconditional right} to defer settlement for at least 12~months after the reporting date. The right must exist \emph{at the end of the reporting period}; the likelihood of exercising it is irrelevant. For loans subject to covenants, the right to defer settlement is affected by the entity's compliance with those covenants at reporting date. For the St~George facility, no unconditional right to defer exists beyond the maturity date of 15~June~2026 --- the loan must be classified as current. Additional disclosures are required for non-current liabilities subject to covenants. & See \S\ref{sec:est-loan-reclass} & \applies \\
\textbf{AASB~110} \textit{Events After the Reporting Period} & Post-balance date events: adjusting events (recognised in the current period) vs.\ non-adjusting events (disclosed only). Property sale exchange, insurance settlement, loan refinancing, and PCO deed execution after balance date are potential triggering events. & See \S\ref{sec:est-other} & \applies \\
\rowcolor{warmgray}
\textbf{AASB~124} \textit{Related Party Disclosures} & CPG management fees, disposal fees, leasing commissions, and any inter-company balances are related party transactions requiring disclosure at arm's-length pricing. The trust's relationship with CPG (as responsible entity) must be disclosed. & See \S\ref{sec:background} & \applies \\
\textbf{AASB~137} \textit{Provisions, Contingent Liabilities and Contingent Assets} & Contingent liabilities (warranty claims from the sale, pending litigation, environmental remediation, make-good obligations) must be disclosed if possible but not probable; recognised if probable. Contingent assets (insurance claim receivable) must be disclosed if probable. & See \S\ref{sec:est-bankguarantees}; \S\ref{sec:est-insurance} & \applies \\
\rowcolor{warmgray}
\textbf{AASB~16} \textit{Leases} & Lease incentive straight-lining and STI calculation. A change in lease term (renewal, early termination, or modification) requires recalculation of the straight-line rent revenue profile. & See \S\ref{sec:est-leasing} & \applies \\
\end{longtable}
\endgroup


% ============================================================================
\subsection{Held-For-Sale and Asset Presentation Standards}
\label{sec:ref-hfs}
% ============================================================================

\begingroup
\small\sffamily
\setlength{\LTpre}{4pt}
\setlength{\LTpost}{4pt}
\renewcommand{\arraystretch}{1.2}
\begin{longtable}{@{}L{3.0cm} p{\dimexpr\linewidth-3.0cm-3.3cm-1.0cm-6\tabcolsep\relax} L{3.3cm} C{1.0cm}@{}}
\caption{Held-for-sale and asset presentation references --- applicability to 22 Constance Street} \label{tab:ref-hfs} \\
\rowcolor{navy}
\textcolor{white}{\bfseries Reference} &
\textcolor{white}{\bfseries Relevance to this SOP} &
\textcolor{white}{\bfseries SOP Section} &
\textcolor{white}{\bfseries\scriptsize\applies} \\
\midrule
\endfirsthead
\multicolumn{4}{@{}l@{}}{\small\itshape\textcolor{navy!70!black}{\tablename~\thetable{} (cont.)}} \\[2pt]
\rowcolor{navy}
\textcolor{white}{\bfseries Reference} &
\textcolor{white}{\bfseries Relevance to this SOP} &
\textcolor{white}{\bfseries SOP Section} &
\textcolor{white}{\bfseries\scriptsize\applies} \\
\midrule
\endhead
\midrule
\multicolumn{4}{@{}r@{}}{\itshape\textcolor{navy!70!black}{Continued on next page\ldots}} \\
\endfoot
\bottomrule
\endlastfoot
\textbf{AASB~5\dag} \textit{Non-current Assets Held for Sale and Discontinued Operations} (Aug.\ 2015, compiled to include amendments effective 1~July~2021) & Specifies the criteria for classifying a non-current asset as ``held for sale'': (a)~the asset must be available for immediate sale in its present condition; and (b)~the sale must be highly probable (management committed, active marketing at a reasonable price, expected completion within 12~months). \textbf{Critical carve-out for investment property:} AASB~5 (para.~5(d)) explicitly excludes investment property measured at fair value under AASB~140 from the AASB~5 \emph{measurement} requirements (lower of carrying amount and FVLCTS). Accordingly, 22~Constance Street, carried at fair value under AASB~140, is not reclassified to a held-for-sale balance sheet line and continues to be measured at fair value until derecognition. However, the \emph{classification criteria} (paras.~6--14) remain informative for determining the disclosure obligations that arise from the property being actively marketed. AASB~5 also governs the accounting for a ``discontinued operation'' --- not currently applicable because the fund's sole activity is the operation and disposal of a single investment property, which does not constitute a separate major line of business for AASB~5 purposes. & See \S\ref{sec:est-hfs}; \S\ref{sec:est-hfs-accounting} & \applies \\
\rowcolor{warmgray}
\textbf{AASB~140} \textit{Investment Property} (para.\ 57--58 and 75) & When an entity decides to dispose of an investment property without development, AASB~140 (para.~58) requires the property to continue to be classified as investment property until it is derecognised (removed from the balance sheet at settlement). There is no reclassification to a held-for-sale category. The fair value movement recognised in P\&L under para.~35 continues to apply at each reporting date using the best available evidence --- for a property under a binding sale contract, the net contract price is the most reliable measure of fair value. Para.~75 requires disclosure of investment property assets that are the subject of disposal agreements or classified as held for sale (per AASB~5) in the movement reconciliation. This disclosure must identify the property, the sale price, and the expected settlement date. & See \S\ref{sec:est-hfs-accounting}; \S\ref{sec:est-reval} & \applies \\
\textbf{AASB~5 (para.~6--9) --- ``Highly Probable'' Criterion} & A sale is ``highly probable'' for AASB~5 purposes when: (a)~management is committed to a plan to sell; (b)~an active programme to locate a buyer and complete the plan is underway; (c)~the asset is being actively marketed at a price reasonable in relation to its current fair value; (d)~the sale is expected to qualify for recognition as a completed sale within 12~months; and (e)~it is unlikely the plan will be significantly changed or withdrawn. For 22~Constance Street, with a binding sale contract in place, all five criteria are satisfied. The 12-month completion requirement must be re-assessed at each reporting date if settlement is delayed. & See \S\ref{sec:est-hfs} & \applies \\
\rowcolor{warmgray}
\textbf{AASB~101} \textit{Presentation of Financial Statements} (para.\ 54--55, 69--76) \dag & Para.\ 54(j) requires non-current assets held for sale (classified under AASB~5) to be presented as a separate line in current assets on the face of the balance sheet. Because AASB~5 measurement requirements do not apply to investment property under the AASB~140 fair value model, 22~Constance Street is not presented in this line. However, AASB~101 (para.\ 69--76 as amended by AASB~2020-1 and AASB~2022-6) governs the classification of the St~George loan as a current liability, which is separately affected by the HFS context (sale proceeds will discharge the loan). The fund must ensure the current/non-current classification of all liabilities is consistent with the expected settlement timeline given the approaching disposal. & See \S\ref{sec:est-hfs-accounting}; \S\ref{sec:est-loan-reclass} & \applies \\
\textbf{AASB~2020-1 and AASB~2022-6\dag} \textit{Amendments to AASB~101 --- Classification of Liabilities as Current or Non-current; Non-current Liabilities with Covenants} (effective for annual periods beginning on or after 1~January~2024) & These amendments clarify the rules for classifying liabilities as current or non-current under AASB~101. Key changes: (a)~a liability is non-current only if the entity has an \emph{unconditional right} to defer settlement for at least 12~months after the reporting date; (b)~the classification is unaffected by the likelihood that the entity will exercise the right; (c)~rights conditional on future covenant compliance do not establish a right at reporting date if the covenants must be complied with \emph{after} reporting date; and (d)~additional note disclosures are required where a non-current liability is subject to covenants that could require early repayment within 12~months after reporting date. For the St~George facility, no right to defer beyond 15~June~2026 exists --- the loan must be classified as current under these amendments (which were mandatory from 1~July~2024 for entities with a 30~June year end). & See \S\ref{sec:est-loan-reclass} & \applies \\
\rowcolor{warmgray}
\textbf{AASB~110} \textit{Events After the Reporting Period} & Post-balance date events arising from the sale process require classification as adjusting (recognised) or non-adjusting (disclosed only) under AASB~110. An unconditional contract of sale entered into \emph{after} the balance date is a non-adjusting event (does not trigger derecognition in the current period) but must be disclosed if material. Settlement of the property after balance date is also a non-adjusting event requiring full disclosure: description of the nature and financial effect (derecognition of investment property, repayment of St~George, gain/loss on disposal, and net distribution to unitholders). & See \S\ref{sec:est-hfs-accounting}; \S\ref{sec:est-other} & \applies \\
\textbf{AASB~137} \textit{Provisions, Contingent Liabilities and Contingent Assets} & In the HFS context, several contingent liabilities arise from vendor obligations under the sale contract: warranty and indemnity provisions; make-good obligations for leases terminating on sale; potential vendor bank guarantees provided to the purchaser. AASB~137 governs whether these are (a)~recognised as provisions (probable, reliably estimable); (b)~disclosed as contingent liabilities (possible but not probable); or (c)~neither (remote). Each vendor warranty exposure must be individually assessed by K\&L Gates and CPG and treated accordingly at each reporting date from the time of exchange. & See \S\ref{sec:est-hfs-legal}; \S\ref{sec:est-bankguarantees} & \applies \\
\rowcolor{warmgray}
\textbf{s104-10 ITAA~1997 --- CGT Event~A1 Timing} & For a conditional contract, CGT Event~A1 occurs when the last condition is satisfied (not at exchange). For an unconditional contract, A1 occurs at the \emph{contract date}, not settlement date. This means the fund may recognise the capital gain in an income year \emph{earlier} than cash settlement --- a common source of tax surprises in property fund disposals. The CGT event date must be determined by K\&L Gates and confirmed with the tax agent before finalising distribution components for the year. & See \S\ref{sec:est-hfs-tax}; \S\ref{sec:est-cgt} & \applies \\
\textbf{s40-295 ITAA~1997 --- Div~40 Balancing Adjustment on Disposal} & When a depreciating asset (Div~40 plant and equipment) is disposed of as part of a property sale, a balancing adjustment equal to the termination value less the adjustable value is included in assessable income (positive balance) or allowed as a deduction (negative balance). The termination value is the portion of the sale proceeds allocated to the plant and equipment (determined by independent valuation of plant component or by apportionment). The tax agent must prepare the final Div~40 depreciation schedule for the disposal year and allocate the sale proceeds between land, building (Div~43), and plant (Div~40). & See \S\ref{sec:est-hfs-tax}; \S\ref{sec:est-capex-taxdivision} & \applies \\
\rowcolor{warmgray}
\textbf{GSTR~2002/5 and GSTR~2015/1 --- GST: Sale as Going Concern} & The sale of a commercial property may be GST-free if it qualifies as the supply of a ``going concern'' under s38-325 GST Act~1999 (the property is supplied with everything necessary to carry on the enterprise; the vendor and purchaser agree in writing to treat it as a going concern). For 22~Constance Street, with partial occupancy and multiple leases, the going-concern exemption requires careful analysis of whether the enterprise being conducted (property investment and leasing) is supplied with all its essential components. This must be determined before the sale contract is executed. Alternatively, if the margin scheme applies (Div~75 GST Act), the GST is calculated on the margin (sale price less acquisition cost) rather than the full price. & See \S\ref{sec:est-hfs-tax} & \applies \\
\textbf{\textit{Corporations Act 2001} s601FC --- Responsible Entity Duties} & As responsible entity, CPG must act in the best interests of unitholders at all times, including in connection with the sale of the trust's single asset. CPG must: (a)~obtain independent valuations to confirm the sale price is at or above fair value; (b)~document the decision-making process for the sale (board resolutions, disclosure to unitholders); (c)~ensure the sale contract is arm's length; and (d)~comply with any unitholder notification or consent requirements under the trust deed and ASIC regulatory guides. Failure to comply with these obligations may expose CPG to liability and the sale to challenge. & See \S\ref{sec:est-hfs-legal} & \applies \\
\end{longtable}
\endgroup



\vfill
\begin{center}
\textcolor{gold}{\rule{4cm}{0.5pt}}\hspace{0.5em}\textcolor{navy}{\rule{4cm}{1.2pt}}\hspace{0.5em}\textcolor{gold}{\rule{4cm}{0.5pt}}\\[0.7em]
{\sffamily\large\bfseries\textcolor{navy}{End of Document}}\\[0.3em]
{\sffamily\footnotesize\textcolor{textgray}{22 Constance St Unit Trust \textbullet{} SOP v3.7 \textbullet{} March 2026 \textbullet{} Confidential \& Internal}}\\[0.4em]
\textcolor{gold}{\rule{2cm}{0.3pt}}\hspace{0.3em}{\sffamily\tiny\textcolor{steel}{$\blacklozenge$}}\hspace{0.3em}\textcolor{gold}{\rule{2cm}{0.3pt}}
\end{center}

\end{document}
